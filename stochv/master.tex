\documentclass[
 a4paper,
 12pt,
 parskip=half
 ]{scrreprt}

\usepackage{../.tex/settings}

\usepackage{../.tex/mathpkgs}
\usepackage{../.tex/mathcmds}

\usepackage[numbers_per_chapter]{../.tex/fancy_thm}

\usepackage{siunitx}
\sisetup{locale = DE}

\theoremstyle{plain}

\theoremstyle{definition}
\newtheorem{defn}[thm]{Definition}
\newtheorem{folg}[thm]{Folgerung}
\newtheorem{rmrk}[thm]{Bemerkung}
%\newtheorem{deno}[thm]{Bezeichnungen}
\newtheorem{exmp}[thm]{Beispiel}
%\peripherynewtheorem{aufg}[thm]{Aufgabe} 
%\newtheorem{prgp}[thm]{} % Numbered paragraph

\newtheorem*{rmrk*}{Bemerkung}
%\newtheorem*{exmp*}{Beispiel}
%\newtheorem*{defn*}{Definition}
%\newtheorem*{deno*}{Bezeichnungen}

\numberwithin{equation}{chapter}

\hypersetup{
  pdftitle={STOCHV},
  pdfauthor={Jonas Hippold},
  hidelinks
}

%opening
\title{Vorlesung\\Vertiefung Stochastik - Statistik}
\subtitle{Wintersemester 2017}
\author{Vorlesung: Anita Behme\\Mitschrift: Jonas Hippold}

\begin{document}

\maketitle

\tableofcontents

\clearpage

\section{Einleitung}
\subsection{Aufgabenstellung und Grundbegriffe}
Es seien $G \subset \real^n$ und $f: G \to \real$ gegeben. Optimierungsproblem:
\begin{equation}
  f(x) \to \min \quad \text{bei } x \in G.
\end{equation}
$f$ ... Zielfunktion \\
$G$ ... Zulässiger Bereich \\
$x \in G$ ... Zulässiger Punkt, zulässiges Element, zulässige Lösung

$x^* \in G$  heißt \emph{potimal} oder \emph{optimale Lösung} oder
\emph{Lösung}, falls
\begin{equation}
  f(x^*) \le f(x) \quad \text{für alle } x \in G.
\end{equation}
$f^* = f_{\min} := f(x^*)$ ... Optimalwert

Falls $G = \real^n$, so heißt (1.1) \emph{freies} oder \emph{unrestringiertes}
Optimierungsproblem (OP).

(1.1) ist ein \emph{diskretes} OP, falls $G$ eine
\emph{diskrete} Menge ist, zum Beispiel $G = \integer^n$.

(1.1) ist ein \emph{stetiges} oder \emph{kontinuierliches} OP, falls alle
Variablen ``stetig'' sind. Sonst gemischt-ganzzahliges Problem.

(1.1) ist ein \emph{lineares} OP, falls $f(x) = c^T x$ und $G$ durch lineare
Restklassen gegeben ist:
\begin{equation}
  G = \{ x \in \real^n : g_i(x) \le 0, i \in I, h_j(x) = 0, j \in J \},
\end{equation}
wobei $g_i, h_j$ für alle $i,j$ affin linear sind. In diesem Fall kann (1.1) als
\[ c^T x \to \min \quad \text{bei } x \in G:= \{ x \in \real^n, Ax = a, Bx \le b
  \} \]
geschrieben werden.

\begin{defn}
  Betrachtet werden die beiden OP
  \begin{align*}
    f(x) &\to \min \quad \text{bei } x \in D \cap E, \tag{P} \\
    g(x) &\to \min \quad \text{bei } x \in E. \tag{Q}
  \end{align*}
  (Q) heißt \emph{Relaxation} zu (P), falls $g(x) \le f(x)$ auf $D \cap E$. Der
  Optimalwert von (Q) kann als (untere) Schranke bzw. Schrankenwert für (P)
  bezeichnet werden.
\end{defn}

\begin{thm}
  Sei $\obar{x}$ Lösung von (Q) und gelte $\obar{x} \in D$ sowie $f(\obar{x}) =
  g(\obar{x})$. Dann ist $\obar{x}$ Lösung von (P).
\end{thm}

\begin{proof}
  Übung.
\end{proof}

\begin{defn}
  Seien (Q1) und (Q2) Relaxationen zu (P). (Q1) heißt \emph{stärker} (strenger)
  als (Q2), wenn die Schranke von (Q1) größer oder gleich der Schranke von (Q2)
  für \emph{jede} Instanz von (P) ist.
\end{defn}

\subsection{Beispiele zur kontinuierlichen Optimierung}
\subsubsection{Transportoptimierung}
Lineare Optimierung

Es seien Erzeuger $i \in I := \{1, \ldots, n\}$ und Verbraucher $j \in J := \{1,
\ldots, n\}$ gegeben. Weiterhin seien die Kosten $c_{ij}$ für den Transport
\emph{einer} Einheit von $i$ nach $j$ sowie der Vorrat $a_i > 0$ und der
Verbrauch $b_j > 0$ für alle $i \in I$ und $j \in J$ bekannt. Wie muss der
Transport organisiert werden, damit die Gesamtkosten minimal sind?

Variable $x_{ij} \ge 0$ ... Transportmenge von $i$ nach $j$.
\[ \sum_{i \in I} \sum_{j \in J} c_{ij} \cdot x_{ij} \to \min \quad \text{bei }
  \sum_{j \in J} x_{ij} = a_i, \quad i \in I. \]

\subsubsection{Kürzeste euklidische Entfernung eines Punktes zu einer Menge}
Nichtlineare Optimierung

Gegeben sind ein Punkt $\tilde{x}$ und Menge $G \subset \real^n$ mit $\tilde{x}
\notin G$.
\[ f(x) := \rez{2} \| \tilde{x} - x \|_2^2 \to \min \quad \text{bei } x \in
G. \]
Falls $G$ konvex ist, dann ist es der Spezialfall der \emph{konvexen
  Optimierung}.

\subsubsection{Tschebyscheff-Approximation}
Semi-infinite Optimierung

Seien $f: \real \to \real$ und $S: \real^{n+1} \to \real$ stetig, zum Beispiel
\[ S(y, x_1, \ldots, x_n) = \sum_{i=1}^n x_i s_i(y), \]
wobei $s_i$ ein Ansatz ist für
\[ \max_{y \in [a,b]} | f(y) - S(y, x_1, \ldots, x_n) | \to \min_{x \in
    \real^n}. \]
Umformulierung:
\[ \tilde{f}(x,\lambda) := \lambda \to \min \quad \text{bei } - \lambda \le
  f(y) - S(y, x_1, \ldots, x_n) \le \lambda \text{ für alle } y \in [a,b]. \]
Das ist ein Beispiel mit endlich vielen Variablen und unendlich vielen
Restriktionen.

\subsection{Beispiele zur diskreten Optimierung}
\subsubsection{Rucksackproblem}
Gegeben sind ein Rucksack mit Volumen $b$, Teile mit Volumen $a_i$ und Bewertung
$c_i$, $i \in I$.

Voraussetzung: $a_i, c_i, b \in \integer_{>0}$, $0 < a_i \le b$ für alle $i \in
I = \{1, \ldots, n\}$.

\paragraph{0/1-Rucksackproblem}
Entscheidungsvariable $x_i \in \boole := \{0,1\}$.
\[ \sum_{i \in I} c_i x_i \to \max \quad \text{bei } \sum_{i \in I} a_i x_i \le
  b, x_i \in \boole. \]
Alternative Formulierung:
\[ \sum_{i \in \tilde{I}} \to \max \quad \text{bei } \sum_{i \in \tilde{I}} a_i
  \le b, \tilde{I} \subset I. \]

\paragraph{Klassisches (Standard-)Rucksackproblem}
\[ \sum_{i \in I} c_i x_i \to \max \quad \text{bei } \sum_{i \in I} a_i x_i \le
  b, x_i \in \integer_+ \text{ für alle } i. \]
Hier bezeichnet $x_i$ die Anzahl, wie oft Teil $i$ mitgenommen wird.

\subsubsection{Das eindimensionale Zuschnittproblem}
Cutting Stock Problem, CSP

Aus möglichst wenig Ausgangsmaterial der Länge $L$ sind $b_i$ Teile der Länge
$l_i$, $i \in I := \{ 1, \ldots, n \}$ zuzuschneiden.

Zuschnittvariante $a_j = (a_{1j}, \ldots, a_{nj})^T \in \integer_+^n$ mit
$\sum_{i \in I} l_i a_{ij} \le L$ für alle $j \in J$.

\paragraph{Kantorovich-Modell}
Variablen: $a_{ij}$ Anzahl, wie oft Teil $i$ aus ZV $j$ erhalten wird, $y_i
\in \boole$ Entscheidungsvariable.
\[ \begin{aligned}
    z &= \sum_{j \in J} y_j \to \min \quad \text{bei } &
    &\sum_{i \in I} l_i \cdot a_{ij} \le L \cdot y_j, & \forall j \in J, \\
    & &
    &\sum_{j \in J} a_{ij} = b_i, & i \in I, \\
    & &
    &a_{ij} \in \integer_+, y_j \in \boole, & \forall i,j.
  \end{aligned} \]
Der Optimalwert $z_{LP}^*$ der stetigen (bzw. LP) Relaxation ist gleich der
Materialschranke $= \rez{L} \sum_{i \in I} l_i b_i$. Bei LP-Relaxation gilt
\[ a_{ij} \ge 0, \quad 0 \le y_i \le 1, \quad \forall i, j. \]

\paragraph{Gilmore/Gomory-Modell (1961)}
Annahme: \emph{Alle} ZV sind bekannt, das heißt die $a_{ij}$ sind Koeffizienten.

Variablen: $x_j \in \integer_+$ ... Anzahl, wie oft Variante $j$ genommen wird.
\[ \begin{aligned}
    z &= \sim_{i \in J} x_j \to \min \quad \text{bei } &
    \sum_{j \in J} a_{ij} x_j = b_i, & \forall i \in I \\
    & &
    x_j &\ge 0, \text{ ganzzahlig}, & \forall j \in J.
  \end{aligned} \]

Vermutung: $z^* - z_{LP} < 2$ für alle Instanzen.

Bemerkung: Das größte bekannte Gap ist 1.

\subsubsection{Facility Location Problem}
Ein großer Dienstleistungsbetrieb möchte neue Filialen aufbauen, um den Kunden
$k \in K$ mit Leistungen zu dienen. Aus der Menge $S := \{1, \ldots, n\}$ der
möglichen Standorte $s$ ist eine Auswahl zu treffen, an denen die Filialen
aufgebaut werden.

Ziel ist, die Gesamtkosten, das heißt die Kosten zum Aufbau
und während des Betriebs, zu minimieren.

$d_{ks} \in \real_+$ ... Kosten, um Kunde $k$ von Standort $s$ aus zu bedienen,
\\
$c_s \in \real_+$ ... einmalige Kosten zum Einrichten der Filiale $s$, \\
$x_j \in \boole$ ... Entscheidungsvariable, \\
$y_{ks} \in [0,1]$ ... Teil des Bedarfs von $k$, der von $s$ erledigt wird.
\[ \begin{aligned}
    z &= \sum_{s \in S} c_s x_s + \sum_{k \in K} \sum_{s \in S} d_{ks} y_{ks} \to
    \min \quad \text{bei } &
    &\sum_{s \in S} y_{ks} = 1, & k \in K, \\
    & &
    &y_{ks} - x_s \le 0, & \forall k, s, \\
    & &
    &y_{ks} \ge 0, x_s \in \boole, & \forall k, s.
  \end{aligned}
\]

\subsubsection{Quadratic Alignment Problem}
In einem Gebäude müssen $m$ Personen auf $n$ Räume verteilt werden. Person $i$
muss Person $j$ persönlich treffen und zwar $c_{ij}$-mal täglich ($c_{ij} \ge
0$). Büro $k$ hat von Büro $l$ die Entfernung $d_{kl} \ge 0$.

Wird Person $i$ dem Raum $k$ und Person $j$ dem Raum $l$ zugeordnet, so muss $i$
die Entfernung $2 d_{kl} c_{ij}$ zurücklegen.

\paragraph{Permutationsmodell}
\[ f(x) = \sum_{i=1}^n \sum_{j=1}^n 2 c_{ij} d_{\pi(i) \pi(j)} \to \min \quad
  \text{bei } \pi \in \Pi(1, \ldots, n). \]

\paragraph{Modellierung mit bodeschen Variablen}
\[ x_{ik} := \begin{cases}
    1, &\text{falls Person $i$ in Zimmer $k$}, \\
    0, &\text{sonst.}
  \end{cases} \]
\[ z = 2 \cdot \sum_{i \in I} \sum_{j \in J} \sum_{k \in K} \sum_{l \in L} c_{ij}
  x_{ik} \cdot x_{jl} \to \min \]
mit den Zuordnungsbedingungen
\[ \begin{aligned}
    \sum_{k = 1}^n x_{ik} &= 1 & \text{für alle } i, \\
    \sum_{i = 1}^n x_{ik} &= 1 & \text{für alle } k
  \end{aligned} \]
bei $x_{ik} \in \boole$.

\subsubsection{Ganzzahlige lineare Optimierung}
\[ z = c^T x \to \min \quad \text{bei} \quad Ax \le b, x \ge 0, x \in
  \integer. \]


\chapter{Deskriptive Statistik}
Gegeben sei stets $(x_1, \ldots, x_n)$, eine Stichprobe vom Umfang $n \in \nat$.
Die $x_i$ können dabei als Realisierungen der i.i.d. Zufallsvariablen $X_i$, $i
= 1, \ldots, n$ mit Verteilungsfunktion $F$ (unbekannt) aufgefasst werden.

\section{Empirische Verteilung und ihre Darstellung}
Frage: Wie können wir $F$ beschreiben und darstellen?

\begin{defn}
  Im Fall eines diskreten Merkmals mit den Ausprägungen $a_1, \ldots, a_m$
  bezeichnet
  \[ n_j := \sharp \{ x_i, i = 1, \ldots, n : x_i = a_j \}, \quad j =1, \ldots,
    m \]
  die \emph{absolute Häufigkeit},
  \[ r_j := \frac{n_j}{n} \]
  die \emph{relative Häufigkeit}
  der $a_j$.
\end{defn}

Diese Häufigkeiten lassen sich gut in Balken-, Säulen- oder Kreisdiagrammen
veranschaulichen.

Bei stetigen Merkmalen gilt typischerweise $n \approx m$, deshalb werden diese
Daten zunächst gruppiert: Bilde Klassen $I_j := (b_j, b_{j+1}]$, $j = 1, \ldots,
l$, wobei
\begin{itemize}
\item $b_1 \le \min \{ x_i, i = 1, \ldots, n \}$ und $b_{l+n} \ge \max\{x_i, i =
  1, \ldots, n\}$.
\item Die Klassenbreite $b_{j+1}-b_j$ ist konstant.
\item Faustregel: $5 \le l \le 25$ und $l \approx \sqrt{n}$.
\end{itemize}

%% TODO: Hier fehlt ein Unterpunkt. Im Skript nachschauen!
\addtocounter{thm}{1}
\begin{defn} %2.3
  Für ein stetiges Merkmal, welches in die Klassen $I_j = (b_j, b_{j+1}]$, $j =
  1, \ldots, l$ gruppiert wurde, sind
  \[ n_j := \sharp \{ x_i \in (b_j, b_{j+1} ], \quad j = 1, \ldots, l \} \]
  die \emph{absoluten Häufigkeiten} und
  \[ r_j = \frac{n_j}{n} \]
  die \emph{relativen Häufigkeiten} der Klassen $I_j$.
\end{defn}

Diese werden typischerweise in Histogrammen visualisiert.

\begin{exmp}[Histogramme in R]
  Die bisher genannten Methoden dienten zur Visualisierung der empirischen
  (Zähl-)Dichte des beobachteten Merkmals. Aus diesem können die ersten
  Rückschlüsse auf die unbekannte Verteilung $F$ bzw. ihre Dichte gezogen
  werden. Zum Beispiel können wir bei quantitativen Daten erkennen, ob $F$
  \emph{symmetrisch}, \emph{rechtsschief/linkssteil},
  \emph{linksschief,rechtssteil} ist, oder ob $F$ \emph{unimodal},
  \emph{bimodal}, \emph{multimodal} ist.
\end{exmp}

Neben der Dichte kann auch die empirische Verteilungsfunktion direkt betrachtet
werden:
\begin{defn}
  Für ordinalskalierte Daten $(x_1, \ldots, x_n)$ mit den Ausprägungen $a_1 <
  a_2 < \ldots < a_n$ sind die \emph{kumulierten Häufigkeiten}
  \[ H(x) := \sharp\{ x_i : x_i \le x \} = \sum_{i=1}^n \ind_{x_i \le x} =
    \sum_{j : a_j < x} n_j \]
  definiert. Die Funktion
  \[ \hat{F}_n(x) := \frac{H(x)}{n} \]
  ist dann die \emph{empirische Verteilungsfunktion} der Stichprobe $(x_1,
  \ldots, x_n)$.
\end{defn}

\begin{exmp}
  Die empirische Verteilungsfunktion in R: \verb+ecdf()+ (empirical cumulative
  distribution function). Geplottete Ausgabe mit \verb+plot(ecdf())+.
\end{exmp}

\begin{prp}[Eigenschaften der empirischen Verteilungsfunktion]
  Die empirische Verteilungsfunktion $\hat{F}_n(x) : \real \to [0,1]$ einer
  Stichprobe $(x_1, \ldots, x_n) \in \real^n$  erfüllt:
  \begin{enumerate}
  \item $\hat{F}_n(x)$ ist monoton wachsend.
  \item $\hat{F}_n(x)$ ist rechtsseitig stetig.
  \item Es gilt
    \[ \lim_{x \to - \infty} \hat{F}_n(x) = 0, \qquad \lim_{x \to \infty}
      \hat{F}_n(x) = 1. \]
  \end{enumerate}
\end{prp}
Das heißt, $\hat{F}_n$ ist eine Verteilungsfunktion.

\begin{proof}
  Übung.
\end{proof}

\section{Kenngrößen empirischer Verteilungen}

\begin{center}
  \includegraphics[width=.95\textwidth]{img/kenngroessen}
\end{center}

Gegeben ist: Stichprobe $(x_1, \ldots, x_n)$, wobei die $x_i$ als Realisierungen
von i.i.d. Zufallsvariablen $X_i \sim F$ aufgefasst werden können.

\subsection{Lagemaße}
\paragraph{Empirischer Erwartungswert bzw. arithmetisches Mittel}
Für quantitative Daten:
\[ \bar{x} := \rez{n} \sum_{i=1}^n x_i. \]
Das arithmetische Mittel reagiert empfindlich auf Ausreißer in den Daten,
speziell bei kleinen Datensätzen.

R-Befehl: \verb+mean(.)+

\paragraph{Geometrisches Mittel}
Für Daten in (Q$\infty$) definiert:
\[ \bar{x}_g = \sqrt[n]{ x_1 \cdot \ldots \cdot x_n }. \]
Wird hauptsächlich für Wachstum und Zinsfaktoren verwendet. Es gilt
\[ \log \bar{x}_g = \rez{n} \sum_{i=1}^n \log x_i \le \log \left( \rez{n}
    \sum_{i=1}^n x_i \right) = \log \bar{x}, \]
also auch $\bar{x}_g \le \bar{x}$, mit Gleichheit genau dann, wenn $x_1 = x_2 =
\ldots = x_n$.

R-Befehl: \verb+exp(mean(log(.)))+

\paragraph{Harmonisches Mittel}
Für Daten in $\real \setminus \{0\}$, so dass $\rez{n} \sum x_i^{-1} \ne 0$:
\[ \bar{x}_h := \left( \rez{n} \sum_{i=1}^n x_i^{-1} \right)^{-1} \]
(z.B. für gemittelte Geschwindigkeiten)

Sind zum Beispiel die $x_i$ Geschwindigkeiten, mit denen Bauteile eine
Produktionslinie der Länge $l$ durchlaufen, dann ergibt sich die
Gesamtbearbeitungszeit $\frac{l}{x_1} + \ldots + \frac{l}{x_n}$. Also ist
\[ \bar{x}_h := \rez{\rez{n} \sum_{i=1}^n x_i^{-1}} =
  \frac{l + \ldots + l}{\frac{l}{x_1} + \ldots + \frac{l}{x_n}} \]
die Durchschnittsgeschwindigkeit der Bauteile.

\paragraph{Gewichtete/getrimmte Mittel}
Zur Begrenzung des Einflusses von Ausreißern oder um Teildatensätze stärker zu
gewichten, verwendet man getrimmte oder gewichtete Mittel:
\begin{itemize}
\item Beim 10 \%1-getrimmten Mittel werden die 10 \% der niedrigsten und die 10
  \% der höchsten Ausprägungen (insgesamt also 20 \% der Daten) weggelassen.
\item Für gewichtete Mittel definiere zunächst Gewichte $w_i \ge 0$, $i = 1,
  \ldots, n$ mit $\sum_{i=1}^n w_i = 1$ und berechne zum Beispiel
  \[ \bar{x}_w = \sum_{i=1}^n w_i x_i. \]
\end{itemize}

\paragraph{Ordnungsstatistik, Median und empirische Quantile}
Die Ordnungsstatistiken $x_{(i)}$, $i = 1, \ldots, n$ für die Stichprobe $(x_1,
\ldots, x_n)$ sind durch die Permutation der Stichprobe gegeben, für welche
\[ x_{(1)} \le \ldots \le x_{(n)} \]
gilt.

Der empirische Median ist dann
\[ x_{\text{med}} := \begin{cases}
    \rez{2} (x_{(n/2)} + x_{(n/2 + 1)}), & n \text{ gerade,} \\
    x_{((n+1)/2)}, & n \text{ ungerade,}
  \end{cases} \]
so dass also mindestens 50 \% der Daten kleiner oder gleich bzw. größer oder
gleich $x_{\text{med}}$ sind.

Allgemeiner ist für $\alpha \in [0,1]$;
\[ x_{\alpha} := \begin{cases}
    \rez{2} (x_{(\alpha n)} + x_{(\alpha n + 1)}), & \alpha n \in \nat, \\
    x_{(\lfloor \alpha n \rfloor + 1)}, & \alpha n \notin \nat,
  \end{cases} \]
das \emph{empirische $\alpha$-Quantil}.

Insbesondere: $x_{1/2} = x_{\text{med}}$; $x_{0,25}$ und $x_{0,75}$ sind das
\emph{untere} und \emph{obere Quantil}.

Die Informationen aus den Quantilen lassen sich gut in einem \emph{Boxplot}
visualisieren.

Der Median ist ein robustes Lagemaß, da er gegenüber Ausreißern nicht sensibel
ist. beachte zudem, dass auf der Modellebene entsprechende Definitionen gelten:
\begin{defn} %2.8
  Sei $X$ reellwertige Zufallsvariable mit Verteilungsfunktion $F$. Die
  verallgemeinerte Inverse von $F$
  \[ F^{\leftarrow}(y) = \inf \{ x \in \real, F(x) \ge y \}\]
  heißt auch \emph{Quantilfunktion} von $F$ bzw. $X$.

  Für $\alpha \in [0,1]$ heißt $F^{\leftarrow}(\alpha)$ \emph{$\alpha$-Quantil}
  von $F$ bzw. $X$. Das $0.5$-Quantil wird auch als \emph{Median} bezeichnet.
\end{defn}

\begin{exmp}%2.9
  Boxplots für Dax-Daten.
\end{exmp}

\paragraph{Empirischer Modus / Modalwert}
$x_{\mathrm{mod}}$ ist die Ausprägung mit der größten Häufigkeit im Datensatz.

Der Modus ist sowohl für quantitative als auch qualitative Daten definiert,
sofern die Häufigkeitsverteilung ein eindeutiges Maximum besitzt.

Bei stetigen Merkmalen kann der Modus als Klassenmittelpunkt der Klasse mit der
größten Häufigkeit angegeben werden, hängt aber damit von der Klassenbildung ab.

Mittels Modus, Median und arithmetischem Mittel lassen sich Rückschlüsse auf die
Schiefe der Verteilung ziehen:\\
Bei einer symmetrischen Verteilung gilt etwa
\[ \obar{x} \approx x_{\mathrm{med}} \approx x_{\mathrm{mod}}, \]
bei einer rechtssteilen Verteilung
\[ \obar{x} < x_{\mathrm{med}} < x_{\mathrm{mod}}, \]
bei einer linkssteilen Verteilung
\[ \obar{x} > x_{\mathrm{med}} > x_{\mathrm{mod}}. \]

\subsection{Streuungsmaße}
Folgende Maßzahlen sind nur für quantitative Daten definiert. Sei also $(x_1,
\ldots, x_n)$ eine Stichprobe mit Werten in $\real$.

\paragraph{Empirische Varianz}
\[ s^2 := \rez{n-1} \sum_{i=1}^n (x_i - \obar{x})^2. \]
Bei der Berechnung helfen:
\begin{lem}[Verschiebungssatz] \label{lem:verschiebung} %2.10
  Für alle $a \in \real$ gilt
  \[ \sum_{i=1}^n (x_i - a)^2 = \sum_{i=1}^n (x_i - \obar{x})^2 + n(\obar{x} -
    a)^2, \]
  sodass für $a = 0$ folgt
  \[ \frac{n-1}{n} s^2 = \left( \rez{n} \sum_{i=1}^n x_i^2 \right) - \obar{x}^2. \]
\end{lem}

\begin{proof}
  Nachrechnen:
  \begin{align*}
    \sum_{i=1}^n (x_i - a)^2
    &= \sum_{i=1}^n (x_i - \obar{x} + \obar{x} - a)^2 \\
    &= \sum_{i=1}^n ((x_i - \obar{x})^2
      + 2(x_i-\obar{x})(\obar{x}-a) + (\obar{x}-a)^2) \\
    &= \sum_{i=1}^n (x_i - \obar{x})^2 + n(\obar{x}-a)^2). \qedhere
  \end{align*}
\end{proof}

\begin{lem}[Transformationsregel] %2.11
  Für $y_i = a x_i + b$ folgt
  \[ s_y^2 = a^2 s_x^2, \]
  wobei $s_x^2$ und $s_y^2$ die empirischen Varianzen der $x_i$ bzw. $y_i$ sind.
\end{lem}

\begin{proof}
  Übung.
\end{proof}

R-Befehl: \verb+var(data)+.

\paragraph{Empirische Standardabweichung}
\[ s := \sqrt{s^2} \]
Hat, anders als die Varianz, dieselbe Maßeinheit wie die Daten. Analog zu Lemma
2.11 gilt die Transformationsregel
\[ s_y = |a| s_x \]
für $y_i = a x_i + b$.

R-Befehl: \verb+sd(data)+.

\paragraph{Mittlere lineare Streuung}
(auch mean absolute deviation, MAD)
\[ s_L := \rez{n} \sum_{i=1}^n | x_i - \obar{x} | \]

\paragraph{Interquartilabstand}
(inter quartile range)
\[ \operatorname{IQR} = x_{0.75} - x_{0.25} \]
wurde bereits im Boxplot verwendet. Der IQR ist ein robustes Streuungsmaß.

\paragraph{Variationsbreite/Spannweite}
\[ x_{(n)} - x_{(1)} \]
Das ist offensichtlich nicht robust gegenüber Ausreißern.

\paragraph{Empirischer Variationskoeffizient}
\[ \operatorname{CV} := \frac{s}{x} \]
Dieser ist maßstabsunabhängig und erlaubt damit den Vegleich verschiedener
Stichproben.

\subsection{Maße für Schiefe und Wölbung}
\paragraph{Empirische Schiefe}
(skewness)
\[ b_3 := \frac{m_3}{s^3} \quad \text{mit} \quad m_3 := \rez{n} \sum_{i=1}^n
  (x_i - \obar{x})^3 \]
ist durch die Normierung mit $s^3$ maßstabsunabhängig. Typischerweise gilt \\
$b_3 \approx 0$ für symmetrische Verteilungen, \\
$b_3 < 0$ für linksschiefe Verteilungen, \\
$b_3 > 0$ für rechtsschiefe Verteilungen.

\paragraph{Empirische Wölbung}
(auch Exzess, Kurtosis)
\[ b_4 := \frac{m_4}{s^4} - 3 \quad \text{mit} \quad m_4 := \rez{n} \sum_{i=1}^n
  (x_i - \obar{x})^4 \]
charakterisiert, wie stark oder schwach Randbereiche und der zentrale Bereich
der Daten besetzt sind.

\subsection{Konzentrationsmaße}
Wir gehen in der Folger davon aus, dass uns ordinalskalierte Daten vorliegen,
wobei alle Ausprägungen $x_1, \ldots, x_n$ nicht-negativ und bereits geordnet
sind:
\[ 0 \le x_1 \le x_2 \le \ldots \le x_n. \]

\paragraph{Lorenzkurve}
Die Lorenzkurve der Daten $x_1 \le \ldots \le x_n$ ist der Streckenzug durch die
Punkte
\[ (0,0), (u_1, v_1), (u_2, v_2), \ldots, (u_n, v_n) = (1,1), \]
wobei
\[ u_j := \frac{j}{n}, \qquad j = 1, \ldots, n, \]
der Anteil der Merkmalsträger ist, welcher die kumulierte relative
Merkmalssumme
\[ v_j := \frac{ \sum_{i=1}^j x_i }{ \sum_{i=1}^n x_i}, \qquad j = 1, \ldots,
  n, \]
auf sich konzentriert.

Entsprechend lässt sich aus der Lorenzkurve direkt ablesen, dass auf $u_j \cdot
100 \%$ der kleinsten Merkmalsträger $v_j \cdot 100 \%$ der Merkmalssumme
entfallen.

Zum Beispiel ``Auf 20 \% der Haushalte im Land entfallen 80 \% des
Gesamteinkommens.''

Eine Interpretation ist aber nur an den Knoten $(u_j, v_j)$ möglich!

Die Lorenzkurve ist stets monoton wachsend und konvex.

\paragraph{Gini-Koeffizient}
\[ \begin{aligned}
    G &:= \frac{\text{Fläche zwischen Diagonale und Lorenzkurve}}
    {\text{Fläche zwischen Diagonale und $u$-Achse}} \\
    &= 2 \cdot \text{Fläche zwischen Diagonale und Lorenzkurve}
  \end{aligned} \]
Nimmt Werte zwischen 0 (Nullkonzentration) und $\frac{n-1}{n}$ (maximale
Konzentration\footnote{%
  Das heißt, die gesamte Merkmalssumme konzentriert sich auf einen
  Merkmalsträger.}%
) an.

\section{PP- und QQ-Plots}
Nach der bisherigen Analyse der Stichprobe $(x_1, \ldots, x_n)$ haben wir
vielleicht eine Vermutung, mit welcher Verteilung die Stichprobe modelliert
werden könnte.

Derartige Vermutungen überprüft man mit PP-Plots (probality-probality plot).
\begin{defn} %2.12
  Sei $(x_1, \ldots, x_n)$ eine ordinalskalierte Stichprobe mit
  Ordnungsstatistik $(x_{(1)}, \ldots, x_{(n)})$ und sei $G$ eine
  Verteilungsfunktion. Dann ist
  \[ \left\{ \left( G(x_{(i)}), \frac{i}{n+1} \right), \quad i = 1, \ldots, n
    \right\} \subset [0,1]^2 \]
  der \emph{PP-Plot von $(x_1, \ldots, x_n)$ bezüglich $G$}.
\end{defn}

Sind die Daten gemäß $G$ verteilt\footnote{%
  Das heißt, die Stichprobe ist eine Realisierung von i.i.d. Zufallsvariablen
  $X_i$, $i = 1, \ldots, n$ mit Verteilung $G$}%
, so sollte $G$ etwa der empirischen Verteilungsfunktion $\hat{F}_n$ der Daten
entsprechen. Für diese gilt (Def. 2.5):
\[ \hat{F}_n(x_{(i)} = \rez{n} H( x_{(i)} ) = \rez{n} \sum_{j=1}^n
  \ind_{\{ x_j \le x_{(i)} \}} = \frac{i}{n}. \]
Es sollte also für große $n$
\[ G(x_{(i)}) \approx \frac{i}{n} \approx \frac{i}{n+1} \]
gelten, so dass der PP-Plot etwa auf der Winkelhalbierenden liegt.

Beachte: Wir betrachten nicht die Paare $(G(x_{(i)}), i/n)$, um den Punkt
$(G(x_{(n)}), 1)$ zu vermeiden. Dieser kann nur dann auf der Winkelhalbierenden
liegen, wenn $G$ ein endliches Maximum besitzt und dieses von $x_{(n)}$
angenommen wird.

\begin{defn} %2.13
  Sei $(x_1, \ldots, x_n)$ eine ordinalskalierte Stichprobe mit
  Ordnungsstatistik $(x_{(1)}, \ldots, x_{(n)})$ und sei $G$ eine
  Verteilungsfunktion. Dann ist
  \[ \left\{ \left( G^{\leftarrow}\left( \frac{i}{n+1} \right), x_{(i)} \right),
      \quad i = 1, \ldots, n \right\} \]
  der \emph{QQ-Plot von $(x_1, \ldots, x_n)$ bezüglich G} mit $G^\leftarrow$ der
  Quantilfunktion aus Definition 2.8.
\end{defn}

Sind die Daten gemäß $G$ verteilt, dann sollte auch der QQ-Plot etwa auf der
Winkelhalbierenden liegen.

QQ-Plots werden insbesondere verwendet, wenn die Ränder der Verteilung stärker
gewichtet werden sollen. Für $i \approx 1$ oder $i \approx n$ gilt
\[ G(x_{(i)}) \approx 0 \approx \rez{n} \quad \text{bzw.} \quad G(x_{(i)})
  \approx 1 \approx \frac{i}{n+1}, \]
egal, ob $G$ die Daten gut beschreibt oder nicht. Der PP-Plot liegt in den
Randbereichen also immer nahe der Winkelhalbierenden, der QQ-Plot aber nicht.

\section{Multivariate Deskription}
Wir betrachten nicht ein, sondern mehrere Merkmale der statistischen Einheiten
$\rightsquigarrow$ bivariat.

Gegeben sein also eine Stichprobe $((x_1, y_1), \ldots, (x_n,y_n))$, wobei die
$(x_i,y_i)$ als Realisierungen der i.i.d. Zufallsvariablen $(X_i,Y_i)$ mit
gemeinsamer Verteilung $F$ (unbekannt) aufgefasst werden können.

\subsection{Darstellung bivariater Stichproben}
\paragraph{Kontingenztabelle}
Trage die (absoluten) Häufigkeiten der möglichen Ausprägungspaare in einer
Tabelle ein:
\begin{center}
  \begin{tabular}{l|ccc|r}
    & $b_1$ & $\cdots$ & $b_l$ \\
    \hline
    $a_1$ & $n_{1,1}$ & $\cdots$ & $n_{1,l}$ & $n_{1,\cdot}$ \\
    $\vdots$ & $\vdots$ & & $\vdots$ & $\vdots$ \\
    $a_k$ & $n_{k,1}$ & $\cdots$ & $n_{k,l}$ & $n_{k,\cdot}$ \\
    \hline
    & $n_{\cdot,1}$ & $\cdots$ & $n_{\cdot,l}$ & n
  \end{tabular}
\end{center}
Dabei sind $a_1, \ldots, a_k$ und $b_1, \ldots, b_l$ mögliche Ausprägungen der
beobachteten Merkmale und
\[ n_{ij} = \sharp \{ (x_m, y_m), m = 1, \ldots, n : (x_m,y_m) = (a_i, b_j)
  \}. \]
Die Summen
\[ n_{i,\cdot} = \sum_{j=1}^l n_{i,j}, \qquad n_{\cdot,j} = \sum_{i=1}^k
  n_{i,j} \]
werden als \emph{Randhäufigkeiten} bezeichnet und entsprechen den Häufigkeiten
der Merkmalsausprägungen der einzelnen Merkmale (Def. 2.1)
\begin{itemize}
\item In einer \emph{relativen Kontingenztabelle} stehen anstelle der absoluten
  die relativen Häufigkeiten.
\item Offenbar lässt sich die Kontingenztabelle auch für gruppierte stetige
  Merkmale erstellen.
\end{itemize}

R-Befehl: \verb+table(.)+ bzw \verb+ftable(.)+ (flat table)

\paragraph{Streudiagramm}
Trage die Datenpaare $(x_i, y_i)$ in ein
Koordinatensystem ein. Die Form der entstehenden Punktwolke erlaubt
gegebenenfalls Rückschlüsse auf den Zusammenhang der Daten (linear, polynomiell,
...). Diese Zusammenhänge werden in der Regressionstheore weiter untersucht.

\subsection{Zusammenhangsmaße}
\paragraph{Empirische Kovarianz}
\[ s_{xy}^2 = \rez{n-1} \sum_{i=1}^n (x_i - \bar{x})(y_i - \bar{y}). \]
Für $x=y$ ist das die empirische Varianz.

\paragraph{Empirische Korrelation}
(Bravais-Pearson-Korrelationskoeffizient)
\[ \rho_{xy} = \frac{ s_{xy}^2 }{ \sqrt{ s_{xx}^2 \cdot s_{yy}^2 } } \]
hat dieselben Eigenschaften wie die $\pW$-theoretische Korrelation:
\begin{itemize}
\item $|\rho_{xy}| \le 1$,
\item $\rho_{xy} = \pm 1$ impliziert einen linearen Zusammenhang, das heißt alle
  $(x_i, y_i)$ liegen auf einer Geraden mit positivem ($\rho_{xy} = 1$) bzw.
  negativem ($\rho_{xy} = -1$) Anstieg.
\end{itemize}
Einen alternativen Korellationskoeffizienten erhält man, wenn man die
Stichprobenwerte $x_i$ und $y_i$ durch die \emph{Ränge} $\rg(x_i)$ bzw.
$\rg(y_i)$ ersetzt. Dabei gilt:
\[ \rg(x_i) = j, \quad \text{falls } x_i = x_{(j)} \text{ für ein } j \in \{ 1,
  \ldots, n \}, i = 1, \ldots, n. \]
Insbesondere folgt
\[ \rg(x_{(i)}) = i \text{ für alle } i = 1, \ldots, n \text{ falls } x_i \ne
  x_j \text{ für } i \ne j. \]
Falls die Stichprobe $(x_1, \ldots, x_n)$ mehrere identische Werte $x_i$
(sogenannte \emph{Bindungen}) enthält, wird diesen Werten der Durchschnittsrang
zugewiesen, also das arithmetische Mittel der in Frage kommenden Ränge.
\begin{center}
  \begin{tabular}{l|ccccccc}
    $x_i$ & 7,2 & 1,3 & 2,6 & 1,3 & 2,7 & 1,3 & 9,1 \\
    \hline
    $x_{(i)}$ & 1,3 & 1,3 & 1,3 & 2,6 & 2,7 & 7,2 & 9,1 \\
    $\rg(x_i)$ & 2 & 2 & 2 & 4 & 5 & 6 & 7
  \end{tabular}
\end{center}
Damit kann der \emph{Spearman-Korellationskoeffizient} definiert werden als
Bravais-Pearson-Korellationskoeffizient der Rangstichproben:
\[ \rho_{xy}^{\mathrm{SP}} := \frac{
    \sum_{i=1}^n (\rg(x_i) - \obar{\rg}(x)) (\rg(y_i) - \obar{\rg}(y))
  }
  {
    \sqrt{
      \sum_{i=1}^n (\rg(x_i) - \obar{\rg}(x))^2
      \sum_{i=1}^n (\rg(j_i) - \obar{\rg}(y))^2
    }
  },
\]
wobei
\[ \obar{\rg}(x) = \rez{n} \sum_{i=1}^n \rg(x) = \rez{n} \sum_{i=1}^n i =
  \rez{n} \frac{n(n+1)}{2} = \frac{n+1}{2} = \obar{\rg}(y). \]
Auch Spearmans Korellationskoeffizient erbt die Eigenschaften der
$\pW$-theoretischen Korrelation:
\begin{itemize}
\item $|\rho_{xy}^{\mathrm{SP}}| \le 1$,
\item $| \rho_{xy}^{\mathrm{SP}} | = 1$ bedeutet, dass die Paare $(\rg(x_i),
  \rg(y_i))$ auf einer Geraden liegen. Es gilt also
  \[ x_i < x_j \qLRq y_i < y_j \quad (\rho_{xy}^{\mathrm{SP}} = 1)
    \qquad \text{bzw.} \qquad
    x_i < x_j \qLRq y_i > y_j \quad (\rho_{xy}^{\mathrm{SP}} = -1). \]
\end{itemize}
Beachte: Ein im Absolutbetrag hoher Korrelationskoeffizient ist nur ein Hinweis
auf einen \emph{möglichen} Zusammenhang zwischen zwei Merkmalen. Um auf einen
tatsächlich vorliegenden (kausalen) Zusammenhang zu schließen, müssen
sachlogische Überlegungen heran gezogen werden!

\paragraph{Scheinkorrelation}
Es wird eine hohe Korrelation gemessen, die jedoch inhaltlich nicht
gerechtfertigt ist. Ursache kann ein unberücksichtigtes drittes Merkmal sein.
Beispiel: Wortschatz $\leftrightarrow$ Größe des Kindes (ohne Berücksüchtigung
des Alters).

\paragraph{Verdeckte Korrelation}
Kausaler Zusammenhang ist vorhanden, aber der Korrelationskoeffizient ist
niedrig. Ursache kann sein, dass die untersuchte Teilpopulation in Gruppen mit
gegenläufigem Verhalten zerfällt.


\chapter{Parametrische Statistik}
\emph{$\pW$-Theorie:} Gegeben ist ein $\pW$-Raum mit gegebenen
Zufallsvariablen/Verteilungen. \\
\emph{Ziel:} Aussagen über ``Funktionen'' der Zufallsvariablen.

\emph{Statistik:} Gegeben ist eine Stichprobe (einer Zufallsvariablen). \\
\emph{Ziel:} Finde die Verteilung.

Sei also $(x_1, \ldots, x_n)$ eine Realisierung der Zufallsstichprobe $(X_1,
\ldots, X_n)$, wobei $X_1, \ldots, X_n$ i.i.d. mit Verteilung $F$ (unbekannt).

\emph{Parametrische Statistik:} Wir gehen davon aus, dass $F$
\emph{parametrisierbar} ist. Das heißt $F$ gehört zu einer vorgebenen
parametrischen Familie $\{ F_\theta : \theta \in \Theta \}$ von
Verteilungsfunktionen mit Parameter $\theta$. Dabei ist $\Theta$ der
\emph{Parameterraum}, das heißt eine Borel-Teilmenge des $\real^n$, welche alle
zulässigen Parameterwerte enthält. $\theta = (\theta_1, \ldots, \theta_n)$ ist
der $n$-dimensionale \emph{Parametervektor} von $F_\theta$.

Wir setzen voraus, dass die Parametrisierung $\theta \mapsto F_\theta$
\emph{identifizierbar} ist, das heißt für $\theta \ne \tilde{\theta}$ gilt
$F_\theta \ne F_{\tilde{\theta}}$.

Die Wahl der Verteilungsfamilie erfolgt aufgrund von Vorüberlegungen wie zum
Beispiel Erfahrungswerten, Methoden der deskriptiven Statistik,
``Verteilungstests''.

Als $\pW$-Raum $(\Omega, \mF, \pP)$, auf dem die Zufallsstichprobe definiert ist,
können wir den \emph{kanonischen $\pW$-Raum} wählen:
\[ \Omega = \real^\infty, \quad \mF = \borel( \real^\infty ) = \borel(\real)
  \times \borel(\real) \times \cdots, \]
\[ \pP( \{ \omega = (\omega_1, \omega_2, \ldots) \in \real^\infty: \omega_{i_1}
  \le x_{i_1}, \ldots, \omega_{i_k} \le x_{i_k}\} ) = F_\theta(x_{i_1}) \cdot
  \ldots \cdot F_\theta(x_{i_k}) \]
für alle $k \in \nat$, $1 \le i_1 \le \ldots \le i_k$.

Da $\pP$ von $\theta$ abhängt, schreibe auch $\pP_\theta$ bzw. $\pE_\theta$,
$\var_\theta$, ...

Gesucht ist nun ein \emph{Schätzer} $\hat{\theta}$ für den unbekannten
Parameter(vektor) $\theta$, das heißt eine Abbildung
\[ T: \real^n \to \real^m : (X_1, \ldots, X_n) \mapsto \hat{\theta}, \]
welche die Zufallsstichprobe auf einen Zufallsparameter abbildet.

Dabei ist $T$ Borel-messbar, $\hat{\theta}$ ist also eine Zufallsvariable bzw.
ein Zufallsvektor.

Üblicherweise wird zudem angenommen, dass
\[ \pP( T(X_1, \ldots, X_n) \in \Theta ) = 1. \]

Die konkrete Auswertung von $\hat{\theta}$ für die Realisierung $(x_1, \ldots,
x_n)$, also $T(x_1, \ldots, x_n)$, wird als \emph{Schätzwert} bezeichnet.

\section{Einige parametrische Verteilungsfamilien}
\paragraph{Normalverteilung}
mit Parametern ($\mu, \sigma^2$), $\mu \in \real$, $\sigma^2 > 0$ ist
absolutstetig mit Dichte 
\[ f_{(\mu,\sigma^2)} = \rez{\sqrt{2 \pi \sigma^2}} \exp \left( -
    \frac{(x-\mu)^2}{2 \sigma^2} \right). \]
Ist $X$ normalverteilt mit Parameter $(\mu,\sigma^2)$ (schreibe $X \sim N(\mu,
\sigma^2$), so gelten
\[ \pE[X] = \mu, \qquad \var(X) = \sigma^2. \]

\paragraph{Chi-Quadrat-Verteilung}
Sind $X_1, \ldots, X_k \sim N(0,1)$ i.i.d., so besitzt
\[ X = X_1^2 + \ldots + X_k^2 \]
eine \emph{$\chi^2$-Verteilung mit $k$ Freiheitsgraden} ($X \sim \chi_k^2$). Es
gelten
\[ \pE[X] = \pE[X_1] + \ldots + \pE[X_k] = k \]
und
\[ \begin{aligned}
    \var(X)
    &= \var(X_1^2 + \ldots + X_k^2) = \var(X_1^2) + \ldots + \var(X_k^2) \\
    &= k \var(X_1^2) = k( \pE[X_1^4] - \pE[X_1^2]^2 ) \\
    &= k \cdot (3-1) = 2k.
  \end{aligned} \]
Die Kovarianzen sind alle $=0$, weil die $X_i$ unabhängig sind.

\paragraph{Exponentialverteilung}
mit Parameter $\lambda > 0$ ist absolutstetig mit Dichte
\[ f_{\lambda} (x) = \begin{cases}
    \lambda e^{-\lambda x}, & x \ge 0, \\
    0, & x < 0.
  \end{cases} \]
Für $X \sim \Exp(\lambda)$ gelten
\[ \pE[X] = \lambda^{-1}, \qquad \var(X) = \lambda^{-2}. \]

\paragraph{Gamma-Verteilung}
mit Parametern $(\lambda, p)$, $\lambda > 0$, $p > 0$ ist durch die Dichte
\[ f_{(\lambda, p)} (x) = \begin{cases}
    \frac{\lambda^p x^{p-1}}{\Gamma(p)} e^{-\lambda x}, & x \ge 0, \\
    0, & x < 0, 
  \end{cases} \]
definiert, wobei
\[ \Gamma(p) = \int_0^\infty x^{p-1} e^{-x} \diffop x, \quad p > 0, \]
die Gammafunktion ist.

Ist $X$ Gamma-verteilt mit Parameter $(\lambda, p)$ ($X \sim \Gamma(\lambda,
p)$), so berechnet sich die momenterzeugende Funktion von $X$ als
\[ \begin{aligned}
    m_X(u)
    &= \pE[e^{uX}] \\
    &= \int_0^\infty e^{ux} \frac{\lambda^p x^{p-1}}{\Gamma(p)} e^{-\lambda x}
    \diffop x \\
    &= \frac{\lambda^p}{\Gamma(p)} \int_0^\infty x^{p-1} e^{-(\lambda-u)x}
    \diffop x \\
    &= \frac{\lambda^p}{\Gamma(p)} \int_0^\infty \frac{y^{p-1}}{(\lambda-u)^{p-1}}
    e^{-y} \diffop y \\
    &= \frac{\lambda^p}{\Gamma(p)} \cdot \frac{\Gamma(p)}{(\lambda-u)^p} \\
    &= \left( \frac{\lambda}{\lambda -u} \right)^2
  \end{aligned}
\]
für $u < \lambda$. Damit
\[ \pE[X^k] = m_X^{(k)}(0) = \cdots = \frac{p \cdot (p+1) \cdots
    (p+k-1)}{\lambda^k}. \]
Sind $X_1, \ldots, X_k$ unabhängig $\Exp(\lambda)$-verteilt, so ist
\[ X_1 + \ldots + X_k \sim \Gamma(\lambda, k). \]
In diesem Fall (für ganzzahliges $p$) spricht man auch von einer
\emph{Erlang-Verteilung} ($\operatorname{Erl}(\lambda, k)$).

Insbesondere gilt $\Gamma( \lambda, 1 ) = \Exp(\lambda)$. Ist $X \sim \chi^2_k$,
so gilt $X \sim \Gamma( 1/2, k/2 )$.

\paragraph{Student-Verteilung (t-Verteilung)}
Sei
\[ X := \frac{U}{\sqrt{V/r}}, \]
mit $r \in \nat$, $U \sim N(0,1)$, $V \sim \chi^2_r$, $U$ und $V$ unabhängig.
Dann ist $X$ Student-verteilt mit $r$ Freiheitsgraden ($X \sim t_r$).

Die Dichte der $t$-Verteilung ergibt sich mittels Dichtetransformationssatz als
\[ f_r(x) = \rez{\sqrt{r} B(r/2,1/2)} \left(1 + \frac{x^2}{r}
  \right)^{(r+1)/2}, \]
wobei
\[ B(p,q) := \int_0^1 t^{p-1} (1-t)^{q-1} \diffop t, \qquad p,q > 0 \]
die \emph{Betafunktion} ist.

Für $r = 1$ ergibt sich die Cauchyverteilung, welche weder Erwartungswert noch
Varianz besitzt.

Für $X \sim t_r$, $r \ge 2$ gilt
\[ \pE[X] = 0, \qquad \var(X) = \begin{cases}
    \infty, & r = 2, \\
  \frac{r}{r-2}, & r \ge 3. 
\end{cases} \]

\paragraph{Fisher-Snedecor-Verteilung (F-Verteilung)}
Sei $X$ definiert durch
\[ X := \frac{U/r}{V/s} \]
für $r,s \in \nat$ und zwei unabhängige Zufallsvariablen $U \sim \chi_r^2$ und
$V \sim \chi_s^2$, dann hat $X$ eine F-Verteilung mit $r,s$ Freiheitsgraden ($X
\sim F_{r,s}$). Die Dichte von $X$ ist gegeben durch
\[ f_{r,s}(x) = \begin{cases}
    \frac {x^{r/2 - 1}}{B\left( \frac{r}{2}, \frac{s}{2} \right) (r/s)^{-r/2}
      \left( 1 + \frac{r}{s} x \right)^{(r+s)/2}}, & x \ge 0, \\
    0, & x < 0.
  \end{cases}
\]
und die Momente (sofern existent) durch
\[ \begin{aligned}
    \pE[X] &= \frac{s}{s-2}, & s &\ge 3 \\
    \var(X) &= \frac{2s^2 (r+s-2)}{r(s-4)(s-2)^2}, & s &\ge 5.
  \end{aligned}
\]

\section{Eigenschaften von Schätzern}
\begin{exmp}
  Angenommen, es liegt eine Stichprobe $(X_1, \ldots, X_n)$ von i.i.d.
  Zufallsvariablen mit Normalverteilung $N(\mu,\sigma^2)$ vor. Entsprechend dem
  parametrischen Modell möchten wir $\theta = (\mu,\sigma^2)$ schätzen. Mögliche
  Schätzer hierfür sind zum Beispiel
  \begin{align*}
    \hat{\mu}_1 &= T_1(X_1, \ldots, X_n) := \rez{n} \sum_{i=1}^n X_i \quad
    \text{oder} \\
    \hat{\mu}_2 &= T_2(X_1, \ldots, X_n) := X_1, \\
    \hat{\sigma}^2_1 &= T_3(X_1, \ldots, X_n) := \rez{n-1} \sum_{i=1}^n \left( X_i - \rez{n} \sum_{j=1}^n X_j \right)^2 \quad
                       \text{oder} \\
    \hat{\sigma}^2_2 &= T_4(X_1, \ldots, X_n) := \rez{n} \sum_{i=1}^n \left( X_i - \rez{n} \sum_{j=1}^n X_j \right)^2.
  \end{align*}
  Wie können wir nun entscheiden, welcher von den je zwei Schätzern im obigen
  Beispiel der ``Bessere'' ist? Welche Eigenschaften kann und soll ein ``guter''
  Schätzer haben?
\end{exmp}

\subsection{Erwartungstreue}
\begin{defn}
  Ein Schätzer $\hat{\theta} = T(X_1, \ldots, X_n)$ für den Parameter $\theta$
  ist \emph{erwartungstreu} (auch \emph{unverzerrt}, \emph{unbiased}), falls
  \[ \pE_\theta[\hat{\theta}] = \theta, \quad \text{für alle } \theta \in \Theta. \]
  Die \emph{Verzerrung} (der \emph{Bias}) für den Parameter $\theta$ heißt
  \emph{asymptotisch erwartungstreu}, falls sein Bias für große Datenmenge
  gegen 0 konvergiert, also
  \[ \pE_\theta[T(X_1, \ldots, X_n)] \xrightarrow{n \to \infty} \theta. \]
\end{defn}

\begin{exmp}
  Die Schätzer $\hat{\mu_1}$ und $\hat{\mu}_2$ in Beispiel 3.1 sind
  erwartungstreu:
  \[ \begin{aligned}
      \pE_\theta[ T_1(X_1, \ldots, X_n) ]
      &= \pE_\theta \left[ \rez{n} \sum_{i=1}^n X_i \right] \\
      &= \rez{n} \sum_{i=1}^n \pE_\theta[X_i] = \mu, \\
      \pE_\theta[ T_2(X_1, \ldots, X_n) ]
      &= \pE_\theta[X_1] = \mu.
    \end{aligned}
  \]
  $\hat{\sigma}^2_1$ ist ebenso erwartungstreu, $\hat{\sigma}^2_2$ ist nicht
  erwartungstreu, aber asymptotisch erwartungstreu.

  Beachte: Da an dieser Stelle die Eigenschaften der Normalverteilung nicht
  eingehen (bis auf Existenz von Erwartungswert und Varianz), sind
  $\hat{\mu}_1$, $\hat{\mu}_1$, $\hat{\sigma^2_1}$ und $\hat{\sigma^2_1}$
  (asymptotisch) erwartungstreue Schätzer für den Erwartungswert/die Varianz
  einer beliebigen Verteilung mit endlichem Erwartungswert/endlicher Varianz.
\end{exmp}

\subsection{Konsistenz}
\begin{defn}
  Ein Schätzer $\hat{\theta} = T(X_1, \ldots, X_n)$ für den Parameter $\theta$
  ist \emph{konsistent} im \emph{quadratischen Mittel/schwachen Sinn/starken
    Sinn},  falls
  \[ T(X_1, \ldots, X_n) \xrightarrow{n \to \infty} \theta \]
  in $L^2$/in Wahrscheinlichkeit/fast sicher, das heißt
  \begin{itemize}
  \item $\hat{\theta}$ ist $L^2$-konsistent: Für $\pE_\theta[\hat{\theta}^2] <
    \infty$ gilt
    \[ \hat{\theta} \xrightarrow{L^2} \theta \qLRq \pE_\theta|T(X_1, \ldots,
      X_n) - \theta|^2 \xrightarrow{n \to \infty} 0, \quad \theta \in \Theta. \]
  \item $\hat{\theta}$ ist (schwach) konsistent:
    \[ \hat{\theta} \xrightarrow{\pP} \theta \qLRq \pP_\theta( |T(X_1, \ldots,
      X_n ) - \theta | > \eps ) \xrightarrow{n \to \infty} 0, \quad \theta \in
      \Theta. \]
  \item $\hat{\theta}$ ist stark konsistent:
    \[ \hat{\theta} \xrightarrow{n \to \infty, \text{ f. s.}} \theta \qLRq
      \pP_\theta \left( \lim_{n \to \infty} T(X_1, \ldots,
      X_n ) = \theta \right) = 1, \quad \theta \in
      \Theta. \]
  \end{itemize}
  Entsprechend gilt
  \[ L^2\text{-Konsistenz} \qRq \text{Schwache Konsistenz} \quad \Leftarrow \quad
    \text{Starke Konsistenz}. \]
\end{defn}

\begin{exmp}
  Seien $X_1, \ldots, X_n$ i.i.d. mit $\pE[X_i] = \mu$, $\var(X_i) = \sigma^2 <
  \infty$. Dann gilt nach dem starken Gesetz der großen Zahlen
  \[ \hat{\mu} := \bar{X} = \rez{n} \sum_{i=1}^n X_i \xrightarrow{\text{f.s.}}
    \mu, \]
  das heißt, das arithmetische Mittel ist ein stark konsistenter Schätzer für
  den Erwartungswert der $X_i$.
\end{exmp}

\begin{lem}
  Sei $\hat{\theta} = T(X_1, \ldots, X_n)$ ein erwartungstreuer Schätzer für den
  Parameter $\theta$, so dass $\var( \hat{\theta} ) \xrightarrow{n \to \infty}
  0$,  dann ist $\hat{\theta}$ schwach konsistent.
\end{lem}

\begin{proof}
  Nach der Tschebyscheff-Ungleichung gilt
  \[ \pP(|\hat{\theta} - \theta| \ge \eps) \le
    \frac{\var(\hat{\theta})}{\eps^2}. \]
  Mit der Erwartungstreue von $\hat{\theta}$ folgt die Behauptung.
\end{proof}

\begin{exmp}
  Seien $X_1, \ldots, X_n$ i.i.d. Zufallsvariablen mit Normalverteilung
  $N(\mu,\sigma^2)$ und definiere die Schätzer
  \begin{align*}
    \sigma^2_1
    &:= \rez{n-1} \sum_{i=1}^n
      \left( X_i - \rez{n} \sum_{i=1}^n X_j \right)^2 \\
    \sigma^2_2
    &:= \rez{n} \sum_{i=1}^n
      \left( X_i - \rez{n} \sum_{i=1}^n X_j \right)^2 \\
    \sigma^2_3
    &:= \rez{n} \sum_{i=1}^n
      \left( X_i - \mu \right)^2 \quad \text{($\mu$ bekannt)}.
  \end{align*}
  Dann sind $\sigma^2_1$, $\sigma^2_2$ und $\sigma^2_3$ schwach konsistent. Zum
  Beispiel gilt für $\sigma^2_3$:
  \begin{align*}
    \var{\sigma^2_3}
    &= \rez{n^2} \sum_{i=1}^n \var((X_i-\mu)^2) \\
    &= \rez{n} \var((X_1 - \mu)^2) \xrightarrow{n \to \infty} 0.
  \end{align*}
  und die Behauptung folgt mittels des obigen Lemmas.
  
Für $\sigma^2_1$ lässt sich ebenfalls die Varianz berechnen und das Lemma
anwenden (Übungsaufgabe); $\sigma^2_2 = \frac{n-1}{n} \sigma^2_1$ (nicht
erwartungstreu) lässt sich auf $\sigma^2_1$ zurückführen.

Beachte: Auch hier gingen die Eigenschaften der Normalverteilung (abgesehen von
der Endlichkeit der Momente) nicht ein.
\end{exmp}

\subsection{Standardfehler und mittlere quadratische Abweichung}
\begin{defn}
  Die \emph{mittlere quadratische Abweichung} (\emph{mean squared error}) eines
  Schätzers $\hat{\theta} = T(X_1, \ldots, X_n)$ für den Parameter $\theta$ ist
  \[ \MSE(\hat{\theta}) = \pE_\theta| \hat{\theta} - \theta |^2. \]
  Der \emph{Standardfehler} von $\hat{\theta}$ ist
  \[ \sqrt{\var_\theta(\hat{\theta})}. \]

  Beachte: Offenbar ist $\hat{\theta}$ konsistent im quadratischen Mittel genau
  dann, wenn
  \[ \MSE( \hat{\theta} ) \xrightarrow{n \to \infty} 0. \]
\end{defn}

\begin{lem}
  Sei $\hat{\theta} = T(X_1, \ldots, X_n)$ ein Schätzer für den eindimensionalen
  Parameter $\theta$, so dass $\pE[\hat{\theta}^2] < \infty$, $\theta \in
  \Theta$. Dann gilt
  \[ \MSE( \hat{\theta}) = \var_\theta(\hat{\theta}) +
    \operatorname{Bias}_\theta^2(\hat{\theta}) \ge \var_\theta(\hat{\theta}). \]
\end{lem}

\begin{proof}
  \begin{align*}
    \MSE(\hat{\theta})
    &= \pE_\theta|\hat{\theta}-\theta|^2 \\
    &= \pE_\theta \left| \hat{\theta} - \pE_\theta[\hat{\theta}]
      + \pE_\theta[\hat{\theta}] - \theta \right|^2 \\
    &= \pE_\theta \left| \hat{\theta} - \pE_\theta[\hat{\theta}] \right|^2
      + 2 \pE_\theta
      \left[ \hat{\theta} - \pE_\theta[\hat{\theta}]\right]
      \left[ \pE_\theta[\hat{\theta}] - \theta \right]
      + \pE_\theta \left| \pE_\theta[\hat{\theta}] - \theta \right|^2 \\
    &= \pE_\theta \left| \hat{\theta} - \pE_\theta[\hat{\theta}] \right|^2
      + 0 + \left( \pE_\theta[\hat{\theta}] - \theta \right)^2 \\
    &= \var_\theta(\hat{\theta}) + \operatorname{Bias}_\theta^2 (\theta).
  \end{align*}
  Die Ungleichung ist klar.
\end{proof}

Aus dem obigen Lemma wird ersichtlich, dass der Standardfehler nur in
Kombination mit dem Bias ein sinnvolles Gütemaß für die Qualität eines Schätzers
darstellt. Betrachtet man jedoch nur unverzerrte Schätzer, so genügt der
Standardfehler als Kriterium. Entsprechend definiert man:

\begin{defn}
Sei $\hat{\theta} = T(X_1, \ldots, T_n)$ ein unverzerrter Schätzer für $\theta$,
so dass 
\[ \var_\theta \left( T(X_1, \ldots, T_n) \right) \le
  \var_\theta \left( S(X_1, \ldots, X_n) \right) \]
für alle unverzerrten Schätzer $S(X_1, \ldots, X_n)$ von $\theta$, dann ist
$\hat{\theta}$ der \emph{UMVUE} (\emph{uniformly minimum variance unbiased
  estimator}) für $\theta$.
\end{defn}

Unverzerrte Schätzer müssen jedoch im Allgemeinen nicht existieren.

\clearpage

\section{Eigenschaften bereits bekannter Statistiken}
\subsection{Ordnungsstatistik}
In Kapitel 2.2: Beschreibe Quantile einer Stichprobe. \\
Modellebene: Entsprechende Definition der Ordnungsstatistik$(X_{(1)}, \ldots,
X_{(n)})$ der Zufallsstichprobe $(X_1, \ldots, X_n)$.

\begin{thm}
  Die Verteilungsfunktion $F_{X_{(i)}}$ der Ordnungsstatistik $X_{(i)}$, $i = 1,
  \ldots, n$ ist gegeben durch
  \[ F_{X_{(i)}}(x) := \sum_{k=i}^n \binom{n}{k} F^k(x) (1-F(x))^{n-k}, \quad x
    \in \real. \]
  Insbesondere:
  \begin{enumerate}
  \item Sind alle $X_i$ disktret mit Wertebereich $E = \{ \ldots, a_{j-1}, a_j,
    a_{j+1}, \ldots \}$ mit $a_j < a_{j+1}$, $j \in \nat$, dann gilt für die
    Zähldichte von $X_{(i)}$
    \[ \pP( X_{(i)} = a_j ) = \sum_{k=i}^n \binom{n}{k} \left(
        F^k(a_j)(1-F(a_j))^{n-k} - F^k(a_{j-1})(1-F(a_{j-1}))^{n-k} \right), \]
    wobei
    \[ F(a_j) = \sum_{\substack{a_k \in E \\ k \le j}} \pP(X_i = a_k). \]
  \item Sind die $X_i$ absolutstetig mit stückweise stetiger Dichtefunktion, so
    ist auch $X_{(i)}$, $i = 1, \ldots, n$, absolutstetig verteilt mit Dichte
    \[ f_{X_{(i)}}(x) = \frac{n!}{(i-1)!(n-1)!} f(x)
      F^{i-1}(x)(1-F(x))^{n-1}. \]
  \end{enumerate}
\end{thm}

\begin{proof}
  Für $x \in \real$ definiere
  \[ Y := \sharp \{ i : X_i \le x \} = \sum_{i=1}^n \ind_{\{X_i \le x\}}. \]
  Dann ist $Y \sim \operatorname{Bin}(n, F(x))$, da die $X-i$ i.i.d. mit
  Verteilungsfunktion $F$ sind. Zudem
  \[ F_{X_{(i)}}(x) = \pP( X_{(i)} \le x ) = \pP( Y \ge i ) \]
  und damit folgt die Behauptung.

  Für diskrete $X_i$ gilt weiter
  \[ \pP( X_{(i)} = a_j ) = \pP( a_{j-1} < X_{(i)} \le a_j ) = F_{X_{(i)}}(a_j)
    - F_{X_{(i)}}(a_{j-1}). \]

  Für stetige $X_i$ folgt die Dichte durch Ableiten der Verteilungsfunktion.
\end{proof}

\subsection{Empirische Verteilungsfunktion}
Nach Definition 2.5
\[ \hat{F}_n(x) = \frac{ \sharp \{ X_i : X_i \le x \}}{n}. \]
Betrachte nun $(x_1, \ldots, x_n)$ als Realisierung einer Zufallsstichprobe
$(X_1, \ldots, X_n)$ mit $X_i$ i.i.d., $X_i \sim F$, so kann $\hat{F}_n(x)$ als
Schätzer\footnote{Das ist eigentlich ein Beispiel für nichtparametrische
  Statistik, soll aber trotzdem hier in der Einführung enthalten sein.} für $F$
verwendet werden.

\begin{thm}
  Es gelten
  \begin{enumerate}
  \item $n \hat{F}_n(x) \sim \operatorname{Bin}(n, F(x))$.
  \item $\hat{F}_n(x)$ ist ein erwartungstreuer, stark konsistenter Schätzer für
    $F(x)$ mit
    \[ \var \hat{F}_n(x) = \frac{F(x) (1 - F(x))}{n}. \]
  \end{enumerate}
\end{thm}

\begin{proof}
  Zu 1: Es gilt
  \begin{align*}
    \hat{F}_n(x)
    &= \rez{n} \sharp \{ X_i : X_i \le x\} \\
    &= \rez{n} \sum_{i=1}^n \ind_{\{X_i \le x\}}
  \end{align*}
  mit $\ind_{\{X_i \le x\}} \sim \operatorname{Ber}(F(x))$. Damit folgt 1., da
  die $X_i$ unabhängig sind.

  Zu 2: Aus 1. folgt direkt für alle $x \in \real$
  \begin{align*}
    \pE[n \hat{F}_n(x) ] &= n F(x), \\
    \var(n \hat{F}_n(x)) &= n F(x) (1-F(x)),
  \end{align*}
  woraus Erwartungstreue und die Varianz folgen.

  Da zudem für $x \in \real$ fix die Zufallsvariablen
  \[ Y_i := \ind_{\{X_i \le x \}}\]
  i.i.d. sind, folgt nach dem starken Gesetz der großen Zahlen
  \[ \hat{F}_n(x) = \rez{n} \sum_{i=1}^n Y_i \xrightarrow{\text{f.s.}} \pE[Y_i]
    = F(x) \]
  für alle $x \in \real$, also ist $\hat{F}_n(x)$ stark konsistent.
\end{proof}

\subsection{Empirische Momente}
Aus Kapitel 2.2: Empirischer Erwartungswert, empirische Varianz als Lage- und
Streuungsmaß einer Stichprobe. Auf der Modellebene:

\begin{defn}
  Seien $X_1, \ldots, X_n$ i.i.d. Zufallsvariablen, so dass $\pE|X_i|^r <
  \infty$ für ein $r \in \nat$. Dann ist für $k = 1, \ldots, r$
  \begin{equation}
    \hat{\mu}_k := \rez{n} \sum_{i=1}^n X_i^k
  \end{equation}
  das \emph{k-te empirische Moment} der $X_i$.
\end{defn}

Die empirischen Momente können zur Schätzung der Momente der $X_i$ verwendet
werden.

\begin{thm}
  Sei $(X_1, \ldots, X_n)$ eine Zufallsstichprobe von i.i.d. Zufallsvariablen
  $X_i$ mit $\pE|X_i|^r < \infty$ für ein $r \in \nat$. Sei $\mu_k := \pE
  X_i^k$, $k = 1, \ldots, r$. Dann ist $\hat{\mu}_k$ ein erwartungstreuer, stark
  konsistenter Schätzer für $\mu_k$.
\end{thm}

\begin{proof}
  Wegen
  \begin{align*}
    \pE[\hat{\mu}_k] = \pE\left[ \rez{n} \sum_{i=1}^n X_i^k \right]
    = \rez{n} \sum_{i=1}^n \pE[X_i^k] = \pE[X_i^k] = \mu_k
  \end{align*}
  gilt die Erwartungstreue. Die starke Konsistenz folgt aus dem starken Gesetz
  der großen Zahlen, denn danach gilt
  \[ \rez{n} \sum_{i=1}^n X_i^k \xrightarrow{\text{f.s.}} \pE[X_i^k], \quad n
    \to \infty. \qedhere \]
\end{proof}

\begin{kor}
  Seien $(X_1, \ldots, X_n)$ i.i.d.
  \begin{enumerate}
  \item Gilt $\pE|X_i| < \infty$, so ist $\bar{X}$ ein erwartungstreuer, stark
    konsistenter Schätzer für $\mu = \pE[X_i]$.
  \item Gilt $\pE X_i^2 < \infty$, so ist $s^2 = \rez{n-1} \sum_{i=1}^n (X_i -
    \bar{X})^2$ ein erwartungstreuer, stark konsistenter Schätzer für $\sigma^2
    = \var X_i$.
  \end{enumerate}
\end{kor}

\begin{proof}
  \begin{enumerate}
  \item Klar, da $\bar{X} = \hat{\mu}_1$.
  \item Folgt aus dem Verschiebungssatz (Lemma \ref{lem:verschiebung}). \qedhere
  \end{enumerate}
\end{proof}

\section{Methoden zur Konstruktion von Schätzern}
Sei stets $(X_1, \ldots, X_n)$ eine Zufallsstichprobe von i.i.d.
Zufallsvariablen mit Verteilungsfunktion $F \in \{ F_\theta, \theta \in \Theta
\}$, $\Theta \subseteq \real^n$ mit identifizierbarer Parametrisierung.

\emph{Gesucht:} Ein Schätzer $\hat{\theta} = T(X_1, \ldots, X_n)$ für $\theta
\in \real^m$

\subsection{Methode der kleinsten Quadrate}
(Least squares estimation, LSE)

\emph{Idee:} Minimiere die quadratischen Abweichungen zwischen den
Beobachtungswerten und dem geschätzen Wert (oder einer Funktion davon).

\begin{exmp}
  Sei $\mu = \pE X_i$ der gesuchte Parameter. Dann ist der LSE für $\mu$ gerade
  \[ \hat{\mu} = \argmin_{\mu \in \real} \sum_{i=1}^n (X_i - \mu)^2. \]
  Zu minimieren ist also
  \[ f(\mu) := \sum_{i=1}^n (X_i - \mu)^2 = \sum_{i=1}^n (X_i^2 - 2 \mu X_i +
    \mu^2 ).\]
  Da $f'(\mu) = -2 \sum_{i=1}^n X_i + 2 n \mu$ folgt, dass
  \[ \mu_0 = \rez{n} \sum_{i=1}^n X_i \]
  die einzige Extremalstelle von $f$ ist. Da zudem $f''(\mu) = 2n > 0$ handelt
  es sich bei $\mu_0$ um ein Minimum.

  Also ist
  \[ \hat{\mu} := \mu_0 = \bar{X} \]
  der LSE für $\mu$.
\end{exmp}
LSE verwendet man besonders oft in der Regressionsanalyse (siehe Kapitel
\ref{ch:regression}).

Anstelle der kleinsten Quadrate sind auch andere ``Verlustfunktionen'' wie zum
Beispiel der absolute Abstand verwendbar.

\subsection{Momentenmethode}
(Method of Moments, MoM)

\emph{Idee:} Im Allgemeinen charakterisiert die Folge $(\mu_k)_{k \in \nat}$ der
Momente einer Verteilung die Verteilung nicht eindeutig (Momentenproblem).

Innerhalb einer vorgegebenen Klasse von Verteilungen ist das Momentenproblem
jedoch oft eindeutig lösbar. Zum Beispiel ist bei der Normalverteilung bereits
Erwartungswert und Varianz ausreichend für eine eindeutige Zuordnung.

Annahmen:
\begin{itemize}
\item Es existiert ein $r \ge n$, so dass $\pE_\theta |X_i|^r < \infty.$
\item Für $k = 1, \ldots, r$ sind die Momente
  \[ \pE_\theta [X_i^k] = g_k( \theta ) \]
  als Funktionen vonn $\theta = (\theta_1, \ldots, \theta_m)$ gegeben.
\end{itemize}

Ansatz: Löse das Gleichungssystem
\begin{equation} %% 3.2
  \hat{\mu}_k = g_k( \theta ), \qquad k = 1, \ldots, r,
\end{equation}
wobei $\hat{\mu}_k = \rez{n} \sum_{i=1}^n X_i^k$ das $k$-te empirische Moment
ist.

\begin{defn} %%3.17
  Ist (3.2) eindeutig lösbar, so heißt die Lösung $\hat{\theta} = T(X_1, \ldots,
  X_n)$ der \emph{Momentenschätzer} von $\theta$.
\end{defn}

\begin{exmp} %%3.18
  Es seien $X_i \sim N(\mu, \sigma^2)$ unabhängig mit unbekannten Parametern
  $\theta:(\mu, \sigma^2)$. Dann gilt $\pE_\theta | X_i |^k < \infty$ für alle
  $k \in \nat$.

  Wir wählen $r$ so klein, wie möglich, um wenige Gleichungen lösen zu müssen,
  aber groß genug um die unbekannten Parameter eindeutig festzulegen.

  Da $\theta \in \real^2$ wähle $r = 2$ und
  \begin{align*}
    g_1( \mu, \sigma^2) &= \pE[X_1] = \mu, \\
    g_2( \mu, \sigma^2) &= \pE[X_1^2] = \hat{\sigma}^2 + \hat{\mu}^2,
  \end{align*}
  so dass nach (3.2)
  \[ \rez{n} \sum_{i=1}^n X_i = \hat{\mu} \quad \text{und} \quad \rez{n}
    \sum_{i=1}^n X_i^2 = \sigma^2 + \mu^2 \]
  zu lösen ist. Es folgt
  \[ \begin{aligned}
      \hat{\mu} &= \rez{n} \sum_{i=1}^n X_i = \bar{X}, \\
      \hat{\sigma}^2 &= \rez{n} \sum_{i=1}^n X_i^2 - \hat{\mu}^2
      \overset{\text{Lem. } (2.10)}{=} \rez{n} \sum_{i=1}^n (X_i - \bar{X})^2.
    \end{aligned}
  \]
  Diese Schätzer sind aus den Beispielen 3.1 und 3.3 bekannt. Insbesondere ist
  $\hat{\sigma}^2$ nicht erwartungstreu.
\end{exmp}

\begin{lem} %% 3.19
  Falls die Funktion $g = (g_1, \ldots, g_r) : \Theta \to S \subset \real^r$
  bijektiv ist und ihre Inverse $g^{-1}: S \to \real^m$ stetig ist, so ist der
  Momentenschätzer $\hat{\theta}$ von $\theta$ stark konsistent.
\end{lem}

\begin{proof}
  Es gilt
  \[ \hat{\theta} = T(X_1, \ldots, X_n) = g^{-1}( \hat{\mu}_1, \ldots,
    \hat{\mu}_r ) \xrightarrow{\text{f.s.}, n \to \infty} \theta, \]
  da $g^{-1}$ stetig ist und nach Satz 3.14 gilt
  \[ \hat{\mu}_k \xrightarrow{\text{f.s.}, n \to \infty} g_k(\theta) \]
  für $k = 1, \ldots, r$.
\end{proof}

\begin{exmp}
  Es seien $X_i \sim U[-\theta, \theta]$ unabhängig und gleichverteilt mit
  unbekanntem Parameter $\theta > 0$. Insbesondere gilt $\pE_\theta |X_i|^k <
  \infty$ für alle $k \in \nat$ und
  \begin{align*}
    g_1( \theta ) &= \pE_\theta[X_i] = 0 \\
    g_2( \theta ) &= \pE_\theta[X_i^2]
                    = \rez{2 \theta} \int_{\theta}^\theta x^2 \diffop x \\
                  &= \rez{2 \theta} \left[ \frac{x^2}{3} \right]_{-\theta}^\theta \\
                  &= \rez{6 \theta} (\theta^3 - (-\theta)^3) = \frac{\theta^2}{3}.
  \end{align*}
  Es folgt das Gleichungssystem
  \[ \rez{n} \sum_{i=1}^n X_i = 0 \quad \text{und} \quad
    \rez{n} \sum_{i=1}^n X_i^2 = \frac{\theta^2}{3}, \]
  so dass
  \[ \hat{\theta} = \sqrt{\frac{3}{4} \sum X_i^2}. \]
  Es sind also in manchen Fällen $r > m$ Gleichungen in (3.2) nötig, um den
  Momentenschätzer eindeutig zu bestimmen.
\end{exmp}

\subsection{Maximum-Likelihood-Methode}
(Maximum-Likelihood estimation, MLE)

\emph{Idee:} Wähle den gesuchten Parameter so, dass die Wahrscheinlichkeit der
angenommenen Realisierung maximal ist.

\emph{Annahmen:} Die Verteilungen der Familie $\{F_\theta, \theta \in \Theta \}$
seien entweder alle diskret oder absolutstetig.

\begin{defn} %% 3.21
  \renewcommand{\thefootnote}{\fnsymbol{footnote}}
  \begin{enumerate}
  \item Sind die Verteilungen $\{F_\theta, \theta \in \Theta \}$ absolutstetig
    mit Dichten $\{ f_\theta, \theta \in \Theta \}$, so ist
    \[ L( X_1, \ldots, X_n; \theta) := \prod_{i=1}^n f_\theta(x_i), \quad
      \theta \in \Theta \]
    die \emph{Likelihood-Funktion} der Stichprobe $(X_1, \ldots, X_n) \in
    \real^n$.
  \item Sind die Verteilungen $\{ F_\theta, \theta \in \Theta \}$ diskret mit
    Zähldichten $\{ p_\theta, \theta \in \Theta \}$, $p_\theta(x) = \pP(X_i =
    x)$, $x \in S = \operatorname{supp}\footnotemark (X_i)$, so ist
    \[ L( X_1, \ldots, X_n; \theta) = \prod_{i=1}^n p_\theta(x_i), \quad \theta
      \in \Theta \]
    die \emph{Likelihood-Funktion} der Stichprobe $(X_1, \ldots, X_n) \in S^n$.
    \footnotetext{%
    $\operatorname{supp}$ ... support, Träger $= \{ x : p_\theta(x) > 0 \}$.}
  \end{enumerate}
  \renewcommand{\thefootnote}{\arabic{footnote}}
\end{defn}

\emph{Ansatz:} Wähle den Schätzer $\hat{\theta}$ so, dass die
Likelihood-Funktion maximiert wird.

\begin{defn} %% 3.22
  Besitzt die Likelihood-Funktion $L(X_1, \ldots, X_n; \theta)$ ein eindeutiges
  Maximum in $\hat{\theta} \in \Theta$, so heißt
  \[ \hat{\theta} = T(X_1, \ldots, X_n ) = \argmax_{\theta \in \Theta} L(X_1,
    \ldots, X_n; \theta ) \]
  der \emph{Maximum-Likelihood-Schätzer} für $\theta$.

  Da das Ableiten der Likelihood-Funktion häufig mühsam ist (Produktregel!)
  betrachtet man stattdessen meist die \emph{$log$-Likelihood-Funktion}
  \[ \begin{aligned}
      l( X_1, \ldots, X_n; \theta )
      &= \log L( X_1, \ldots, X_n; \theta ) \\
      &= \begin{cases}
        \sum_{i=1}^n \log f_\theta(x_i), &F_\theta \text{ absolutstetig,} \\
        \sum_{i=1}^n \log p_\theta(x_i), &F_\theta \text{ diskret.}
      \end{cases}
    \end{aligned}
  \]
  Da der Logarithmus streng monoton wachsend ist, gilt
  \[ \hat{\theta} = \argmax_{\theta \in \Theta} l(X_1, \ldots, X_n; \theta ) \]
  und diese Maximalstelle ist in der Regel deutlich leichter zu berechnen.
\end{defn}

\begin{exmp} %% 3.23
  Es seien $X_i \sim B(\theta)$ unabhängig mit $\theta \in [0,1]$ unbekannt, das
  heißt
  \[ p_\theta(x) = \theta^x (1-\theta)^{1-x}, \quad x \in [0,1].\]
  Dann folgt
  \[ L(X_1, \ldots, X_n; \theta) = \prod_{i=1}^n \theta^{x_i}(1-\theta)^{x_i}
    = \theta^{\sum_{i=1}^n x_i} (1-\theta)^{n - \sum_{i=1}^n x_i}. \]
  \begin{enumerate}[i)]
  \item Falls $\bar{x} = 0$, so folgt
    \[ L(X_1, \ldots, X_n; \theta) = (1-\theta)^n, \]
    also
    \[ \hat{\theta} = \argmax_{\theta \in \Theta} L(X_1, \ldots, X_n; \theta) =
      0. \]
  \item Falls $\bar{x} = 1$, so folgt
    \[ L(X_1, \ldots, X_n; \theta) = \theta^n, \]
    also
    \[ \hat{\theta} = \argmax_{\theta \in \Theta} L(X_1, \ldots, X_n; \theta) =
      1. \]
  \item Für $\bar{x} \in (0,1)$ bilde die $log$-Likelihood-Funktion
    \begin{align*}
      l(X_1, \ldots, X_n; \theta)
      &= n \bar{x} \log \theta + n (1-\bar{x}) \log (1-\theta) \\
      \Rightarrow \quad \pdiff{}{\theta} l(X_1, \ldots, X_n; \theta)
      &= \frac{n \bar{x}}{\theta} - \frac{n(1-x)}{1-\theta} \overset{!}{=} 0 \\
      0
      &= \bar{x}(1-\theta) - (1-\bar{x})\theta = \bar{x} - \theta.
    \end{align*}
    Also ist $\hat{\theta} = \bar{x}$ der einzige Kandidat für den MLE.
    \[ \frac{\partial^2}{\partial \theta^2} l(X_1, \ldots, X_n; \theta)
      = - \frac{n \bar{x}}{\theta^2} - \frac{n(1 - \bar{x})}{(1-\theta)^2} <
      0. \]
    Also liegt in $\bar{x}$ tatsächlich ein Maximum vor, $\hat{\theta} =
    \bar{x}$ ist der MLE für $\theta$.
  \end{enumerate}
  Nun seien $X_i \sim U[0, \theta]$ unabhängig und $\theta > 0$ unbekannt. Dann
  gilt
  \begin{align*}
    L(X_1, \ldots, X_n; \theta)
    &= \prod_{i=1}^n f_\theta(x_i)
      = \prod_{i=1}^n \rez{\theta} \ind_{\{ x_i \in [0,\theta]\}} \\
    &= \begin{cases}
      \theta^{-n}, &0 \le x_1, \ldots, x_n \le \theta, \\
      0, &\text{sonst}
    \end{cases} \\
    &= \begin{cases}
      \theta^{-n}, &\min \{x_1, \ldots, x_n \} \ge 0
      \wedge \max \{ x_1, \ldots, x_n \}\le \theta, \\
      0, &\text{sonst.}
    \end{cases}
  \end{align*}
  Es folgt als Maximum-Likelihood-Schätzer
  \[ \hat{\theta} = \argmax_{\theta \in \Theta} L(X_1, \ldots, X_n; \theta) =
    X_{(n)} = \max \{ x_1, \ldots, x_n \}. \]
\end{exmp}

\emph{Beachte:}
\begin{itemize}
\item Maximum-Likelihood-Schätzer müssen im Allgemeinen nicht existieren.
\item Eine explizite Formel für den ML-Schätzer existiert nur in Spezialfällen.
  Oft werden daher numerische Verfahren zur Bestimmung des Schätzers angewandt
  (zum Beispiel Newton-Raphson, Fisher Scoring, usw.).
\end{itemize}

\begin{thm} %% 3.24
  Sei $m = 1$ und $\Theta$ ein offenes Intervall in $\real$. Sei $\{ F_\theta:
  \theta \in \Theta \}$ eine Verteilungsfamilie, welche nur absolutstetige oder
  diskrete Verteilungen beinhaltet und welche identifizierbar ist. Die
  Likelihood-Funktion $L$ sei unimodal, das heißt für
  \[ \hat{\theta} = \argmax_{\theta \in \Theta} L(X_1, \ldots, X_n; \theta) \]
  gilt
  \[ \left\{ \begin{aligned}
        &L(X_1, \ldots, X_n; \theta) & &\text{ist steigend für alle } \theta
        < \hat{\theta}, \\
        &L(X_1, \ldots, X_n; \theta) & &\text{ist fallend für alle } \theta
        > \hat{\theta}.
      \end{aligned}
    \right.
  \]
  Dann gilt
  \[ \hat{\theta} = T(X_1, \ldots, X_n) \xrightarrow{\pP, n \to \infty}
    \theta. \]
\end{thm}

\begin{defn} %% 3.25
  Seien $\theta, \theta' \in \Theta$ und sei
  \[ L(x; \theta) = \begin{cases}
      f_\theta(x) &\text{im absolutstetigen Fall,} \\
      p_\theta(x) &\text{im diskreten Fall}
    \end{cases}
  \]
  die (Zähl-)dichte von $F_\theta$ bzw. $\pP_\theta$. Dann ist die
  \emph{Kullbach-Leibler-Divergenz} $\D( \pP_\theta \| \pP_{\theta'} )$ von
  $\pP_\theta$ und $\pP_{\theta'}$ definiert als
  \[ \D( \pP_\theta \| \pP_{\theta'} ) := \pE_\theta [ \log L(X; \theta) ] -
    \pE_\theta[ \log L(X; \theta' )]. \]
  Insbesondere ergibt sich für zwei absolutstetige Verteilungen
  \[ \D( \pP_\theta \| \pP_{\theta'} ) =
    \begin{cases}
      \int_\real \log \left( \frac{f_\theta(x)}{f_{\theta'}(x)} f_\theta(x)
        \diffop x \right), &\pP_\theta( L(X; \theta') = 0 ) = 0, \\
      \infty & \pP_\theta( L(X; \theta') = 0 ) > 0
    \end{cases}
  \]
  und für zwei diskrete Verteilungen
  \[ \D( \pP_\theta \| \pP_{\theta'} ) =
    \begin{cases}
      \sum_{x \in I} \log \left( \frac{p_\theta(x)}{p_{\theta'}(x)} p_\theta(x)
        \diffop x \right), &\pP_\theta( L(X; \theta') = 0 ) = 0, \\
      \infty & \pP_\theta( L(X; \theta') = 0 ) > 0.
    \end{cases}
  \]
  Offenbar ist $\D( \pP_\theta \| \pP_{\theta'} )$ nicht symmetrisch, also keine
  Metrik.
\end{defn}

\begin{lem} %% 3.26
  Seien $\theta, \theta' \in \Theta$, dann gelten
  \begin{enumerate}
  \item $\D( \pP_\theta \| \pP_{\theta'} ) \ge 0$,
  \item $\D( \pP_\theta \| \pP_{\theta'} ) = 0$ $\Rightarrow$ $\pP_\theta =
    \pP_{\theta'}$, also $\theta = \theta'$.
  \end{enumerate}
\end{lem}

\begin{proof}
  Wir betrachten den absolutstetigen Fall, der diskrete läuft analog.

  Zu 1.: Falls $\pP_\theta( L(X; \theta') = 0 ) > 0$, so ist nichts zu zeigen.
  Sei also $\pP_\theta( L(X; \theta') = 0 ) = 0$. Setze
  \[ g(x) := \begin{cases}
      \frac{f_\theta(x)}{f_{\theta'}(x)}, & f_{\theta'}(x) \ne 0, \\
      1, &\text{sonst.}
    \end{cases}
  \]
  Dann gilt mit Wahrscheinlichkeit 1, dass
  \[ L(x; \theta) = g(x) L( x; \theta' ). \]
  Definiere $h(x) := 1 - x + x \log x$, $x > 0$. Dann ist $h$ konvex, denn
  \begin{align*}
    h'(x) &= -1 + \log x +1 = \log x, \\
    h''(x) &= \rez{x} > 0.
  \end{align*}
  Zudem besitzt $h$ nur ein Minimum bei $x = 1$, welches auch die einzige
  Nullstelle von $h$ ist. Es gilt also $h(x) \ge 0$. Sei $X$ Zufallsvariable mit
  $X \sim F_{\theta'}$, also mit Dichte $f_{\theta'}$. Dann ist
  \begin{align*}
    0 &\le \pE_{\theta'}[ h(g(x)) ] \\
      &= 1 - \pE_{\theta'}[g(x)] + \pE_{\theta'}[g(x) \log(g(x))] \\
      &= 1- \int \frac{f_\theta(x)}{f_{\theta'}(x)} f_{\theta'} \diffop x +
        \int \frac{f_\theta(x)}{f_{\theta'}(x)}
        \log \left( \frac{f_\theta(x)}{f_{\theta'}(x)} \right)
        f_{\theta'}(x) \diffop x \\
      &= 1 - 1 + \int
        \log \left( \frac{f_\theta(x)}{f_{\theta'}(x)} \right)
        f_\theta(x)\diffop x \\
      &= \D( \pP_\theta || \pP_{\theta'} ).
  \end{align*}

  Zu 2.: Falls $\D( \pP_\theta || \pP_{\theta'} ) = 0$, so folgt
  \[ \pE_{\theta'}[ h(g(X))] = 0. \]
  Da jedoch $h(g(X)) \ge 0$, muss also fast sicher $h(g(X)) = 0$ gelten. Damit
  ist $g(X) = 1$ fast sicher und damit gilt entweder $f_\theta(x) =
  f_{\theta'}(x)$ oder $f_{\theta'} = 0$. Also folgt $\pP_\theta = \pP_{\theta'}$.
\end{proof}

\begin{proof}[Beweis von Satz 3.24]
  Zu zeigen: Für alle $\eps > 0$ gilt
  \begin{equation}
    \pP_\theta \left( | \hat{\theta} - \theta | > \eps \right)
    \xrightarrow{n \to \infty} 0.
  \end{equation}
  Da die Familie $\{ F_\theta, \theta \in \Theta \}$ ist, gilt dann
  \[ \D( \pP_\theta || \pP_{\theta \pm \eps} ) > \delta > 0 \]
  nach Lemma 3.26.

  Um (3.3) zu zeigen, genügt es, eine untere Schranke für $\pP(|\hat{\theta} -
  \theta| \le \eps)$ zu finden, welche gegen 1 konvergiert.
  \[ \begin{aligned}
      \{ |\hat{\theta} - \theta| < \eps \}
      &\supseteq \left\{
        L(X_1, \ldots, X_n; \theta) > L(X_1, \ldots, X_n; \theta \pm \eps)
      \right\} \\
      &= \left\{
        \frac{L(X_1, \ldots, X_n; \theta)}{L(X_1, \ldots, X_n; \theta \pm \eps)}
        > 1
      \right\} \\
      &\supseteq \left\{
        \frac{L(X_1, \ldots, X_n; \theta)}{L(X_1, \ldots, X_n; \theta \pm \eps)}
        > e^{n \delta}
      \right\} \\
      &= \left\{
        \rez{n} \log \left( 
          \frac{L(X_1, \ldots, X_n; \theta)}{L(X_1, \ldots, X_n; \theta \pm \eps)}
          > \delta
         \right)
      \right\} =: E_+ \cap E_-.
    \end{aligned}
  \]
  Das erste ``$\supseteq$'' folgt aus der Unimodalität.

  Es gilt also
  \[ \begin{aligned}
      \pP_\theta( |\hat{\theta} - \theta| < \eps )
      &\ge \pP_\theta( E_+ \cap E_- ) \\
      &= \pP_\theta( E_+ ) + \pP_\theta( E_- ) - \pP_\theta( E_+ \cup E_- ).
    \end{aligned}
  \]
  Wenn
  \begin{equation}
    \lim_{n \to \infty} \pP_\theta( E_\pm ) = 1,
  \end{equation}
  dann gilt
  \[ 1 = \lim_{n \to \infty} \pP_\theta( E_\pm ) \le
    \lim_{n \to \infty} \pP_\theta( E_+ \cup E_- ) \le 1, \]
  also
  \[ \lim_{n \to \infty} \pP_\theta( E_+ \cup E_-) = 1 \]
  so dass
  \[ 1 \ge \lim_{n \to \infty} \pP_\theta( |\hat{\theta} - \theta| < \eps ) \ge
    1 + 1 - 1 = 1 \]
  und damit
  \[ \lim_{n \to \infty} \pP_\theta( |\hat{\theta} - \theta| \le \eps) \ge
    \lim_{n \to \infty} \pP_\theta( |\hat{\theta} - \theta < \eps ) = 1, \]
  also die Behauptung (3.3).

  Zeige (3.4): Wir nehmen an, dass $\{ F_\theta, \theta \in \Theta \}$ nur
  absolutstetige Verteilungen enthält. Wir betrachten $\pP_\theta(E_+)$ mit
  \[ E_+ = \left\{
      \rez{n} \log \left(
        \frac{L(X_1, \ldots, X_n; \theta)}{L(X_1, \ldots, X_n; \theta + \eps)} 
      \right) > \delta
    \right\}. \]
  Setze
  \[ g(x) := \begin{cases}
      \frac{f_\theta(x)}{f_{\theta + \eps}(x)}, &f_{\theta + \eps}(x) > 0, \\
      1, &\text{sonst.}
    \end{cases}
  \]
  Wir machen eine Fallunterscheidung basierend auf $\D( \pP_\theta \|
  \pP_{\theta + \eps})$,
    \[ \D( \pP_\theta \| \pP_{\theta'} ) =
    \begin{cases}
      \int_\real \log \left( \frac{f_\theta(x)}{f_{\theta + \eps}(x)} f_\theta(x)
        \diffop x \right), &\pP_\theta( L(X; \theta + \eps) = 0 ) = 0, \\
      \infty & \pP_\theta( L(X; \theta + \eps) = 0 ) > 0
    \end{cases}
  \]

  \begin{enumerate}
  \item $\D( \pP_\theta \| \pP_{\theta + \eps}) < \infty$. Dann ist
    \[ \pP_\theta( L(X; \theta + \eps) > 0) = 1. \]
    Zudem ist
    \[ \begin{aligned}
        \rez{n} \log \frac{L(X_1, \ldots, X_n; \theta)}{L(X_1, \ldots, X_n;
          \theta + \eps)}
        &= \rez{n} \sum_{i=1}^n \log
        \frac{L(X_i; \theta)}{L(X_i; \theta + \eps)} \\
        &= \rez{n} \sum_{i=1}^n \log g(X_i) \\
        &\xrightarrow{\text{f.s.}}
        \pE_\theta[\log g(X_1)]
      \end{aligned}
    \]
    nach dem starken Gesetz der großen Zahlen, denn
    \[ \pE_\theta[ \log g(X_1)] = \int \log(g(x)) f_\theta(x) \diffop x =
      \D( \pP_\theta \| \pP_{\theta + \eps}) < \infty. \]
    Da $\D( \pP_\theta \| \pP_{\theta + \eps}) > \delta > 0$ folgt damit
    \[ \pP_\theta (E_+) \xrightarrow{n \to \infty} 1. \]
  \item $\D( \pP_\theta \| \pP_{\theta + \eps}) = 0$ und $\pP_\theta(L(X, \theta
    + \eps) = 0) = 0$: Es gilt
    \[ g(x) = \frac{f_\theta(x)}{f_{\theta + \eps}(x)} \quad \text{ fast
        sicher.} \]
    Wir nutzen ein Abschneideargument: Für $c > 0$ fix gilt
    \[ \pE_\theta \left[
        \log \left(
          \min \{ g(X_1), c \}
        \right)
      \right] < \infty \]
    und damit wie unter 1.:
    \[ \rez{n} \sum_{i=1}^n \log \left(
        \min \{ g(X_i), c \}
      \right)
      \xrightarrow{\text{f.s.}}
      \pE_\theta \left[
        \log \left(
          \min \{ g(X_1), c \}
        \right)
      \right]
      \xrightarrow{c \to \infty}
      \D( \pP_\theta \| \pP_{\theta + \eps}).
    \]
    Da jedoch
    \[ E_+ \supset \left\{
        \rez{n} \sum_{i=1}^n \log
        \left(
          \min \{ g(X_1), c \}
        \right)
        > \delta
      \right\}, \]
    folgt
    \[ \pP_\theta( E_+ ) \ge \pP
      \left(
        \left\{ \rez{n} \sum_{i=1}^n \log
        \left(
          \min \{ g(X_1), c \}
        \right)
        > \delta
      \right\}
    \right)
    \xrightarrow{n \to \infty} 1.
  \]
\item $\D( \pP_\theta \| \pP_{\theta + \eps}) = \infty$ und $\pP_\theta(L(X, \theta
  + \eps) = 0) = a > 0$: In diesem Fall gilt
  \[ \begin{aligned}
      &\phantom{=} \pP_\theta \left(
        \rez{n} \log
        \frac{L(X_1, \ldots, X_n; \theta)}{L(X_1, \ldots, X_n; \theta + \eps)}
        \infty
      \right) \\
      &= 1 - \pP_\theta \left(
        \rez{n} \log
        \frac{L(X_1, \ldots, X_n; \theta)}{L(X_1, \ldots, X_n; \theta + \eps)}
        < \infty
      \right) \\
      &= 1 - \pP_\theta \left(
        \bigcap_{i=1}^n \{ L(X_i, \theta + \eps) > 0 \}
      \right) \\
      &= 1 - (1-a)^n \xrightarrow{n \to \infty} 1,
    \end{aligned}
  \]
  also insbesondere $\pP_\theta(E_+) \xrightarrow{n \to \infty} 1$. \qedhere
  \end{enumerate}
\end{proof}

\section{Intervallschätzung}
\emph{Idee:} Beziehe die Präzision der Schätzung mit ein.

\begin{defn}
  Sei $\alpha \in (0,1)$ fix, dann liefern die Statistiken
  \[ g_u = G_u(X_1, \ldots, X_n)
    \quad \text{und} \quad
    g_o = G_o(X_1, \ldots, X_n) \]
  der Zufallsstichprobe $(X_1, \ldots, X_n)$ ein
  \emph{$1-\alpha$-Konfidenzintervall} für den Parameter $\theta$, wenn
  \[ \pP_\theta( g_u \le g_o ) \le 1
    \quad \text{und} \quad
    \pP_\theta( g_u \le \theta \le g_o ) = 1 - \alpha. \]
  Für eine gegebene Realisierung $(X_1, \ldots, X_n) = (x_1, \ldots, x_n)$ ist
  \[ [G_u(x_1, \ldots, x_n), G_o(x_1, \ldots, x_n)] \]
  das \emph{realisierte $1-\alpha$-Konfidenzintervall} für $\theta$.
\end{defn}

\emph{Beachte:}
\begin{itemize}
\item $[g_u, g_o]$ ist ein zufälliges Intervall.
\item \textbf{Nicht}: $\theta$ liegt mit Wahrscheinlichkeit $1-\alpha$ im
  Konfidenzintervall, denn $\theta$ ist \emph{nicht} zufällig.
\item Sondern: $\theta$ wird mit Wahrscheinlichkeit $1-\alpha$ vom
  Konfidenzintervall überdeckt.
\item Übliche Werte für die \emph{Irrtumswahrscheinlichkeit} sind
  \[ \alpha = \num{0.1}, \quad
    \alpha = \num{.05}, \quad
    \alpha = \num{.01}. \]
  Die resultierenden \emph{Überdeckungswahrscheinlichkeiten} sind
  \[ 1 - \alpha = \num{.9} = 90 \%, \quad
    1 - \alpha = \num{.95} = 95 \%, \quad 
    1 - \alpha = \num{.99} = 99 \%. \]
\item Das Konfidenzintervall wie oben definiert ist nicht eindeutig, da die
  Irrtumswahrscheinlichkeit auf das Über- oder Unterschreiten des Intervalls
  aufgeteilt wird. Daher betrachtet man meist \emph{symmetrische}
  Konfidenzintervalle, bei denen
  \[ \pP_\theta( \theta < g_u ) = \frac{\alpha}{2}
    = \pP_\theta( \theta > g_o ) \]
  gilt.

  Bei \emph{einseitigen} Konfidenzintervallen wird eine Grenze auf $\pm \infty$
  gesetzt, betrachte also $(-\infty, g_o]$ oder $[g_u,\infty)$. Es muss dann
  \[ \pP_\theta( \theta \le g_o ) = 1-\alpha \quad \text{oder} \quad
    \pP_\theta(\theta \ge g_u) = 1-\alpha \]
  gelten.
\end{itemize}

\begin{exmp}
  Sei $(X_1,\ldots,X_n)$ eine Zufallsstichprobe normalverteilter Daten mit $X_i
  \sim \ndist(\mu, \sigma^2)$. Gesucht ist das Konfidenzintervall für den
  unbekannten Parameter $\mu$.
  
  1. Fall: $\sigma^2$ ist bekannt. Verwende das arithmetische Mittel:
  \[ \bar{X} = \rez{n} \sum_{i=1}^n X_i \sim \ndist\left(\mu,
      \frac{\sigma^2}{n}\right). \]
  \[ \Rightarrow \quad U := \frac{\bar{X}-\mu}{\sigma / \sqrt{n} }
    \sim \ndist(0,1), \]
  das heißt, die Verteilung von $U$ ist bekannt und unabhängig von $\mu$. Also
  gilt
  \[ \pP( -z_{1 - \frac{\alpha}{2}}) \le U \le z_{1 - \frac{\alpha}{2}}) = 1
    - \alpha \]
  mit $z_\alpha = \alpha$-Quantil der Standard-Normalverteilung. Damit
  \[ \begin{aligned}
      1-\alpha
      &= \pP\left( - z_{1 - \frac{\alpha}{2}} <
        \frac{\bar{X}-\mu}{\sigma/\sqrt{n}} \le z_{1 - \frac{\alpha}{2}}
      \right) \\
      &= \pP\left( - z_{1 - \frac{\alpha}{2}} \frac{\sigma}{\sqrt{n}} <
        \bar{X}-\mu
        \le z_{1 - \frac{\alpha}{2}} \frac{\sigma}{\sqrt{n}}
      \right) \\
      &= \pP \left( \bar{X} - z_{1 - \frac{\alpha}{2}} \frac{\sigma}{\sqrt{n}} \le
        \mu < \bar{X} +
        z_{1 - \frac{\alpha}{2}} \frac{\sigma}{\sqrt{n}}
      \right).
    \end{aligned}
  \]
  Also gilt:
  \[ \left[ \bar{X} - z_{1 - \frac{\alpha}{2}} \frac{\sigma}{\sqrt{n}},
      \bar{X} + z_{1 - \frac{\alpha}{2}} \frac{\sigma}{\sqrt{n}} \right] \]
  ist ein $1-\alpha$-Konfidenzintervall für $\mu$.
  \begin{itemize}
  \item Das Konfidenzintervall ist symmetrisch um $\bar{X}$ (den
    Punktschätzer).
  \item Wächst $\alpha$, wird das Konfidenzintervall kleiner.
  \item Wächst $n$, so wird das Intervall kleiner (also die Schätzung
    präziser).
  \end{itemize}
  
  2. Fall: $\sigma^2$ unbekannt. Verwende anstelle von $U = \frac{\bar{X} -
    \mu}{\sigma / \sqrt{n}}$
  \[ T := \frac{\bar{X} - \mu}{s/\sqrt{n}} = U \cdot \sqrt{
      \frac{s^2}{\sigma^2}} = \frac{U}{\sqrt{V / (n-1)}}\]
  mit der empirischen Varianz $s^2$ und
  \[ V = \frac{(n-1)s^2}{\sigma^2} = \sum_{i=1}^n
    \left( \frac{X_i - \bar{X}}{\sigma} \right)^2. \]
  Schreibe $Z_i = \frac{X_i - \mu}{\sigma} \sim \ndist(0,1)$, so gilt
  \[ V = \sum_{i=1}^n (Z_i - \bar{Z})^2 = \sum_{i=1}^n Z_i^2 - n \bar{Z}^2 \]
  mit
  \[ \bar{Z} = \rez{n} \sum_{i=1}^n Z_i = \rez{\sigma n}
    \left( \sum_{i=1}^n (X_i - \mu) \right)
    = \frac{n \bar{X} - n \mu}{n \sigma} = \rez{\sqrt{n}} U. \]
  Setze nun
  \[ Y = (Y_1, \ldots, Y_n)^\top = A(Z_1, \ldots, Z_n)^\top = A Z^\top \]
  mit $A \in \realmat{n}{n}$ orthogonal, so dass
  \[ Y_1 = \rez{n} \sum_{i=1}^n Z_i = \sqrt{n} \bar{Z} = U. \]
  Dann sind $Y_1, \ldots, Y_n$ unabhängig mit $\ndist(0,1)$ und
  \[ \sum_{i=1}^n Y_i^2 = Y Y^\top = (AZ)(AZ)^\top = ZZ^\top = \sum_{i=1}^n
    Z_i^2. \]
  Es folgt
  \[ V = \sum_{i=1}^n Z_i^2 - n \bar{Z}^2 = \sum_{i=1}^n Y_i^2 - Y_1^2 =
    \sum_{i=2}^n Y_i^2 \sim \chi_{n-1}^2 \]
  und dass $V$ und $U$ unabhängig sind. Also ist $T$ t-verteilt mit $n-1$
  Freiheitsgraden. Damit gilt
  \[ \pP \left( -t_{n-1,1-\alpha/2} <
      \frac{\bar{X}-\mu}{s/\sqrt{n}} \le
      t_{n-1,1-\alpha/2} \right)
    = 1 - \alpha \]
  mit $t_{n-1,1-\alpha/2}$ dem $1-\alpha/2$-Quantil der t-Verteilung mit $n-1$
  Freiheitsgraden. Als Konfidenzintervall für $\mu$ folgt (wie unter 1.):
  \[ \left[ \bar{X} - t_{n-1,1-\alpha/2} \frac{s}{\sqrt{n}},
      \bar{X} + t_{n-1,1-\alpha/2} \frac{s}{\sqrt{n}} \right]. \]
  Quantilberechnung in R mit \verb+qt(.,n)+
\end{exmp}

\section{Asymptotische Normalität}
\begin{defn}
  Eine Folge von Schätzern $\hat{\theta}_n = T(X_1, \ldots, X_n) \in \real$ für
  einen Parameter $\theta$ ist \emph{asymptotisch normalverteilt}, falls für
  alle $\theta \in \Theta$ Folgen $\mu_n(\theta) \in \real$ und
  $\sigma_n(\theta) > 0$ existieren, so dass
  \[ \frac{\hat{\theta}_n - \mu_n(\theta)}{\sigma_n(\theta)} \xrightarrow{d} Z
    \sim \ndist(0,1). \]
\end{defn}

\emph{Beachte:}
\begin{itemize}
  \item Meist gilt $\mu_n(\theta) = \pE_\theta[T(X_1, \ldots, X_n)]$ und
    $\sigma_n^2(\theta) = \var_\theta(T(X_1, \ldots, X_n))$. Das muss aber nicht
    der Fall sein.
  \item Falls $\mu_n(\theta) = \pE_\theta[T(X_1, \ldots, X_n)]$ und
    $\sigma_n^2(\theta) = \var_\theta(T(X_1, \ldots, X_n))$ gewählt werden kann
    und $\lim_{n \to \infty} \sigma_n = 0$, so folgt aus der asymptotischen
    Normalität auch asymptotische Erwartungstreue und schwache Konsistenz (Lemma
    3.6).
\end{itemize}

\begin{exmp}
  \begin{enumerate}[(i)]
  \item Seien $X_1, \ldots, X_n$ i.i.d. Zufallsvariablen, so dass
    $\pE_\theta|X|^{2k} < \infty$, dann ist das $k$-te empirische Moment (Def.
    3.13) asymptotisch normalverteilt, dann nach dem zentralen Grenzwertsatz
    gilt
    \begin{align*}
      \frac{\hat{\mu} - \pE_\theta[\hat{\mu}_k]}{\sqrt{\var (\hat{\mu}_k)}}
      &= \frac{\rez{n} \sum_{i=1}^n X_i^k - \rez{n} \sum_{i=1}^n \pE[X_i]^k}
      {\sqrt{\var \left( \rez{n} \sum_{i=1}^n X_i^k \right)}} \\
      &= \frac{\hat{\mu}_k - \pE[X_1^k]}{\sqrt{\rez{n} \var{(X_1^k)}}} \\
      &= \sqrt{n} \cdot \frac{\hat{\mu}_k - \pE_\theta(X_1^k)}{\sqrt{\var(X_1^k)}}
        \to Z \sim \ndist(0,1)
    \end{align*}
  \item Die empirische Verteilungsfunktion $\hat{F}_n(x)$ (Def. 2.5) ist
    asymptotisch normalverteilt. Aus Satz 3.12 ist bekannt, dass für alle $x \in
    \real$
    \[ \hat{F}_n(x) = \rez{n} \sum_{i=1}^n \ind_{\{X_i \le x\}}, \]
    wobei die $Y_i = \ind_{\{X_i \le x\}} \sim \mathrm{Bernoulli}(F(x))$ i.i.d.
    sind. Also folgt aus dem zentralen Grenzwertsatz (angewandt auf $Y_i$):
    \[ \sqrt{n} \cdot \frac{\hat{F}_n(x) - F(x)}{\sqrt{F(x)(1-F(x))}}
    \to Z \sim \ndist(0,1), \quad n \to \infty. \]
  \end{enumerate}
\end{exmp}

\begin{rmrk}
  Für asymptotisch normalverteilte Schätzer lassen sich \emph{asymptotische
    Konfidenzintervalle} bestimmen: Sei $(X_1, \ldots, X_n)$ Zufallsstichprobe,
  sodass $\pE[X_i] = \mu$, $\var(X_i) = \sigma^2 < \infty$. Dann folgt aus
  Beispiel 3.30(i), dass $\bar{X}$ asymptotisch normalverteilt ist, es gilt also
  \[ \sqrt{n} \frac{\bar{X} - \mu}{\sigma} \xrightarrow{d} Z \sim
    \ndist(0,1). \]
  Zudem gilt nach Korollar 3.15 starke Konsistenz von $s^2$, also
  \[ s^2 \xrightarrow{\text{f.s.}} \sigma^2 \qRq
    \frac{\sigma^2}{s} \xrightarrow{\text{f.s.}} 1, \quad
    n \to \infty.
  \]
  Damit folgt nach dem Lemma von Slutzky
  \[ \sqrt{n} \cdot \frac{\bar{X} - \mu}{s}
    = \sqrt{n} \cdot \frac{\bar{X} - \mu}{\sigma} \cdot \frac{\sigma}{s}
    \xrightarrow{d} Z \cdot 1 \sim \ndist(0,1) \]
  und es folgt
  \[ \pP \left( -z_{1-\alpha/2} \le
      \frac{\bar{X} - \mu}{\sigma / \sqrt{n}} \le
      z_{1-\alpha/2} \right) \to 1 - \alpha, \quad n \to \infty, \]
  so dass
  \[ \left[ \bar{X} - z_{1-\alpha/2} \frac{s}{\sqrt{n}},
      \bar{X} + z_{1-\alpha/2} \frac{s}{\sqrt{n}} \right] \]
  ein asymptotisches Konfidenzintervall für $\mu$ ist.
\end{rmrk}

\begin{lem}[Delta-Methode]
  Sei $\hat{\theta}$ eine Folge von Schätzern, so dass
  \[ \frac{\hat{\theta}_n - \mu}{\sigma_n} \xrightarrow{d}
    Z \sim \ndist(0,1), \quad n \to \infty \]
  wobei $\lim_{n \to \infty} \sigma_n = 0$.

  Sei $g$ eine Funktion, welche in einer Umgebung von $\mu$ definiert ist und in
  $\mu$ differenzierbar ist mit $g'(\mu) \ne 0$. Dann gilt:
  \[ \frac{g(\hat{\theta}_n) - g(\mu)}{g'(\mu) \sigma_n}
    \xrightarrow{d} Z \sim \ndist(0,1), \quad n \to \infty. \]
\end{lem}

\begin{proof}
  Verwende eine Taylorentwicklung von $g$ um $\mu$:
  \[ g(\hat{\theta}_n) = g(\mu) + (\hat{\theta}_n) g'(\mu) + R_n. \]
  Also
  \[ \frac{g(\hat{\theta}_n) - g(\mu)}{g'(\mu) \sigma_n} =
    \underbrace{\frac{\hat{\theta}_n - \mu}{\sigma_n}}_{
      \xrightarrow{d} Z \sim \ndist(0,1)}
    + \frac{R_n}{g'(\mu) \sigma_n}. \]
  da $R_n$ nur höhere Potenzen von $(\hat{\theta}_n - \mu)$ enthält, gilt
  $\frac{R_n}{\sigma_n} \xrightarrow{d} 0$ und damit folgt die Behauptung.
\end{proof}

\begin{exmp}
  Seien $X_i \sim U[-\theta, \theta]$ i.i.d. mit $\theta > 0$ unbekannt. Dann
  ist nach Beispiel 3.30 der Momentenschätzer für $\theta$
  \[ \theta_n = T(X_1, \ldots, X_n) = \sqrt{\frac{3}{n} \sum_{i=1}^n X_i^2}
    = \sqrt{3 \hat{\mu}_2}, \]
  wobei
  \[ \frac{\mu_2 - \pE[\hat{\mu_2}]}{\sqrt{\var_\theta(\hat{\mu}_2)}}
    \xrightarrow{d} Z \sim \ndist(0,1). \]

  Hierbei ist
  \[ \pE[\hat{\mu}_2] = \pE[X_1^2] = \rez{2\theta} \int_{-\theta}^\theta x^2
    \diffop x = \frac{\theta^2}{3} \]
  und
  \[ \var_\theta(\hat{\mu}_2) = \rez{n} \var_\theta (X_1^2) =
    \rez{n} (\pE[X_1^4] - \pE_\theta[X_1^2]^2) = \ldots =
    \rez{n} \left( \frac{\theta^4}{5}-\frac{\theta^4}{9} \right)
    = \rez{n} \frac{4}{45} \theta^4, \]
  sodass
  \[ \frac{\hat{\mu}_2
      - \frac{\theta}{3}}{\sqrt{\rez{n} \frac{4}{45} \theta^4}}
    \xrightarrow{d} Z \sim \ndist(0,1). \]
  Da
  \[ \sigma_n = \sqrt{\frac{4}{45} \rez{n} \theta^4} \to 0, \quad n \to
    \infty \]
  ist Lemma 3.32 anwendbar. Setze $g: \real_+ \to \real : x \mapsto \sqrt{3x}$,
  dann ist $g$ differenzierbar mit $g'(x) = \rez{2} \sqrt{\frac{3}{x}} \ne 0$
  und es folgt
  \[ \frac{g(\hat{\mu}_2) -
      g(\pE_\theta[\hat{\mu}_2])}
    {g'(\pE_\theta[\hat{\mu}_2]) \sqrt{\var(\hat{\mu}_2)}}
    = \sqrt{5n} \frac{\hat{\theta}_n-\theta}{\theta}
    \xrightarrow{d} Z \sim \ndist(0,1), \quad n \to \infty. \]
\end{exmp}

Im Allgemeinen erhält man mit der Beweistechnik von Lemma 3.32:
\begin{prp}[Delta-Methode]
  Sei $\hat{\theta}_n$ eine Folge von Schätzern in $\real^m$, so dass
  \[ \frac{\hat{\theta}_n - \mu}{c_n} \xrightarrow{d} Z \sim \ndist(0, \Sigma),
  \quad n \to \infty, \]
  wobei $\mu \in \real^m$, $c_n \in \real$ mit $\lim_{n \to \infty} c_n = 0$,
  $\Sigma \in \realmat{m}{m}$ symmetrisch und nicht-negativ definit (positiv
  semidefinit).

  Sei $g : \real^n \to \real^k$ eine Funktion definiert durch
  \[ g(x) = (g_1(x), \ldots, g_k(x))^\top, \quad x \in \real^m, \]
  welche komponentenweise in einer Umgebung von $\mu$ stetig differenzierbar ist
  und setze
  \[ D = \left( \pdiff{g_i}{x_j}(\mu) \right)_{
      \substack{i=1,\ldots,k \\ j=1,\ldots,m}}. \]
  Sind alle Diagonalelemente von $D \Sigma D^\top$ nicht null, so gilt:
  \[ \frac{g(\hat{\theta}_n) - g(\mu)}{c_n} \xrightarrow{d}
  Z' \sim \ndist(0, D \Sigma D^\top), \quad n \to \infty. \]
\end{prp}

\begin{exmp}
  Seien $X_i \sim \ndist(\mu,\sigma^2)$ unabhängig mit unbekannten Parametern
  $(\mu,\sigma^2) = \theta$. Dann sind die Momentenschätzer für $\mu$ und
  $\sigma^2$ (Bsp. 3.18):
  \[ \hat{\mu} = \bar{X} - \hat{\mu}_1 \quad \text{und} \quad
    \hat{\sigma}^2 = \rez{n} \sum_{i=1}^n X_i^2 - \hat{\mu}_2
    = \hat{\mu}_2 - (\hat{\mu}_1)^2. \]
  Definiere $Y_i = \pmat{X_i \\ X_i^2}$, dann sind die $Y_i$ i.i.d. und nach dem
  zentralen Grenzwertsatz gilt
  \[ \sqrt{n} ( \bar{Y} - \pE_\theta[Y_1]) \xrightarrow{d} Z \sim(0,\Sigma) \]
  mit
  \[ \Sigma = \pE_\theta[Y_1 Y_1^\top] = \pE
    \left[ \pmat{ X_1^2 & X_1^3 \\ X_1^3 X_1^4} \right] \cdot
    \begin{pmatrix}
      \mu^2 + \sigma^2 & \mu^3 + 3 \mu \sigma^2 \\
      \mu^3 + 3 \mu \sigma^2 & \mu^4 + 6 \mu^2 \sigma^2 + 3 \sigma^4
    \end{pmatrix},
  \]
  also mit $\pmat{\hat{\mu}_1 \\ \hat{\mu}_2} = \bar{Y}$ asymptotisch
  normalverteilt.

  Definiere nun:
  \[ g : \real^2 \to \real^2, (x,y)^\top \mapsto (x, y+x^2)^\top, \]
  \[ D := \left( \pdiff{g}{x_j} \pE_\theta[Y] \right)
    = \pmat{ 1 & 0 \\ -2 \mu & 1 }, \]
  \[ D \Sigma D^\top =
    \begin{pmatrix}
      \mu^2 + \sigma^2 & \mu( \sigma^2 - \mu^2 ) \\
      \mu( \sigma^2 - \mu^2 ) & (\sigma^2 - \mu^2) + 2 \sigma^4
    \end{pmatrix}
  \]
  Die Delta-Methode (Satz 3.34) liefert nun
  \[ \sqrt{n} \left( \pmat{ \hat{\mu} \\ \hat{sigma}^2} -
      \pmat{\mu \\ \sigma^2} \right)
    \xrightarrow{d} Z \sim \ndist(0, D \Sigma D^\top )\]
\end{exmp}

\begin{defn}
  Sei $(X_1, \ldots, X_n)$ eine Zufallsstichprobe von i.i.d. Zufallsvariablen
  $X_i \sim F_\theta$, $\theta \in \Theta$, wobei $\Theta$ ein offenes Intervall
  in $\real$ und $\{ F_\theta, \theta \in \Theta \}$ eine Verteilungsfamilie
  ist, welche nur absolutstetige oder nur diskrete Verteilungen beinhaltet.
  Sei
  \[ \begin{aligned}
      L(x_1, \ldots, x_n; \theta) = \begin{cases}
        \prod_{i=1}^n f_\theta(x_i), &
        \text{im absolutstetigen Fall,} \\
        \prod_{i=1}^n p_\theta(x_i), &
        \text{im diskreten Fall,}
      \end{cases}
    \end{aligned}
  \]
  die Likelihood-Funktion (und damit die (Zähl-)Dichte) von $(X_1, \ldots,
  X_n)$. Dann ist die \emph{Fisher-Information} der Stichprobe gerade
  \[ I_n(\theta) := \pE_\theta \left[ 
      \left( \pdiff{}{\theta} \log L(X_1, \ldots, X_n; \theta) \right)^2
    \right], \quad \theta \in \Theta. \]
\end{defn}

\begin{thm} [Asymptotische Normalität der Maximum-Likelihood-Schätzer]
  Sei $m=1$ und $\Theta$ ein offenes Intervall in $\real$. Sei $\{ F_\theta,
  \theta \in \Theta \}$ eine Verteilungsfamilie, welche nur absolutstetige oder
  nur diskrete Verteilungen beinhaltet und identifizierbar ist. Sei
  $(X_1, \ldots, X_n)$ eine Zufallsstichprobe aus i.i.d. Zufallsvariablen $X_i
  \sim F_\theta$, so dass die folgenden Bedingungen erfüllt sind:
  \begin{enumerate}
  \item $0 < I_1(\theta) < \infty$.
  \item $B := \operatorname{supp} L(x; \theta) = \{ x \in \real : L(x,\theta) >
    0 \}$ hängt nicht von $\theta$ ab.
  \item $L(x;\theta)$ ist dreimal stetig differenzierbar in $\theta$ und es
    gelten für $k = 1,2$ und $\theta \in \Theta$ die Regularitätsbedingungen
    \[ \int_B \frac{\partial^k}{\partial \theta^k} L(x;\theta) \diffop x
      = \frac{\partial^k}{\partial \theta^k} \int_B L(x;\theta) \diffop x =
      0. \]
  \item Für jedes $\theta_0 \in \Theta$ existieren eine Konstante $\delta =
    \delta( \theta_0 )$, sowie eine messbare Funktion $g_{\theta_0} : B \to
    [0, \infty]$ mit $\pE[g_{\theta_0}(X_1)] < \infty$, so dass
    \[ \left| \frac{\partial^3}{\partial \theta^3} \log L(x;\theta) \right|
      \le g_{\theta_0} (x), \quad \forall x \in B, |\theta-\theta_0| < \delta. \]
  \end{enumerate}
  Ist $\hat{\theta} = T(X_1, \ldots, X_n)$ ein schwach konsistenter ML-Schätzer
  für $\theta$, so ist $\hat{\theta}$ asymptotisch normalverteilt:
  \[ \sqrt{n \cdot I_1(\theta)} (\hat{\theta} - \theta) \xrightarrow{d} Z \sim
    \ndist( 0, 1 ), \quad n \to \infty. \]
\end{thm}

Für den Beweis benötigt man das folgende Lemma:
\begin{lem}
  Sei $(X_1, \ldots, X_n)$ eine Zufallsstichprobe mit Fisher-Information
  $I^{(n)}(\theta)$ entsprechend Definition 3.36. Es gelten die Bedingungen 1.
  bis 3. von Satz 3.37. Dann gilt
  \[ n \cdot I_1(\theta) = I_n(\theta) = \var_\theta\left(
    \pdiff{}{\theta} \log L(X_1, \ldots, X_n; \theta) \right). \]
\end{lem}

\begin{proof}
  Es gilt
  \begin{align*}
    \pdiff{}{\theta} \log L(X_1, \ldots, X_n; \theta)
    &= \pdiff{}{\theta} \sum_{i=1}^n \log L(X_i; \theta) \\
    &= \sum_{i=1}^n \pdiff{}{\theta}  \log L(X_i; \theta) \\
    &= \sum_{i=1}^n \frac{\pdiff{}{\theta} L(X_i; \theta)}{L(X_i; \theta)},
  \end{align*}
  so dass (exemplarisch im absolutstetigen) Fall
  \begin{align*}
    \pE_\theta \left[ \pdiff{}{\theta} \log L(X_1, \ldots, X_n; \theta) \right]
    &= \sum_{i=1}^n \pE_\theta
      \left[ \frac{\pdiff{}{\theta} L(X_i; \theta)}{L(X_i; \theta)} \right] \\
    &= \sum_{i=1}^n \int_B
      \left[ \frac{\pdiff{}{\theta} L(X_i; \theta)}{L(X_i; \theta)} \right]
      \diffop x \\
    &= 0
  \end{align*}
  nach Bedingung 3.

  Insgesamt folgt damit
  \begin{align*}
    I_n(\theta)
    &= \var_\theta \left( \pdiff{}{\theta}
      \log L(X_1, \ldots, X_n; \theta) \right) \\
    &= \var_\theta \left( \sum_{i=1}^n \pdiff{}{\theta}
      \log L(X_i; \theta) \right) \\
    &= \sum_{i=1}^n \var_\theta \left( \pdiff{}{\theta}
      \log L(X_i; \theta) \right)
      \quad \text{($X_i$ unabhängig)} \\
    &= n \cdot \var_\theta \left( \pdiff{}{\theta}
      \log L(X_1; \theta) \right)
      \quad \text{($X_i$ gleichverteilt)} \\
    &= n \cdot \pE_\theta \left( \pdiff{}{\theta}
      \log L(X_1; \theta) \right)^2
      \quad \text{(analog zu obiger Rechnung)} \\
    &= n \cdot I_1(\theta). \qedhere
  \end{align*}
\end{proof}

\begin{proof}[Beweis zu Satz 3.37]
  Wir setzen
  \[ \ell_n(\theta) := \log L(X_1, \ldots, X_n; \theta), \quad \theta \in
    \Theta, \]
  sowie
  \[ \ell^{(k)}_n(\theta) := \frac{\diffop^k}{\diffop \theta^k \ell_n(\theta)},
    \quad k = 1, 2, 3. \]
  Ist $\hat{\theta}$ ein ML-Schätzer für $\theta$, so gilt
  $\ell_n^{(1)}(\hat{\theta}) = 0$. Andererseits folgt mittels Taylorentwicklung
  \[ \ell_n^{(1)}(\hat{\theta}) = \ell_n^{(1)} + (\hat{\theta} - \theta
    \ell_n^{(2)}(\theta)) + (\hat{\theta} - \theta)^2 \cdot
    \frac{\ell_n^{(3)}(\theta^*)}{2}, \]
  mit $\theta^*$ zwischen $\theta$ und $\hat{\theta}$, so dass insgesamt
  \begin{align*}
    \ell_n^{(1)}(\theta)
    &= -(\hat{\theta}-\theta) \left( \ell_n^{(2)}(\theta)
      + (\hat{\theta}-\theta) \cdot \frac{\ell_n^{(3)}(\theta^*)}{2} \right) \\
    (\hat{\theta}-\theta)
    &= \frac{\ell_n^{(1)}(\theta)}{\ell_n^{(2)}(\theta)
      + (\hat{\theta}-\theta) \cdot \frac{\ell_n^{(3)}(\theta^*)}{2}} \\
    \sqrt{n} (\hat{\theta}-\theta)
    &= \frac{\frac{\ell_n^{(1)}(\theta)}{\sqrt{n}}}{
      - \frac{\ell_n^{(2)}(\theta)}{n}
      + (\hat{\theta}-\theta) \cdot \frac{\ell_n^{(3)}(\theta^*)}{2n}}.
  \end{align*}
  Falls nun gelten
  \[ \frac{\ell_n^{(1)}(\theta)}{\sqrt{n}} \xrightarrow[n \to \infty]{d} Z'
    \sim \ndist(0,I_1(\theta)), \tag{i} \]
  \[ - \frac{\ell_n^{(2)}(\theta)}{n} \xrightarrow[n \to \infty]{\text{f.s.}}
    I_1(\theta), \tag{ii} \]
  \[ (\hat{\theta}-\theta) \cdot \frac{\ell_n^{(3)}(\theta^*)}{n}
    \xrightarrow[n \to \infty]{\pP} 0, \tag{iii} \]
  so folgt nach dem Satz von Slutzky auch
  \[ \sqrt{n} (\hat{\theta}-\theta) \xrightarrow[n \to \infty]{d}
    Z'' \sim \ndist(0, I_1(\theta)^{-1})\]
  und damit die Behauptung.

  Zu (i): Es gilt
  \[ \ell_n^{(1)}(\theta) = \pdiff{}{\theta} \log \prod_{i=1}^n L(X_i;\theta)
    = \pdiff{}{\theta} \sum_{i=1}^n \log L(X_i;\theta)
    = \sum_{i=1}^n \pdiff{}{\theta} \log L(X_i;\theta), \]
  wobei die $\pdiff{}{\theta} \log L(X_i;\theta)$ i.i.d. Zufallsvariablen mit
  Erwartungswert 0 und Varianz $I_1(\theta)$ sind (siehe Lemma 3.38). Also folgt
  nach dem zentralen Grenzwertsatz
  \[ \frac{\ell_n^{(1)}(\theta)}{\sqrt{n}} = \rez{\sqrt{n}} \sum_{i=1}^n
    \pdiff{}{\theta} \log L(X_i;\theta) \xrightarrow[n \to \infty]{d}
    Z' \sim \ndist(0,I_1(\theta)). \]

  Zu (ii): Nach dem starken Gesetz der großen Zahlen gilt mit
  $L^{(k)}(X_i;\theta) := \frac{\partial^k}{\partial \theta^k} L(X_i;\theta)$
  \begin{align*}
    - \frac{\ell_n^{(2)}(\theta)}{n}
    &= - \rez{n} \sum_{i=1}^n \frac{\partial^2}{\partial \theta^2}
      \log L(X_i; \theta) \\
    &= - \rez{n} \sum_{i=1}^n \frac{\partial}{\partial \theta}
      \frac{L^{(1)}(X_i; \theta)}{L(X_i; \theta)} \\
    &= \rez{n} \sum_{i=1}^n
      \left( \frac{L^{(1)}(X_i; \theta)}{L(X_i; \theta)}  \right)^2
      - \rez{n} \sum_{i=1}^n
      \frac{L^{(2)}(X_i; \theta)}{L(X_i; \theta)} \\
    &\xrightarrow[n \to \infty]{\text{f.s.}}
      \pE_\theta \left[ \left(
      \frac{L^{(1)}(X_i; \theta)}{L(X_i; \theta)}
      \right)^2 \right]
      - \pE_\theta \left[
      \frac{L^{(2)}(X_i; \theta)}{L(X_i; \theta)}
      \right] \\
    &= I_1(\theta),
  \end{align*}
  denn
  \[ \pE_\theta \left[ \frac{L^{(2)}(X_i; \theta)}{L(X_i; \theta)} \right]
    = \int_B \frac{\partial^2}{\partial \theta^2} L(x;\theta) \diffop x = 0 \]
  nach Bedingung 3.

  Zu (iii): Es gilt $\hat{\theta} \xrightarrow[n \to \infty]{\pP} \theta$, da
  $\hat{\theta}$ schwach konsistent ist, das heißt für alle $\eps > 0$ gilt
  \[ \pP_\theta \left( |\hat{\theta}-\theta| \le \eps \right) \to 1, \quad
    n \to \infty. \]
  Andererseits gilt nach Bedingung 4 für alle $\theta$, so dass
  $|\hat{\theta}-\theta| < \delta$:
  \begin{align*}
    \left| \frac{\ell_n^{(3)}(\theta^*)}{n} \right|
    &\le \rez{n} \sum_{i=1}^n \left| \frac{\partial^3}{\partial \theta^3}
      \log L(X_i; \theta^*) \right| \\
    &\le \rez{n} \sum_{i=1}^n g_\theta(X_i) \\
    &\xrightarrow[n \to \infty]{\text{f.s.}}
      \pE_\theta [g_\theta(X_1)] < \infty.
  \end{align*}
  Es existiert also eine Konstante $c > 0$, so dass
  \[ \pP_\theta \left( \left|
        \frac{\ell_n^{(3)}(\theta^*)}{n}
      \right| < c \right) \xrightarrow{n \to \infty} 1, \]
  und insgesamt folgt
  \[ (\hat{\theta}-\theta) \cdot \frac{\ell_n^{(3)}(\theta^*)}{2n}
    \xrightarrow[n \to \infty]{\pP} 0. \qedhere \]
\end{proof}

\section{Suffizienz und Effizienz}
\paragraph{Suffizienz} Schätzer/Statistiken bilden (Zufalls-)Stichproben auf
einen Schätzer/Schätzwert ab und reduzieren damit die enthaltene Information.
Suffiziente Schätzer vergeben dabei keine relevante Information. Formal:
\begin{defn} %3.39
  Sei $X=(X_1,\ldots, X_n)$ eine Zufallsstichprobe mit Verteilung $F_\theta$,
  $\theta \in \Theta$. $T(X)$ sei ein Schätzer für $\theta$. Dann ist $T$
  \emph{suffizient für $\theta$}, falls für alle $\theta_1, \theta_2 \in \Theta$
  \[ \pP_{\theta_1}( X = (x_1,\ldots, x_n) | T(X) = t ) =
    \pP_{\theta_2}( X = (x_1, \ldots, x_n) | T(X)=t ) \]
  gilt.
\end{defn}

\begin{thm}[Faktorisierungssatz von Neyman-Fisher] %% 3.40
  Sei $X = X_1, \ldots, X_n$ eine Zufallsstichprobe mit Verteilung $F_\theta$,
  $\theta \in \Theta$. $\{F_\theta, \theta \in \Theta\}$ enthalte nur diskrete
  oder absolutstetige Verteilungen. $T: \real^n \to \real^m$ sei ein Schätzer
  für $\theta$. Dann ist $T$ suffizient für $\theta$ genau dann, wenn
  $g : \real^m \times \Theta \to \real$ und $h: \real^n \to \real$ existieren,
  so dass
  \begin{equation} %% 3.5
    L(x; \theta) = g(T(x);\theta) \cdot h(x)
  \end{equation}
  für alle $x \in \real^n$, $\theta \in \Theta$.
\end{thm}

\begin{exmp} %% 3.41
  Sei $X=(X_1,\ldots,X_n)$ mit $X_i \sim \mathrm{Bernoulli}(\theta)$, $\theta
  \in [0,1]$. $T(X) = \bar{X}$ ist der ML-Schätzer für $\theta$ (Bsp. 3.23). Es
  gilt
  \begin{align*}
  L(x_1, \ldots, x_n; \theta)
    &= \prod_{i=1}^n \theta^{x_i} (1-\theta)^{x_i}
      \ind_{x_i \in \{0,1\}} \\
    &= \theta^{\sum_{i=1}^n x_i} (1-\theta)^{n - \sum_{i=1}^n x_i}
      \ind_{x \in \{0,1\}^n} \\
    &= \theta^{n T(x)} (1-\theta)^{n(1-T(x))}
      \ind_{x \in \{0,1\}^n} \\
    &= g(T(x), \theta) \cdot h(x)
  \end{align*}
  und $T$ ist nach Satz 3.40 suffizient.
\end{exmp}

\begin{proof}[Beweis zu Satz 3.40]
  Hier nur diskret, der absolutstetige Fall läuft analog.

  Da $X$ diskret ist, ist auch $T(X)$ diskret und wir schreiben
  \[ q(t, \theta) = \pP_\theta(T(X) = t), \quad t =(t_1,\ldots, t_m) \in
    \real^m. \]
  Sei $T$ suffizient. Setze
  \[ g(t,\theta) := q(t;\theta), \quad h(x) = \pP_\theta(X=x | T(X) = T(x)), \]
  dann ist $h$ unabhängig von $\theta$, da $T$ suffizient ist. Zudem
  \begin{align*}
    g(T(x), \theta) \cdot h(x)
    &= \pP_\theta(T(X)=T(x)) \cdot \pP_\theta(X=x|T(X) = T(x)) \\
    &= \pP_\theta(T(X)=T(x), X=x) \\
    &= \pP_\theta(X=x) \\
    &= L(x,\theta).
  \end{align*}
  
  Umgekehrt sei eine Faktorisierung wie in (3.5) gegeben. Sei $T(x) = t$, dann
  \begin{align*}
    \pP_\theta(X=x|T(X) = t)
    &= \frac{\pP_\theta(X=x}{\pP_\theta(T(X)=t)} \\
    &= \frac{\pP_\theta(X=x)}{\pP_\theta(T(X)=T(x))} \\
    &= \frac{L(x,\theta)}{\sum_{y:T(y) = T(x)} L(y; \theta)} \\
    &\overset{(3.5)}{=} \frac{g(T(x),\theta) \cdot h(x)}
      {\sum_{y:T(y) = T(x)} g(T(y),\theta) \cdot h(y)} \\
    &= \frac{h(x)}
      {\sum_{y:T(y) = T(x)} h(y)}
  \end{align*}
  unabhängig von $\theta$.

  Ist $T(x) \ne t$, so gilt
  \[ \pP_\theta(X=x|T(x)=t) = 0, \]
  also unabhängig von $\theta$. Damit ist $T$ suffizient.
\end{proof}

Entsprechend zu Definition 3.36 definiere die \emph{Fisher-Information einer
  Statistik} $T: \real^n \to \real^m$ mit (Zähl-)dichte $q(t,\theta)$ als
\[ I_T(\theta) := \pE_\theta \left[ \left( \pdiff{}{\theta} \log q(T;\theta)
    \right)^2 \right], \quad \theta \in \Theta. \]

\begin{lem} %% 4.42
  Sei $X = X_1, \ldots, X_n$ eine Zufallsstichprobe mit Verteilung $F_\theta$,
  $\theta \in \Theta$. $\{F_\theta, \theta \in \Theta\}$ enthalte nur diskrete
  oder absolutstetige Verteilungen. $T: \real^n \to \real^m$ sei ein Schätzer
  für $\theta$ mit (Zähl-)dichte $q(t;\theta)$. Ist $T$ suffizient, so gilt
  \[ I_T(\theta) = I_n(\theta) \]
  für alle $\theta \in \Theta$.
\end{lem}

\begin{proof}
  Da $T$ suffizient ist, gilt nach Satz 3.40
  \[ L(x; \theta) = g(T(x);\theta) \cdot h(x) \]
  für alle $x \in \real^n$, $\theta \in \Theta$, wobei $g(t;\theta) =
  q(t; \Theta)$ gewählt werden kann. Es folgt also für alle $x \in \real^n$,
  $\theta \in \Theta$ und $t = T(x) \in \real^m$
  \[ \log L(x; \theta) = \log q(t; \theta) + \log h(x). \]
  Also
  \begin{align*}
    \pdiff{}{\theta} \log L(x;\theta)
    &= \pdiff{}{\theta} \log q(t;\theta) + \pdiff{}{\theta} \log h(x) \\
    &= \pdiff{}{\theta} \log q(t;\theta),
  \end{align*}
  so dass
  \[ I_n(\theta) =
    \pE_\theta \left[ \left( \pdiff{}{\theta} \log L(X;\theta) \right)^2 \right]
    =
    \pE_\theta \left[ \left( \pdiff{}{\theta} \log q(T;\theta) \right)^2 \right]
    = I_T(\theta). \qedhere
  \]
\end{proof}

\paragraph{Effizienz}
In Def 3.8: Standardfehler als Gütemaß eines Schätzers.

\begin{thm}[Cramer-Rao-Ungleichung] %% 3.43
  Sei $m=1$ und $\Theta$ ein offenes Intervall in $\real$. $\{F_\theta,\theta
  \in \Theta\}$ beinhalte nur diskrete oder nur absolutstetige Verteilungen und
  sei identifizierbar. Sei $X=(X_1,\ldots,X_n)$ eine Zufallsstichprobe mit $X_i
  \sim F_\theta$, so dass
  \begin{enumerate}[(i)]
  \item $0 < I_1(\theta) < \infty$,
  \item $B := \mathrm{supp} L(x,\theta)$ hängt nicht von $\theta$ ab,
  \item $L(x;\theta)$ ist stetig differenzierbar in $\theta$ und
    \[ \int_B \tilde{T}(x) \pdiff{}{\theta} L(x;\theta) \diffop x =
      \pdiff{}{\theta} \int_B \tilde{T}(x) L(x;\theta) \diffop x \]
    für alle $\theta \in \Theta$ und alle Schätzer $\tilde{T}(X)$ mit
    $\pE_\theta[|\tilde{T}(X)|] < \infty$.
  \end{enumerate}
  Sei $T(X)$ Schätzer für $\theta$ mit $\var_\theta(T(X)) < \infty$ und
  $\psi(\theta) := \pE_\theta[T(X)]$,  dann gilt
  \[ \var_\theta(\theta) \ge \frac{\psi'(T(X))^2}{I_n(\theta)} \]
  für alle $\theta \in \Theta$.

  Ist $T$ erwartungstreu, so gilt
  \[ \var_\theta(T(X)) \ge \rez{I_n(\theta)}. \]
\end{thm}

\begin{defn} %% 3.44
  Es gelten die Bedingungen von Satz 3.43. Dann heißt $T$
  \emph{(Cramer-Rao)-effizient}, falls
  \[ \var_\theta(T(X)) = \frac{\psi'(\theta)^2}{I_n(\theta)}. \]
\end{defn}

\begin{proof}[Beweis von Satz 3.43]
  Bemerke, dass Lemma 3.38 unter den gegebenen Bedingungen gilt, so dass
  \begin{equation} %% 3.6
    \pE_\theta \left[ \pdiff{}{\theta} \log L(X;\theta) \right] = 0
  \end{equation}
  und
  \begin{equation} %% 3.7
    \var_\theta \left( \pdiff{}{\theta} \log L(X;\theta) \right)
    = I_n(\theta) = n \cdot I_1(\theta).
  \end{equation}
  Damit gilt
  \begin{align*}
    \psi'(\theta)
    &= \pdiff{}{\theta} \int_B T(x) L(x;\theta) \diffop x \\
    &\overset{(\text{iii})}{=} \int_B T(x)
      \pdiff{}{\theta} L(x;\theta) \diffop x \\
    &= \int_B T(x) L(x;\theta) \pdiff{}{\theta} \log L(x;\theta) \diffop x \\
    &= \int_B (T(x) - \psi(\theta)) L(x;\theta)
      \pdiff{}{\theta} \log L(x;\theta) \diffop x
      + \int_B \psi(\theta) L(x;\theta)
      \pdiff{}{\theta} \log L(x;\theta) \diffop x \\
    &= \pE_\theta \left[
      (T(X) - \psi(\theta)) \pdiff{}{\theta} \log L(X;\theta)
      \right]
      + \psi(\theta) \underbrace{\pE_\theta \left[ 
      \pdiff{}{\theta} \log L(X;\theta) 
      \right]}_{=0\, (3.6)}
  \end{align*}
  Cauchy-Schwarz-Ungleichung:
  \begin{align*}
    (\psi'(\theta))^2
    &= \left( \pE_\theta \left[
      (T(X)-\psi(\theta)) \pdiff{}{\theta} \log L(X;\theta)
      \right] \right)^2 \\
    &\le \pE_\theta \left[
      (T(X)-\psi(\theta))^2
      \right] \cdot \pE_\theta \left[ \left(
      \pdiff{}{\theta} \log L(X;\theta)
      \right)^2 \right] \\
    &= \var_\theta(T(X)) \cdot I_n(\theta). \qedhere
  \end{align*}
\end{proof}

\chapter{Hypothesentests}
\emph{Ziel:} Behauptungen über die Beschaffenheit der (unbekannten) Verteilung
$F$ überprüfen.

\emph{Parametrische Tests:} Behauptungen können in Abhängigkeit vom Parameter
$\theta$ der Verteilung $F_\theta$ formuliert werden.

Es sei also stets $(X_1, \ldots, X_n)$ eine Zufallsstichprobe mit $X_i \sim F =
F_\theta$ mit $F_\theta \in \{ F_{\theta'} : \theta' \in \Theta\}$. $(x_1,
\ldots, x_n)$ ist eine Realisierung von $(X_1, \ldots, X_n)$.

\section{Das statistische Testproblem}
Allgemeine Vorgehensweise bei parametrischen Tests:
\begin{itemize}
\item \emph{Aufstellen von Hypothesen und Gegenhypothesen.} Formuliere die zu
  überprüfende Behauptung über $F_\theta$ in Form der sogenannten
  \emph{Nullhypothese}:
  \[ H_0 : \theta \in \Theta_0 \subset \Theta. \]
  Entsprechend ergibt sich als \emph{Gegenhypothese}:
  \[ H_0 : \theta \in \Theta_1 = \Theta \setminus \Theta_0 \qLRq \theta \notin
    \Theta_0. \]
\item \emph{Definition einer Entscheidungsregel.} Lege fest, für welche
  Bedingung der Zufallsstichprobe die Nullhypothese abgelehnt wird.

  Dazu: Definiere eine geeignete \emph{Teststatistik} $T$, also eine messbare
  Abbildung
  \[ T: \real^n \to \real: (X_1, \ldots, X_n) \mapsto T(X_1, \ldots, X_n)\]
  und einen \emph{Ablehnungsbereich} $K \subset \real$ (kritischer Bereich,
  rejection region) und damit die \emph{Entscheidungsregel}:
  \begin{align*}
    \{ (X_1, \ldots, X_n) : \text{ Test lehnt $H_0$ ab } \}
    &= \{ (X_1, \ldots, X_n) : T(X_1, \ldots, X_n) \in K \} \\
        \{ (X_1, \ldots, X_n) : \text{ Test lehnt $H_1$ ab } \}
    &= \{ (X_1, \ldots, X_n) : T(X_1, \ldots, X_n) \in K^c \}
  \end{align*}
\end{itemize}

\begin{exmp}
  Angenommen $X_1, \ldots, X_5 \sim \mathrm{Bernoulli}(\theta)$ mit $\theta \in
  [0,1]$ unbekannt. Wir wollen testen
  \[ H_0 : \theta \le \rez{2}, \qquad H_1 : \theta > \rez{2}. \]
  Als Teststatistik verwenden wir $Y = \sum_{i=1}^5 X_i$ mit den Realisierungen
  $\{0, 1, \ldots, 5\}$.
  \begin{itemize}
  \item Test 1: $H_0$ wird abgelehnt, falls $Y = 5$, also $X_i = 1$ für $i =
    1, \ldots, 5$.
  \item Test 2: $H_0$ wird abgelehnt, falls $Y \in \{3,4,5\}$.
  \end{itemize}
  Was ist ``besser''?
\end{exmp}

\paragraph{Bestimmen der Präzision der Tests}
Die Präzision wird mit der \emph{Gütefunktion}
\[ \beta( \theta ) = \pP_\theta( T(X_1, \ldots, X_n) \in K ), \quad \theta
  \in \Theta \]
bewertet. Mit dieser lassen sich die möglicherweise auftretenden Fehler des
Tests bestimmen:
\begin{center}
  \begin{tabular}{r|l|l}
    & Test lehnt $H_0$ ab. & Test lehnt $H_0$ nicht ab. \\
    \hline & \\
    $H_0$ wahr & $\alpha(\theta) = \beta(\theta)$, $\theta \in \Theta_0$ 
                           & Kein Fehler \\
    & Fehler 1. Art \\
    \hline & \\
    $H_1$ wahr & Kein Fehler & $1 - \beta(\theta)$, $\theta \in \Theta_1$ \\
    & & Fehler 2. Art
  \end{tabular}
\end{center}

\begin{itemize}
\item \emph{Macht} (power) des Tests: $\beta(\theta)$, $\theta \in \Theta_1$
\item \emph{Signifikanzniveau}, \emph{Testniveau}:
  \[ \alpha := \sup_{\theta \in \Theta_0} \alpha(\theta) = \sup_{\theta \in
      \theta_0} \beta(\theta) \]
\end{itemize}

Im Beispiel ergeben sich:
\begin{footnotesize}
  \begin{center}
    \begin{tabular}{r|l|l}
      & Test 1
      & Test 2 \\
      \hline & \\
      Gütefunktion
      & $\beta_1(\theta) = \pP_\theta(Y = 5) = \theta^5$
      & $\begin{aligned}
        \beta_2 &= \pP_\theta(Y \ge 3) \\
        &= \theta^5 + \binom{5}{4} \theta^4 (1 - \theta)
        + \binom{5}{3} \theta^3 (1-\theta)^2 \end{aligned}$ \\
      \hline & \\
      Fehler 1. Art
      & $\alpha_1(\theta) = \theta^5$, $\theta \le \rez{2}$
      & $\begin{aligned}
        \alpha_2 &= \theta^5 + \binom{5}{4} \theta^4 (1 - \theta)
        + \binom{5}{3} \theta^3 (1-\theta)^2 \\
        \theta &\le \rez{2}
      \end{aligned}$ \\
      \hline & \\
      Signifikanzniveau
      & $\begin{aligned}
        \alpha_1 &= \sup \alpha_1(\theta) = \alpha_1(\rez{2}) \\
        &= 2^{-5} = \num{.03}, \, \theta \le \rez{2}
      \end{aligned}$,
      & $\begin{aligned}
        \alpha_2 &= \sup \alpha_2(\theta) = \rez{2} \\
        \theta &\le \rez{2}
      \end{aligned}$ \\
      \hline & \\
      Fehler 2. Art
      & $1 - \theta^5$, $\theta > \rez{2}$
      & $\begin{aligned}
        &1 - \left( \theta^5 + \binom{5}{4} \theta^4 (1-\theta)
          + \binom{5}{3} \theta^3 (1-\theta)^2 \right) \\
        &\theta > \rez{2}
      \end{aligned}$ \\
      \hline & \\
      \shortstack[r]{Maximale \\ Fehlerwahrscheinlichkeit \\ 2. Art}
      & $\begin{aligned}
        &\sup (1-\beta_1(\theta)) = 1 - 2^{-5} \approx \num{.97} \\
        &\theta > \rez{2}
      \end{aligned}$
      & $1 - \num{.5} = \num{.5}$
    \end{tabular}
  \end{center}
\end{footnotesize}

Test 1 hat also ein sehr gutes Signifikanzniveau, jedoch einen großen Fehler
2. Art. Bei Test 2 ist der Fehler 2. Art kleiner, das Signifikanzniveau ist
aber schlecht.

\clearpage

\emph{Beachte:}
\begin{itemize}
\item Ein idealer Test hätte einen kleinen Fehler 1. Art und einen kleinen
  Fehler 2. Art. Solch ein Test existiert aber im Allgemeinen nicht.
  
  Man kontrolliert daher den Fehler 1. Art: Ein Signifikanzniveau wird
  vorgegeben und der Test wird passend konstruiert (mit möglichst kleinem
  Fehler 2. Art).
\item Ein statistischer Test kann die Nullhypothese nur
  \emph{ablehnen}/\emph{verwerfen} oder \emph{nicht ablehnen}/\emph{nicht
    verwerfen}, jedoch \emph{niemals} ``annehmen''.
\item Die eigentliche Vermutung (die man zeigen möchte) muss daher als
  Gegenhypothese formuliert werden.
\end{itemize}
  
%\emph{Tests}
%\begin{itemize}
%\item $H_0 : \theta \in \Theta_0$ gegen $H_1: \theta \notin \Theta_0$.
%\item Test lehnt $H_0$ ab, falls $T(X_1, \ldots, X_k) \in K$ mit $T$
%  Teststatistik, $K$ Ablehnungsbereich.
%\item $\alpha = \sup_{\theta \in \Theta_0} \beta(\theta) \in
%  K)$ Signifikanzniveau (max. Fehler 1. Art) mit Gütefunktion $\beta(\theta) :=
%  \pP_\theta(T(X_1, \ldots, X_n)$.
%\end{itemize}

\section{Einstichprobentests}
\subsection{Tests für normalverteilte Stichproben}
\begin{exmp}[Gauss-Test]
  Gegeben sei eine Zufallsstichprobe $(X_1, \ldots, X_n)$ mit i.i.d.
  Zufallsvariablen mit $X_i \sim \ndist(\mu, \sigma^2)$, wobei $\sigma^2$
  bekannt sei, $\mu$ unbekannt.

  Es wird vermutet, dass $\mu > \mu_0$ für $\mu_0 \in \real$  gegeben. Wir
  setzen also 
  \[ H_0 : \mu \le \mu_0 \qquad \text{und} \qquad H_1: \mu > \mu_0. \]
  Das ist ein \emph{einseitiges} Testproblem.

  Als Teststatistik verwenden wir das standardisierte arithmetische Mittel
  \[ T = T(X_1, \ldots, X_n) = \frac{\bar{X}-\mu_0}{\sigma / \sqrt{n}}. \]

  Unter $H_0$ nimmt $T$ mehrheitlich negative Werte an. Eine sinnvolle
  Entscheidungsregel ist daher
  \[ \text{Verwerfe } H_0 \qLRq \frac{\bar{X}-\mu_0}{\sigma / \sqrt{n}} > c \]
  für eine Konstante $c > 0$.

  Wollen wir einen Test zum vorgegebenen Signifikanzniveau $\alpha$ erstellen,
  so muss gelten
  \[ \alpha \ge \sup_{\mu \le \mu_0} \beta(\mu), \]
  wobei
  \begin{align*}
    \beta(\mu)
    &= \pP_\mu \left( \frac{\bar{X}-\mu_0}{\sigma / \sqrt{n}} > c \right) \\
    &= \pP_\mu \left( \frac{\bar{X}-\mu}{\sigma / \sqrt{n}} > c
      + \frac{\mu_0 - \mu}{\sigma / \sqrt{n}} \right) \\
    \intertext{Es gilt
    $\frac{\bar{X}-\mu}{\sigma / \sqrt{n}} \sim \ndist(0,1)$
    unter $\pP_\mu$.}
    &= 1 - \pP_\mu \left( \frac{\bar{X}-\mu}{\sigma / \sqrt{n}} \le c
      + \frac{\mu_0 - \mu}{\sigma / \sqrt{n}} \right) \\
    &= 1 - \Phi \left(c + \frac{\mu_0 - \mu}{\sigma / \sqrt{n}} \right)
      \overset{\text{unter } H_0}{\le} 1 - \Phi(c)
  \end{align*}
  mit $\Phi$ Verteilungsfunktion der Standardnormalverteilung.

  Gilt also
  \[ \alpha \ge 1 - \Phi(c), \]
  so erhalten wir einen Test zum Niveau $\alpha$. Setze also $c = z_{1-\alpha}$
  ($1-\alpha$-Quantil der Standardnormalverteilung).

  Wir erhalten also
  \begin{mdframed}
    \textbf{Einseitiger Gauss-Test}
    \[ H_0 : \mu \le \mu_0, \qquad H_1 : \mu > \mu_0 \]
    Verwerfe $H_0$ zum Niveau $\alpha$ $\Leftrightarrow$
    \[ \frac{\bar{X}-\mu_0}{\sigma/\sqrt{n}} > z_{1-\alpha} \]
  \end{mdframed}
  
  Beachte, dass die Gütefunktion auch von $n$ abhängt: Für wachsendes $n$ sinkt
  die Fehlerwahrscheinlichkeit zweiter Art.

  Angenommen, wir möchten nun einen einseitigen Gausstest zur Nullhypothese $\mu
  \le \mu_0$, konstruieren, der einen Fehler erster Art $\le \num{.1}$ und einen 
  maximalen Fehler zweiter Art von \num{.2} für $\mu \ge \mu_0 + \sigma$ hat.
  Wie müssen $\alpha$ und $n$ gewählt werden?

  Es muss gelten
  \[ \begin{aligned}
      \beta(\mu) &\le \num{.1}, & \mu &\le \mu_0, \\
      1 - \beta(\mu) &\le \num{.2}, & \mu &\ge \mu_0 + \sigma.
    \end{aligned}
  \]
  Aus der ersten Gleichung folgt direkt $\alpha = \num{.1}$. Zudem soll
  $\beta(\mu) \ge \num{.8}$ gelten für alle $\mu \ge \mu_0 + \sigma$, wobei
  $\beta(\mu)$ in $\mu$ wächst. Also muss gelten
  \begin{align*}
    \num{.8}
    &\le \beta(\mu_0 + \sigma) \\
    &= 1 - \Phi \left(
      c + \frac{\mu_0 - (\mu_0 + \sigma)}{\sigma / \sqrt{n}}
      \right) \\
    &= 1 - \Phi( c - n^{-1/2})
  \end{align*}
  mit $c = z_{1-\alpha} = z_{\num{.9}} = \num{1.28}$. Da jedoch
  \[ \num{.8} = 1 - \Phi(\num{-.84}) \]
  folgt
  \[ \num{1.28} - n^{-1/2} \le \num{-.84} \]
  und damit $n \ge \num{4.49}$, also $n > 4$.
\end{exmp}

\clearpage

Weitere Tests für normalverteilte Daten:
\begin{mdframed}
  \textbf{Zweiseitiger Gauss-Test}
  (Test von $\mu$ bei bekanntem $\sigma$).
  
  Hypothesen:
  \[ H_0 : \mu = \mu_0, \qquad H_1 : \mu \ne \mu_0, \]
  Teststatistik:
  \[ T = \frac{\bar{X}-\mu_0}{\sigma/\sqrt{n}}. \]

  Unter $H_0$ gilt $T \sim \ndist(0,1)$. Entscheidungsregel:
  \[ \text{Verwerfe $H_0$ zum Niveau $\alpha$} \qLRq |T| > z_{1-\alpha/2}. \]
\end{mdframed}

\begin{mdframed}
  \textbf{t-Test} (Zweiseitiger Test von $\mu$ bei unbekanntem $\sigma$)

  Hypothesen:
  \[ H_0 : \mu = \mu_0, \qquad H_1 : \mu \ne \mu_0, \]
  Teststatistik:
  \[ T = \frac{\bar{X}-\mu_0}{s/\sqrt{n}} \]
  mit der empirischen Varianz $s^2$.

  Unter $H_0$ gilt $T \sim t_{n-1}$ (Bsp. 3.28). Entscheidungsregel:
  \[ \text{Verwerfe $H_0$ zum Niveau $\alpha$} \qLRq |T| > t_{n-1,1-\alpha/2}. \]
\end{mdframed}

\begin{mdframed}
  Zweiseitiger Test von $\sigma^2$ bei bekanntem $\mu$

  Hypothesen:
  \[ H_0 : \sigma^2 = \sigma_0^2, \qquad H_1 : \sigma^2 \ne \sigma_0^2, \]
  Teststatistik:
  \[ T = T(X_1, \ldots, X_n) =
    \sum_{i=1}^n \frac{X_i-\mu^2}{\sigma_0^2}
    = \frac{n \tilde{s}^2}{\sigma_0^2}. \]

  Unter $H_0$ gilt $T \sim \chi_n^2$. Entscheidungsregel:
  \[ \text{Verwerfe $H_0$ zum Niveau $\alpha$} \qLRq T \notin
    [\chi_{n,\alpha/2}^2, \chi_{n,1-\alpha/2}^2]. \]  
\end{mdframed}

\clearpage

\begin{mdframed}
  Zweiseitiger Test von $\sigma^2$ bei unbekanntem $\mu$

  Hypothesen:
  \[ H_0 : \sigma^2 = \sigma_0^2, \qquad H_1 : \sigma^2 \ne \sigma_0^2, \]
  Teststatistik:
  \[ T = T(X_1, \ldots, X_n) = \frac{(n-1)s^2}{\sigma_0^2} \]
  
  Unter $H_0$ gilt $T \sim \chi_{n-1}^2$ (Beispiel 3.28). Entscheidungsregel:
  \[ \text{Verwerfe $H_0$ zum Niveau $\alpha$} \qLRq T \notin
    [\chi_{n-1,\alpha/2}^2, \chi_{n-1,1-\alpha/2}^2]. \]  
\end{mdframed}

\subsection{Asymptotische Tests}
Für eine Zufallsstichprobe $(X_1, \ldots, X_n)$ mit i.i.d. Zufallsvariablen $X_i
\sim F_\theta$, $\theta \in \Theta \subseteq \real$ testen wir die Hypothesen
\[ H_0 : \theta = \theta_0 \quad \text{gegen} \quad H_1 : \theta \ne
  \theta_0. \]
Gegeben sei ein erwartungstreuer, asymptotisch normalverteilter Schätzer
$\hat{\theta}$ für $\theta$, sowie ein konsistenter Schätzer $\hat{\theta}^2$
für die Varianz von $\hat{\theta}$.

Dann gilt unter $H_0$
\[ \frac{\hat{\theta}-\theta_0}{\hat{\sigma}} \xrightarrow{d} 
  Y \sim \ndist(0,1), \quad n \to \infty. \]
Als Teststatistik verwendet man daher
\[ T(X_1, \ldots, X_n) := \frac{\hat{\theta}-\theta_0}{\hat{\sigma}} \]
und die Entscheidungsregel lautet: Verwerfe $H_0$ zum asymptotischen Niveau
$\alpha$ $\Leftrightarrow$
\[ |T| > z_{1-\alpha/2}. \]

\begin{rmrk*}
  \begin{itemize}
  \item Basiert auf dem Wald-Test für ML.
  \item Das Testniveau $\alpha$ wird nur asymptotisch erreicht! Der Test sollte
    also nur für große $n$ verwendet werden.
  \end{itemize}
\end{rmrk*}

\begin{exmp}[Binomialverteilung]
  $X_i \sim \mathrm{Bernoulli}( \theta )$, $\theta \in (0,1)$. Wir testen
  \[ H_0 : \theta = \theta_0 \quad \text{gegen} \quad
    H_1 : \theta \ne \theta_0. \]
  Entspechend Beispiel 3.23.1: $\bar{X}$ ist ML-Schätzer für $\theta$. $\bar{X}$
  ist erwartungstreu und asymptotisch normalverteilt nach Satz 3.37. Unter $H_0$
  gilt:
  \[ \sqrt{n} (\bar{X}-\theta_0) \xrightarrow{d} \ndist(0,
    \theta_0(1-\theta_0)), \quad n \to \infty. \]
  Zudem verwenden wir als Schätzer für die Varianz von $\hat{\theta}$
  \[ \hat{\sigma}^2 := \rez{n} \bar{X} (1-\bar{X})
    = \rez{n} (\hat{\mu}_2^2 - \hat{\mu}_1^2) \]
  und dieser Schätzer ist nach Satz 3.16 sogar stark konsistent, denn
  \[ \var \bar{X} = \var \left( \rez{n} \sum_{i=1}^n X_i \right)
    = \rez{n^2} \sum_{i=1}^n \var(X_i)
    = \rez{n} (\pE[X_1^2] - \pE[X_1]^2). \]
  Es folgt unter $H_0$
  \[ \sqrt{n} \frac{\bar{X}-\theta_0}{\sqrt{\bar{X}(1-\bar{X})}}
    \xrightarrow{d} Z \sim \ndist(0,1), \quad n \to \infty, \]
  falls $\bar{X} \notin \{0,1\}$.

  Die Entscheidungsregel lautet also: Verwerfe $H_0$ zum asymptotischen Niveau
  $\alpha$ $\Leftrightarrow$
  \[ \left| \sqrt{n} \frac{\bar{X}-\theta_0}{\sqrt{\bar{X}(1-\bar{X})}} \right|
  > Z_{1-\alpha/2}. \]
\end{exmp}

\subsection{Die Wahl von $\alpha$ und der $p$-Wert}
Die Wahl von $\alpha$ hängt vom Sachzusammenhang ab. Bei Präzisionsmessungen
wird $\alpha$ sehr klein gewählt. In Sozialwissenschaften genügt häufig
$\alpha = \num{.1}$. Oft berechnet man daher den $p$-Wert.

\begin{defn}
  Seien $(x_1, \ldots, x_n)$ eine Realisierung von $(X_1, \ldots, X_n)$ und
  $T(X_1, \ldots, X_n)$ eine Teststatistik für einen gegebenen Hypothesentest.
  Der \emph{$p$-Wert} des Tests ist das kleinste Signifikanzniveau, bei welchem
  \[ t = T(x_1, \ldots, x_n) \]
  zur Ablehnung der Nullhypothese führt.
\end{defn}

\begin{exmp}
  Ein Analyst behauptet, die jährlichen Returns der DAX30-Unternehmen im
  Finanzsektor liegen über 35.

  Wir verwenden die Daten aus Tabelle 1 und nehmen an, dass diese (approximativ)
  normalverteilt sind.

  Wir wollen die Aussage des Analysten bestätigen und verwenden einen
  einseitigen t-Test mit
  \[ H_0 : \mu \le 35 = \mu_0, \qquad H_1 : \mu > 35. \]
  Die Teststatistik ist
  \[ T = \frac{\bar{X}-\mu_0}{s / \sqrt{n}} \]
  mit $n = 30$ (fünf Unternehmen, sechs Jahre).

  Auswerten ergibt
  \[ \bar{X} = \num{39.54}, \quad s = \num{15.2} \qRq t = \num{1.65}. \]

  Die Nullhypothese wird abgelehnt, falls $t > t_{29,1-\alpha}$, so dass
  \[ p = \inf \{ \alpha : t_{29, 1-\alpha} < \num{1.63} \}. \]
  Da
  \[ t_{29,\num{.94}} = \num{1.6} < t < \num{1.7} < t_{29,\num{.95}}, \]
  folgt für den $p$-Wert $\num{.05} < p < \num{.06}$ (genau $p = \num{.05649}$).
  In R hätte man auch direkt einen t-Test durchführen können.
\end{exmp}

\textbf{Beachte:}
\begin{itemize}
\item Der $p$-Wert ist immer datenabhängig.
\item Man ist meist an kleinen $p$-Werten interessiert.
\item Der $p$-Wert sagt jedoch nichts über den Fehler 2. Art aus.
\end{itemize}

\section{Zweistichprobentests}
Gegeben sind Zufallsstichproben
\begin{align*}
  X &= (X_1, \ldots, X_n),
      \quad X_i \sim F_X \in \{ F_theta, \theta \in \Theta \}, \\
  Y &= (Y_1, \ldots, Y_n),
      \quad Y_i \sim F_Y \in \{ F_theta, \theta \in \Theta \}
\end{align*}
mit Realisierungen $(x_1, \ldots, x_n)$ und $(y_1, \ldots, y_n)$.

Sind $X$ und $Y$ kausal unabhängig, so sprechen wir von \emph{unverbundenen}
Stichproben. \emph{Verbundene} Sichproben liegen vor, wenn zum Beispiel
verschiedene Merkmale an denselben statistischen Einheiten betrachtet werden.

\subsection{Tests für unverbundene, normalverteilte Sichproben}
Es seien stets $X_i \sim \ndist(\mu_X, \sigma_X^2)$, $Y_j \sim \ndist(\mu_Y,
\sigma_Y^2)$, $X$ und $Y$ unabhängig.

\begin{mdframed}
  \textbf{Test auf Gleicheit der Erwartungswerte bei Varianzhomogenität}

  Es gelte $\sigma_X^2 = \sigma_Y^2 = \sigma^2$.

  Hypothesen:
  \[ H_0 : \mu_X = \mu_Y \quad \text{gegen} \quad
    H_1 : \mu_X \ne \mu_Y. \]
  Teststatistik:
  \[ T = T(X_1, \ldots, X_n, Y_1, \ldots, Y_k) =
    \frac{\bar{X}-\bar{Y}}{s_p \sqrt{\rez{n}+\rez{k}}} \]
  mit
  \[ s_p^2 := \rez{n+k-2} \left(
      \sum_{i=1}^n (X_i - \bar{X})^2
      + \sum_{j=1}^k (Y_j - \bar{Y})^2
    \right). \]
  Unter $H_0$ gilt: $T \sim t_{n+k-2}$ ($\ast$). Also verwerfe $H_0$ zum Niveau
  $\alpha$ $\Leftrightarrow$
  \[ |T| > t_{n+k-2, 1 - \alpha / 2}. \]
\end{mdframed}

Um ($\ast$) zu zeigen, beachte:
\[ \bar{X} \sim \ndist \left( \mu_x, \frac{\sigma^2}{n} \right)
  \text{ und }
  \bar{Y} \sim \ndist \left( \mu_y, \frac{\sigma^2}{k} \right)
\]
und unter $H_0$ gilt also
\[ \bar{X} - \bar{Y} \sim \ndist 
  \left(
    0, \sigma^2 \left( \rez{n} + \rez{k} \right)
  \right) \]
Zudem ist $s_p$ ein unverzerrter Schätzer\footnote{%
  $s^2 = \rez{n-1} \sum \left( X_i - \bar{X} \right)^2$, 
  $\pE(s^2) = \sigma^2.$
} für $\sigma$ (Nachrechnen bzw.
vergleiche Beispiel 3.3), für welchen gilt
\[ \frac{n+k-2}{\sigma^2} s_p^2
  = \underbrace{\sum_{i=1}^n \frac{(X_i - \bar{X})^2}{\sigma^2}}_{\sim
    \chi^2_{n-1}}
  + \underbrace{\sum_{j=1}^k \frac{(Y_i - \bar{Y})^2}{\sigma^2}}_{\sim
    \chi^2_{k-1}}
  \sim \chi^2_{n+k-2}.
\]
Damit folgt
\[ T = \frac{\bar{X}-\bar{Y}}{\sigma \sqrt{\rez{n}+\rez{k}}}
  \sqrt{(n+k-2) \frac{\sigma^2}{(n+k-2)s_p^2}}
  \sim t_{n+k-2}. \]

\begin{mdframed}
  \textbf{Welch-Test} Test auf Gleichheit der Erwartungswerte bei
  Varianzinhomogenität

  Es gelte $\sigma_x^2 \ne \sigma_y^2$ (beide unbekannt).

  Hypothesen:
  \[ H_0 : \mu_x = \mu_y, \qquad H_1 = \mu_x \ne \mu_y. \]
  Teststatistik:
  \[ T = T(X_1, \ldots, X_n, Y_1, \ldots, Y_k)
    := \frac{\bar{X}-\bar{Y}}{s_{XY} \sqrt{\rez{n}+\rez{k}}} \]
  mit
  \[ s_{XY}^2 := \frac{(n-1) s_X^2 + (k-1) s_Y^2}{n+k-2}. \]
  Unter $H_0$ ist $T \sim t_{n+k-2}$, also verwerfe $H_0$ zum Niveau $\alpha$
  $\Leftrightarrow$
  \[ |T| > t_{n+k-2,1 - \alpha/2}. \]
\end{mdframed}

Vorgeschaltet wird meistens:
\begin{mdframed}
  \textbf{Test auf Varianzinhomogenität}

  Hypothesen:
  \[ H_0 : \sigma_x^2 = \sigma_y^2, \qquad H_1 : \sigma_x^2 \ne \sigma_y^2. \]
  Teststatistik:
  \[ T = T(X_1, \ldots, X_n, Y_1, \ldots, Y_k) := \frac{s_x^2}{s_y^2}. \]
  Unter $H_0$ gilt $T \sim  F_{n-1,k-1}$. Verwerfe also $H_0$ zum Niveau
  $\alpha$ $\Leftrightarrow$
  \[ |T| > t_{n+k-2,1 - \alpha/2}. \]
\end{mdframed}

\subsection{Verbundene Stichproben (Matched Pairs)}
Es gelte $n=k$ und
\[ W_i = X_i - Y_i \sim \ndist(\mu_w,\sigma_w^2). \]
Hier erhält man parallel zum klassischen t-Test:

\begin{mdframed}
  Hypothesen:
  \[ H_0 : \mu_w < 0 \quad \text{gegen} \quad H_1 \mu_w \ge 0. \]
  Teststatistik:
  \[ T := \frac{\bar{W}}{s_w / \sqrt{n}}. \]
  
  Unter $H_0$ gilt $T \sim t_{n-1}$. Verwerfe daher $H_0$ zum Niveau $\alpha$
  $\Leftrightarrow$
  \[ T > t_{n-1,1-\alpha}. \]
\end{mdframed}

\chapter{Einfache lineare Regression}\label{ch:regression}
Sei $(X,Y) = ((X_1, Y_1), \ldots, (X_n,Y_n))$ eine bivariate Zufallsstichprobe
mit Realisierungen $((x_1,y_1), \ldots, (x_n, y_n))$.

R-Beispiel: Es gilt offenbar
\[ y_i = a + b x_i + \eps_i \]
bzw. allgemein
\[ y_i = f(x_i) + \eps_i. \]

Derartige Beziehungen nennt man \emph{Regressionen}.

\emph{Regressionsanalyse}: Finde eine Funktion $f$, so dass ein möglichst großer
Anteil der Variabilität der Daten durch $f$ erklärt werden kann und nur wenig
Variabilität auf die Fehler zurück zu führen ist.

Bei der Suche nach $f$ beginnt man meist mit einer linearen Funktion, das heißt
man legt eine \emph{Ausgleichsgerade} durch die Punktwolke.

Wir betrachten daher das \emph{Standardmodell der einfachen linearen
  Regression}. Es gelte
\begin{equation}
  y_i = a + b x_i + \eps_i, \quad i = 1, \ldots, n
\end{equation}
mit
\begin{itemize}
\item $Y_1, \ldots, Y_n$ beobachtbare, reelle Zufallsvariablen,
\item $x_1, \ldots, x_n$ deterministische Werte oder Realisierungen reeller
  Zufallsvariablen $X_i$,
\item $\eps_1, \ldots, \eps_n$ Fehlervariablen, das heißt i.i.d.
  Zufallsvariablen mit $\pE[\eps_i] = 0$ und $\var(\eps_i) = \sigma^2$
  (unbekannt),
\item $a,b$ unbekannte Parameter in $\real$.
\end{itemize}

Die Parameter $a$, $b$ und $\sigma^2$ sind aus den Daten zu schätzen.

\begin{rmrk}
  \begin{itemize}
  \item Sind die $x_i$ Realisierungen reeller Zufallsvariablen, zum Beispiel,
    weil $X$ und $Y$ zwei statistische Merkmale sind, die an den Einheiten $i$
    erhoben werden, so spricht man von einem Regressionsmodell mit
    \emph{stochastischem Regressor}. In diesem Fall sind die angegebenen
    Eigenschaften im obigen Modell nur unter der Bedingung $X_i = x_i$ gültig
    (also zum Beispiel $\pE[\eps_i | X_i = x_i] = 0$).
  \item Die Voraussetzung $\var(\eps_i) = \sigma^2$ (geg. $X_i = x_i$) heißt
    \emph{Homoskedastizität}. Für heteroskedastische Fehler (das heißt $\var
    (\eps_i)=\sigma_i^2$) sind die folgenden Methoden der linearen Regression
    nur modifiziert anwendbar.
  \item Aus (5.1) folgt für die Beobachtungsgrößen: Es gilt (gegeben $X_i =
    x_i$)
    \[ \pE[Y_i] = a + b x_i, \qquad \var(Y_i) = \sigma^2 \]
    und die $Y_i$ sind unabhängig.

    Insbesondere gilt für $\eps_i \sim \ndist(0,\sigma^2)$: $Y_i \sim \ndist( a
    + b x_i, \sigma^2)$.
  \end{itemize}
\end{rmrk}

\section{Punktschätzung}
Offensichtlich ist eine zentrale Aufgabe der linearen Regression das Finden von
Punktschätzern für die Parameter $a$, $b$, $\sigma^2$.
\[ (\hat{a}, \hat{b}) = \argmin_{(a,b)} \sum_{i=1}^n (Y_i - a - b x_i )^2 =:
  \argmin_{(a,b)} f(a,b). \]
Die Funktion $f$ besitzt partielle Ableitungen
\begin{align*}
  \pdiff{}{a} f(a,b) &= - 2 \sum_{i=1}^n (Y_i - a - b x_i), \\
  \pdiff{}{b} f(a,b) &= - 2 \sum_{i=1}^n  x_i (Y_i - a - b x_i). \\
\end{align*}
Setze die Ableitungen null, das heißt
\begin{equation}
 0 = \sum_{i=1}^n Y_i - n \hat{a} - \hat{b} \sum_{i=1}^n x_i = n \bar{Y} -
 \sum_{i=1}^n \hat{Y}_i,
\end{equation}
wobei $\hat{Y}_i = \hat{a} + \hat{b} x_i$ der Schätzer für $Y_i$ ist.

Aus der ersten Gleichung folgt
\[ \hat{a} = \bar{Y} - \hat{b} \bar{X}. \]
Setzen wir diesen Wert in die zweite Gleichung ein, erhalten wir
\begin{align*}
 0 &= \sum_{i=1}^n x_i Y_i - (\bar{Y} - \hat{b} \bar{X}) \sum_{i=1}^n x_i -
     \hat{b} \sum_{i=1}^n x_i^2 \\
   &= \rez{n} \sum_{i=1}^n x_i Y_i
     - \bar{Y} \bar{X} + \hat{b} (\bar{X}^2 - \obar{X^2}) \\
  \hat{b} &= \frac{\rez{n} \sum_{i=1}^n x_i Y_i - \bar{Y} \bar{X}}
            {\bar{X}^2 - \obar{X^2}}
            = \frac{\tilde{s}^2_{XY}}{\tilde{s}^2_X},
\end{align*}
wobei
\[ \tilde{s}^2_{XY}
  := \rez{n} \sum_{i=1}^n (x_i - \bar{X})(Y_i - \bar{Y}),  \qquad
  \tilde{s}^2_X := \rez{n} \sum_{i=1}^n (x_i - \bar{X})^2. \]
Damit gilt $\hat{Y}_i = \hat{a} + \hat{b} x_i$, $f(x) = \hat{a} + \hat{b} x_i$,
die Ausgleichsgerade.

Die Differenzen
\[ \hat{\eps_i} = Y_i - \hat{Y}_i \]
bezeichnet man als \emph{Residuen}. Nun lässt sich $\sigma^2$ schätzen als
\[ \hat{\sigma}^2 = \rez{n-2} \sum_{i=1}^n \hat{\eps}_i^2. \]

\begin{prp}
  Im einfachen linearen Regressionsmodell sind
  \[ \hat{b} = \frac{{s_{XY}}}{s_X}, \quad \hat{a} = \bar{Y} - \hat{b} \bar{x},
    \quad \hat{\sigma}^2 = \rez{n-2} \sum_{i=1}^n \hat{\eps}_i^2 \]
  erwartungstreu. Falls
  \begin{equation}
    \lim_{n \to \infty} \sum_{i=1}^n (x_i - \bar{x})^2 = \infty,
  \end{equation}
  so sind $\hat{a}$, $\hat{b}$ schwach konsistent.
\end{prp}

\begin{proof}
  Folgende Beziehungen lassen sich elementar nachrechnen:
  \[ \pE \hat{a} = a, \quad \pE \hat{b} = b, \quad \pE \hat{\sigma}^2 =
    \sigma^2 \]
  sowie
  \begin{equation}
  \var(\hat{a}) = \sigma^2 \frac{\sum x_i^2}{n \sum(x_i - \bar{x})^2} =:
    \sigma_a^2, \quad
    \var(\hat{b}) = \frac{\sigma^2}{\sum(x_i - \bar{x})^2} =: \sigma_b^2.
  \end{equation}
  Damit gilt Erwartungstreue. Mit Lemma 3.6 folgt die schwache Konsistenz von
  $\hat{a}$ und $\hat{b}$.
\end{proof}

\begin{thm}[Satz von Gauss-Markov]
  Die Schätzer $\hat{a}$, $\hat{b}$ sind die besten\footnotemark linearen,
  unverzerrten Schätzer für $a$, $b$ im Standardmodell der linearen Regression.

  Gilt zusätzlich die Normalverteilungsannahme, so sind $\hat{a}$, $\hat{b}$
  UMVUE, das heißt, es gibt keine unverzerrten Schätzer mit kleinerer Varianz.
\end{thm}
\footnotetext{Das heißt, es gibt keinen linearen, unverzerrten Schätzer mit
  geringerer Varianz.}

Ziel: Bewerte die Güte des erhaltenen Regressionsmodells
\begin{itemize}
\item Plotte $\hat{\eps}_i$ über $i$ bzw. über $x_i$. Sie sollten klein und
  symmetrisch sein sowie keinen Trend aufweisen.
\item Standardfehler für $\hat{a}$ und $\hat{b}$:
  \begin{align*}
  \[ \mathrm{SE}(\hat{a}) &= \sqrt{\MSE\hat{a}}
    = \sqrt{\hat{\sigma}^2 \frac{\sum x_i^2}{n \sum (x_i - \bar{x})^2}}, \\
    \mathrm{SE} (\hat{b}) &= \sqrt{\MSE\hat{b}}
    = \sqrt{\hat{\sigma}^2 \frac{1}{n \sum (x_i - \bar{x})^2}}
  \end{align*}
  nach Prop. 5.2.
\item Streuungszerlegung
\end{itemize}

\clearpage

\begin{lem}
  Die \emph{Gesamtstreuung} (sum of squares total) der $y_i$
  \[ \operatorname{SQT} = \sum_{i=1}^n (y_i - \bar{y})^2 \]
  lässt sich in \emph{erklärte Streuung} (sum of squares explained)
  \[ \operatorname{SQE} = \sum_{i=1}^n (\hat{y}_i - \bar{y})^2 \]
  und \emph{Residualstreuung} (sum of squares residuals)
  \[ \operatorname{SQR} = \sum_{i=1}^n (y_i - \hat{y}_i)^2 \]
  zerlegen:
  \[ \operatorname{SQT} = \operatorname{SQE} + \operatorname{SQR}. \]
\end{lem}

\begin{proof}
  \begin{align*}
    \operatorname{SQT}
    &= \sum_{i=1}^n (y_i - \bar{y})^2 \\
    &= \sum_{i=1}^n (y_i - \hat{y}_i + \hat{y}_i - \bar{y})^2 \\
    &= \sum_{i=1}^n (y_i - \hat{y}_i)^2
      + 2 \sum_{i=1}^n (y_i - \hat{y}_i)(\hat{y}_i-\bar{y})
      + \sum_{i=1}^n (\hat{y}_i - \bar{y})^2 \\
    &= \operatorname{SQR} + E + \operatorname{SQE}.
  \end{align*}
  Es gilt
  \begin{align*}
    \rez{2} E
    &= \sum_{i=1}^n \hat{y}_i(y_i - \hat{y}_i) - \bar{y}
      \underbrace{\sum_{i=1}^n (y_i - \hat{y}_i )}_{= 0 \text{ nach 5.2}} \\
    &= \hat{a} \underbrace{\sum_{i=1}^n (y_i - \hat{y}_i)}_{=0}
      + \hat{b} \sum x_i (y_i - \hat{a} - \hat{b} x_i) \\
    &= \hat{b} \left(
      \sum_{i=1}^n x_i y_i - \hat{a} \sum_{i=1}^n x_i
      - \hat{b} \sum_{i=1}^n x_i^2    
      \right) \\
    &= \hat{b} \Big(
      \underbrace{\sum x_i y_i - n \bar{x} \bar{y}}_{(n-1)s_{xy}^2} +
      \underbrace{n \hat{b} \bar{x}^2 - \hat{b} \sum x_i^2}_{-(n-1)\hat{b} s_x^2}
      \Big) \\
    &= \hat{b} (n-1) (s_{XY}^2 - \hat{b} s_X^2)
      = \hat{b} (n-1) (s_{XY}^2 - \frac{s_{XY}^2}{s_X^2} s_X^2) = 0. \qedhere
  \end{align*}
\end{proof}

\begin{defn}
  Das \emph{Bestimmheitsmaß der Regressionsgeraden} ist definiert als
  \[ R^2 := \frac{\operatorname{SQE}}{\operatorname{SQT}}. \]
\end{defn}

Interpretation:
\begin{itemize}
\item $R^2 = 1$: $\operatorname{SQR} = \sum \hat{\eps}_i = 0$, also perfekter
  Fit.
\item $R^2 = 0$: $\operatorname{SQR} = \operatorname{SQT}$, also $\hat{y}_i =
  \bar{y}$ für alle $i$. Damit auch $\hat{b} = 0$, also $s_{XY}^2 = 0$ und $X$
  und $Y$ unkorreliert.
\item Faustregel: Es besteht ein linearer Zusammenhang, falls
  \[ R^2 > \frac{16}{n+2} \]
  für genügend großes $n$.
\end{itemize}

\begin{lem}
  ES gilt
  \[ R^2 = \rho_{XY}^2
    = \left( \frac{s_{XY}^2}{\sqrt{s_X^2 s_Y^2}} \right)^2. \]
  mit dem Bravais-Pearson-Korrelationskoeffizient $\rho_{XY}^2$.
\end{lem}

\begin{proof}
  Es gilt
  \[ \bar{\hat{y}} = \rez{n} \sum \hat{y}_i = \rez{n} \sum (\hat{a} + \hat{b}
    x_i) = \hat{a} + \hat{b} \bar{x}
    = (\bar{y}-\hat{b} \bar{x}) + \hat{b} \bar{x} = \bar{y}. \]
  Damit
  \begin{align*}
    \sum (\hat{y}_i - \bar{y})^2
    &= \sum (\hat{y}_i - \bar{\hat{y}} )^2
      = \sum( \hat{a} + \hat{b} x_i - \hat{a} + \hat{b} \bar{x})^2
      = \hat{b}^2 \sum (x_i - \bar{x})^2
  \end{align*}
  und damit
  \[ R^2 = \frac{\sum( \hat{y}_i - \bar{y})^2}{\sum (y_i - \bar{y})^2}
    = \frac{\hat{b}^2 \sum (x_i - \bar{x})^2}{\sum(y_i - \bar{y})^2}
    = \frac{ \left(  \frac{s_{XY}^2}{s_X^2} \right)^2 s_X^2}{s_Y^2}
    = \frac{(s_{XY}^2)^2}{s_X^2 s_Y^2} = \rho_{XY}^2. \qedhere
  \]
\end{proof}

\section{Intervallschätzung}
\begin{lem}
  Unter der Normalverteilungsannahme $\eps_i \sim \ndist(0,\sigma^2)$ oder
  äquivalent $Y_i \sim \ndist(a+b x_i, \sigma^2)$
  gilt für die Schätzer $\hat{a}$ und $\hat{b}$ im Standardmodell der linearen
  Regression
  \[ \hat{b} \sim \ndist(b, \sigma_b^2), \qquad
    \hat{a} \sim \ndist(a, \sigma_a^2) \]
  mit $\sigma_a^2$ und $\sigma_b^2$ wie in (5.4).

  Zudem gilt
  \[ \frac{\hat{a} - a}{\hat{\sigma}_a} \sim t_{n-2}, \qquad
    \frac{\hat{b} - b}{\hat{\sigma}_b} \sim t_{n-2}, \]
  wobei
  \[ \hat{\sigma}_a^2 = \hat{\sigma}^2 \frac{\sum x_i^2}{n \sum (x_i -
      \bar{x})^2}, \qquad
    \hat{\sigma}_b^2 = \hat{\sigma}^2 \frac{1}{n \sum (x_i -
      \bar{x})^2}.
  \]
\end{lem}

Beweis durch Nachrechnen.

Damit folgen als $1-\alpha$-Konfidenzintervalle für $a$ und $b$ (unter der
Normalverteilungsannahme):
\[ \big[ \hat{a} - t_{n-2,1-\alpha/2} \cdot \hat{\sigma}_a,
    \hat{a} + t_{n-2,1-\alpha/2} \cdot \hat{\sigma}_a \big], \quad
 \big[ \hat{b} - t_{n-2,1-\alpha/2} \cdot \hat{\sigma}_b,
 \hat{b} + t_{n-2,1-\alpha/2} \cdot \hat{\sigma}_b \big]. \]

\begin{rmrk*}
  \begin{itemize}
  \item Für $n > 30$ lassen sich die Quantile der t-Verteilung durch
    $\ndist(0,1)$-Quantile ersetzen.
  \item Die Normalverteilungsannahme gilt oft nur approximativ oder gar nicht.
    Die Aussage von Lemma 5.7 gelten näherungsweise, falls die
    Konsistenzbedingung (5.3) erfüllt ist, egal wie die $\eps_i$ verteilt sind.
  \end{itemize}
\end{rmrk*}

\section{Tests}
Klassische Tests im Standardmodell der einfachen linearen Regression:
\begin{mdframed}
  \begin{align*}
    H_0 &: a = a_0, & H_1 &: a \ne a_0, \\
    H_0 &: a \ge a_0, & H_1 &: a < a_0, & T_{a_0}
                      &:= \frac{\hat{a} - a_0}{\hat{\sigma}_a} \\
    H_0 &: a \le a_0, & H_1 &: a > a_0, \\[1em]
    H_0 &: b = b_0, & H_1 &: b \ne b_0, \\
    H_0 &: b \ge b_0, & H_1 &: b < b_0, & T_{b_0}
                      &:= \frac{\hat{b} - b_0}{\hat{\sigma}_b} \\
    H_0 &: b \le b_0, & H_1 &: b > b_0.
  \end{align*}
  Verwerfe $H_0$ zum (approximativen)\footnotemark Niveau $\alpha$, falls
  \[ |T_{a_0}| > t_{n-2, 1 - \alpha/2},
    T_{a_0} < - t_{n-2, 1 - \alpha}
    T_{a_0} > t_{n-2, 1 - \alpha}
    |T_{b_0}| > t_{n-2, 1 - \alpha/2}
    T_{b_0} < - t_{n-2, 1 - \alpha}
    T_{b_0} > t_{n-2, 1 - \alpha}
  \]
\end{mdframed}
\footnotetext{%
  Falls die Normalverteilungsannahme nicht gilt, aber die Konsistenzbedingung.
  Sonst ``approximativ'' streichen.
}

Speziell für $H_0 : b = 0$ ist die Teststatistik
\[ F := (n-2) \frac{R^2}{1-R^2}, \]
denn
\begin{align*}
  (n-2) \frac{R^2}{1-R^2}
  &= (n-2) \left( \rez{R^2} - 1 \right)^{-1} \\
  &= (n-2) \left( \frac{\operatorname{SQT} - \operatorname{SQE}}
    {\operatorname{SQE}} \right)^{-1} \\
  &= \frac{\operatorname{SQE}}{\operatorname{SQR} / (n-2)} \\
  &= \frac{\sum (\hat{y}_i -\bar{y})^2}{\rez{n-2} \sum(y_i - \hat{y}_i)^2} \\
  &= \frac{\hat{b}^2 \sum (x_i - \bar{x})^2}{\rez{n-2} \sum(y_i - \hat{y}_i)^2} \\
  &= \frac{\hat{b}^2 \sum (x_i - \bar{x})^2}{\hat{\sigma}^2} \\
  &= \left( \frac{\hat{b}}{\hat{\sigma}_b^2} \right) \\
  &= T_{b_0 = 0)^2.
\end{align*}

Gilt die Normalverteilungsannahme, so folgt unter $H_0$
\[ T_{b_0} \sim t_{n-2} \]
und damit
\[ F \sim F_{1,n-2} \]
und es folgt als Test
\begin{mdframed}
  \[ H_0 : b = 0 \quad \text{gegen} \quad H_1 : b \ne 0 \]
  Teststatistik:
  \[ F = (n-2) \frac{R^2}{1-R^2}. \]
  Verwerfe $H_0$ zum Niveau $\alpha$, falls
  \[ F > F_{1,n-2,1-\alpha}. \]
\end{mdframed}
\end{document}