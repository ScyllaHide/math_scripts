\documentclass[
 a4paper,
 12pt,
 parskip=half
 ]{scrreprt}

\usepackage{../.tex/settings}

\usepackage{../.tex/mathpkgs}
\usepackage{../.tex/mathcmds}

\usepackage[numbers,with_chapter]{../.tex/fancy_thm}

\swapnumbers
\theoremstyle{plain}
%\newtheorem{thm}{Satz}[section] % reset theorem numbering for each chapter
%\newtheorem{lem}[thm]{Lemma}    

\theoremstyle{definition}
\newtheorem{defn}[thm]{Definition} 
\newtheorem{folg}[thm]{Folgerung} 
\newtheorem{rmrk}[thm]{Bemerkung} 
\newtheorem{deno}[thm]{Bezeichnungen}
\newtheorem{exmp}[thm]{Beispiel} 
\newtheorem{prgp}[thm]{} % Numbered paragraph

\newtheorem*{rmrk*}{Bemerkung}
\newtheorem*{exmp*}{Beispiel}
\newtheorem*{defn*}{Definition}

\numberwithin{equation}{section}

\hypersetup{
  pdftitle={Maß und Integral},
  pdfauthor={Jonas Hippold},
  hidelinks
}

%opening
\title{Vorlesung\\Maß und Integral}
\subtitle{Wintersemester 2016/2017}
\author{Vorlesung: Prof. Dr. Zoltán Sasvári\\Mitschrift: Jonas Hippold}

\begin{document}

\maketitle

\tableofcontents

\clearpage

\section*{Organisatorisches}
 \begin{itemize}
  \item \emph{Aufgaben:} Auf der Vorlesungsseite
  \item \emph{Prüfungsvorleistung:} 50 \% der Punkte in den Pflichtaufgaben und eine vorgetragene Übungslösung in der Übung
  \item \emph{Abgabe der Pflichtaufgaben:} Dienstag, 9:20 Uhr, Briefkasten 
  \item \emph{Schriftliche Prüfung:} Definitionen, Sätze, Beweise \\
   Aufgaben: ähnlich wie in den Übungen \\
   Einschreibung: 09.01.2017 bis eine Woche vor Prüfungstermin
  \item \emph{Literatur:}
   \begin{itemize}
    \item G. B. Folland: Real Analysis
    \item H. Bauer: Wahrscheinlichkeitstheorie und Grundzüge der Maßtheorie
    \item R. L. Schilling: Maß und Integral (2015)
   \end{itemize}
 \end{itemize}

\chapter{Maße}
\section{Einführung}
\begin{prgp}
 Eine sehr alte Aufgabe der Geometrie: Man bestimme Länge, Flächeninhalt,
 Volumen von gewissen Gebieten, Körpern in $\real$, in der Ebene, im Raum.
  
 Mit Hilfe des Riemann-Integrals kann man dieses Problem für \emph{gewisse}
 Mengen in $\real^2$ und $\real^3$ lösen: Mengen, die durch glatte Kurven oder
 Flächen berandet sind.
 
  Mit diesem Zugang kann man aber kompliziertere Mengen nicht behandeln.
\end{prgp}
 
\subsection*{Axiomatischer Zugang}
Sei $d \in \nat$ beliebig. 

Anstelle von Länge, Flächeninhalt oder Volumen sprechen wir vom \emph{Maß} einer
Menge.

Welche Eigenschaften sollte ein Maß haben?

Ein Maß $\mu$ soll \emph{jeder Menge} $E \subset \real^d$ eine Zahl $\mu(E) \in
[0, \infty]$ zuordnen, so dass
\begin{enumerate}[(i)]
\item Ist $E_1, E_2, \ldots$ eine endliche oder unendliche Folge von
  \emph{disjunkten} Mengen $E_j \subset \real^d$, so gilt:
  \[ \mu(E_1 \cup E_2 \cup \ldots ) = \mu(E_1) + \mu(E_2) + \ldots \]
\item Sind $E$ und $F$ \emph{kongruent}\footnote{%
    $F$ entsteht aus $E$ mit Hilfe von Verschiebung, Spiegelung oder Rotation},
  so ist $\mu(E) = \mu(F)$.
\item Der $d$-dimensionale Einheitswürfel soll das Maß 1 haben:
  \[ \mu([0,1]^d) = 1. \]
\end{enumerate}

Wir zeigen, dass ein solches Maß nicht existiert! Wir betrachten nur den Fall
$d=1$, der allgemeine Fall lässt sich analog behandeln.

Wir definieren eine Äquivalenzrelation $\sim$ auf $[0,1)$ durch
\[ x \sim y \qRq x-y \in \rat. \]
Sei $N$ eine Teilmenge von $[0,1)$, die genau ein Element von jeder
Äquivalenzklasse enthält. Sei $R := \rat \cap [0,1)$ und für jedes $r \in R$ sei
\[ N_r := \{ x + r: x \in N \cap [0,1-r) \} \cup \{ x + r - 1: x \in N \cap [1-r,1) \}. \] 
Um $N_r$ zu erhalten, wird $N$ um $r$ nach rechts verschoben; der Teil, der
dabei $[0,1)$ verlässt, wird um 1 nach links verschoben. 

Wir zeigen:
\[ [0,1) = \bigcup_{r \in R} N_r \text{ und } N_r \cap N_s = \emptyset,\, r \ne s. \]

\begin{proof}
 Sei $x \in [0,1)$ beliebig und sei $y$ das Element von $N$, das zur
 Äquivalenzklasse von $x$ gehört. Dann ist $x \in N_r$, wobei
 \[ r := \begin{cases}
          x-y, & \text{wenn } x \ge y, \\
          x-y+1 & \text{wenn } x < y.
         \end{cases} \]
 Also ist
 \[ [0,1) = \bigcup_{r \in R} N_r. \]
 
 Ist $x \in N_r \cap N_s$, so sind $x-r$ (oder $x-r+1$) und $x-s$ (oder $x-s+1$)
 verschiedene Elemente von $N$, die zur selben Äquivalenzklasse gehören. Das ist
 nicht möglich. Also ist 
 \[ N_r \cap N_s = \emptyset,\, r \ne s. \]

 Wir zeigen nun noch:
 \[ \mu( N ) = \mu(N_r),\, r \in R. \]

 Aus den geforderten Eigenschaften von $\mu$ folgt
 \[ \mu(N) \overset{\text{(i)}}{=} \mu( N \cap [0,1-r) ) + \mu( N \cap [1-r, 1) ) \overset{\text{(ii)}}{=} \mu(N_r). \] 
 
 Nun ist
 \[ 1 \overset{\text{(iii)}}{=} \mu( [0,1) ) \overset{\text{(i)}}{=} \sum_{r \in R} \mu( N_r ). \]
 Das ist nicht möglich, da
 \[ \sum_{r \in R} \mu( N_r ) = 
   \begin{cases}
    \infty, & \text{falls } \mu(N) > 0, \\
    0, & \text{falls } \mu(N) = 0.
   \end{cases} \qedhere \]
\end{proof}

Sei $I \ne \emptyset$ eine beliebige Menge und $a_i \in \real$. Was versteht man
unter $\sum_{i \in I} a_i$? 

Ein Ausweg wäre (i) nur für endliche Folgen zu fordern (was die Nützlichkeit von
$\mu$ sehr einschränken würde). Das geht auch nicht! 

\newpage

\begin{thm}[Banach-Tarski, 1924]
 Seien $U$ und $V$ beliebige, nichtleere, beschränkte, offene Mengen in
 $\real^d$, $d \ge 3$. Dann gibt es ein $k \in \nat$ und Teilmengen $E_1,
 \ldots, E_k, F_1, \ldots, F_k$, so dass 
 \begin{enumerate}[(i)]
  \item Die $E_j$ sind disjunkt und ihre Vereinigung ist $U$.
  \item Die $F_j$ sind disjunkt und ihre Vereinigung ist $V$.
  \item $E_j$ ist kongruent zu $F_j$ für alle $j$.
 \end{enumerate}
\end{thm}

Das bedeutet: Die Mengen $U$ und $V$ haben mit den obigen Forderungen das selbe
Maß, egal wie man $U$ und $V$ wählt. 

Folgerung: $\real^d$ enthält Teilmengen, die so seltsam sind, dass man kein
\emph{geometrisch} vernünf\-tiges Maß für \emph{alle} Teilmengen definieren
kann. 

Wir geben die Forderung auf, dass das Maß für \emph{alle} Teilmengen definiert
werden soll. 

\section{Aufgaben}
 Siehe \verb+Aufgaben-1-2.pdf+.

\section{\texorpdfstring{$\sigma$}{Sigma}-Algebren}
Sei $X \ne \emptyset$ eine beliebige Menge. Das Komplement von $A \subset X$
bezüglich $X$ bezeichnet $A^C := X \setminus A$.

\begin{defn}
 Eine Familie $\mA \subset \pot(X)$ heißt eine \emph{$\sigma$-Algebra} (in $X$),
 wenn
 \begin{enumerate}[(i)]
  \item $X \in \mA$.
  \item $A \in \mA$ $\Rightarrow$ $A^C \in \mA$.
  \item $A_1, A_2, \ldots \in \mA$ $\Rightarrow$ $\bigcup_{i=1}^\infty A_i \in \mA$.
 \end{enumerate}
 $\mA$ heißt eine \emph{Algebra}, wenn (i), (ii) und (iii) mit endlich vielen
 $A_i$ gelten.
\end{defn}

Einfache Beispiele:
\begin{itemize}
 \item $\mA = \{ X, \emptyset \}$.
 \item $\mA = \pot( X )$.
 \item $\mA = \{ X, \emptyset, A, A^C \}$ mit $A \subset X$.
\end{itemize}

\clearpage

\begin{lem}
 Ist $\mA$ eine Algebra, so gilt
 \begin{enumerate}[(i)]
  \item $\emptyset \in \mA$,
  \item $A \setminus B \subset \mA$, wenn $A,B \subset \mA$,
  \item $\bigcap_{j=1}^n A_j \in \mA$, wenn $A_1, \ldots, A_n \in \mA$.
  \enumeratext{(i)}{Ist $\mA$ eine $\sigma$-Algebra, so gilt auch}
  \item $\bigcap_{j=1}^\infty A_j \in \mA$, wenn $A_1, \ldots, A_n \in \mA$.
 \end{enumerate}
\end{lem}

\begin{proof}
 (i): $\emptyset = X^C \in \mA$.
  
 (iii) folgt aus (1.3.1.ii), (1.3.1.iii) und 
 \[ \left( \bigcap_{j=1}^n A_j \right)^C = \bigcup_{j=1}^n A_j^C \in \mA. \]
 
 (ii) $A \setminus B = A \cap B^C \overset{\text{(iii)}}\in \mA$.
 
 (iv) wird analog zu (iii) bewiesen.
\end{proof}

\begin{thm}
 Ist eine Algebra abgeschlossen bezüglich abzählbarer \emph{disjunkter} Vereinigungen, so ist sie eine $\sigma$-Algebra.
\end{thm}

\begin{proof}
 Sei $\mA$ eine Algebra, $E_1, E_2, \ldots \in \mA$ und für $k = 1, 2, \ldots$ definiere
 \[ F_k := E_k \setminus \left[ \bigcup_{j=1}^{k-1} E_j \right] = E_k \cap \left[ \bigcup_{j=1}^{k-1} E_j \right]^C \]
 Die $F_k$ gehören zu $\mA$, da $\mA$ eine Algebra ist und $F_k$ durch endlich viele Vereinigungen entsteht. Sie sind disjunkt und
 \[ \bigcup_{j=1}^\infty F_j = \bigcup_{j=1}^\infty E_j \qRq \bigcup_{j=1}^\infty E_j \in \mA, \]
 weil nach Voraussetzung $\mA$ abgeschlossen bezüglich abzählbarer disjunkter Vereinigungen ist.
\end{proof}

\begin{lem}
 Jeder Durchschnitt von $\sigma$-Algebren in $X$ ist selber eine $\sigma$-Algebra.
\end{lem}

\begin{proof}
 Einfaches Nachprüfen der Eigenschaften.
\end{proof}

\begin{folg}
 Zu jeder Familie $E \subset \pot(X)$ existiert eine kleinste, $E$ enthaltende $\sigma$-Algebra. Wir bezeichnen sie mit $\sigma(E)$.
\end{folg}

\begin{proof}
 $\pot(X)$ ist eine $\sigma$-Algebra, die $E$ enthält. Man bilde den Durchschnitt mit allen E enthaltenden $\sigma$-Algebren.
\end{proof}

Man nennt $\sigma(E)$ die \emph{von $E$ erzeugte $\sigma$-Algebra} und $E$ einen \emph{Erzeuger von $\sigma(E)$}.

Beispiel: $E = \{A\}$, $\sigma(E) = \{ X, \emptyset, A, A^C \}$.

\begin{lem}
 Ist $E \subset \sigma(F)$, so gilt $\sigma(E) \subset \sigma(F)$.
\end{lem}

\begin{proof}
 $\sigma(F)$ ist eine $\sigma$-Algebra, die $E$ enthält $\Rightarrow$ sie enthält auch $\sigma(E)$.
\end{proof}

\begin{rmrk}
 Sei $\mA_1 \subset \mA_2 \subset \ldots$ eine unendliche Folge von $\sigma$-Algebren, so dass $\mA_n \ne \mA_{n+1}$ für alle $n$. Dann ist $\bigcup_{n=1}^{\infty} \mA_n$ keine $\sigma$-Algebra!\footnote{Zum Beweis siehe \texttt{UnionsOfSigmaFields.pdf}}
\end{rmrk}

\begin{defn}
 Sei $X$ ein metrischer Raum (oder ein topologischer Raum). Die $\sigma$-Algebra, die durch alle \emph{offenen Mengen} von $X$ erzeugt wird, heißt \emph{Borelsche $\sigma$-Algebra} und wird mit $\borel(X)$ bezeichnet. Die Elemente von $\borel(X)$ heißen \emph{Borelmengen}.
\end{defn}

Ist $\borel(\real) = \pot(\real)$? Nein! Beispiele folgen später.

\section{Aufgaben}
Siehe \verb+Aufgaben-1-4.pdf+.

\section{Maße}
Sei $X \ne \emptyset$ eine beliebige Menge, $\mA$ eine $\sigma$-Algebra in $X$.

\begin{defn}
 Ein \emph{Maß} auf $\mA$ ist eine Abbildung $\mu: \mA \to [0,\infty]$ mit
 \begin{enumerate}[(i)]
  \item $\mu( \emptyset ) = 0$.
  \item Sind die Mengen $A_1, A_2, \ldots \in \mA$ paarweise disjunkt, so gilt
  \[ \mu\left( \bigcup_{i=1}^\infty A_i \right) = \sum_{i=1}^\infty \mu(A_i). \]
  Diese Eigenschaft nennt man \emph{$\sigma$-Additivität}\footnote{Auch möglich: Nur endliche Additivität, aber solche Maße kommen selten vor.}.
 \end{enumerate}
\end{defn}

\begin{rmrk*}
 Auch üblich: Maß auf $(X, \mA)$ (redundant, weil $X \in \mA$) oder einfach Maß auf $X$, wenn klar ist, welche $\sigma$-Algebra gemeint ist.
\end{rmrk*}

\begin{prgp}
 Sei $\mu$ ein Maß auf einer $\sigma$-Algebra in $X$. Die folgenden Bezeichnungen sind üblich:
 \begin{itemize}
  \item $(X,\mA)$: \emph{Messraum},
  \item $(X,\mA,\mu)$: \emph{Maßraum},
  \item die Elemente von $\mA$ heißen \emph{messbare Mengen}.
  \item Im Fall $\mu(X) < \infty$ heißt $\mu$ \emph{endlich}.
  \item Wenn $X = \bigcup_{j=1}^\infty E_j$, wobei $E_j \in \mA$ und $\mu(E_j) < \infty$, dann heißt $\mu$ \emph{$\sigma$-endlich}.
  \item Wenn $E = \bigcup_{j=1}^\infty E_j$, $E_j$ wie oben, dann heißt $E$ \emph{$\sigma$-endlich} (bezüglich $\mu$).
 \end{itemize}
\end{prgp}

\begin{exmp}
 \begin{enumerate}[(i)]
  \item Das \emph{Zählmaß} 
   \[ \mu(A) := \text{ Anzahl der Elemente von } A \in \pot(X) = \mA \]
   und das \emph{Dirac-Maß} 
   \[ \delta_x(A) := \mu(A \cap \{x\}) \text{ in } x \in X, A \in \pot(X) \]
   sind Maße auf $\pot(X)$. Das Dirac-Maß ist endlich, da $\mu(x) = 1$. Das Zählmaß ist genau dann endlich, wenn $X$ endlich ist; es ist genau dann $\sigma$-endlich, wenn $X$ abzählbar ist.
  \item Sei $X$ unendlich, $\mu(A) := 0$, wenn $A$ endlich ist, sonst sei $\mu(A) := \infty$. $\mu$ ist \emph{kein Maß}: Seien $x_1, x_2, \ldots \in X$ eine Folge in $X$, $x_i \ne x_j$ für $i \ne j$, dann ist
  \[ \mu( \{ x_1, x_2, \ldots \} ) \ne \sum_{j=1}^\infty \mu( \{ x_j \} ) = 0. \]
  Allerdings ist $\mu$ endlich additiv.
 \end{enumerate}
\end{exmp}

\begin{thm}
 Sei $(X, \mA, \mu)$ ein Maßraum und $E, F, E_j \in \mA$, $j \in \nat$. 
 \begin{enumerate}[(i)]
  \item \emph{Monotonie.} Wenn $E \subset F$, dann ist $\mu(E) \le \mu(F)$.
  \item \emph{Subadditivität.} $\mu \left( \bigcup_{j=1}^\infty E_j \right) \le \sum_{j=1}^\infty \mu(E_j)$.
  \item \emph{Stetigkeit von unten.} Wenn $E_1 \subset E_2 \subset \ldots$, dann
   \[ \mu \left( \bigcup_{j=1}^\infty E_j \right) = \lim_{j \to \infty} \mu( E_j ).  \]
  \item \emph{Stetigkeit von oben.} Wenn $E_1 \supset E_2 \supset \ldots$ und $\mu(E_n) < \infty$ für ein $n \in \nat$, dann
   \[ \mu \left( \bigcap_{j=1}^\infty E_j \right) = \lim_{j \to \infty} \mu( E_j ).  \]
 \end{enumerate}
\end{thm}

\begin{proof}
 \begin{enumerate}[(i)]
  \item $F \setminus E \in \mA$; $E \cap (F \setminus E) = \emptyset$, also
   \[ \mu(F) = \mu(E) + \underbrace{\mu(F \setminus E)}_{\ge 0} \ge \mu(E). \]
  \item Sei $F_1 = E_1$ und (wie im Beweis von Satz 1.3.3)
   \[ F_k := E_k \setminus \left( \bigcup_{j=1}^\infty E_j \right), \quad k > 1. \]
   $F_k \in \mA$, $F_k \cap F_j = \emptyset$ für $k \ne j$, $F_k \subset E_k$
   \[ \bigcup_{j=1}^\infty F_j = \bigcup_{j=1}^\infty E_j. \]
   Damit folgt unter Verwendung der $\sigma$-Additivität:
   \[ \mu \left( \bigcup_{j=1}^\infty E_j \right) = \mu \left( \bigcup_{j=1}^\infty F_j \right) \overset{\sigma}{=} \sum_{j=1}^\infty \mu( F_j ) \overset{\text{(i)}}{\le} \sum_{j=1}^\infty \mu( E_j ). \]
  \item Sei $E_0 := \emptyset$. Wegen $E_{j-1} \subset E_j$ gilt für $j \in \nat$
   \[ \mu( E_j \setminus E_{j-1} ) = \mu(E_j) - \mu(E_{j-1}) \]
   und
   \[ \bigcup_{j=1}^\infty E_j = \bigcup_{j=1}^\infty(E_j \setminus E_{j-1}), \]
   wobei die Mengen auf der rechten Seite disjunkt sind.
   \begin{align*}
    \mu \left( \bigcup_{j=1}^\infty E_j \right) 
      &\overset{\sigma}{=} \sum_{j=1}^\infty \mu( E_j \setminus E_{j-1}) \\
      &= \lim_{n \to \infty} \sum_{j=1}^n ( \mu( E_j ) - \mu( E_{j-1} ) ) \\
      &= \lim_{n \to \infty} \mu( E_n ).
   \end{align*}
  \item Sei $F_j := E_n \setminus E_j$, $j > n$. Dann $F_{n+1} \subset F_{n+2} \subset \ldots$ und $\mu( E_n )  = \mu( F_j ) + \mu( E_j )$ für $j > n$ sowie
  \[ E_n = \left( \bigcap_{j=n+1}^\infty E_j \right) \cup \left( \bigcup_{j=n+1}^\infty F_j \right). \]
  Damit folgt
  \begin{align*}
   \mu(E_n) 
    &\overset{\text{(iii)}}{=} \mu \left( \bigcap_{j=n+1}^\infty E_j \right) + \lim_{j \to \infty} \mu (F_j) \\
    & = \mu \left( \bigcap_{j=n+1}^\infty E_j \right) + \lim_{j \to \infty} ( \mu( E_n ) - \mu(E_j) ).
  \end{align*}
  Durch Subtraktion von $\mu(E_n)$ folgt die Behauptung. Dabei nutzen wir die Voraussetzung $\mu(E_n) < \infty$! \qedhere
 \end{enumerate}
\end{proof}

\begin{deno}
 Sei $(X, \mA, \mu)$ ein Maßraum. Wenn $E \in \mA$ und $\mu(E) = 0$, dann heißt $E$ eine \emph{Nullmenge}, oder eine \emph{$\mu$-Nullmenge}. Die Vereinigung von abzählbar vielen Nullmengen ist wieder eine Nullmenge\footnote{Das folgt aus der Subadditivität von $\mu$.}.
 
 Gilt eine Eigenschaft für alle $x \in X$ bis auf eine Nullmenge, so sagt man, sie gilt \emph{fast überall} oder \emph{$\mu$-fast überall}.
 
 Ist $E$ eine Nullmenge $F \subset E$, so ist auch $F$ eine Nullmenge, vorausgesetzt $F$ ist messbar. Das ist im Allgemeinen nicht so\footnote{Einfaches Beispiel: $\mA = \{ \emptyset, X \}$, $X \ne \emptyset$, $\mu( \emptyset ) = \mu(X) = 0$; Dann ist $X$ eine Nullmenge, aber Teilmengen von $X$ gehören nicht zu $\mA$ und sind daher nicht messbar.}.
 
 Enthält $\mA$ alle Teilmengen von Nullmengen, so heißt $\mA$ \emph{$\mu$-vollständig}.
\end{deno}

\begin{thm}[Vervollständigung von Maßen]
 Sei $(X, \mA, \mu)$ ein Maßraum, $\mathcal{N} := \{ N \in \mA: \mu(N) = 0 \}$ die Menge der Nullmengen in $\mA$ und 
 \[ \obar{\mA} := \{ E \cup F : E \in \mA, F \subset N \text{ für ein } N \in \mathcal{N} \}. \]
 Dann ist $\obar{\mA}$ eine $\sigma$-Algebra und $\mu$ lässt sich eindeutig zu einem vollständigen Maß $\obar{\mu}$ auf $\obar{\mA}$ fortsetzen.
\end{thm}

\begin{proof}
 $\obar{\mA}$ ist eine $\sigma$-Algebra: Es ist klar, dass $\obar{\mA}$ abgeschlossen ist bezüglich abzählbarer Vereinigungen\footnote{Betrachte die Folgen $(E_j)_{j \in \nat} \subset \mA$ und $(F_j)_{j \in \nat}, F_j \subset N_j, \mu(N_j) = 0$. Es gilt $\bigcup E_j \in \mA$, weil $\mA$ eine $\sigma$-Algebra ist. Die abzählbare Vereinigung von Nullmengen ist ebenfalls wieder eine Nullmenge, also $(\bigcup E_j) \cup (\bigcup F_j) \in \obar{\mA}$.}.
 
 Noch zu zeigen: $A = E \cup F \in \obar{\mA} \Rightarrow A^C \in \obar{\mA}$ (wobei $F \subset N, \mu(N) = 0$). Wir dürfen annehmen, dass $E \cap N = \emptyset$ und damit $E \cap F = \emptyset$ (sonst ersetzen wir wir $F$ und $N$ durch $F \setminus E$ bzw. $N \setminus E$). Es gilt
 \begin{align*}
  E \cup F &= ( E \cup N ) \cap (N^C \cup F) \\
  (E \cup F)^C &= \underbrace{( E \cup N )^C}_{\in \mA} \cup \underbrace{(N \setminus F)}_{\in \mathcal{N}} \in \obar{\mA}. \qedhere
 \end{align*}
\end{proof}

\subsection*{Fortsetzung auf $\obar{\mA}$}
 Für $E \cup F \in \obar{\mA}$ sei $\obar{\mu}(E \cup F) := \mu(E)$. Wir müssen zeigen, dass diese Definition korrekt ist. Nehmen wir an, dass 
 \[ E_1 \cup F_1 = E_2 \cup F_2, \quad F_j \subset N_j \in \mathcal{N}. \]
 Dann $E_1 \subset E_2 \cup N_2$ $\Rightarrow$ $\mu(E_1) \le \mu(E_2) + \mu(N_2) = \mu(E_2)$ wegen der Monotonie und Subadditivität von $\mu$. Genauso sehen, wir dass $\mu(E_2) \le \mu(E_1)$ $\Rightarrow$ $\mu(E_1) = \mu(E_2)$.
 
 Es ist leicht zu sehen,  dass $\obar{\mu}$ ein vollständiges Maß ist und die Fortsetzung eindeutig ist (siehe Aufgabe 1.6.4).

\section{Aufgaben}
Siehe \verb+Aufgaben-1-6.pdf+.

\section{Äußere Maße}
Ziel des Abschnitts: Konstruktion von Maßen; $X \ne \emptyset$ beliebige Menge

\begin{defn}
 Ein \emph{äußeres Maß} auf $\pot(X)$ ist eine Abbildung $\mu^* : \pot(X) \to [0,\infty]$ mit den folgenden Eigenschaften:
 \begin{enumerate}[(i)]
  \item $\mu^*(\emptyset) = 0$,
  \item $\mu^*( A ) \le \mu^*(B)$, wenn $A \subset B \subset X$,
  \item $\mu^*( \bigcup_{j=1}^\infty A_j ) \le \sum_{j=1}^\infty A_j$, $A_j \in \pot(X)$.
 \end{enumerate}
\end{defn}

\begin{lem}
 Seien $\mE \subset \pot(X)$ und $\rho: \mE \to [0, \infty]$ so, dass $\emptyset, X \in \mE$ und $\rho(\emptyset) = 0$. Für $A \subset X$ definieren wir
 \[ \mu^*(A) := \inf \left\{ \sum_{j=1}^\infty \rho(E_j) : E_j \in \mE, A \subset \bigcup_{j=1}^\infty E_j \right\}. \]
 Dann ist $\mu^*$ ein äußeres Maß.
\end{lem}

\begin{exmp*}
 $X = \real$, $\mE$: Intervalle $(a,b)$, $\rho((a,b)) := b-a$ oder $X = \real^2$, $\mE$: Rechtecke $E = (a_1, b_1) \times (a_2, b_2)$, $\rho(E) := (b_1 - a_1) \cdot (b_2 - a_2)$.
\end{exmp*}

\begin{proof}
 Die Definition von $\mu^*$ ist sinnvoll, da $X \in \mE$. Es gilt: $\mu^*(\emptyset) = 0$ (man nehme $E_j := \emptyset$).
 
 1.7.1(ii) ist klar.
 
 Subadditivität: Seien $A_1, A_2, \ldots \in \pot(X)$, $A:= \bigcup_{j=1}^\infty A_j$ und $\eps > 0$. Für $j, k \in \nat$ wählen wir $E_j^k \in \pot(X)$, so dass\footnote{Wir können solche Mengen immer finden, weil $\mu^*$ als Infimum definiert ist.}
 \[ A_j \subset \bigcup_{k=1}^\infty E_j^k, \quad \text{und} \quad \sum_{k=1}^\infty \rho( E_j^k) \le \mu^* (A_j) + \frac{\eps}{2^j}. \]
 Es gilt $A \subset \bigcup_{j,k} E_j^k$ und
 \[ \sum_{j=1}^\infty \sum_{k=1}^\infty \rho( E_j^k ) \le \sum_{j=1}^\infty \mu^*(A_j) + \eps. \]
 Folglich $\mu^*(A) \le \sum_{j=1}^\infty \mu^*(A_j) + \eps$. Da $\eps > 0$ beliebig ist, folgt die Aussage.
\end{proof}

\begin{defn}
 Sei $\mu^*$ ein äußeres Maß auf $\pot(X)$. Eine Menge $A \subset X$ heißt \emph{$\mu^*$-messbar}, wenn
 \[ \mu^*(S) \ge \mu^*( S \cap A ) + \mu^*( S \setminus A ) \tag{i} \]
 gilt für alle $S \subset X$. Die Familie aller $\mu^*$-messbaren Mengen wird mit $\meas_{\mu^*}$ bezeichnet.
\end{defn}

\begin{rmrk*}
 Aus (i) folgt, dass
 \[ \mu^*(S) = \mu^* (S \cap A) + \mu^* ( S \setminus A ), \quad S \subset X \]
 wegen der Subadditivität 1.7.1(iii).
\end{rmrk*}

\begin{thm}[Caratheodory]
 Die Familie $\meas_{\mu^*}$ ist eine $\sigma$-Algebra und $\mu^*$ ist ein Maß auf $\meas_{\mu^*}$.
 
 Ist $\mu^*(A) = 0$, so gilt $A \in \meas_{\mu^*}$.
\end{thm}

\begin{proof}
 siehe \verb+Beweis-Satz-1-7-4.pdf+.
\end{proof}

\begin{defn}
 Ein \emph{Prämaß} auf einer Algebra $\mA$ ist eine Abbildung $\mu: \mA \to [0, \infty]$ mit 
 \begin{enumerate}[(i)]
  \item $\mu(\emptyset) = 0$.
  \item Für beliebige disjunkte Mengen $A_1, A_2, \ldots \in \mA$ mit $\bigcup_{j=1}^\infty A_j \in \mA$ gilt
   \[ \mu \left( \bigcup_{j=1}^\infty A_j \right) = \sum_{j=1}^\infty \mu( A_j ). \]
 \end{enumerate}
\end{defn}

\begin{rmrk*}
 $\mu$ ist monoton, also für $A, B \in \mA, A \subset B$ gilt $\mu(A) \le \mu(B)$. Das folgt aus $B = A \cup (B \setminus A)$.
\end{rmrk*}

Nach Lemma 1.7.2 wird durch
\[ \mu^*(A) := \inf \left\{ \bigcup_{j=0}^\infty \mu(E_j) : E_j \in \mA, A \subset \bigcup_{j=0}^\infty E_j \right \} \]
ein äußeres Maß  auf $\pot(X)$ definiert.

\begin{lem}
 Ist $\mu$ ein Prämaß auf einer Algebra $\mA$, so gilt
 \begin{enumerate}[(i)]
  \item $\mA \subset \meas_{\mu^*}$,
  \item $\mu^* |_\mA = \mu$ (das heißt $\mu^*$ ist die Fortsetzung von $\mu$).
 \end{enumerate}
\end{lem}

\begin{proof}
 (ii) Sei $E \in \mA$ und $E \subset \bigcup_{j=1}^\infty A_j$ wobei $A_j \in \mA$. Wir setzen für $n \in \nat$
 \[ B_n := E \cap \left( A_n \setminus \bigcup_{j = 1}^{n-1} A_j \right). \]
 Dann sind die $B_n$ disjunkte Elemente von $\mA$, $B_n \subset A_n$. Es gilt $\bigcup_{j=1}^\infty B_j = E$ wegen $E \subset \bigcup_{j=1}^\infty A_j$. Also folgt
 \[ \mu(E) = \sum_{j=1}^\infty \mu(B_j) \overset{\text{Mon.}}{\le} \sum_{j=1}^\infty \mu(A_j) \qRq \mu(E) \le \mu^*(E). \]
 Die Ungleichung $\mu^*(E) \le \mu(E)$ folgt aus $E \subset \bigcup_{j=1}^\infty A_j$, wobei $A_1 = E$ und $A_j = \emptyset$ für $j > 1$. $\mu^*(E) \le \mu(E)$ gilt jetzt, weil $\mu^*$ als Infimum definiert ist.
 
 Damit folgt $\mu^*(E) = \mu(E)$.
 
 (i) Seien $A \in \mA$, $S \subset X$ und $\eps > 0$. Wir wählen eine Folge $\{ B_j \}_{j=1}^\infty \subset \mA$ mit $S \subset \bigcup_{j=1}^\infty B_j$ und $\sum_{j=1}^\infty B_j \le \mu^*(S) + \eps$.
 \begin{align*}
  \mu^*(S) + \eps &\ge \sum_{j=1}^\infty \mu(B_j \cap A) + \sum_{j=1}^\infty \mu(B_j \cap A^C ) \\
  \intertext{(Das folgt aus der Additivität von $\mu$, weil $B_j \cap A$ und $B_j \cap A^C$ disjunkt sind.)}
  &\ge \mu^*( S \cap A ) + \mu^*( S \cap A^C )
 \end{align*}
 Weil $\eps$ beliebig war, ist
 \[ \mu^*(S) \ge \mu^*(S \cap A) + \mu^*(S \cap A^C), \]
 also folgt $A$ ist $\mu^*$-messbar.
\end{proof}

\begin{thm}
 Seien $\mA \subset \pot(X)$ eine Algebra, $\mu$ ein Prämaß auf $\mA$; $\meas := \sigma(\mA)$ und $\obar{\mu} := \mu|_\meas$. Dann gilt
 \begin{enumerate}[(i)]
  \item $\obar{\mu}$ ist ein Maß auf $\meas$ mit $\obar{\mu}|_\mA = \mu$.
  \item Ist $\nu$ ein Maß auf $\meas$ mit $\nu |_\mA = \mu$, so gilt $\nu(E) \le \obar{\mu}(E)$ für alle $E \in \meas$, wobei im Fall $\obar{\mu}(E) < \infty$ sogar Gleichheit gilt\footnotemark.
  \item Ist $\mu$ $\sigma$-endlich, so ist $\obar{\mu}$ die \emph{einzige} Fortsetzung von $\mu$ zu einem Maß auf $\meas$.
 \end{enumerate}
\end{thm}
\footnotetext{Das heißt für Mengen mit endlichem Maß ist die Fortsetzung eindeutig.}

\begin{proof}
 \begin{enumerate}[(i)]
  \item Folgt aus Satz 1.7.4 (Caratheodory) und aus Lemma 1.7.6.
  \item Sei $E \in \meas$, $E \subset \bigcup_{j=1}^\infty A_j$, wobei $A_j \in \mA$. Dann gilt
  \[ \nu(E) \overset{\text{Subadd.}}{\le} \sum_{j=1}^\infty \nu(A_j) \overset{\text{Ann.}}{=} \sum_{j=1}^\infty \mu(A_j) \qRq \nu(E) \overset{\text{Def. von }\obar{\mu}}{\le} \obar{\mu}(E). \]
  Für $A := \bigcup_{j=1}^\infty A_j$ gilt
  \[ \nu(A) \overset{\text{Stet.}}{=} \lim_{n \to \infty} \nu \left( \bigcup_{j=1}^n A_j \right) = \lim_{n \to \infty} \mu \left( \bigcup_{j=1}^n A_j \right) \overset{\text{Stet.}}{=} \obar{\mu}(A), \]
  wobei ``Stet.'' für ``Stetigkeit von unten'' steht. 
  
  Im Falle $\obar{\mu}(E) < \infty$ können wir die $A_j$ so wählen, dass
  \[ \obar{\mu}(A) < \obar{\mu}(E) + \eps \qRq \obar{\mu}( A \setminus E ) < \eps \]
  und
  \begin{align*}
   \obar{\mu}(E) \le \obar{\mu}(A) = \nu(A) &= \nu(E) + \nu(A \setminus E) \\
   &\le \nu(E) + \obar{\mu}(A \setminus E) \\
   &\le \nu(E) + \eps.
  \end{align*}
  Weil $\eps$ beliebig, folgt $\obar{\mu}(E) = \nu(E)$.
  \item Sei $X = \bigcup_{j=1}^\infty A_j$, wobei $\mu(A_j) < \infty$. O.B.d.A. seien die $A_j$ disjunkt. Dann gilt für alle $E \in \meas$
  \[ \obar{\mu}(E) = \sum_{j=1}^\infty (E \cap A_j) \overset{\text{(ii)}}{=} \sum_{j=1}^\infty \nu(E \cap A_j) = \nu(E), \]
  das heißt $\obar{\mu} = \nu$ auf $\meas$. \qedhere
 \end{enumerate}
\end{proof}

Zusammenfassung:
\begin{itemize}
 \item Maß $\mu$ auf $\sigma$-Algebra
 \item Äußeres Maß $\mu^*$ auf $\pot(X)$ $\rightarrow$ Einschränkung auf die $\sigma$-Algebra von messbaren Mengen, dort Maß 
 \item Prämaß auf einer Algebra $\rightarrow$ Äußeres Maß $\rightarrow$ Maß auf $\sigma$-Algebra.
\end{itemize}

\section{Aufgaben}
Siehe \verb+Aufgaben-1-8.pdf+.

\section{Produkt von \texorpdfstring{$\sigma$}{Sigma}-Algebren}
Bezeichnungen in diesem Abschnitt:
\begin{itemize}
 \item $A \ne \emptyset$ eine Indexmenge,
 \item $\{ X_\alpha \}$, $\alpha \in A$ eine Familie von nichtleeren Mengen,
 \item $X := \prod_{\alpha \in A} X_\alpha = \{ f:A \to \bigcup_{\alpha \in A} X_\alpha$ mit $f( \alpha ) \in X_\alpha,$ für alle $\alpha \in A \}$.
 \item Falls $X_\alpha = Y$ für alle $\alpha$, dann ist $X = Y^A$ die Menge aller Funktionen von $A$ nach $Y$.
 \item Die Elemente von $X$ werden mit $X = (X_\alpha) = (X_\alpha)_{\alpha \in A} \in X$ bezeichnet.
 \item $\Pi_\alpha : X \to X_\alpha$ sind die \emph{Koordinatenabbildungen} $\Pi_\alpha(X) = X_\alpha$, zum Beispiel $A = \{ 1, 2 \}$, $X_1 = X_2 = \real$, $X = \real^2 = \real \times \real$, $\Pi_1(X) = X_1$.
 \item $\meas_\alpha$ eine $\sigma$-Algebra auf $X_\alpha$.
\end{itemize}

\begin{exmp*}
 $A = \{ 1,2 \}$, $X_1 = X_2 = \real$, $X_1 \times X_2 = \real^2$, $\meas_1, \meas_2 = \borel( \real )$.
\end{exmp*}

\begin{defn}
 Die \emph{Produkt-$\sigma$-Algebra} $\bigotimes_{\alpha \in A} \meas_\alpha$ auf $X$ ist die $\sigma$-Algebra, die durch
 \[ \{ \Pi_\alpha^{-1}( E_\alpha ) : E_\alpha \in \meas_\alpha, \alpha \in A \} \]
 erzeugt wird.

\[ \Pi_\alpha^{-1} ( E_\alpha ) = \prod_{\beta \in A} E_\beta, \]
wobei $E_\beta = X_\alpha$, $\alpha \ne \beta$.
 
Im Falle $A= \{ 1, 2, \ldots, n \}$ schreiben wir $\bigotimes_1^n \meas_j$ oder $\meas_1 \otimes \cdots \otimes \meas_n$.
\end{defn}

\begin{lem}
 Ist $A$ abzählbar, so wird $\bigotimes_{\alpha \in A} \meas_\alpha$ durch
 \[ \left\{ \prod_{\alpha \in A} E_\alpha : E_\alpha \in \meas_\alpha \right\} \]
 erzeugt.
\end{lem}

\begin{proof}
 Wenn $E_\alpha \in \meas_\alpha$, dann 
 \[ \Pi_\alpha^{-1}(E_\alpha) = \prod_{\beta \in A} E_\beta, \]
 wobei $E_\beta = X_\beta$, $\beta \ne \alpha$. Andererseits gilt
 \[ \prod_{\alpha \in A} E_\alpha = \bigcap_{\alpha \in A} \Pi^{-1}_\alpha (E_\alpha). \]
 Das ist ein abzählbarer Durchschnitt nach Voraussetzung für $A$. Die Aussage folgt deshalb aus Lemma 1.3.6. Wir können die Erzeuger der einen $\sigma$-Algebra durch die Erzeuger der anderen ausdrücken. Folglich müssen die erzeugten $\sigma$-Algebren identisch sein.
\end{proof}

\begin{lem}
 Sei $\mE_\alpha$ Erzeuger für $\meas_\alpha$, $\alpha \in A$. Dann ist
 \[ \mF_1 := \{ \Pi_\alpha^{-1} (E_\alpha) : E_\alpha \in \mE_\alpha, \alpha \in A \} \]
 Erzeuger für $\bigotimes_{\alpha \in A} \meas_\alpha$. Ist $A$ abzählbar und $X_\alpha \in \mE_\alpha$, so ist auch
 \[ \mF_2 := \left\{ \prod_{\alpha \in A} E_\alpha : E_\alpha \in \mE_\alpha \right\} \]
 Erzeuger für $\bigotimes_{\alpha \in A} \meas_\alpha$.
\end{lem}

\begin{proof}
 Offensichtlich gilt $\sigma(\mF_1) \subset \bigotimes_{\alpha \in A} \meas_\alpha$, da $\mE_\alpha \subset \meas_\alpha$. Andererseits ist 
 \[ \{ E \subset X_\alpha : \Pi_\alpha^{-1} (E) \in \sigma( \mF_1 ) \} \]
 eine $\sigma$-Algebra (Nachweis ist Übungsaufgabe), die $\mE_\alpha$ und damit auch $\meas_\alpha$ enthält. Das heißt $\Pi_\alpha^{-1}( E ) \in \sigma ( \mF_1 )$ für alle $E \in \meas_\alpha$, $\alpha \in A$, also
 \[ \bigotimes_{\alpha \in A} \meas_\alpha \subset \sigma ( \mF_1 ). \]
 
 Die zweite Aussage folgt aus der ersten genauso wie in Lemma 1.9.2.
\end{proof}

\begin{defn*}
 Seien $X_1, \ldots, X_n$ metrische Räume und $X = \prod_1^n X_j$. Für $x = (x_1, \ldots, x_n) \in X$, $y = (y_1, \ldots, y_n) \in X$ ist $x_j, y_j \in X_j$. Also existiert eine Metrik $d_j(x_j,y_j)$, die für jede Komponente eine andere sein kann, da $X$ ein Produkt metrischer Räume ist. Wir definieren die \emph{Produktmetrik}
 \[ d(x,y) := \max_{1 \le j \le n} d_j(x_j,y_j), \]
 also der größte Abstand von $x$ und $y$.
\end{defn*}
 
\begin{thm}
 Seien $X_1, \ldots, X_n$ metrische Räume und $X = \prod_1^n X_j$ der metrische Raum mit der Produktmetrik. Dann ist $\bigotimes_1^n \borel(X_j) \subset \borel(X)$. Sind die $X_j$ separabel\footnotemark, so gilt 
 \[ \bigotimes_1^n \borel(X_j) = \borel(X). \]
\end{thm}
\footnotetext{Das heißt es existiert eine höchstens abzählbare Teilmenge, die in diesen Räumen dicht liegt. Zum Beispiel ist $\real^n$ für alle $n \in \nat$ separabel, da $\rat^n$ abzählbar ist und dicht in $\real^n$ liegt. Allerdings ist $\real^n$ mit der diskreten Metrik nicht separabel!}

\begin{proof}
 Nach Lemma 1.9.3 wird $\bigotimes_1^n \borel(X_j)$ erzeugt durch die Mengen $\Pi_j^{-1}( U_j )$, $1 \le j \le n$, wobei $U_j \subset X_j$ offen ist. Da diese Mengen offen sind, folgt nach Lemma 1.3.6, dass $\bigotimes_1^n \borel(X_j) \subset \borel(X)$.
 
 Wir beweisen noch die zweite Aussage. Sei $D_j := \{ x_j^k \}_{k=1}^\infty$ eine dichte Teilmenge von $X_j$ und sei $\mE_j$ die (abzählbare) Familie aller Kugeln mit rationalem Radius und Mittelpunkt $x_j^k$ für ein $j$ und $k$. Dann ist jede offene Menge in $X_j$ eine abzählbare Vereinigung von Mengen aus $\mE_j$. Die Menge aller Punkte in $X$ mit Koordinaten aus $\bigcup D_j$ ist abzählbar und dicht in $X$ (Nachweis Übung). $\mE_j$ ist ein Erzeuger für $\borel(X_j)$ und nach Lemma 1.9.3 gilt $\bigotimes_1^n \borel(X_j) = \borel(X)$.
\end{proof}

\begin{folg}
 $\bigotimes_1^d \borel(\real) = \borel(\real^d)$.
\end{folg}

\section{Aufgaben}
Siehe \verb+Aufgaben-1-10.pdf+.

\section{Borel-Maße auf \texorpdfstring{$\real$}{IR}}
Ein Maß auf der Borel-$\sigma$-Algebra $\borel(X)$, wobei $X$ ein metrischer (oder topologischer) Raum ist, heißt \emph{Borel-Maß} (auf $X$).

In diesem Abschnitt werden wir Borel-Maße auf $\real$ mit Hilfe von reellen Funktionen konstruieren. Dabei werden wir zuerst das Maß von Intervallen festlegen. Intervalle der Form 
\[ (a,b], (a, \infty), \emptyset \]
wobei $- \infty \le a < b < \infty$ werden wir als H-Intervalle\footnote{``H'' für \emph{halboffen}.} bezeichnen. Der Durchschnitt von H-Intervallen ist ein H-Intervall und das Komplement eines H-Intervalls ist die disjunkte Vereinigung von zwei H-Intervallen, wobei eines der beiden auch leer sein kann.

Wir bezeichnen mit $\mA_H$ die Familie aller endlichen, disjunkten Vereinigungen von H-Intervallen. Nach dem folgenden Lemma ist $\mA_H$ eine Algebra.

\begin{lem}
 Sei $X \ne \emptyset$ eine beliebige Menge, $\mE \in \pot(X)$ mit
 \begin{enumerate}[(i)]
  \item $\emptyset \in \mE$,
  \item $E,F \in \mE \Rightarrow E \cap F \in \mE$,
  \item $E \in \mE \Rightarrow E^C$ ist eine Vereinigung von endlich vielen disjunkten Mengen aus $\mE$
 \end{enumerate}
und bezeichne $\mA$ die Familie aller endlichen, disjunkten Vereinigungen von Mengen aus $\mE$. Dann ist $\mA$ eine Algebra.
\end{lem}

\begin{proof}
 Wir betrachten nur den Spezialfall, dass $E^C$ die Vereinigung von genau zwei
 disjunkten Mengen ist. Der allgemeine Fall lässt sich analog behandeln.
 
 Wir zeigen zuerst: $\mA$ ist abgeschlossen bezüglich endlicher Vereinigungen.
 Sei $A,B \in \mE$ und $B^C = C_1 \cup C_2$, wobei $C_1, C_2 \in \mE$ disjunkt
 sind. Dann
 \[ A \setminus B = A \cap B^C = A \cap( C_1 \cup C_2 ) = (A \cap C_1) \cup (A \cap C_2) \in \mA, \]
 da $(A \cap C_1), (A \cap C_2) \in \mE$ nach Voraussetzung (ii) und Definition
 von $\mA$ als Familie von Vereinigungen aus $\mE$. Also ist
 \[ A \cup B = \underbrace{(A \setminus B)}_{\in \mA} \cup \underbrace{B}_{\in \mE} \in \mA \]
 nach Definition von $\mA$. Wegen
 \[ \bigcup_{j=1}^n A_j = \left( \bigcup_{j=1}^{n-1}( A_j \setminus A_n ) \right) \cup A_n \]
 folgt durch Induktion über $n$, dass $\bigcup_1^n A_j \in \mA$, wenn $A_j \in
 \mE$, also ist $\mA$ abgeschlossen bezüglich endlicher Vereinigungen. 
 
 Noch zu zeigen: $\mA$ ist abgeschlossen bezüglich Komplementbildung. Sei $A =
 \bigcup_1^n A_j$, wobei $A_j \in \mE$ und die $A_j$ seien disjunkt. Sei
 weiterhin $A_j^C = B_j^1 \cup B_j^2, \quad B_j^1, B_j^2 \in \mE$ disjunkt. 
 \[ A^C = \bigcap_{j=1}^n A_j^C = \bigcap_{j=1}^n (B_j^1 \cup B_j^2) \overset{\footnotemark}{=} \bigcup \{ B_1^{k_1} \cap \cdots \cap B_n^{k_n} : k_1, \ldots, k_n = 1,2 \} \in \mA, \]
 da $B_1^{k_1} \cap \cdots \cap B_n^{k_n} \in \mE$ und ihre Vereinigung disjunkt
 ist.
 \footnotetext{``Ausmultiplizieren'', es ergibt sich die Vereinigung aller Kombinationen von $B_j^1, B_j^2$.}
 
 Also ist $\mA$ eine Algebra.
\end{proof}

\begin{lem}
 Sei $F: \real \to \real$ monoton wachsend und rechtsstetig\footnotemark. Sind
 $(a_j, b_j]$, $j= 1, \ldots, n$, disjunkte H-Intervalle, so sei
 \[ \mu \left( \bigcup_{j=1}^n (a_j, b_j] \right) := \sum_{j=1}^n [ F(b_j) - F(a_j) ] \]
 und $\mu( \emptyset ) := 0$. Dann ist $\mu$ ein Prämaß auf $\mA_H$.
\end{lem} 
\footnotetext{Das heißt: Für den rechsseitigen Grenzwert gilt $\lim_{h \downarrow 0} f(x + h) = f(x)$.}

Zum Beispiel $F(x) = x$, $\mu( (0,1] ) = 1$, $\mu( (a,b] ) = b - a$;
Vereinbarung $F( \pm \infty ) := \lim_{x \to \pm \infty} F(x)$. Dann ist $-
\infty \le F(-\infty) \le F(\infty) \le \infty$. 

\begin{proof}
 Zuerst zeigen wir, dass die obige Definition korrekt ist (die Elemente von
 $\mA_H$ lassen sich auf unterschiedliche Weise als disjunkte Vereinigung von
 H-Intervallen darstellen). Sei $(a,b] := \bigcup_1^n (a_j, b_j] \in \mA_H$,
 wobei die Intervalle $(a_j, b_j]$ disjunkt sind. O.B.d.A. sei $a = a_1 < b_1 =
 a_2 < b_2 = \ldots < b_n = b$. Dann ist 
 \[ \sum_{j=1}^n [ F( b_j ) - F(a_j) ] = F(b) - F(a), \]
 also unabhängig von der Darstellung. 
 
 Allgemeiner: Sind $\{ I_i \}_1^n$ und $\{ J_j \}_1^m$ Folgen von disjunkten
 H-Intervallen mit $\bigcup_1^n I_i = \bigcup_1^m J_j$, so gilt
 \[ I_i = I_i \cap \left( \bigcup_{j=1}^m J_j \right) = \bigcup_{j=1}^m ( I_i \cap J_j ). \]
 Daraus folgt
 \[ \sum_{i=1}^n \mu(I_i) = \sum_{i=1}^n \sum_{j=1}^m \mu( I_i \cap J_j ) = \sum_{j=1}^m \sum_{i=1}^n \mu( I_i \cap J_j ) =  \sum_{j=1}^m \mu(J_j). \]
 Also ist die Definition von $\mu$ korrekt.
 
 Noch zu zeigen:
 \[ \mu \left( \bigcup_{j=1}^\infty I_j \right) = \sum_{j=1}^\infty \mu(I_j) \]
 für eine beliebige Folge $\{ I_j \}_1^\infty$ von disjunkten H-Intervallen $I_j
 = (a_j, b_j]$ mit $\bigcup_1^\infty I_j \in \mA_H$.
 
 Wegen der endlichen Additivität von $\mA_H$ dürfen wir annehmen, dass
 $\bigcup_1^\infty I_j = (a,b] =: I$. Aus der Definition von $\mu$ folgt dann
 \[ \mu(I) = \mu \left( \bigcup_{j=1}^n I_j \right) + \mu \left( I \setminus \bigcup_{j=1}^n I_j \right)
    \ge \mu\left( \bigcup_{j=1}^n I_j \right) = \sum_{j=1}^n \mu( I_j ), n \in \nat. \]
 Mit $n \to \infty$ erhalten wir $\mu(I) \ge \sum_1^\infty \mu( I_j )$. 
 
 Wir zeigen noch die umgekehrte Ungleichung. Sei zuerst $- \infty < a < b <
 \infty $ und $\eps > 0$. Da $F$ rechtsstetig ist, existieren $\delta, \delta_j
 > 0$ mit
 \begin{align*}
    F( a + \delta ) - F(a) &< \eps, \tag{$*$} \\
    F(b_j + \delta_j) - F(b_j) &< \frac{\eps}{2^j}. \tag{$**$}
 \end{align*}
 Die offenen Intervalle $(a_j, b_j + \delta_j)$ überdecken die kompakte Menge
 $[a + \delta, b]$. Wir wählen eine \emph{endliche} Überdeckung.
 
 Wir lassen dabei Intervalle weg, die in einem größeren Intervall enthalten
 sind. Mit geeigneter Nummerierung dürfen wir annehmen, dass
 \begin{enumerate}[(1)]
  \item $(a_1, b_1 + \delta_1), \ldots, (a_N, b_N + \delta_N)$ das Intervall $[a+\delta,b]$ überdecken,
  \item $a_1 < \ldots < a_2$,
  \item $b_j + \delta_j \in (a_{j+1}, b_{j+1} + \delta_{j+1})$ für $j= 1, \ldots, N$.
 \end{enumerate}
 Dann gilt
 \begin{align*}
  \mu(I) &= F(b) - F(a) & &\text{Def. von $\mu$} \\
         &\le F(b) - F(a+\delta) + \eps & &\text{Wegen } (*) \\
         &\le F(b_N + \delta_N) - F( a_1 ) + \eps & &\text{Monotonie und (3)} \\
         &= F(b_N + \delta_N) - F( a_N ) + \sum_{j=1}^{N-1} (F(a_{j-1}) - F(a_j)) + \eps \\
         &\le F(b_N + \delta_N) - F( a_N ) + \sum_{j=1}^{N-1} (F(b_j + \delta_j) - F(a_j)) + \eps & &\text{Monotonie und (3)} \\
         &=\sum_{j=1}^N (F(b_j + \delta_j) - F(a_j)) + \eps \\
         &=\sum_{j=1}^N (\underbrace{F(b_j + \delta_j) - F(b_j)}_{< \eps/2^j} + \underbrace{F(b_j) - F(a_j)}_{=\mu(I_j)}) + \eps & &\text{Wegen } (**)\\
         &\le \sum_{j=1}^N \mu(I_j) + 2\eps.
 \end{align*}
 Da $\eps$ beliebig war, folgt die Behauptung.

 Im Falle $a= -\infty$ liefert dieselbe Überlegung, dass 
 \[ F(b) - F(-M) \le \sum_{j=1}^\infty \mu(I_j) + 2 \eps \]
 für alle $M < \infty$. Im Falle $b = \infty$ erhalten wir analog
 \[ F(M) - F(a) \le \sum_{j=1}^\infty \mu(I_j) + 2 \eps. \]
 Die Aussage folgt nun mit $\eps \to 0$ und $M \to \infty$.
\end{proof}

\begin{thm}
 Für jede monoton wachsende und rechtsstetige Funktion $F: \real \to \real$
 existiert ein \emph{eindeutig bestimmtes} Maß $\mu_F$ auf $\borel(\real)$ mit
 $\mu_F( (a,b] ) = F(b) - F(a)$, $a,b \in \real$. Ist $G$ eine weitere solche
 Funktion, so gilt $\mu_F = \mu_G$ genau dann, wenn $F-G$ konstant ist.
 
 Sei umgekehrt $\mu$ ein Maß auf $\borel(\real)$, das endlich ist auf
 beschränkten Borel-Mengen. Dann ist die Funktion $F$ mit $F(x) = \mu((0,x))$
 für $x>0$, $F(0) = 0$ und $F(x) = - \mu((x,0))$ für $x<0$ monoton wachsend,
 rechtsstetig und $\mu = \mu_F$.
\end{thm}

\begin{proof}
 Nach Lemma 1.11.2 definiert $F$ ein Prämaß auf $\mA_H$. Es ist klar, dass $F$
 und $G$ genau dann das selbe Prämaß definieren, wenn $F-G$ konstant ist. Wegen
 \[ \real = \bigcup_{j \in \integer} (j, j+1] \]
 sind diese Maße $\sigma$-endlich. Die ersten zwei Aussagen folgen deshalb aus
 Satz 1.7.7.
 
 \emph{Dritte Aussage:} Die Monotonie von $F$ folgt aus der Monotonie von $\mu$,
 aus der Stetigkeit von oben von $\mu$ folgt die Rechsstetigkeit von $F$. Es ist
 klar, dass \fbox{$\mu = \mu_F$ auf $\mA_H$}. Nach der Eindeutigkeitsaussage von
 Satz 1.7.7 gilt \fbox{$\mu = \mu_F$ auf $\borel(\real)$}.
\end{proof}

\begin{rmrk}
 Wir hätten auch Intervalle der Form $[a,b)$ verwenden können, dann wäre
 $\tilde{F}$ linksstetig.
 
 Ist $\mu$ ein endliches Maß auf $\borel(\real)$, so gilt $\mu = \mu_F$ mit
 $F(x) = \mu((-\infty, x])$. 
 
 Zu jeder rechsstetigen monotonen Funktion $F$ gehört ein vollständiges Maß
 $\obar{\mu}_F$, dessen Definitionsbereich im Allgemeinen größer ist als
 $\borel(\real)$. Wir werden dieses Maß auch mit $\mu_F$ bezeichnen, es heißt
 das zu $F$ gehörige \emph{Lebesgue-Stieltjes-Maß}.
\end{rmrk}

\textbf{Das ``wichtigste'' Maß auf $\real$:} Das Lebesgue-Maß. Das ist das
vollständige Maß, das zu $F(x) = x$ gehört. Es wird mit $\lambda$ bezeichnet.
Für $a,b \in \real$, $a \le b$ ist 
\[ \begin{aligned}
    \lambda( [a,b) ) &= b - a, \\ 
    \lambda( (a,b) ) &= \lambda( [a,b] ) = b-a,
   \end{aligned} \]
da $\lambda(\{x\}) = 0$ für alle $x \in \real$, $\{ x \} \subset (x - \rez{n}, x
+ \rez{n}]$, also $\mu( \{ x \} ) \le \frac{2}{n} \xrightarrow{n \to \infty} 0$.
Das gilt auch für $\mu_F$, wenn $F$ in 0 stetig ist.

Die Elemente von $\meas_\lambda$ heißen \emph{Lebesgue-messbar}. Die
Einschränkung von $\lambda$ auf $\borel(\real)$ wird auch Lebesgue-Maß genannt.

Im Weiteren in diesem Abschnitt:
\begin{itemize}
 \item $F: \real \to \real$: Eine monoton wachsende, rechtsstetige Funktion,
 \item $\mu$: Das zugehörige Lebesgue-Stieltjes-Maß,
 \item $\meas$: Die zugehörige $\sigma$-Algebra.
\end{itemize}

Für $E \in \meas$ gilt (Definition von $\mu^*$):
\[ \begin{aligned} \mu^*(E) 
    :&= \inf \left\{ \sum_{j=1}^\infty [ F(b_j) - F(a_j) ] : E \subset \bigcup_{j=1}^\infty (a_j,b_j] \right\} \\
     &= \inf \left\{ \sum_{j=1}^\infty \mu((a_j,b_j]) : E \subset \bigcup_{j=1}^\infty (a_j,b_j] \right\}
   \end{aligned} \]
H-Intervalle können hier durch offene Intervalle ersetzt werden.

\begin{lem}
 Für $E \in \meas$ gilt:
 \[ \mu^*(E) = \inf \left\{ \sum_{j=1}^\infty \mu((a_j,b_j)) : E \subset \bigcup_{j=1}^\infty (a_j,b_j) \right\}. \]
\end{lem}

\begin{proof}
 Bezeichne $\nu(E)$ die rechte Seite und sei $E \subset \bigcup_1^\infty
 (a_j,b_j)$. Wir schreiben $\ell_j := b_j - a_j$ und $I_{jk} := (b_j -\ell_j
 2^{1-k}, b_j - \ell_j 2^{-k}]$, $k \in \nat$. Dann ist $(a_j,b_j) =
 \bigcup_1^\infty I_{jk}$ und damit
 \[ E \subset \bigcup_{j,k} I_{jk} \quad \text{und} \quad \sum_{j=1}^\infty \mu( (a_j, b_j) ) = \sum_{j,k=1}^\infty \mu(I_{jk}) \ge \mu(E), \]
 also $\nu(E) \ge \mu(E)$.
 
 Wir zeigen nun noch, dass $\nu(E) \le \mu(E)$. Für beliebiges $\eps > 0$
 existiert $\{ (a_j, b_j] \}_1^\infty$ mit $E \subset \bigcup_1^\infty
 (a_j,b_j]$ und nach Definition von $\mu(E)$ gilt
 \[ \sum_{j=1}^\infty \mu((a_j,b_j]) \le \mu(E) + \eps. \]
 Für jedes $j$ existiert $\delta_j > 0$ mit
 \[ F( b_j + \delta_j ) - F(b_j) < \frac{\eps}{2^j} \]
 wegen der Rechsstetigkeit und Monotonie von $F$. Dann ist $E \subset
 \bigcup_1^\infty (a_j, b_j + \delta_j )$ und
 \[ \begin{aligned}
     \sum_{j=1}^\infty \mu((a_j,b_j+\delta_j)) 
     &\le \sum_{j=1}^\infty \mu((a_j,b_j+\delta_j]) \\
     &=   \sum_{j=1}^\infty ( F(b_j + \delta_j) - F(a_j) ) \\
     &=   \sum_{j=1}^\infty ( \underbrace{F(b_j + \delta_j) - F(b_j)}_{< \eps/2^j} + F(b_j) - F(a_j) ) \\
     &\le \eps + \mu(E) + \eps.
    \end{aligned} \]
 Da $\eps$ beliebig, folgt die Behauptung.
\end{proof}

\begin{thm}
 Für $E \in \meas$ gilt
 \[ \begin{aligned}
    \mu(E) &= \inf \left\{ \mu(U) : E \subset U, U\, \text{offen} \right\} \\
           &= \sup \left\{ \mu(K) : K \subset E, K\, \text{kompakt} \right\} 
    \end{aligned} \]
\end{thm}

\begin{proof}
 \textbf{Erste Gleichung.} Ist $U$ offen und $U \supset E$, so gilt $\mu(U) \ge
 \mu(E)$. Andererseits ist $U$ die Vereinigung abzählbar vieler Intervalle
 $(a_j,b_j)$ mit $\mu(U) \le \sum_{j=1}^\infty \mu((a_j, b_j))$. Die erste
 Gleichung folgt deshalb aus Lemma 1.11.5.
 
 \textbf{Zweite Gleichung.} Sei $E$ zuerst beschränkt. Für beliebige $\eps > 0$
 existiert nach der ersten Gleichung eine offene Menge $U \supset \obar{E}
 \setminus E$, die den Abschluss\footnotemark von $E$ enthält und
 \[ \mu(U) \le \mu( \obar{E} \setminus E ) + \eps. \]
 Nach Voraussetzung ist $E \in \meas$. $\obar{E}$ ist eine Borel-Menge, also ist
 $\obar{E} \setminus E \in \meas$. Die Menge $K := \obar{E} \setminus U \subset
 E$ ist kompakt und
 \[ \begin{aligned}
     \mu(K) 
     &= \mu(E) - \mu(E \cap U) \\
     &= \mu(E) - ( \mu(U) - \mu(U \setminus E) ) \\
     &\ge \mu(E) - \mu(U) - \mu(\obar{E} \setminus E) \\
     &\ge \mu(E) - \eps.
    \end{aligned} \]
 Damit ist die Gleichung für beschränkte $E$ gezeigt. Sei nun $E$ unbeschränkt
 und sei $E_j := E \cap (j,j+1)$, $j \in \integer$. Die $E_j$ sind disjunkt und
 für jedes $\eps > 0$ existiert eine kompakte Menge $K_j \subset E_j$ mit
 \[ \mu(K_j) \ge \mu(E_j) - \frac{\eps}{2^{|j|}}. \]
 Die Menge $H_n := \bigcup_{-n}^n K_j \subset E$ ist kompakt und
 \[ \mu( H_n ) \ge \mu \left( \bigcup_{j=-n}^n E_j \right) - 3 \eps. \]
 Die Aussage folgt nun aus
 \[ \mu(E) = \lim_{n \to \infty} \mu \left( \bigcup_{j=-n}^n E_j \right). \qedhere\]
\end{proof}
\footnotetext{Abschluss von $E$: Durchschnitt aller abgeschlossenen Mengen, die $E$ enthalten; $\obar{E} \setminus E = \partial E$.}

Für $r,s \in \real$, $E \subset \real$ ist $rE := \{ r\cdot e : e \in E \}$ und $E+s:= \{ e+s : e \in E \}$.

\begin{thm}
 Für beliebige $r,s \in \real$ und $E \subset \real$ gilt
 \begin{align*}
  \lambda^* (E+s) &= \lambda^* (E), & \lambda^* (rE) &= |r| \lambda^* (E).
 \end{align*}
 Ist $E \subset \real$ $\lebesgue$-messbar, so sind die Mengen $E+s$ und $rE$ ebenfalls $\lebesgue$-messbar und
 \begin{align*}
  \lambda (E+s) &= \lambda (E), & \lambda(rE) &= |r| \lambda(E).
 \end{align*}
\end{thm}

\begin{rmrk}
 \item $\lambda(E) = 0$ wenn $E$ abzählbar ist. 
 
 \textbf{Konstruktion einer Nullmenge mit der Mächtigkeit des Kontinuums.} Jede Zahl $x \in [0,1]$ lässt sich darstellen als
 \[ x = \sum_{j=1}^\infty a_j \cdot \rez{3^j}, \quad \text{wobei } a_j \in \{ 0,1,2 \}. \]
 Diese Darstellung ist nicht ganz eindeutig, weil $\cdots 0 2 2 2 \cdots = \cdots 1 0 0 0 \cdots$. Daher vereinbaren wir, dass $\cdots 1 0 0 0 \cdots$ ausgeschlossen wird.
 
 Die Cantorsche Menge\footnote{\url{https://de.wikipedia.org/wiki/Cantor-Menge}} $C$ ist die Menge aller $x \in [0,1]$ mit $a_j \ne 1$ für alle $j$. Sie hat die Eigenschaften
 \begin{enumerate}[(a)]
  \item $C$ ist kompakt, weil $C$ beschränkt und abgeschlossen ist (Durchschnitt abgeschlossener Mengen)
  \item Wenn $x,y \in C$ mit $x < y$, dann existiert $z \notin C$ mit $x<z<y$. Also ist $C$ \emph{total unzusammenhängend}\footnote{Ist $X$ ein metrischer Raum, dann ist $X$ \emph{zusammenhängend}, wenn $X = U \cup V$, $U \cap V = \emptyset$, $U,V$ offen folgt, dass entweder $U = \emptyset$ oder $V = \emptyset$. $X$ heißt \emph{total unzusammenhängend}, wenn es neben der leeren und den einelementigen Teilmengen keine weiteren zusammenhängenden Teilmengen gibt.} (vollständig disjunkt) und \emph{nirgends dicht}\footnote{Eine Menge $E \subset X$ heißt \emph{nirgends dicht}, wenn das Innere ihres Abschlusses leer ist, $(\obar{E})^\circ = \emptyset$.}.
  \item $C$ hat keine isolierten Punkte.
  \item $\lambda(C) = 0$.
  \item $C$ hat die Mächtigkeit des Kontinuums.
 \end{enumerate}
 Hinweis: $C = \{ x \in [0,1] : a_j \ne 1$ für alle $j \}$. 
\end{rmrk}

\section{Aufgaben}
siehe \verb+Aufgaben-1-12.pdf+. 

%%% Local Variables:
%%% TeX-master: "skript_mint"
%%% End:


\clearpage

\chapter{Integrale}
\section{Messbare Abbildungen}
Zu jeder Abbildung $f:X \to Y$ zwischen zwei Mengen gehört eine \emph{inverse Abbildung} $f^{-1}:\pot(Y) \to \pot(X)$ definiert durch $f^{-1}(E) = \{ x \in X : f(x) \in E \}.$ Die inverse Abbildung kommutiert mit Vereinigung, Durchschnitt und Komplement:
\begin{equation}
 \begin{aligned}
 f^{-1} \left( \bigcap_\alpha E_\alpha \right) &= \bigcap_\alpha f^{-1}(E_\alpha) \\
 f^{-1} \left( \bigcup_\alpha E_\alpha \right) &= \bigcap_\alpha f^{-1}(E_\alpha) \\
 f^{-1} ( E^C ) &= (f^{-1}(E) )^C
 \end{aligned}
 \tag{$\ast$}
\end{equation}

\begin{lem}
 Ist $\mN \subset \pot(Y)$ eine $\sigma$-Algebra, so ist auch
 \[ f^{-1}(\mN) := \{ f^{-1}(E) : E \in \mN \} \]
 eine $\sigma$-Algebra.
\end{lem}

\begin{proof}
 Das folgt aus den drei Gleichungen ($\ast$).
\end{proof}

\begin{defn}
 Sind $(X,\mM)$ und $(Y,\mN)$ Messräume, so heißt eine Abbildung $f: X \to Y$ \emph{$(\mM,\mN)$-messbar}\footnote{Das heißt $f^{-1}(\mN) \subset \mM$.} (oder einfach messbar), wenn $f^{-1}(E) \in \mM$, $E \in \mN$.
 
 \emph{Vergleich zur Stetigkeit:} Urbilder von offenen Mengen sind offen.
\end{defn}

\begin{lem}
 Sind $(X,\mM)$, $(Y,\mN)$ und $(Z,\mO)$ Messräume und $f:X \to Y$, $g: Y \to Z$ $(\mM,\mN)$-messbare bzw. $(\mN,\mO)$-messbare Abbildungen, so ist $f \circ g = g(f)$ $(\mM,\mO)$-messbar.
\end{lem}

\begin{proof}
 Einfach.
\end{proof}

\clearpage

\begin{lem}
 Ist $\mE$ ein Erzeuger für die $\sigma$-Algebra $\mN$, so ist $f$ genau dann $(\mM,\mN)$-messbar, wenn $f^{-1}(\mE) \subset \mM$.
\end{lem}

\begin{proof}
 Die eine Richtung folgt aus der Definition der Messbarkeit von $f$. Nehmen wir an, dass $f^{-1}(\mE) \subset \mM$. Die Familie $\{ E \subset Y : f^{-1}(E) \in \mM \}$ ist wegen ($\ast$) eine $\sigma$-Algebra, die $\mE$ enthält, folglich enthält sie auch $\mN$.
\end{proof}

\begin{folg}
 Sind $X$ und $Y$ metrische Räume, so ist jede stetige Abbildung $f: X \to Y$ $(\borel(X),\borel(Y))$-messbar.
\end{folg}

\begin{proof}
 $f$ stetig $\Leftrightarrow$ $f^{-1}(U)$ offen für beliebige offene Menge $U \subset Y$. Die Behauptung folgt deshalb aus Lemma 2.1.4.
\end{proof}

Sei $(X,\mM)$ ein Messraum. Eine \emph{reellwertige} oder \emph{komplexwertige} Funktion $f$ auf $X$ heißt $\mM$-messbar (oder einfach messbar), wenn sie $(\mM,\borel(\real))$- oder $(\mM,\borel(\complex))$-messbar ist. Standardmäßig betrachten wir also die Borelmengen im Wertebereich.

\textbf{Spezialfälle.}
\begin{itemize}
 \item $f: \real \to \real$ heißt \emph{Lebesgue}-messbar, wenn sie $(\mM_\lambda, \borel(\real))$-messbar ist.
 \item $f: \real \to \real$ heißt \emph{Borel}-messbar, wenn sie $(\borel(\real), \borel(\real))$-messbar ist.
 \item Analog für komplexwertige Funktionen.
\end{itemize}

\textbf{Warnung.}
Sind $f,g: \real \to \real$ Lebesgue-messbar, so braucht $g(f)$ nicht Lebesgue-messbar zu sein\footnote{Lemma 2.1.3 ist hier aufgrund der Definition der Lebesgue-Messbarkeit nicht anwendbar.}!

\begin{lem}
 Sei $f$ eine rechtsstetige Funktion auf dem Messraum $(X,\mM)$. Die folgenden Aussagen sind äquivalent:
 \begin{enumerate}[(i)]
  \item $f$ ist $\mM$-messbar (genauer $(\mM,\borel(\real))$-messbar).
  \item $f^{-1}((a,\infty)) \in \mM$, $a \in \rat$.
  \item $f^{-1}((a,\infty)) \in \mM$, $a \in \rat$.
  \item $f^{-1}([a,\infty)) \in \mM$, $a \in \rat$.
  \item $f^{-1}((-\infty,a)) \in \mM$, $a \in \rat$.
  \item $f^{-1}((-\infty,a]) \in \mM$, $a \in \rat$.
 \end{enumerate}
\end{lem}

\begin{proof}
 Aufgabe (Folgt aus Lemma 2.1.4 und Aufgabe 1.4.4).
\end{proof}

\clearpage

\begin{lem}
 Seien $(X,\mM)$ und $(Y_\alpha, \mN_\alpha)$, $\alpha \in A$, Messräume, $Y := \prod_{\alpha \in A} Y_\alpha$ und $\mN := \bigotimes_{\alpha \in A} \mN_\alpha$.
 \begin{enumerate}[(i)]
  \item Die Koordinatenprojektionen $\Pi_\alpha: Y \to Y_\alpha$ sind $(\mN,\mN_\alpha)$-messbar.
  \item Eine Abbildung $f: X \to Y$ ist genau dann $(\mM,\mN)$-messbar, wenn die Abbildung $f_\alpha := \Pi_\alpha(f)$ $(\mM,\mN_\alpha)$-messbar ist für alle $\alpha$.
 \end{enumerate}
\end{lem}

Zum Beispiel $Y = Y_1 \times Y_2$, $f(x) = ( f_1(x), f_2(x) )$, $\Pi_1(f(x)) = f_1(x)$.

\begin{proof}
 (i) folgt aus der Definition von $\mN$ und Lemma 2.1.4.
 \[ \Pi^{-1}_\alpha ( N_\alpha ) = \prod_{\beta \in A} Z_\beta \in \mN, \quad \text{für alle } N_\alpha \in \mN_\alpha \]
 $Z_\alpha = N_\alpha$, $Z_\beta = Y_\beta$ für $\beta \ne \alpha$.
 (ii) Ist $f$ messbar, so ist auch $f_\alpha$ messbar nach Lemma 2.1.3. Ist umgekehrt $f_\alpha$ messbar für alle $\alpha$, so gilt
 \[ f^{-1}( \Pi_\alpha^{-1}( E_\alpha ) ) = f_\alpha^{-1} (E_\alpha) \in \mM, \quad \text{für alle } E_\alpha \in \mN_\alpha. \]
 Nach Lemma 2.1.4 ist $f$ messbar.
\end{proof}

\begin{folg}
 $f: X \to \complex$ ist genau dann $\mM$-messbar, wenn $\Re f$ und $\Im f$ messbar sind.
\end{folg}

\begin{proof}
 $\complex = \real \times \real$, also ist $\borel(\complex) = \borel( \real \times \real ) = \borel( \real ) \otimes \borel(\real)$ nach 1.9.5.
\end{proof}

\begin{folg}
 Seien $f_1, \ldots, f_d$ reell- oder komplexwertige Funktionen auf $X$. Die Abbildung $f := (f_1, \ldots, f_d) \in \real^d$ bzw. $\complex^d$ ist genau dann Borel-messbar, wenn $f_1, \ldots, f_d$ Borel-messbar sind.
\end{folg}

\begin{proof}
 Das folgt auch aus 1.9.5.
\end{proof}

\begin{folg}
 Sind $f,g: X \to \complex$ $\mM$-messbar, so sind auch $f+g$, $fg$ und $f/g$ (falls definiert) $\mM$-messbar.
\end{folg}

\begin{proof}
 $F := (f,g)$ ist $(\mM,\borel(\complex \times \complex))$-messbar nach 2.1.9. $\psi(x,y) := x + y$, $x,y \in \complex$ ist stetig und damit Borel-messbar. Also ist $f+g = \psi(F)$ $\mM$-messbar.
 
 $fg$ und $f/g$ folgen analog.
\end{proof}

Es ist manchmal zweckmäßig, Funktionen zu betrachten, deren Werte aus
\[ \realext := [-\infty,\infty] := \real \cup \{-\infty, \infty\} \]
sind. Wir definieren die Borel-$\sigma$-Algebra in $\realext$ durch 
\[ \borel(\realext) := \{ E \subset \realext : E \cap \real \in \borel( \real ) \} \]
Es ist leicht zu sehen, dass die Familie der Mengen $(a, \infty]$ oder $[-\infty,a)$, $a \in \real$ (oder $\rat$), Erzeuger für $\borel(\realext)$ sind.

\begin{lem}
 Ist $\{f_n\}$ eine Folge messbarer, $\realext$-wertiger Funktionen auf $(X, \mM)$, so sind die Funktionen 
 \begin{align*}
  g_1(x) &= \sup_j f_j(x), & g_2(x) &= \inf_j f_j(x), \\
  g_3(x) &= \limsup_{j} f_j(x), & g_4(x) &= \liminf_j f_j(x)
 \end{align*}
 messbar.
 
 Existiert $f(x) = \lim_j f_j(x)$ für alle $x$, so ist auch $f$ messbar.
\end{lem}

\begin{proof}
 Es gilt
 \begin{align*}
   g_1^{-1}( (a, \infty] ) &= \bigcup_{j} f_j^{-1} ((a,\infty]), & g_2^{-1}([-\infty,a)) = \bigcup_j ([-\infty,a)).
 \end{align*}
 Nach 2.1.4 sind $g_1$ und $g_2$ messbar. Hieraus folgt, dass $h_k(x) := \sup_{j > k} f_j(x)$ messbar ist, also auch $g_3 = \inf_{k \ge 0} h_k(x)$. Analog für $g_4$.
 
 Wenn der Grenzwert $\lim_j f_j(x)$ existiert, dann ist $f = g_3 = g_4$ und damit $f$ messbar.
\end{proof}

\begin{folg}
 Sind $f,g : X \to \realext$ messbar, so auch $\max(f,g)$ und $\min(f,g)$.
\end{folg}

\begin{folg}
 Ist $\{f_i\}$ eine Folge komplexwertiger, messbarer Funktionen und existiert $\lim_i f_i(x) = f(x)$ für alle $x$, so ist $f$ messbar.
\end{folg}

\begin{proof}
 Das folgt aus 2.1.8.
\end{proof}

\begin{deno}
 \begin{itemize}
  \item Für eine Funktion $f: X \to \realext$ definieren wir den \emph{positiven} und den \emph{negativen} Teil von $f$ durch 
  \[ f^+ = \max( f(x), 0 ), \qquad f^- = \max( -f(x), 0 ). \]
  \item Ist $f$ messbar, so auch $f^+$ und $f^-$ (siehe Folgerung 2.1.12). 
  \item Ist $f:X \to \complex$, so gilt die \emph{Polarzerlegung}
  \[ f = \sgn f \cdot |f|, \]
  wobei $\sgn z = \frac{z}{|z|}$, $z \ne 0$ und $\sgn 0 = 0$.
  \item Ist $f$ messbar, so auch $\sgn(f)$ und $|f|$.
  \item Sei $X, \meas$ ein Messraum und $E \subset X$. Die Funktion $\ind_E$, die auf $E$ den Wert 1 und sonst den Wert 0 annimmt, heißt \emph{Indikatorfunktion} von $E$. Sie ist genau dann messbar, wenn $E \in \meas$.\footnote{Es gilt $\ind^{-1}( B ) \in \{ \emptyset, E, E^C, X \}$.}
  \item Eine \emph{einfache Funktion} (oder Treppenfunktion) $f$ auf $X$ ist eine endliche Linearkombination von Indikatorfunktionen von Mengen aus $\meas$: 
  \[ f = \sum_{j=1}^n c_j \ind_{E_j}, \quad c_j \in \complex, \quad E_j \in \meas. \]
  \item Summe und Produkt von einfachen Funktionen sind einfach.
 \end{itemize}
\end{deno}

\begin{thm}
 Sei $(X, \meas)$ ein Messraum.
 \begin{enumerate}[(i)]
  \item Ist $f: X \to [0, \infty]$ messbar, so existiert eine Folge $\{ \varphi_n \}$ von einfachen Funktionen mit $0 \le \varphi_1 \le \varphi_2 \le \ldots \le f$, $\varphi_n \to f$, punktweise und $\varphi_n \to f$ gleichmäßig auf jeder Menge, wo $f$ beschränkt ist.
  \item Ist $f: X \to \complex$ messbar, so existiert eine Folge $\{ \varphi_n \}$ von einfachen Funktionen mit $0 \le |\varphi_1| \le |\varphi_2| \le \ldots \le |f|$, $\varphi_n \to f$, punktweise und $\varphi_n \to f$ gleichmäßig auf jeder Menge, wo $f$ beschränkt ist.
 \end{enumerate}
\end{thm}

\begin{proof}
 \begin{enumerate}[(i)]
  \item Für $n \in \nat$, $0 \le k \le 2^{2n}-1$ seien $E_n^k := f^{-1} (( k \cdot 2^{-n} , (k+1) \cdot 2^{n-} ] )$, $F_n := f^{-1}((2^n, \infty])$, 
  \[ \varphi := \sum_{k=0}^{2^{2n} - 1} k \cdot 2^{-n} \ind_{E_n^k} + 2^n \ind_{F_n}. \]
  Es ist leicht zu sehen: Für alle $n \in \nat$ gilt: $0 \le f - \varphi_n \le 2^{-n}$ auf der Menge, wo $f \le 2^n$, $\varphi_n \le \varphi_{n+1}$.
  \item Sei $f = g + i \cdot h = (g^+ - g^-) + i( h^+ - h^- )$. Nach (i) wählen wir die entsprechenden Folgen $\psi_n^+$, $\psi_n^-$, $\chi_n^+$ und $\chi_n^-$. Dann setzen wir $\varphi_n := (\psi_n^+  - \psi_n^-) + i(\chi_n^+ - \chi_n^-)$. \qedhere
 \end{enumerate}
\end{proof}

\begin{lem}
 Sei $\mu$ ein vollständiges Maß, $f,f_n,g: X \to \real^d$ oder $\complex^d$.
 \begin{enumerate}[(i)]
  \item Ist $f$ messbar und $f=g$ fast überall, so ist $g$ messbar.
  \item Ist $f_n$, $n \in \nat$ messbar und $f_n \to f$ fast überall, so ist $f$ messbar.
 \end{enumerate}
\end{lem}

\textbf{Anmerkung.}
Ist $\mu$ nicht vollständig, so sind (i) und (ii) falsch!

\begin{proof}
 \begin{enumerate}[(i)]
  \item Es gilt
  \[ g^{-1}(B) = \Big( [g=b] \cap f^{-1}(B) \Big) \cup \Big( [g \ne b] \cap g^{-1}(B) \Big), \]
  die Menge $[g=b] \cap f^{-1}(B)$ ist messbar aufgrund der Voraussetzung und $[g \ne b] \cap g^{-1}(B)$ ist eine Nullmenge und damit messbar, weil $\mu$ vollständig ist.
  \item Verläuft ähnlich. Sei $N$ eine $\mu$-Nullmenge und $N_0 \subset N$ mit $N_0 \notin \meas_\mu$. Dann ist $g := \ind_{N_0} = 0 =: f$ fast überall, aber $\ind_{N_0}$ ist nicht messbar. \qedhere
 \end{enumerate}
\end{proof}

\section{Aufgaben}
Siehe \verb+Aufgaben-2-2-Teil-1.pdf+ und \verb+Aufgaben-2-2-Teil-2.pdf+.

\section{Integration nichtnegativer Funktionen}
In diesem Abschnitt ist $(X,\meas,\mu)$ ein Maßraum.

Die Menge $L^+$ ist die Menge aller messbaren Funktionen $f:X \to [0, \infty]$.

\begin{defn}
 Sei $\varphi \in L^+$ eine einfache Funktion mit Standard-Darstellung $\varphi = \sum_1^n a_j \ind_{E_j}$. Wir definieren das \emph{Integral} von $\varphi$ bezüglich $\mu$ durch
 \[ \int \varphi \diffop \mu := \sum_{j=1}^n a_j \mu( E_j ). \]
\end{defn}

\begin{rmrk*}
 \begin{itemize}
  \item $\int \varphi \diffop \mu = \infty$ ist möglich, zum Beispiel $\int \ind_\real \diffop \lambda = \infty$.
  \item Kurze Schreibweise: $\int \varphi$, wenn keine Verwechselungsgefahr.
  \item Angabe der Funktionsvariable: $\int \varphi(x) \diffop \mu(x)$, zum Beispiel wenn mehrere Variablen vorkommen:
  \[ \int \varphi( x_1, x_2, x_3 ) \diffop \mu(x_2). \]
  Auch üblich: $\int \varphi(x) \mu(\diffop x)$.
 \end{itemize}
\end{rmrk*}

Ist $A \in \meas$ und $\varphi$ wie oben, so ist 
\[ \varphi \cdot \ind_A = \sum_{j=1}^n a_j \ind_{E_j \cap A} \in L^+. \]
Wir schreiben
\[ \int_A \varphi \diffop \mu := \int \varphi \cdot \ind_A \diffop \mu \]
\emph{Integral über $A$}.

\begin{lem}
 Für einfache Funktionen $\varphi, \si \in L^+$ gilt:
 \begin{enumerate}[(i)]
  \item $\int c \varphi = c \int \varphi$, $c \ge 0$,
  \item $\int(\varphi + \psi) = \int \varphi + \int \psi$,
  \item wenn $\varphi \le \psi$, dann $\int \varphi \le \psi$,
  \item $A \to \int_A \varphi$ ist ein Maß auf $\meas$.
 \end{enumerate}
\end{lem}

\begin{proof}
 \begin{enumerate}[(i)]
  \item Ist einfach.
  \item Seien $\varphi = \sum_{j=1}^n a_j \ind_{E_j}$, $\psi = \sum_{j=1}^m b_j \ind_{F_j}$ die Standard-Darstellungen. Aus 
  \[ E_j = \bigcup_{k=1}^m ( E_j \cap F_k ), \quad F_k = \bigcup_{j=1}^n ( E_j \cap F_k ), \]
  wobei die Vereinigungen disjunkt sind, folgt wegen der Additivität von $\mu$ 
  \[ \begin{aligned}
      \int \varphi + \int \psi &= \sum_{j=1}^n a_j \mu( E_j ) + \sum_{k=1}^m b_k \mu( F_k ) \\
      &= \sum_{j,k} (a_j + b_k) \mu (E_j \cap F_k ).
     \end{aligned} \]
  Dieselbe Überlegung zeigt, dass die Summe rechts gleich $\int(\varphi + \psi)$ ist.
  \item Wenn $\varphi \le \psi$ und $E_j \cap F_k \ne \emptyset$, dann $a_j \le b_k$ $\Rightarrow$
  \[ \int \varphi = \sum_{j,k} a_j \mu( E_j \cap F_k ) \le \sum_{j,k} b_k \mu (E_j \cap F_k) = \int \psi. \]
  \item Seien $A_j \in \meas$, $j \in \nat$ disjunkt und $A := \bigcup_1^\infty A_j$.
  \[ \begin{aligned}
      \int_A \varphi &= \sum_j a_j \mu( A \cap E_j ) \\
      &= \sum_{j,k} a_j \mu ( A_j \cap E_j ) \\
      &= \sum_k \int_{A_k} \varphi
     \end{aligned} \]
  und damit folgt die Behauptung. \qedhere
 \end{enumerate}
\end{proof}

\begin{defn}
 Für eine beliebige Funktion $f \in L^+$ sei
 \[ \int f \diffop \mu := \sup \left\{ \int \varphi \diffop \mu : 0 \le \varphi \le f, \varphi \text{ einfach } \right\}. \]
\end{defn}

\textbf{Anmerkungen.} 
Aus 2.3.2(iii) folgt, dass die zwei Definitionen für $\int f$ übereinstimmen, wenn $f$ einfach ist. Aus der Definition ist außerdem klar:
\[ \int f \le \int g, \text{ wenn } f \le g, \qquad \int f = c \cdot \int f, c \ge 0. \]

\begin{thm}[Monotone Konvergenz, Beppo Levi]
 Ist $\{f_n\}$ eine Folge in $L^+$ mit $f_j \le f_{j+1}$ für alle $j$ und $f = \lim_{n \to \infty} f_n$, so gilt
 \[ \int f = \lim_{n \to \infty} \int f_n.\footnotemark \]
\end{thm}
\footnotetext{Beide Grenzwerte existieren wegen der Monotonie, $\infty$ ist möglich.}

\begin{proof}
 Für alle $n$ gilt $\int f_n \le \int f$ $\Rightarrow$ $\lim \int f_n \le \int f$. Es bleibt noch zu zeigen, dass $\lim \int f_n \ge \int f$. 
 
 Sei $\alpha \in (0,1)$ beliebig, sei $\varphi$ eine einfache Funktion mit $0 \le \varphi \le f$ und $E_n := \{ x : f_n(x) \ge \alpha \cdot f(x) \}$. Dann gilt $E_n \subset E_{n+1}$, weil $f_n \le f_{n+1}$, $\bigcup E_n = X$ und
 \[ \int f_n \ge \int_{E_n} f_n \ge \alpha \int_{E_n} \varphi. \]
 Nach 2.3.2(iv) und der Stetigkeit von unten ist
 \[ \lim \int_{E_n} \varphi = \int \varphi \qRq \lim \int f_n \ge \alpha \int \varphi \]
 für alle $\alpha \in (0,1)$, also auch für $\alpha = 1$. 
 
 Wir bilden das Supremum über alle einfachen Funktionen $0 \le \varphi \le f$ und erhalten:
 \[ \lim \int f_n \ge \int f. \qedhere \]
\end{proof}

\begin{thm}
 Ist $\{ f_n \}$ eine endliche oder unendliche Folge in $L^+$ und $f = \sum_n f_n$, so gilt
 \[ \int f =  \sum_n \int f_n. \]
\end{thm}

\begin{proof}
 Wir betrachten zunächst zwei Funktionen $f_1$ und $f_2$. Nach Satz 2.1.15 existieren monoton wachsende  Folgen $\{ \varphi_j \}$ und $\{ \psi_j \}$ von einfachen Funktionen, die gegen $f_1$ bzw. $f_2$ konvergieren. Dann konvergiert die \emph{monoton wachsende Folge} $\{ \varphi_j + \psi_j \}$ gegen $f_1 + f_2$. Nach dem Satz über monotone Konvergenz (m) und der Additivität für einfache Funktionen (a) gilt:
 \[ \int (f_1 + f_2 ) \overset{(\text{m})}{=} \lim \int (\varphi_j + \psi_j) \overset{(\text{a})}{=} \lim \int \varphi_j + \lim \int \psi_j \overset{(\text{m})}{=} \int f_1 + \int f_2. \]
 
 Induktion über $N$ liefert
 \[ \int \sum_{n=1}^N f_n = \sum_{n=1}^N \int f_N. \]
 
 Ist die Folge unendlich, so bilden wir den Grenzwert $N \to \infty$:
 \[ \int \sum_{n=1}^\infty f_n \overset{(\text{m})}{=} \sum_{n=1}^\infty \int f_n. \qedhere \]
\end{proof}

\begin{thm}
 Für eine Funktion $f \in L^+$ gilt $\int f = 0$ genau dann, wenn $f=0$ fast überall.
\end{thm}

\begin{proof}
 Sei zuerst $f = \sum_j a_j \ind_{E_j}$, $a_j \ge 0$ eine einfache Funktion. Dann gilt $\int f = \sum_j a_j \mu( E_j ) = 0$ genau dann, wenn für alle $j$ entweder $a_j = 0$ oder $\mu( E_j ) = 0$, also ist $f = 0$ fast überall.
 
 Allgemein: Wenn $f=0$ fast überall und $\varphi$ eine einfache Funktion ist mit $0 \le \varphi \le f$, dann ist $\varphi = 0$ fast überall. Also
 \[ \int f = \sup \left\{ \int \varphi \right\} = 0. \]
 
 Andere Richtung: Es gilt
 \[ \{ x : f(x) > 0 \} = \bigcup_{n=1}^\infty E_n, \]
 wobei $E_n := \{ x : f(x) > \rez{n} \}$. Wäre die Aussage $f=0$ fast überall falsch, so hätten wir $\mu(E_n) > 0$ für ein $n$. Wegen $f \ge n \cdot \ind_{E_n}$ gilt dann 
 \[ \int f \ge n \mu( E_n ) > 0, \]
 Widerspruch.
\end{proof}

\begin{folg}
 Sei $f \in L^+$ und $\{ f_n\}$ eine monoton wachsende Folge in $L^+$, die fast überall gegen $f$ konvergiert. Dann gilt $\int f = \lim \int f_n.$
\end{folg}

\begin{proof}
 Übungsaufgabe.
\end{proof}

\begin{thm}[Lemma von Fatou]
 Sei $(f_n)_{n \in \nat}$ eine Folge in $L^+$. Dann gilt
 \[ \int \lim_{n \to \infty} f_n \diffop \mu \le \lim_{n \to \infty} \int f_n \diffop \mu. \]
\end{thm}

\begin{proof}
 Für alle $k \in \nat$ gilt
 \[ \inf_{n \ge k} f_n \le f_j \]
 für alle $j \ge k$. Damit folgt 
 \begin{align*}
     \int \inf_{n \ge k} f_n \diffop \mu &\le \int f_j \diffop \mu, \\
     \Rightarrow \int \inf_{n \ge k} f_n \diffop \mu &\le \inf_{j \ge k} \int f_j \diffop \mu. \tag{$*$}
 \end{align*}
 Wir bilden den Grenzwert $k \to \infty$ und wenden den Satz von der monotonen Konvergenz (m) an:
 \begin{align*}
     \int \liminf_{n \to \infty} f_n \diffop \mu &= \int \lim_{n \to \infty} \underbrace{\inf{n \ge k} f_n}_{\text{aufsteigend, messbar, } > 0} \diffop \mu \\
     &\overset{(\text{m})}{=} \lim_{k \to \infty} \int \inf_{n \ge k} f_n \diffop \mu \\
     &\overset{(*)}{=} \lim_{k \to \infty} \inf_{n \ge k} \int f_n \diffop \mu \\
     &= \liminf_{k \to \infty} f_k \diffop \mu. \qedhere
 \end{align*}
\end{proof}

\begin{folg}
 Sei $(f_n)_n$ eine Folge in $L^+$, die fast überall gegen eine Funktion $f \in L^+$ konvergiert. Dann ist
 \[ \int f \diffop \mu \le \liminf_{n \to \infty} \int f_n \diffop \mu. \]
\end{folg}

\begin{proof}
 Gilt die Konvergenz überall, dann ist
 \[ f(x) = \lim_{n \to \infty} f_n(x) = \liminf_{n \to \infty} f_n(x) \]
 für alle $x \in X$ und somit folgt die Behauptung aus dem Lemma von Fatou. 
 
 Konvergenz überall können wir erreichen, indem wir $f_n$ und $f$ auf einer Nullmenge ändern\footnote{Passe $f_n$ und $f$ so an, dass $f'_n$ auch gegen $f'$ konvergiert auf der Nullmenge, wo $f_n$ \emph{nicht} gegen $f$ konvergiert.}, das ändert nicht die Integrale.
\end{proof}

\begin{lem}
 Sei $f \in L^+$ und $\int f \diffop \mu < \infty$. Dann
 \begin{enumerate}[(i)]
  \item $\{ x \in X,  f(x) = \infty \}$ ist eine Nullmenge.
  \item $\{ x \in X, f(x) > 0 \}$ ist $\sigma$-endlich.
 \end{enumerate}
\end{lem}

\begin{proof}
 Übungsaufgabe.
\end{proof}

\section{Aufgaben}
Siehe \verb+Aufgaben-2-4.pdf+.

\section{Integration komplexer Funktionen}
In diesem Abschnitt ist $(X, \meas, \mu)$ ein Maßraum.

\begin{defn}
 Sei $f:X \to \realext$ eine messbare Funktion. Ist $\int f^+ \diffop \mu < \infty$ oder $\int f^- \diffop \mu < \infty$, so definieren wir das \emph{Integral von $f$} bezüglich $\mu$ durch
 \[ \int f \diffop \mu := \int f^+ \diffop \mu - \int f^- \diffop \mu. \]
 Sind $\int f^+ \diffop \mu < \infty$ und $\int f^- \diffop \mu < \infty$, so heißt $f$ \emph{integrierbar}. 
 
 Wegen $0 \le f^2 \le |f| = f^+ + f^-$ ist $f$ genau dann integrierbar, wenn $\int |f| \diffop \mu < \infty.$
\end{defn}

\begin{lem}
 Die Menge der reellwertigen integrierbaren Funktionen bildet einen reellen linearen Raum\footnotemark und das Integral ist ein lineares Funktional\footnotemark auf diesem Raum.
\end{lem}
 \addtocounter{footnote}{-2} %3=n
 \stepcounter{footnote}\footnotetext{Ein linearer Raum $X$ ist abgeschlossen bezüglich Skalarmultiplikation und Addition}
 \stepcounter{footnote}\footnotetext{Sei $V$ ein $\mathbb{K}$-Vektorraum. Ein lineares Funktional ist eine lineare Abb. $T: V \to \mathbb{K}$.}

\begin{proof}
 \textbf{Linearer Raum:} Das folgt aus
 \[ | af + bg | \le |a| \cdot |f| + |n| \cdot |g| \]
 für alle $a,b \in \real$, $f,g$ Funktionen.
 
 Sind $f,g$ integrierbar, so folgt daraus, dass $af + bg$ integrierbar ist.
 
 \textbf{Lineares Funktional:} $\int (af) \diffop \mu = a \int f \diffop \mu$ für alle $a \in \real$ ist trivial.
 
 Seien $f,g$ integrierbar und $h = f+g$. Dann ist $h = h^+ - h^- = (f^+ - f^-) + (g^+ - g^-)$, also
 \[ h^+ + f^- + g^- = h^- + f^+ + g^+ \ge 0. \]
 Nach Satz 2.3.5 ist
 \[ \int h^+ \diffop \mu + \int f^- \diffop \mu + \int g^- \diffop \mu = \int h^- \diffop \mu + \int f^+ \diffop \mu + \int g^+ \diffop \mu \]
 und damit
 \[ \int h \diffop \mu = \int f \diffop \mu + \int g \diffop \mu. \qedhere \]
\end{proof}

\begin{defn}
 Eine messbare Funktion $f: X \to \complex$ heißt \emph{integrierbar}, falls $\int |f| \diffop \mu < \infty$.
 
 \textbf{Allgemeiner:} $f$ heißt auf $E \in \meas$ integrierbar, falls $\int_E |f| \diffop \mu$ endlich ist. Wegen $|f| \le |\Re f| + |\Im f| \le 2 |f|$ folgt, dass $f$ genau dann integrierbar ist, wenn $\Re f$ und $\Im f$ integrierbar sind. In diesem Fall sei
 \[ \int f \diffop \mu = \int \Re f \diffop \mu + i \int \Im f \diffop \mu. \]
 Insbesondere gilt
 \[ \Re \left( \int f \diffop \mu \right) = \int \Re f \diffop \mu, \qquad
    \Im \left( \int f \diffop \mu \right) = \int \Im f \diffop \mu. \]
\end{defn}

\begin{lem}
 Die Menge aller komplexwertigen integrierbaren Funktionen ist ein komplexer linearer Raum und das Integral ist ein lineares Funktional auf diesem Raum.
\end{lem}

\begin{proof}
 Einfach, ähnlich wie der Beweis von 2.5.2.
\end{proof}

Bezeichnung für den obigen Raum: $\intf^1(X)$, $\intf^1(X,\mu)$, $\intf^1(\mu)$ oder einfach $\intf^1$.

\begin{lem}
 Für $f \in \intf^1$ gilt
 \[ \left| \int f \diffop \mu \right| \le \int |f| \diffop \mu. \]
\end{lem}

\begin{proof}
 Ist $\int f \diffop \mu = 0$, dann ist die Behauptung trivial.
 
 \textbf{Fall 1:} $f$ ist reellwertig. Dann
 \[ \begin{aligned}
     \left| \int f \diffop \mu \right| &\overset{\text{Def.}}{=} \left| \int f^+ \diffop \mu - \int f^- \diffop \mu \right| \\
     &\le \left| \int f^+ \diffop \mu \right| + \left| \int f^- \diffop \mu \right|
     = \int f^+ \diffop \mu + \int f^- \diffop \mu \\
     &= \int ( f^+ + f^- ) \diffop \mu = \int |f| \diffop \mu.
    \end{aligned} \]
 \textbf{Fall 2:} $f$ ist komplexwertig. Sei 
 \[ \alpha = \frac{ \obar{\int f  \diffop \mu} }{ | \int f \diffop \mu |}. \]
 Dann gilt $|\alpha| \le 1$ und
 \[ \left| \int f \diffop \mu \right| \overset{(\text{z})}{=} \alpha \int f \diffop \mu = \int \alpha f \diffop \mu, \]
 wobei (z): $z\obar{z} = |z|^2, z := \int f \diffop \mu$. Also
 \begin{align*}
  \left| \int f \diffop \mu \right| &= \Re \left| \int f \diffop \mu \right| = \Re \left( \int \alpha f \diffop \mu \right) \\
  &= \int \Re (\alpha f) \diffop \mu \le \int \underbrace{|\alpha f|}_{|\alpha|\cdot|f| \le |f|} \diffop \mu \\
  &\le \int |f| \diffop \mu. \qedhere
 \end{align*}
\end{proof}
 
\begin{lem}
 Ist $f \in \intf^1$, dann gilt $\{ x, f(x) \ne 0 \}$ ist $\sigma$-endlich.  
\end{lem}

\begin{proof}
 Übungsaufgabe.
\end{proof}

\begin{lem}
 Für $f,g \in \intf^1$ sind äquivalent:
 \begin{enumerate}[(i)]
  \item $\int_E f \diffop \mu =  \int_E g \diffop \mu$ für alle $E \in \meas$.
  \item $\int |f-g| \diffop \mu = 0$.
  \item $f=g$ fast überall.
 \end{enumerate}
\end{lem}

\begin{proof}
 Übungsaufgabe.
\end{proof}

\begin{rmrk}
 Das Integral einer Funktion $f$ ändert sich also nicht, wenn wir $f$ auf einer Nullmenge modifizieren. Demzufolge können wir Funktionen integrieren, die nur fast überall definiert sind. In den nicht definierten Punkten können wir die Funktion zum Beispiel gleich 0 setzen.
 
 In diesem Sinne können wir $\realext$-wertige Funktionen, die fast überall endlich sind, als $\real$-wertig betrachten.
\end{rmrk}

\textbf{Bezeichnung.} 
$L^1(\mu)$ ist die Menge der Äquivalenzklassen von fast überall definierten $\mu$-integrierbaren Funktionen, wobei $f$ und $g$ äquivalent heißen, falls $f = g$ fast überall.

Dann ist $L^1(\mu)$ ein komplexer linearer Raum und sogar ein metrischer Raum mit der Metrik
\[ \rho( f,g ) := \int | f -g | \diffop \mu. \]
Man verwendet die (eigentlich nicht korrekte) Schreibweise $f \in L^1(\mu)$ für die Äqui\-valenz\-klasse von $f$.

\begin{thm}[Dominierte Konvergenz, Satz von Lebesgue]
 Seien $f_n, g \in \intf^1$, $n \in \nat$, sodass $f_n \to f$ fast überall und $|f_n| \le g$ für alle $n \in \nat$. Dann gilt
 \[ \int f \diffop \mu = \int \lim_{n \to \infty} f_n \diffop \mu = \lim_{n \to \infty} \int f \diffop \mu. \]
\end{thm}

\begin{proof}
 Die Funktion $f$ ist messbar und $|f| \le g$ fast überall, also ist $f \in \intf^1$. 
 
 Es genügt, reellwertige Funktionen zu betrachten (man bildet sonst Real- und Imaginärteil). Es gilt
 \[ g + f_n \ge 0 \quad \text{und} \quad g - f_n \ge 0 \]
 für alle $n \in \nat$. Damit folgt mit dem Lemma von Fatou (f):
 \[ \begin{aligned}
  \int(g+f) \diffop \mu &= \int \liminf_{n \to \infty} (g + f_n) \diffop \mu \\
  &\overset{(\text{f})}{\le} \liminf_{n \to \infty} \int (g+f_n) \diffop \mu = \int g \diffop \mu + \liminf_{n \to \infty} \int f_n \diffop \mu 
    \end{aligned} \]
  sowie
 \[ \begin{aligned}
 \int(g-f) \diffop \mu &= \int \liminf_{n \to \infty} (g - f_n) \diffop \mu \\
  &\overset{(\text{f})}{\le} \liminf_{n \to \infty} \int(g - f_n) \diffop \mu = \int g \diffop \mu - \limsup_{n \to \infty} \int f_n \diffop \mu.
    \end{aligned} \]
 Folglich gilt
 \[ \liminf_{n \to \infty} \int f_n \diffop \mu \ge \int f \diffop \mu \ge \limsup_{n \to \infty} \int f_n \diffop \mu. \]
 Daraus folgt
 \[ \lim_{n \to \infty} \int f_n \diffop \mu = \int f \diffop \mu. \]
 Hier wurde verwendet, dass $\lim_{n \to \infty} a_n$ existiert $\Leftrightarrow$ $\liminf_{n \to \infty} a_n = \limsup_{n \to \infty} a_n$.
\end{proof}

%%%

\begin{thm}
 Es sei $\{ f_j \}$ eine Folge in $\intf^1$ mit $\sum_1^\infty |f_j| < \infty$. Dann konvergiert $\sum_1^\infty f_j$ fast überall gegen eine Funktion in $\intf^1$ und es gilt
 \[ \int \sum_{j=1}^\infty f_j = \sum_{j=1}^\infty \int f_j. \]
\end{thm}

\begin{proof}
 Nach Satz 2.3.5 gilt
 \[ \int \sum_{j=1}^\infty | f_j | = \sum_{j=1}^\infty \int |f_j| < \infty \]
 und folglich $\sum_1^\infty | f_j | \in \intf^1$.
 
 Wegen Lemma 2.3.10 ist $\sum_1^\infty |f_j|$ fast überall endlich, also konvergiert $\sum_1^\infty f_j$ fast überall.  Für alle $n$ gilt
 \[ \left| \sum_{j=1}^n f_j \right| \le \sum_{j=1}^n | f_j | \le \sum_{j=1}^\infty |f_j| =: g. \]
 Die Behauptung folgt nun aus dem Satz über dominierte Konvergenz.
\end{proof}

\begin{thm}
 \begin{enumerate}[(i)]
  \item Für jede Funktion $f \in \intf^1$ und jedes $\eps > 0$ existiert eine integrierbare, einfache Funktion $\varphi = \sum a_j \ind_{E_j}$ mit $\int | f -\varphi | < \eps$.\footnotemark (Abstand $d(f,\varphi)$ in $\intf^1$)
  \item Ist $\mu$ ein Lebesgue-Stieltjes-Maß auf $\real$, so können wir für die $E_j$ offene Intervalle nehmen.
  \item Ist $\mu$ wie in (ii), so existiert eine stetige Funktion $g$, die außerhalb eines beschränkten Intervalls verschwindet\footnotemark, mit
  \[ \int |f-g| \diffop \mu < \eps.\footnotemark \]
 \end{enumerate}

\end{thm}
\addtocounter{footnote}{-3} %3=n
 \stepcounter{footnote}\footnotetext{Das heißt die integrierbaren einfachen Funktionen sind \emph{dicht} in $\intf^1$.}
 \stepcounter{footnote}\footnotetext{$g$ hat einen \emph{kompakten Träger}, $\operatorname{supp} g := \obar{\{ x \in \real : g(x) \ne 0 \}}$.}
 \stepcounter{footnote}\footnotetext{Also sind auch die stetigen Funktionen mit kompaktem Träger dicht in $\intf^1$.}

\begin{proof}
 \begin{enumerate}[(i)]
  \item Sei$\{ \varphi_j \}$ wie in Satz 2.1.15(ii). Nach dem Satz über dominierte Konvergenz gilt $\int |\varphi_n - f| < \eps$, wenn $n$ hinreichend groß\footnote{Es ist $|\varphi_n| \le |f|$, $\varphi_j \to f$ und damit $| \varphi_n - f| \le |\varphi_n| + |f| \le 2 |f| =: g$.} ist.
  \item Sei $\varphi = \sum a_j \ind_{E_j}$, wobei die $E_j$ disjunkt und die $a_j \ne 0$ sind. Dann ist 
  \[ \mu(E_j)  = \rez{|a_j|} \int_{E_j} |\varphi| \le \rez{|a_j|} \int |f| < \infty. \]
  Weiterhin gilt $\mu( E \Delta F ) = \int( \ind_E - \ind_F )$. 
  
  Nach Aufgabe 1.12.1 können wir $\ind_{E_j}$ in der $\intf^1$-Metrik beliebig approximieren durch endliche Linearkombinationen von Funktionen $\ind_{I_k}$, wobei die $I_k$ offene Intervalle sind.
  \item Ist $I_k = (a_k, b_k)$ wie in (ii), so können wir $\ind_{I_k}$ in der $\intf^1$-Metrik beliebig durch stetige Funktionen approximieren, die außerhalb von $(a,b)$ verschwinden (auch Differenzierbarkeit ist möglich). \qedhere
 \end{enumerate}
\end{proof}

\begin{exmp*}
 Sei $g(x) = 0$ für $x \notin (a_k,b_k)$, $g(x) = 1$ für $x \in [a_k + \rez{n}, b_k - \rez{n}]$ und sei $g$ linear auf $[a, a + \rez{n}]$ und $[b-\rez{n},b]$. Dann gilt
 \[ \int | g - \ind_{(a_k,b_k)} | = \mu(( a, a+\rez{n})) + \mu( (b-\rez{n},b) )) \xrightarrow{n \to \infty} 0 \]
 nach dem Stetigkeitssatz (von oben).
\end{exmp*}

\begin{thm}[Stetigkeit und Differenzierbarkeit von Parameterintegralen]
 Sei $f:X \times [a,b] \to \complex$, $-\infty < a < b < \infty$, so das $f( \cdot, t ) : X \to \complex$ integrierbar ist. Wir schreiben
 \[ F(t) := \int_X f( x, t ) \diffop \mu(x), \quad t \in [a,b]. \]
 \begin{enumerate}[(i)]
  \item Nehmen wir an, dass ein $g \in \intf^1(\mu)$ existiert mit $|f(x,t) \le g(x)$ für alle $x$ und $t$, und dass $\lim_{t \to t_0} f(x,t) = f(x,t_0)$ für alle $x$. Dann gilt
  \[ \lim_{t \to t_0} F(t) = F(t_0). \]
  Ist speziell $f(x, \cdot)$ stetig für alle $x$, so ist $F$ stetig.
  \item Nehmen wir an, dass $\pdiff{f}{t}$ und ein $g \in \intf^1$ existieren, so dass $\left| \pdiff{f}{t} (x,t) \right| \le g(x)$ für alle $x$ und $t$. Dann ist $F$ differenzierbar und 
  \[ F'(t) = \int \pdiff{f}{t} \diffop \mu(x), \quad t \in [a,b]. \]
 \end{enumerate}
\end{thm}

\begin{proof}
 \begin{enumerate}[(i)]
  \item Wir wenden den Satz über dominierte Konvergenz auf $f_n(x) := f(x, t_n)$ an, wobei $\{ t_n \}$ eine beliebige Folge in $[a,b]$ mit $t_n \to t_0$ ist. 
  \item Sei $\{ t_n \}$ wie oben. Es gilt
  \[ \pdiff{f}{t}(x,t_0) = \lim_{n \to \infty} h_n(x), \qquad \text{wobei } h_n(x) := \frac{f(x,t_n) - f(x,t_0)}{t_n -t_0} \]
  und $t_n \to t_0$. Hieraus folgt, dass $\pdiff{f}{t}$ messbar ist.
  
  Mit dem Mittelwertsatz (m) folgt
  \[ |h_n(x)| \overset{(\text{m})}{\le} \sup_{t \in [a,b]} \left| \pdiff{f}{t} (x,t) \right| \le g(x) \]
  und damit folgt
  \[ F'(t_0) = \lim_{n \to \infty} \frac{F(t_n) - F(t_0)}{t_n - t_0} = \lim_{n \to \infty} \int h_n(x) \diffop \mu(x) \overset{(\text{d})}{=} \int \pdiff{f}{t} (x,t) \diffop \mu(x) \]
  mit dem Satz über dominierte Konvergenz (d). \qedhere
 \end{enumerate}
\end{proof}

\begin{prgp}[Vergleich Lebesgue- und Riemann-Integral auf $\real$]
 Sei $[a,b]$ ein beschränktes Intervall. Eine \emph{Zerlegung} von $[a,b]$ ist eine endliche Folge $P = \{ t_j \}_0^n$ mit $a = t_0 < \ldots < t_n = b$. Sei $f$ eine beliebige, beschränkte, reellwertige Funktion auf $[a,b]$. Für jede Zerlegung $P$ sei
\[ S_P f := \sum_{j=1}^n M_j(t_j-t_{j-1}), \qquad s_P f := \sum_{j=1}^n m_j(t_j-t_{j-1}), \]
wobei $M_j$ bzw. $m_j$ das Supremum bzw. Infimum von $f$ über $(t_{j-1},t_j]$ bezeichnen. Gilt
\[ i(f) := \inf_{p} S_p f = \sup_p s_p f =: I(f), \]
so heißt $f$ \emph{Riemann-integrierbar} und das \emph{Riemann-Integral} von $f$ über $[a,b]$ wird definiert durch
\[ \int_a^b f(x) \diffop x := I(f) = i(f). \]
\end{prgp}

\begin{thm}
 Sei $f$ eine beschränkte, reellwertige Funktion auf $[a,b]$.
 \begin{enumerate}[(i)]
  \item Ist $f$ Riemann-integrierbar, dann ist $f$ Lebesgue-integrierbar und
  \[ \int_a^b f(x) \diffop x = \int_{[a,b]} f(x) \diffop \lambda(x). \]
  \item $f$ ist genau dann Riemann-integrierbar, wenn das Lebesgue-Maß der Menge
  \[ \{ x \in [a,b] : f \text{ ist nicht stetig in } x \} \]
  gleich Null ist.
 \end{enumerate}
\end{thm}

\begin{proof}
 \begin{enumerate}[(i)]
  \item Für jede Partition $P$ sei
  \[ G_P := \sum_{j=1}^n M_j \ind_{(t_{j-1},t_j]}, \qquad g_P := \sum_{j=1}^\infty m_j \ind_{(t_{j-1},t_j]}. \]
  Dann gilt
  \[ \int_{[a,b]} G_P \diffop \lambda = S_P f, \qquad \int_{[a,b]} g_P \diffop \lambda = s_P f. \]
  Wir wählen eine Folge $\{ P_k \}$ von Partitionen, so dass
  \begin{itemize}
   \item ihre Feinheit $\max_j (t_j - t_{j-1}$ gegen Null geht,
   \item $P_{k+1}$ eine Verfeinerung von $P_k$ ist, dann ist $g_{P_k}$ \emph{monoton wachsend} und $G_{P_k}$ \emph{monoton fallend},
   \item $S_{P_k} f$ und $s_{P_k} f$ gegen $\int_a^b f(x) \diffop x$ konvergiert.
  \end{itemize}
  Sei $G := \lim G_{P_k}$ und $g := \lim g_{P_k}$. Dann ist $g \le f \le G$ und nach dem Satz über dominierte Konvergenz gilt
  \[ \int G \diffop \lambda = \int g \diffop \lambda = \int_a^b f(x) \diffop x. \]
  Folglich ist
  \[ \int \underbrace{(G-g)}_{\ge 0} \diffop \lambda = 0 \qRq G \overset{\text{f.ü.}}{=} g \qRq G=g=f \text{ fast überall.} \]
  Da $G$ messbar ist (Grenzwert von einfachen Funktionen) und $\lambda$ vollständig ist, ist auch $f$ messbar und
  \[ \int_{[a,b]} f \diffop \lambda = \int_{[a,b]} G \diffop \lambda = \int_a^b f(x) \diffop x. \]
  \item Wir definieren für $x \in [a,b]$
  \[ h(x) := \liminf_{y \to x} f(y), \qquad H(x) := \limsup_{y \to x} f(y). \]
  Dann gilt $h \le f \le H$ und $h(x) = H(x)$, wenn $f$ in $x$ stetig ist.
  
  Zu zeigen ist also: $f$ ist Riemann-integrierbar $\Leftrightarrow$ $h = H$ fast überall.
  
  Sei $f$ Riemann-integrierbar. Mit den Bezeichnungen von (i) gilt dann $H = G$ fast überall und $h=g$ fast überall (Warum? Übungsaufgabe!). Folglich sind $h$ und $H$ Lebesgue-messbar und
  \[ \int_{[a,b]} h \diffop \lambda = i(f) = I(f) = \int_{[a,b]} H \diffop \lambda. \]
  Da $h \le H$ folgt hieraus, dass $h = H$ fast überall.
  
  Ist $f$ fast überall stetig, so ist $f$ messbar (Warum? Übungsaufgabe!). Folglich sind auch $h$ und $H$ messbar und $h = H$ fast überall. Die Gleichung oben gilt auch in diesem Fall, das heißt $i(f) = I(f)$, also ist $f$ Riemann-integrierbar. \qedhere
 \end{enumerate}
\end{proof}

\begin{prgp}[Uneigentliches Riemann-Integral]
 Ist $f$ Riemann-integrierbar über $[a,b]$ für jedes $b > 0$ und Lebesgue-integrierbar über $[0,\infty)$, so gilt nach dem Satz über dominierte Konvergenz\footnote{Wähle eine Folge $b_n \to \infty$, definiere $f_n := \ind_{[a,b_n]} \cdot f$, also $\int_0^{b_n} f = \int_0^\infty f_n$, $|f_n| \le |f| =: g$.}
 \[ \int_{[0,\infty)} f \diffop \lambda = \lim_{b \to \infty} \int_a^b f(x) \diffop x \overset{\text{Def.}}{=} \int_0^\infty f(x) \diffop x. \]
\end{prgp}

Der Grenzwert oben kann also auch dann existieren, wenn $f$ nicht Lebesgue-integrierbar ist auf $[0,\infty)$, zum Beispiel
\[ \sum_{n=1}^\infty (-1)^{n+1} \cdot \rez{n} \cdot \ind_{(n,n+1)}(x) := f(x). \]
Es gilt: $f$ ist integrierbar $\Leftrightarrow$ $|f|$ ist integrierbar, aber
\[ \int |f| = \int_{[0,\infty)} \sum_{n=1}^\infty \rez{n} \cdot \ind_{(n,n+1)} = \sum_{n=1}^\infty \rez{n} = \infty. \]
Also ist $f$ nicht Lebesgue-integrierbar. 

Wir können aber das uneigentliche Riemann-Integral berechnen:
\[ \int_0^{N+1} f(x) \diffop x = \int_0^{N+1} \sum_{n=1}^\infty (-1)^{n+1} \cdot \rez{n} \cdot \ind_{(n,n+1)}(x) \diffop x = \sum_{n=1}^N (-1)^{n+1} \cdot \rez{n} \xrightarrow{N \to \infty} \ln 2. \]
Die Reihe konvergiert nach dem Leibniz-Kriterium.

Die Indikatorfunktion $f$ der Menge der rationalen Zahlen in $[0,1]$ ist Lebesgue-messbar, $f=0$ fast überall und damit ist sie Lebesgue-integrierbar mit $\int f = 0$. $f$ ist keinem Punkt stetig, also auch nicht Riemann-integrierbar.
\[ 0 = i(f) \ne I(f) = 1. \]

\section{Aufgaben}
Siehe \verb+Aufgaben-2-6.pdf+.

\section{Konvergenzarten}
In diesem Abschnitt ist $X$ eine beliebige Menge, $\{ f_n \}$ eine Folge von komplexen Funktionen auf $X$.

$f_n \to t$, $n \to \infty$ kann bedeuten:
\begin{itemize}
 \item Punktweise Konvergenz: 
  \[ \lim_{n \to \infty} f_n(x) = f(x) \text{ für alle } x \in X, \]
 \item Gleichmäßige Konvergenz:
  \[ \lim_{n \to \infty} \| f_n - f \| = 0, \text{ wobei } \| g \| := \sup_{x \in X} | g(x) |. \]
\end{itemize}

Ist $(X, \mA, \mu)$ ein Maßraum, so können wir auch über \emph{Konvergenz fast überall} oder $L^1$-Konvergenz sprechen:
\[ \lim_{n \to \infty} \| f_n - f \|_1 = 0, \text{ wobei } \| g \|_1 := \int_X |g| \diffop \mu. \]

Die folgenden Beispiele sind nützlich ($X = \real$):
\begin{enumerate}[(i)]
 \item $f_n = \rez{n} \cdot \ind_{(0, n]}$, $\int f_n \diffop \lambda = 1$,
 \item $f_n = \ind_{(n,n+1)}$, $\int f_n \diffop \lambda = 1$,
 \item $f_n = n \cdot \ind_{(0,\rez{n}]}$, $\int f_n \diffop \lambda = 1$,
 \item $f_n = \ind_{[j/2^k,(j+1)/2^k]}$, wobei $k \in \nat$, $0 \le j \le 2^k$, $n = j + 2^k$, $\int f_n \diffop \lambda = \rez{2^k}$ (``Wandernde Hüte'')
\end{enumerate}

In (i), (ii) und (iii) gilt $f_n \to 0$ punktweise, in (i) sogar gleichmäßig, aber nicht in (ii) und (iii).

In (i), (ii) und (iii) konvergiert $f_n$ fast überall, aber keine $L^1$-Konvergenz, da $\| f_n - f \|_1 = \| f_n \|_1 = 1 \nrightarrow 0$ für $n \to \infty$.

In (iv) gilt dagegen $f_n \to 0$ in der $L^1$-Metrik (Norm), aber die Folge konvergiert \emph{in keinem Punkt}.

\textbf{Andererseits:} Wenn $f_n \to f$ fast überall und $| f_n | \le g \in L^1$ für alle $n$, dann $f_n \to f$ in $L^1$.
\[ |f_n - f| \le | f_n | + | f | \le g + g \text{ f. ü. } \qRq \lim_{n \to \infty} \int | f_n - f | = \int \underbrace{\lim_{n \to \infty} | f_n - f |}_{= 0 \text{ f. ü.}} = 0.\]

\begin{defn}
 Eine Folge $\{f_n\}$ messbarer, komplexer Funktionen auf einem Maßraum $(X,\mA,\mu)$ ist eine \emph{Cauchy-Folge nach Maß}, wenn für jedes $\eps > 0$ gilt:
 \[ \mu( \{ x : | f_n(x) - f_m(x) | \ge \eps \} ) \to 0; \quad n, m \to \infty. \]
 Die Folge $\{ f_n \}$ \emph{konvergiert nach Maß} gegen eine messbare Funktion $f$, wenn
 \[ \mu( \{ x : | f_n(x) - f(x) | \ge \eps \} ) \to 0; \quad n \to \infty. \]
\end{defn}

\textbf{Beispiel.} Die Folgen (i), (iii) und (iv) (\emph{nicht} (ii)) konvergieren gegen 0 nach Maß; (ii) ist keine Cauchy-Folge nach Maß.

\begin{lem}
 Gilt $f_n \to f$ in $L^1$, so gilt auch $f_n \to f$ nach Maß.
\end{lem}

\begin{proof}
 Definiere $E_{n,\eps} := \{ | f_n(x) - f(x) | \ge \eps \}$. Es gilt:
 \[ \underbrace{\int | f_n - f |}_{\to 0 \text{ für } n \to \infty} \ge \int_{E_{n,\eps}} | f_n - f | \ge \eps \cdot \mu( E_{n,\eps} ). \qedhere \]
\end{proof}

Die Umkehrung ist falsch, siehe die Beispiele (i) und (iii).

\begin{thm}
 Sei $\{ f_n \}$ eine Cauchy-Folge nach Maß. Dann existiert eine messbare Funktion $f$, sodass $f_n \to f$ nach Maß, und eine Teilfolge $\{ f_{nj} \}$ von $\{ f_n \}$, die fast überall gegen $f$ konvergiert. Gilt $f_n \to g$ nach Maß, so ist $g = f$ fast überall.
\end{thm}

\begin{proof}
 Wir wählen eine Teilfolge $\{ g_j \} := \{ f_{nj} \}$, sodass $\mu( E_j ) \le \rez{2^j}$, wobei 
 \[ E_j := \{ x : | g_j(x) - g_{j+1}(x) | \ge \rez{2^j} \}, \quad j \in \nat. \]
 Mit $F_k := \bigcup_k^\infty E_j$ ist
 \[ \mu( F_k ) \le \sum_{j=k}^\infty \rez{2^j} = 2^{1-k}. \]
 Für $k \le j \le i$ und $x \notin F_k$ gilt
 \[ | g_j(x) - g_i(x) | \le \sum_{l=j}^{i-1} | g_{l+1}(x) - g_l(x) | \le \sum_{l=j}^{i-1} \rez{2^l} = 2^{1-j}. \]
 Also ist $\{g_j\}$ punktweise Cauchy auf $F_k^c$. Mit $F := \bigcap_1^\infty F_k$ ist $\mu(F) = 0$ und damit ist die Funktion
 \[ f(x) := \begin{cases} \lim_{i \to \infty} g_j(x), &x \notin F, \\ 0 &x \in F \end{cases} \]
 messbar und $g_j \to f$ fast überall.
 
 Die obige Ungleichung zeigt, dass
 \[ | g_j(x) - f(x) | \le 2^{1-j}, \quad j \ge k, \quad x \notin F_k. \]
 Wegen $\mu(F_k) \to 0$, $k \to \infty$ folgt hieraus, dass $g_j \to f$ nach Maß. Dann gilt aber auch $f_n \to f$ nach Maß, da
 \[ \{ x : | f_n(x) - f(x) | \ge \eps \} \subset \left\{ x : | f_n(x) - g_j(x) | \ge \frac{\eps}{2} \right\} \cup \left\{ x : | g_j(x) - f(x) | \ge \frac{\eps}{2} \right\} \]
 für alle $n$.
 
 Nehmen wir an, dass $f_n \to g$ nach Maß. Dann gilt
 \[ \{ x : | f(x) - g(x) | \ge \eps \} \subset \left\{ x : | f(x) - f_n(x) | \ge \frac{\eps}{2} \right\} \cup \left\{ x : | f_n(x) - g(x) | \ge \frac{\eps}{2} \right\} \]
 für alle $n$. Also ist
 \[ \mu( \{ x : | f(x) - g(x) | \ge \eps \} ) = 0 \]
 für alle $\eps > 0$. Damit gilt $f = g$ fast überall.
\end{proof}

\begin{thm}[Jegorow]
 Sei $(X, \mA, \mu)$ ein endlicher Maßraum und $f_n, f$ messbar mit $f_n \to f$ fast überall.
 
 Dann gilt: Für alle $\eps > 0$ existiert $E \in \mA$ mit $\mu(E) < \eps$, so dass $f_n \to f$ gleichmäßig auf $E^c$, das heißt
 \[ \sup_{x \in E^c} | f_n(x) - f(x) | \xrightarrow{n \to \infty} n. \]
\end{thm}

\begin{proof}
 Indem wir $f_n$, $f$ gegebenenfalls auf Nullmengen modifizieren, können wir o.B.d.A. annehmen, dass $f_n \to f$ (überall).
 
 Definiere 
 \[ E_n(h) := \bigcup_{m \ge n} \left\{x_i | f_m(x) -f(x) | \ge \rez{h} \right\}. \]
 Es gilt $E_{n+1}(h) \subseteq E_n(h)$ und $\bigcup_{n \ge 1} E_n(h) = \emptyset$ (für $x \in X$ existiert wegen $f_n(x) \to f(x)$ also ein $N \in \nat$, sodass $| f_n(x) - f(x) | < \rez{h}$ für $n \ge N$, folglich ist $x \notin E_{n+1}(h)$).
 
 Da $\mu$ ein endliches Maß ist, folgt aus der Stetigkeit von oben:
 \[ \mu( E_n(h) ) \xrightarrow{n \to \infty} \mu(\emptyset) = 0. \]
 Für $\eps > 0$, $k \in \nat$, wähle $n_k = n_k(\eps)$ mit
 \[ \mu( E_{n_k}(k) < \eps \cdot 2^{-k}. \]
 Für $E := \bigcup_{k \ge 1} E_{n_k}(k)$ gilt $\mu(E) < \eps$, denn
 \[ \mu(E) \le \sum_{k\ge 1} \underbrace{\mu( E_{n_k}(k) )}_{< \eps \cdot 2^{-k}} < \eps \]
 und $| f_n(x) - f(x) | < \rez{k}$ für alle $x \in E^c$ und $n \ge n_k$.
\end{proof}

\section{Aufgaben}
Siehe \verb+Aufgaben-2-8.pdf+.

\section{Produktmaße}
In diesem Abschnitt sind $(X, \mM, \mu)$ und $(Y, \mN, \eta)$ Maßräume.

Wir haben bereits die Produkt-$\sigma$-Algebra
\[ \mM \otimes \mN := \sigma( \{ M \times N; M \in \mM, N \in \mN \} ) \]
auf $X \times Y$ konstruiert. Wir werden jetzt ein Maß auf $\mM \otimes \mN$ konstruieren, das in natürlicher Weise als Produkt von $\mu$ und $\eta$ angesehen werden kann.

Ein Rechteck $R$ ist eine Menge der Form $A \times E$ für $A \in \mM$, $E \in \mN$.

Natürliche Idee: Das Produktmaß eines solchen Rechtecks durch $\mu(A) \cdot \eta(E)$ zu definieren.

\begin{lem}
 Die Familie $\mA$ aller endlichen disjunkten Vereinigungen von Rechtecken ist eine Algebra und ein Erzeuger von $\mM \otimes \mN$.
\end{lem}

\begin{proof}
 Die erste Aussage folgt aus
 \[ (A \times E) \cap (B \times F) = (A \cap B) \times (E \cap F) \]
 wegen Lemma 1.11.1, dass
 \[ (A \times E)^C = (X \times E^C) \cup (A^C \times Y). \]
 
 Die zweite Aussage folgt aus der Definition von $\mM \otimes \mN$.
\end{proof}

\begin{lem}
 Sei $B \in \mA$ die disjunkte Vereinigung der Rechtecke $A_1 \times E_1, \ldots, A_n \times E_n$. Dann hängt die Zahl
 \[ \pi( B ) := \sum_{j=1}^n \mu(A_j) \eta(E_j) \]
 nur von $B$ ab, nicht aber von der speziellen Wahl der $A_j$ und $E_j$. Weiterhin ist $\pi$ ein Prämaß auf $\mA$.
\end{lem}

\begin{proof}
 Nehmen wir an, dass $B = A \times E \ne \emptyset$. Für $x \in X, y \in Y$ gilt
 \[ \ind_A(x) \cdot \ind_E(y) = \ind_{A \times E}(x,y) = \sum_{j=1}^n \ind_{A_j \times E_j} (x,y) = \sum_{j=1}^n \ind_{A_j}(x) \cdot \ind_{E_j}(y). \]
 Wir integrieren bezüglich $x$ und verwenden Satz 2.3.5:
 \[ \begin{aligned}
     \mu(A) \cdot \ind_E (y) 
     &= \int \int_A(x) \diffop \mu(x) \cdot \ind_E(y) \\
     &= \int \sum_{j=1}^n \ind_{A_j} (x) \cdot \ind_{E_j}(y) \diffop \mu(x) \\
     &= \sum_{j=1}^n \mu(A_j) \cdot \ind_{E_j} (y).
    \end{aligned} \]
 Wir integrieren bezüglich $y$:
 \[ \mu(A) \cdot \eta(E) = \sum_{j=1}^n \mu(A_j) \eta(E_j). \]
 Damit ist der Spezialfall bewiesen.
 
 Aus diesem Spezialfall folgt: Betrachten wir zwei Darstellungen einer Menge $B \in \mA$, wobei die eine Darstellung eine Verfeinerung der anderen ist, so sind die Summen gleich.
 
 Der allgemeine Fall folgt nun aus der Tatsache, dass zwei beliebige Darstellungen einer Menge $B \in \mA$ immer eine gemeinsame Verfeinerung besitzen.
 
 Die letzte Aussage folgt daraus, dass die obigen Gleichungen auch für abzählbar viele $A_j$, $E_j$ gelten.
\end{proof}

\begin{defn}
 Nach dem vorliegenden Lemma und Satz 1.7.7 lässt sich $\pi$ zu einem Maß auf $\mM \otimes \mN$ fortsetzen. Dieses Maß heißt das \emph{Produkt} von $\mu$ und $\eta$. Es wird mit $\mu \times \eta$ bezeichnet.
\end{defn}

\begin{rmrk}
 Sind $\mu$ und $\eta$ $\sigma$-endlich, das heißt $X = \bigcup_j A_j$, $Y = \bigcup_j E_j$ mit $\mu(A_j) < \infty$, $\eta(E_j) < \infty$, dann gilt:
 \[ X \times Y = \bigcup_j \bigcup_k (A_j \times E_k), \]
 also $\mu \times \eta$ $\sigma$-endlich.
 
 Nach Satz 1.7.7 ist dann $\mu \times \eta$ das \emph{einzige} Maß auf $\mM \otimes \mN$ mit 
 \[ (\mu \times \eta)(A \times E) = \mu(A) \cdot \eta(E) \]
 für alle Rechtecke $A \times E$.
\end{rmrk}

\begin{rmrk}
 Dieselbe Konstruktion kann man mit endlich vielen Faktoren durch\-führen.
 
 Seien $(X_j, \mM_j, \mu_j)$, $j = 1, \ldots, n$ Maßräume. Unter einem Rechteck verstehen wir Mengen der Form
 \[ A_1 \times \ldots \times A_n, \quad A_j \in \mM_j. \]
 
 Das Analogon von Lemma 2.9.1 bleibt gültig und dieselbe Konstruktion liefert dann ein Maß $\mu_1 \times \ldots \times \mu_n$ auf $\mM_1 \otimes \ldots \otimes \mM_n$ mit
 \[ (\mu_1 \times \ldots \times \mu_n)( A_1 \times \ldots \times A_n ) = \prod_{j=1}^n \mu_j(A_j). \]
 Sind die $\mu_j$ $\sigma$-endlich, so ist auch $\mu_1 \times \ldots \times \mu_n$ $\sigma$-endlich.
 
 Im Falle der Eindeutigkeit gelten die Assoziativgesetze. 
 
 Beispiel: Wenn wir $X_1 \times X_2 \times X_3$ mit $(X_1 \times X_2) \times X_3$ identifizieren, dann gilt
 \[ \mM_1 \otimes \mM_2 \otimes \mM_3 = (\mM_1 \otimes \mM_2) \otimes \mM_3. \]
 Die Gleichheit $\mu_1 \times \mu_2 \times \mu_3 = ( \mu_1 \times \mu_2 ) \times \mu_3$ folgt nun daraus, dass diese Maße auf Mengen der Form $A_1 \times A_2 \times A_3$ übereinstimmen, also wegen der Eindeutigkeit gleich sind.
\end{rmrk}

Im Weiteren betrachten wir nur zwei Faktoren.

Für $E \subseteq X \times Y$, $x \in X$, $y \in Y$ schreiben wir
\[ E_x := \{ y \in Y; (x,y) \in E \}, \quad E_y := \{ x \in X; (x,y) \in E \}. \]

Für eine Funktion $f: X \times Y \to \real$ sei
\[ f_x(y) := f^y(x) := f(x,y). \]

\begin{lem}
 \begin{enumerate}[(i)]
  \item Wenn $E \in \mM \otimes \mN$, dann gilt $E_x \in \mN$ und $E^y \in \mM$ für alle $x \in X$, $y \in Y$.
  \item Ist $f$ $\mM \otimes \mN$-messbar, so ist $f_x$ $\mN$-messbar für alle $x \in X$ und $f^y$ $\mM$-messbar für alle $y \in Y$.
 \end{enumerate}
\end{lem}

\begin{proof}
 \begin{enumerate}[(i)]
  \item Sei 
   \[ \mR := \{ E \subseteq X \times Y, E_x \in \mN, E^y \in \mN \text{ für alle } x,y \}. \]
   Dann enthält $\mR$ alle Rechtecke (so ist $(A \times B)_x = B$ oder $(A \times B)_x = \emptyset$ je nachdem, ob $x \in A$ oder $x \notin A$).
   
   Wegen $(\bigcup_j E_j)_x = \bigcup_j (E_j)_x$ und $(E_x)^C = (E^C)_x$ ist $\mR$ eine $\sigma$-Algebra, also $\mM \otimes \mN \subseteq \mR$.
  \item Folgt aus (i) wegen
   \[ (f_x^{-1})(B) = \{ y \in Y; f(x,y) \in B \} = \{ y: (x,y) \in f^{-1}(B) \} = (\underbrace{f^{-1}(B)}_{\in \mM \otimes \mN}) \in \mN. \]
   Analog für $f^y$. \qedhere
 \end{enumerate}
\end{proof}

\begin{rmrk}
 Auch wenn $\mu$ und $\eta$ vollständig sind, ist das Produktmaß $\mu \times \eta$ im Allgemeinen nicht vollständig.
 
 \textbf{Beispiel:} Nehmen wir an, dass $E \in \pot(Y) \setminus \mN$ und eine nichtleere Menge $A \in \mM$ mit $\mu(A) = 0$ existieren (z.B. für $X=Y=\real$, $\mu = \eta$ Lebesgue-Maß). Nach dem Lemma 2.9.6 ist $A \times E \notin \mM \otimes \mN$, obwohl $A \times E \subseteq A \times Y$ und $(\mu \times \eta)(A \times Y) = 0$.
\end{rmrk}

\begin{defn}
 Eine \emph{monotone Klasse} auf $X$ ist eine Familie $\mC \subseteq \pot(X)$, die abgeschlossen ist bezüglich abzählbarer, monoton wachsender Vereinigungen und abzählbarer, monoton fallender Durchschnitte. Jede $\sigma$-Algebra ist eine monotone Klasse und der Durchschnitt von monotonen Klassen ist eine monotone Klasse. Folglich existiert für eine beliebige Familie $\mE \subseteq \pot(X)$ eine eindeutige kleinste monotone Klasse, die $\mE$ enthält, die durch $\mE$ erzeugte monotone Klasse.
\end{defn}

\begin{lem}
 Ist $\mA$ eine Algebra, so ist die durch $\mA$ erzeugte monotone Klasse $\mC$ gleich der $\sigma$-Algebra $\mM$, die durch $\mA$ erzeugt wird.
\end{lem}

\begin{proof}
 Da $\mM$ eine monotone Klasse ist, die $\mA$ enthält, gilt $\mC \subseteq \mM$. 
 
 Für $E \in \mC$ sei:
 \[ \mC(E) := \{ F \in \mC; E \setminus F, F \setminus E, F \cap E \in \mC \}. \]
 Offenbar sind $\emptyset, E \in \mC(E)$.
 
 Wenn $E \in \mA$, dann gilt $F \in \mC(E)$ für alle $F \in \mA$ (da $\mA$ eine Algebra ist). Dann $E \in \mC(F)$ genau dann, wenn $F \in \mC(E)$. Weiterhin ist $\mC(E)$ eine monotone Klasse. 
 
 Ist $E \in \mA$, dann gilt 
 \[ \mA \subseteq \mC(E) \qRq \mC \subseteq \mC(E), \]
 das heißt wenn $F \in \mC$, dann gilt $F \in \mC(E)$ für alle $E \in \mA$. Also $\mA \subseteq \mC(F)$ für alle $F \in \mC$ und damit $\mC \subseteq \mC(F)$ für alle $F \in \mC$.
 
 Hieraus folgt, dass für alle $E,F \in \mC$ gilt $E \setminus F, F \setminus E, E \cap F \in \mC$. Wegen $X \in \mA \subseteq \mC$ folgt, dass $\mC$ eine Algebra ist. 
 
 Wenn $(E_j) \in \mC$, dann ist auch $F_n := \bigcup_1^n E_j \in \mC$. Damit folgt aus der Abgeschlossenheit von $\mC$ unter monoton wachsenden Vereinigungen, dass $\bigcup_n F_n = \bigcup_j E_j \in \mC$. Folglich ist $\mC$ eine $\sigma$-Algebra, also $\mM \subseteq \mC$.
 
 Aus $\mC \subseteq \mM$ und $\mM \subseteq \mC$ folgt $\mM = \mC$.
\end{proof}

\begin{thm}
 Seien $(X,\mM,\mu)$ und $(Y,\mN,\nu)$ $\sigma$-endliche Maßräume und $E \in \mM \otimes \mN$.
 
 Die Funktionen $x \mapsto \nu(E_x)$ und $y \mapsto \mu(E^y)$ sind messbar auf $X$ bzw. $Y$ und
 \begin{align*}
  (\mu \times\nu)(E) &= \int \nu(E_x) \diffop \mu(x), \tag{1} \\
  (\mu \times\nu)(E) &= \int \mu(E^y) \diffop \nu(y). \tag{2}
 \end{align*}
\end{thm}

\begin{proof}
 \textbf{1. Fall:} $\mu$ und $\nu$ endlich. 
 
 Sei $\mC$ die Menge aller $E \in \mM \otimes \mN$, für welche die Aussage des Satzes gilt.
 
 Rechtecke der From $E= A \times B$, $A \in \mM$, $B \in \mN$, gehören zu $\mC$, da
 \[ \nu(E_x) = \ind_A (x) \nu(B), \qquad \mu(E^y) = \ind_B (y) \mu(A). \]
 Aus der Additivität folgt, dass endliche, disjunkte Vereinigungen von Rechtecken zu $\mC$ gehören.
 
 Nach Lemma 2.9.1 und Lemma 2.9.9 genügt es zu zeigen, dass $\mC$ eine monotone Klasse ist. Ist $(E_n)_{n \in \nat}$ eine monoton wachsende Folge in $\mC$ und $E = \bigcup_{n \in \nat} E_n$, so sind $f_n: y \mapsto \mu(E_n^y)$ messbar und konvergieren punktweise und monoton wachsend gegen $f: y \mapsto \mu(E^y)$.
 
 Folglich ist $f$ messbar und nach dem Satz von der monotonen Konvergenz (m) und wegen $E_n \in \mC$ (c) gilt
 \[ \begin{aligned}
     \int \mu(E^y) \diffop \nu(y) 
     &\overset{(\text{m})}{=} \lim_{n \to \infty} \int \mu((E_n)^y) \diffop \nu(y) \\
     &\overset{(\text{c})}{=} \lim_{n \to \infty} \int (\mu \times \nu)(E_n) \\
     &= (\mu \times \nu)(E).
    \end{aligned} \]
 Analog sieht man, dass
 \[ (\mu \times \nu)(E) = \int \nu(E_x) \diffop \mu(x) \]
 gilt. Also ist $E \in \mC$.
 
 Ist $(E_n)_{n \in \nat}$ eine monoton fallende Folge von $\mC$ und $E = \bigcap_{n \in \nat} E_n$, so gilt $\mu((E_n)^y) \le \mu(X) < \infty$ und $\nu(Y) < \infty$. Daher ist die Funktion $y \mapsto \mu((E_n)^y)$ in $\lebesgue'(\nu)$. Damit liefert analog zum vorhergehenden Schritt der Satz über dominierte Konvergenz, dass $E \in  \mC$ gilt.
 
 \textbf{2. Fall:} $\mu, \nu$ $\sigma$-endlich.
 
 Dann lässt sich $X \times Y$ als Vereinigung einer monoton wachsenden Folge $(X_j,Y_j)_{j \in \nat}$ von Rechtecken mit endlichem Maß darstellen.
 
 Ist $E \in \mM \otimes \mN$, so können wir die Überlegungen des ersten Falls auf $E \cap (X_j \times Y_j)$, $j \in \nat$ anwenden:
 \[ \begin{aligned}
     (\mu \times \nu)(E \cap (X_j \times Y_j)) 
     &= \int \ind_{X_j}(x) \nu( E_x \cap Y_j ) \diffop \mu(x) \\
     &= \int \ind_{Y_j}(y) \mu( E^y \cap X_j ) \diffop \nu(x).
    \end{aligned} \]
 Mit dem Stetigkeitssatz (Anwenden auf die linke Seite) und dem Satz über monotone Konvergenz (rechte Seite) liefert eine Grenzwertbildung $j \to \infty$ die Behauptung.
\end{proof}

\begin{thm}[Fubini-Tonelli]
 Seien $(X,\mM,\mu)$ und $(Y,\mN,\nu)$ $\sigma$-endliche Maßräume.
 
 \begin{enumerate}[(i)]
  \item Tonelli: Ist $f \in \lebesgue^+(X \times Y)$, so sind die Funktionen $g:x \mapsto \int f_x \diffop \nu$ und $h:y \mapsto \int f^y \diffop \mu$ in $\lebesgue^+(X)$ bzw. $\lebesgue^+(Y)$ und
  \begin{align*}
   \int f \diffop (\mu \times \nu)
     &= \int \left[ \int f(x,y) \diffop \nu(y) \right] \diffop \mu(x), \tag{1} \\
     &= \int \left[ \int f(x,y) \diffop \mu(x) \right] \diffop \nu(y). \tag{2}
  \end{align*}
  \item Fubini: Ist $f \in \lebesgue'(\mu \times \nu)$, so gilt $f_x \in \lebesgue'(\nu)$ für fast alle $x \in X$, $f^y \in \lebesgue'(\mu)$ für fast alle $y \in Y$. Die fast überall definierten Funktionen $g:x \mapsto \int f_x \diffop \nu$ und $h:y \mapsto \int f^y \diffop \mu$ sind in $\lebesgue'(X)$ bzw. $\lebesgue'(Y)$ und es gelten die Gleichungen (1) und (2).
 \end{enumerate}
\end{thm}

\begin{proof}
 \begin{enumerate}[(i)]
  \item Ist $f$ Indikatorfunktion, so folgt die Aussage aus dem gerade bewiesenen Satz 2.9.10. Wegen Linearität gilt (i) damit für alle nicht-negativen, einfachen Funktionen.
 
  Sei $f \in \lebesgue^+(X \times Y)$ beliebig. Wir wählen eine Folge $(f_n)_{n \in \nat}$ von nicht-negativen, einfachen Funktionen, die punktweise und monoton wachsend gegen $f$ konvergieren (Satz 2.1.15).
  
  Nach dem Satz über monotone Konvergenz (m) konvergieren die zugehörigen $g_n$ und $h_n$ gegen $g$ und $h$. Folglich sind $g$ und $h$ messbar und es gilt
  \[ \int g \diffop \mu \overset{(\text{m})}{=} \lim_{n \to \infty} g_n \diffop \mu \overset{(1)}{=} \lim_{n \to \infty} \int f_n \diffop (\mu \times \nu) \overset{(\text{m})}{=} \int f \diffop (\mu \times \nu), \]
  \[ \int h \diffop \nu \overset{(\text{m})}{=} \lim_{n \to \infty} h_n \diffop \nu \overset{(2)}{=} \lim_{n \to \infty} \int f_n \diffop (\mu \times \nu) \overset{(\text{m})}{=} \int f \diffop (\mu \times \nu). \]
  \item Aus (i) erhalten wir: Ist $f \in \lebesgue^+ (X \times Y)$ und $\int f \diffop (\mu \times \nu) < \infty$, so gilt fast überall $g < \infty$, $h < \infty$.
  
  Also ist $f_x \in L'(\nu)$ für fast alle $x$ bzw. $f^y \in L'(\mu)$ für fast alle $y$.
  
  Ist nun $f \in L'(\mu \times \nu)$ beliebig, so können wir die gerade gezeigten Aussagen für $(\Re f)^\pm$ bzw. $(\Im f)^\pm$ anwenden.
  \qedhere
 \end{enumerate}
\end{proof}

\begin{rmrk}
 \begin{enumerate}[(a)]
  \item Die Klammern ``$[\,]$'' in (1) und (2) werden gelegentlich auch weggelassen, man schreibt auch
   \[ \int \left[ \int f(x,y) \diffop \mu(x) \right] \diffop \nu(y) 
      = \iint f(x,y) \diffop \mu(x) \diffop \nu(y)
      = \iint f \diffop \mu \diffop \nu. \]
  \item Sei $X = Y = [0,1]$, $\mM = \mN = \borel([0,1])$, $\mu = \lambda |_{\borel([0,1])}$, $\nu$ Zählmaß und $D = \{ (x,y) \in [0,1] \times [0,1] : x = y \}$. 
  
  Dann gilt 
  \[ \iint \ind_D \diffop \lambda \diffop \nu = 0, \qquad \iint \ind_D \diffop \nu \diffop \lambda = 1. \]
  Man kann sogar zeigen, dass
  \[ \int \ind_D \diffop (\lambda \times \nu) = (\lambda \times \nu)(D) = \infty. \]
  
  Also darf die $\sigma$-Additivität für die Anwendbarkeit des Satzes nicht weggelassen werden!
  \item Die Sätze von Fubini und Tonelli werden oft zusammen verwendet. Man möchte häufig die Reihenfolge der Integration in einem Doppelintegral $\iint f \diffop \mu \diffop \nu$ vertauschen. Zuerst zeigt man, dass $\iint |f| \diffop \mu \diffop \nu < \infty$, indem man das Integral nach Tonelli als iteriertes Integral auswertet. Danach wendet man den Satz von Fubini an.
 \end{enumerate}
\end{rmrk}

Eine ``vollständige'' Version des Satzes von Fubini-Tonelli folgt ohne Beweis.
\begin{thm}
 Seien $(X, \mM, \mu)$ und $(Y, \mN, \nu)$ \emph{vollständige} $\sigma$-endliche Maßräume und sei $(X \times Y, \mA, \sigma)$ die Vervollständigung von $(X \times Y, \mM \otimes \mN, \mu \times \nu)$. Ist $f$ $\mA$-messbar und 
 \[ \text{entweder} \quad \text{(a)} \quad f \ge 0 \quad \text{oder} \quad \text{(b)} \quad f \in L'(\sigma), \] 
 so gilt: $f_x$ ist $\mN$-messbar für fast alle $x$ bzw. $f^y$ ist $\mM$-messbar für fast alle $y$. Im Falle (b) sind diese Funktionen fast überall integrierbar.
 
 Weiterhin sind $x \mapsto \int f_x \diffop \nu$ und $y \mapsto \int f^y \diffop \mu$ messbar, im Falle (b) auch integrierbar und
 \begin{align*}
  \int f \diffop \sigma
  &= \int \left[ \int f(x,y) \diffop \nu(y) \right] \diffop \mu(x) \tag{1}\\
  &= \int \left[ \int f(x,y) \diffop \mu(x) \right] \diffop \nu(y). \tag{2}
 \end{align*}
\end{thm}

\section{Aufgaben}
Siehe \verb+Aufgaben-2.10.pdf+.

\section{Das Lebesgue-Integral auf \texorpdfstring{$\real^n$}{IRn}}
\begin{defn}
 Das \emph{Lebesgue-Maß auf $\real^n$} ist die Vervollständigung des Produktmaßes $\lambda \times \ldots \times \lambda$ auf $\borel(\real) \otimes \ldots \otimes \borel(\real)$.
\end{defn}
 
Die $\sigma$-Algebra der $\lambda_n$-messbaren Mengen wird mit $\lebesgue_n$ bezeichnet, ihre Elemente heißen \emph{Lebesgue-messbare Mengen}.
 
Wenn keine Verwechselungsgefahr besteht, werden wir einfach $\lambda$ statt $\lambda_n$ und $\int f(x) \diffop x$ statt $\int f \diffop \lambda_n$ schreiben.
 
Ist $E  = \prod_1^n E_j$ ein Rechteck in $\real^n$, so wird $E_j \subseteq \real$ als \emph{Seite} von $E$ bezeichnet.
\end{document}
