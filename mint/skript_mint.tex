\documentclass[
 a4paper,
 12pt,
 parskip=half
 ]{scrreprt}

\usepackage{../.tex/settings}

\usepackage{../.tex/mathpkgs}
\usepackage{../.tex/mathcmds}

\swapnumbers
\theoremstyle{plain}
\newtheorem{thm}{Satz}[section] % reset theorem numbering for each chapter
\newtheorem{lem}[thm]{Lemma}    

\theoremstyle{definition}
\newtheorem{defn}[thm]{Definition} 
\newtheorem{folg}[thm]{Folgerung} 
\newtheorem{rmrk}[thm]{Bemerkung} 
\newtheorem{deno}[thm]{Bezeichnungen}
\newtheorem{exmp}[thm]{Beispiel} 
\newtheorem{prgp}[thm]{} % Numbered paragraph

\newtheorem*{rmrk*}{Bemerkung}
\newtheorem*{exmp*}{Beispiel}
\newtheorem*{defn*}{Definition}

\numberwithin{equation}{section}

\hypersetup{
  pdftitle={Maß und Integral},
  pdfauthor={Jonas Hippold},
  hidelinks
}

%opening
\title{Vorlesung\\Maß und Integral}
\subtitle{Wintersemester 2016/2017}
\author{Vorlesung: Prof. Dr. Zoltán Sasvári\\Mitschrift: Jonas Hippold}

\begin{document}

\maketitle

\tableofcontents

\section*{Organisatorisches}
 \begin{itemize}
  \item \emph{Aufgaben:} Auf der Vorlesungsseite
  \item \emph{Prüfungsvorleistung:} 50 \% der Punkte in den Pflichtaufgaben und eine vorgetragene Übungslösung in der Übung
  \item \emph{Abgabe der Pflichtaufgaben:} Dienstag, 9:20 Uhr, Briefkasten 
  \item \emph{Schriftliche Prüfung:} Definitionen, Sätze, Beweise \\
   Aufgaben: ähnlich wie in den Übungen \\
   Einschreibung: 09.01.2017 bis eine Woche vor Prüfungstermin
  \item \emph{Literatur:}
   \begin{itemize}
    \item G. B. Folland: Real Analysis
    \item H. Bauer: Wahrscheinlichkeitstheorie und Grundzüge der Maßtheorie
    \item R. L. Schilling: Maß und Integral (2015)
   \end{itemize}
 \end{itemize}

\chapter{Maße}
\section{Einführung}
\begin{prgp}
 Eine sehr alte Aufgabe der Geometrie: Man bestimme Länge, Flächeninhalt, Volumen von gewissen Gebieten, Körpern in $\real$, in der Ebene, im Raum.
 
 Mit Hilfe des Riemann-Integrals kann man dieses Problem für \emph{gewisse} Mengen in $\real^2$ und $\real^3$ lösen: Mengen, die durch glatte Kurven oder Flächen berandet sind.
 
 Mit diesem Zugang kann man aber kompliziertere Mengen nicht behandeln.
\end{prgp}
 
\subsection*{Axiomatischer Zugang}
Sei $d \in \nat$ beliebig. 

Anstelle von Länge, Flächeninhalt oder Volumen sprechen wir vom \emph{Maß} einer Menge.

Welche Eigenschaften sollte ein Maß haben?

Ein Maß $\mu$ soll \emph{jeder Menge} $E \subset \real^d$ eine Zahl $\mu(E) \in [0, \infty]$ zuordnen, so dass
\begin{enumerate}[(i)]
 \item Ist $E_1, E_2, \ldots$ eine endliche oder unendliche Folge von \emph{disjunkten} Mengen $E_j \subset \real^d$, so gilt:
 \[ \mu(E_1 \cup E_2 \cup \ldots ) = \mu(E_1) + \mu(E_2) + \ldots \]
 \item Sind $E$ und $F$ \emph{kongruent}\footnote{$F$ entsteht aus $E$ mit Hilfe von Verschiebung, Spiegelung oder Rotation}, so ist $\mu(E) = \mu(F)$.
 \item Der $d$-dimensionale Einheitswürfel soll das Maß 1 haben:
  \[ \mu([0,1]^d) = 1. \]
\end{enumerate}

Wir zeigen, dass ein solches Maß nicht existiert! Wir betrachten nur den Fall $d=1$, der allgemeine Fall lässt sich analog behandeln.

Wir definieren eine Äquivalenzrelation $\sim$ auf $[0,1)$ durch
\[ x \sim y \qRq x-y \in \rat. \]
Sei $N$ eine Teilmenge von $[0,1)$, die genau ein Element von jeder Äquivalenzklasse enthält. Sei $R := \rat \cap [0,1)$ und für jedes $r \in R$ sei 
\[ N_r := \{ x + r: x \in N \cap [0,1-r) \} \cup \{ x + r - 1: x \in N \cap [1-r,1) \}. \]
Um $N_r$ zu erhalten, wird $N$ um $r$ nach rechts verschoben; der Teil, der dabei $[0,1)$ verlässt, wird um 1 nach links verschoben.

Wir zeigen:
\[ [0,1) = \bigcup_{r \in R} N_r \text{ und } N_r \cap N_s = \emptyset,\, r \ne s. \]

\begin{proof}
 Sei $x \in [0,1)$ beliebig und sei $y$ das Element von $N$, das zur Äquivalenzklasse von $x$ gehört. Dann ist $x \in N_r$, wobei 
 \[ r := \begin{cases}
          x-y, & \text{wenn } x \ge y, \\
          x-y+1 & \text{wenn } x < y.
         \end{cases} \]
 Also ist
 \[ [0,1) = \bigcup_{r \in R} N_r. \]
 
 Ist $x \in N_r \cap N_s$, so sind $x-r$ (oder $x-r+1$) und $x-s$ (oder $x-s+1$) verschiedene Elemente von $N$, die zur selben Äquivalenzklasse gehören. Das ist nicht möglich. Also ist
 \[ N_r \cap N_s = \emptyset,\, r \ne s. \]

 Wir zeigen nun noch:
 \[ \mu( N ) = \mu(N_r),\, r \in R. \]

 Aus den geforderten Eigenschaften von $\mu$ folgt
 \[ \mu(N) \overset{\text{(i)}}{=} \mu( N \cap [0,1-r) ) + \mu( N \cap [1-r, 1) ) \overset{\text{(ii)}}{=} \mu(N_r). \] 
 
 Nun ist
 \[ 1 \overset{\text{(iii)}}{=} \mu( [0,1) ) \overset{\text{(i)}}{=} \sum_{r \in R} \mu( N_r ). \]
 Das ist nicht möglich, da
 \[ \sum_{r \in R} \mu( N_r ) = 
   \begin{cases}
    \infty, & \text{falls } \mu(N) > 0, \\
    0, & \text{falls } \mu(N) = 0.
   \end{cases} \qedhere \]
\end{proof}

Sei $I \ne \emptyset$ eine beliebige Menge und $a_i \in \real$. Was versteht man unter $\sum_{i \in I} a_i$?

Ein Ausweg wäre (i) nur für endliche Folgen zu fordern (was die Nützlichkeit von $\mu$ sehr einschränken würde). Das geht auch nicht!

\begin{thm}[Banach-Tarski, 1924]
 Seien $U$ und $V$ beliebige, nichtleere, beschränkte, offene Mengen in $\real^d$, $d \ge 3$.
 
 Dann gibt es ein $k \in \nat$ und Teilmengen $E_1, \ldots, E_k, F_1, \ldots, F_k$, so dass
 \begin{enumerate}[(i)]
  \item Die $E_j$ sind disjunkt und ihre Vereinigung ist $U$.
  \item Die $F_j$ sind disjunkt und ihre Vereinigung ist $V$.
  \item $E_j$ ist kongruent zu $F_j$ für alle $j$.
 \end{enumerate}
\end{thm}

Das bedeutet: Die Mengen $U$ und $V$ haben mit den obigen Forderungen das selbe Maß, egal wie man $U$ und $V$ wählt.

Folgerung: $\real^d$ enthält Teilmengen, die so seltsam sind, dass kein \emph{geometrisch} vernünftiges Maß für \emph{alle} Teilmengen definieren kann.

Wir geben die Forderung auf, dass das Maß für \emph{alle} Teilmengen definiert werden soll.

\section{Aufgaben}
 Siehe \verb+Aufgaben-1-2.pdf+.

\section{\texorpdfstring{$\sigma$}{Sigma}-Algebren}
Sei $X \ne \emptyset$ eine beliebige Menge. Das Komplement von $A \subset X$ bezüglich $X$ bezeichnet $A^C := X \setminus A$.

\begin{defn}
 Eine Familie $\mA \subset \pot(X)$ heißt eine \emph{$\sigma$-Algebra} (in $X$), wenn
 \begin{enumerate}[(i)]
  \item $X \in \mA$.
  \item $A \in \mA$ $\Rightarrow$ $A^C \in \mA$.
  \item $A_1, A_2, \ldots \in \mA$ $\Rightarrow$ $\bigcup_{i=1}^\infty A_i \in \mA$.
 \end{enumerate}
 $\mA$ heißt eine \emph{Algebra}, wenn (i), (ii) und (iii) mit endlich vielen $A_i$ gelten.
\end{defn}

Einfache Beispiele:
\begin{itemize}
 \item $\mA = \{ X, \emptyset \}$.
 \item $\mA = \pot( X )$.
 \item $\mA = \{ X, \emptyset, A, A^C \}$ mit $A \subset X$.
\end{itemize}

\begin{lem}
 Ist $\mA$ eine Algebra, so gilt
 \begin{enumerate}[(i)]
  \item $\emptyset \in \mA$,
  \item $A \setminus B \subset \mA$, wenn $A,B \subset \mA$,
  \item $\bigcap_{j=1}^n A_j \in \mA$, wenn $A_1, \ldots, A_n \in \mA$.
  \enumeratext{(i)}{Ist $\mA$ eine $\sigma$-Algebra, so gilt auch}
  \item $\bigcap_{j=1}^n A_j \in \mA$, wenn $A_1, \ldots, A_n \in \mA$.
 \end{enumerate}
\end{lem}

\begin{proof}
 (i): $\emptyset = X^C \in \mA$.
  
 (iii) folgt aus (1.3.1.ii), (1.3.1.iii) und 
 \[ \left( \bigcap_{j=1}^n A_j \right)^C = \bigcup_{j=1}^n A_j^C \in \mA. \]
 
 (ii) $A \setminus B = A \cap B^C \overset{\text{(iii)}}\in \mA$.
 
 (iv) wird analog zu (iii) bewiesen.
\end{proof}

\begin{thm}
 Ist eine Algebra abgeschlossen bezüglich abzählbarer \emph{disjunkter} Vereinigungen, so ist sie eine $\sigma$-Algebra.
\end{thm}

\begin{proof}
 Sei $\mA$ eine Algebra, $E_1, E_2, \ldots \in \mA$ und für $k = 1, 2, \ldots$ definiere
 \[ F_k := E_k \setminus \left[ \bigcup_{j=1}^{k-1} E_j \right] = E_k \cap \left[ \bigcup_{j=1}^{k-1} E_j \right]^C \]
 Die $F_k$ gehören zu $\mA$, da $\mA$ eine Algebra ist und $F_k$ durch endlich viele Vereinigungen entsteht. Sie sind disjunkt und
 \[ \bigcup_{j=1}^\infty F_j = \bigcup_{j=1}^\infty E_j \qRq \bigcup_{j=1}^\infty E_j \in \mA, \]
 weil nach Voraussetzung $\mA$ abgeschlossen bezüglich abzählbarer disjunkter Vereinigungen ist.
\end{proof}

\begin{lem}
 Jeder Durchschnitt von $\sigma$-Algebren in $X$ ist selber eine $\sigma$-Algebra.
\end{lem}

\begin{proof}
 Einfaches Nachprüfen der Eigenschaften.
\end{proof}

\begin{folg}
 Zu jeder Familie $E \subset \pot(X)$ existiert eine kleinste, $E$ enthaltende $\sigma$-Algebra. Wir bezeichnen sie mit $\sigma(E)$.
\end{folg}

\begin{proof}
 $\pot(X)$ ist eine $\sigma$-Algebra, die $E$ enthält. Man bilde den Durchschnitt mit allen E enthaltenden $\sigma$-Algebren.
\end{proof}

Man nennt $\sigma(E)$ die \emph{von $E$ erzeugte $\sigma$-Algebra} und $E$ einen \emph{Erzeuger von $\sigma(E)$}.

Beispiel: $E = \{A\}$, $\sigma(E) = \{ X, \emptyset, A, A^C \}$.

\begin{lem}
 Ist $E \subset \sigma(F)$, so gilt $\sigma(E) \subset \sigma(F)$.
\end{lem}

\begin{proof}
 $\sigma(F)$ ist eine $\sigma$-Algebra, die $E$ enthält $\Rightarrow$ sie enthält auch $\sigma(E)$.
\end{proof}

\begin{rmrk}
 Sei $\mA_1 \subset \mA_2 \subset \ldots$ eine unendliche Folge von $\sigma$-Algebren, so dass $\mA_n \ne \mA_{n+1}$ für alle $n$. Dann ist $\bigcup_{n=1}^{\infty} \mA_n$ keine $\sigma$-Algebra!\footnote{Zum Beweis siehe \texttt{UnionsOfSigmaFields.pdf}}
\end{rmrk}

\begin{defn}
 Sei $X$ ein metrischer Raum (oder ein topologischer Raum). Die $\sigma$-Algebra, die durch alle \emph{offenen Mengen} von $X$ erzeugt wird, heißt \emph{Borelsche $\sigma$-Algebra} und wird mit $\borel(X)$ bezeichnet. Die Elemente von $\borel(X)$ heißen \emph{Borelmengen}.
\end{defn}

Ist $\borel(\real) = \pot(\real)$? Nein! Beispiele folgen später.

\section{Aufgaben}
Siehe \verb+Aufgaben-1-4.pdf+.

\section{Maße}
Sei $X \ne \emptyset$ eine beliebige Menge, $\mA$ eine $\sigma$-Algebra in $X$.

\begin{defn}
 Ein \emph{Maß} auf $\mA$ ist eine Abbildung $\mu: \mA \to [0,\infty]$ mit
 \begin{enumerate}[(i)]
  \item $\mu( \emptyset ) = 0$.
  \item Sind die Mengen $A_1, A_2, \ldots \in \mA$ paarweise disjunkt, so gilt
  \[ \mu\left( \bigcup_{i=1}^\infty A_i \right) = \sum_{i=1}^\infty \mu(A_i). \]
  Diese Eigenschaft nennt man \emph{$\sigma$-Additivität}\footnote{Auch möglich: Nur endliche Additivität, aber solche Maße kommen selten vor.}.
 \end{enumerate}
\end{defn}

\begin{rmrk*}
 Auch üblich: Maß auf $(X, \mA)$ (redundant, weil $X \in \mA$) oder einfach Maß auf $X$, wenn klar ist, welche $\sigma$-Algebra gemeint ist.
\end{rmrk*}

\begin{prgp}
 Sei $\mu$ ein Maß auf einer $\sigma$-Algebra in $X$. Die folgenden Bezeichnungen sind üblich:
 \begin{itemize}
  \item $(X,\mA)$: \emph{Messraum},
  \item $(X,\mA,\mu)$: \emph{Maßraum},
  \item die Elemente von $\mA$ heißen \emph{messbare Mengen}.
  \item Im Fall $\mu(X) < \infty$ heißt $\mu$ \emph{endlich}.
  \item Wenn $X = \bigcup_{j=1}^\infty E_j$, wobei $E_j \in \mA$ und $\mu(E_j) < \infty$, dann heißt $\mu$ \emph{$\sigma$-endlich}.
  \item Wenn $E = \bigcup_{j=1}^\infty E_j$, $E_j$ wie oben, dann heißt $E$ \emph{$\sigma$-endlich} (bezüglich $\mu$).
 \end{itemize}
\end{prgp}

\begin{exmp}
 \begin{enumerate}[(i)]
  \item Das \emph{Zählmaß} 
   \[ \mu(A) := \text{ Anzahl der Elemente von } A \in \pot(X) = \mA \]
   und das \emph{Dirac-Maß} 
   \[ \delta_x(A) := \mu(A \cap \{x\}) \text{ in } x \in X, A \in \pot(X) \]
   sind Maße auf $\pot(X)$. Das Dirac-Maß ist endlich, da $\mu(x) = 1$. Das Zählmaß ist genau dann endlich, wenn $X$ endlich ist; es ist genau dann $\sigma$-endlich, wenn $X$ abzählbar ist.
  \item Sei $X$ unendlich, $\mu(A) := 0$, wenn $A$ endlich ist, sonst sei $\mu(A) := \infty$. $\mu$ ist \emph{kein Maß}: Seien $x_1, x_2, \ldots \in X$ eine Folge in $X$, $x_i \ne x_j$ für $i \ne j$, dann ist
  \[ \mu( \{ x_1, x_2, \ldots \} ) \ne \sum_{j=1}^\infty \mu( \{ x_j \} ) = 0. \]
  Allerdings ist $\mu$ endlich additiv.
 \end{enumerate}
\end{exmp}

\begin{thm}
 Sei $(X, \mA, \mu)$ ein Maßraum und $E, F, E_j \in \mA$, $j \in \nat$. 
 \begin{enumerate}[(i)]
  \item Wenn $E \subset F$, dann ist $\mu(E) \le \mu(F)$. \hfill \emph{(Monotonie)}
  \item $\mu \left( \bigcup_{j=1}^\infty E_j \right) \le \sum_{j=1}^\infty \mu(E_j)$. \hfill \emph{(Subadditivität)}
  \item Wenn $E_1 \subset E_2 \subset \ldots$, dann \hfill \emph{(Stetigkeit von unten)}
   \[ \mu \left( \bigcup_{j=1}^\infty E_j \right) = \lim_{j \to \infty} \mu( E_j )  \]
  \item Wenn $E_1 \supset E_2 \supset \ldots$ und $\mu(E_n) < \infty$ für ein $n \in \nat$, dann \hfill \emph{(Stetigkeit von oben)}
   \[ \mu \left( \bigcap_{j=1}^\infty E_j \right) = \lim_{j \to \infty} \mu( E_j )  \]
 \end{enumerate}
\end{thm}

\begin{proof}
 \begin{enumerate}[(i)]
  \item $F \setminus E \in \mA$; $E \cap (F \setminus E) = \emptyset$, also
   \[ \mu(F) = \mu(E) + \underbrace{\mu(F \setminus E)}_{\ge 0} \ge \mu(E). \]
  \item Sei $F_1 = E_1$ und (wie im Beweis von Satz 1.3.3)
   \[ F_k := E_k \setminus \left( \bigcup_{j=1}^\infty E_j \right), \quad k > 1. \]
   $F_k \in \mA$, $F_k \cap F_j = \emptyset$ für $k \ne j$, $F_k \subset E_k$
   \[ \bigcup_{j=1}^\infty F_j = \bigcup_{j=1}^\infty E_j. \]
   Damit folgt unter Verwendung der $\sigma$-Additivität:
   \[ \mu \left( \bigcup_{j=1}^\infty E_j \right) = \mu \left( \bigcup_{j=1}^\infty F_j \right) \overset{\sigma}{=} \sum_{j=1}^\infty \mu( F_j ) \overset{\text{(i)}}{\le} \sum_{j=1}^\infty \mu( E_j ). \]
  \item Sei $E_0 := \emptyset$. Wegen $E_{j-1} \subset E_j$ gilt für $j \in \nat$
   \[ \mu( E_j \setminus E_{j-1} ) = \mu(E_j) - \mu(E_{j-1}) \]
   und
   \[ \bigcup_{j=1}^\infty E_j = \bigcup_{j=1}^\infty(E_j \setminus E_{j-1}), \]
   wobei die Mengen auf der rechten Seite disjunkt sind.
   \begin{align*}
    \mu \left( \bigcup_{j=1}^\infty E_j \right) 
      &\overset{\sigma}{=} \sum_{j=1}^\infty \mu( E_j \setminus E_{j-1}) \\
      &= \lim_{n \to \infty} \sum_{j=1}^n ( \mu( E_j ) - \mu( E_{j-1} ) ) \\
      &= \lim_{n \to \infty} \mu( E_n ).
   \end{align*}
  \item Sei $F_j := E_n \setminus E_j$, $j > n$. Dann $F_{n+1} \subset F_{n+2} \subset \ldots$ und $\mu( E_n )  = \mu( F_j ) + \mu( E_j )$ für $j > n$ sowie
  \[ E_n = \left( \bigcap_{j=n+1}^\infty E_j \right) \cup \left( \bigcup_{j=n+1}^\infty F_j \right). \]
  Damit folgt
  \begin{align*}
   \mu(E_n) 
    &\overset{\text{(iii)}}{=} \mu \left( \bigcap_{j=n+1}^\infty E_j \right) + \lim_{j \to \infty} \mu (F_j) \\
    & = \mu \left( \bigcap_{j=n+1}^\infty E_j \right) + \lim_{j \to \infty} ( \mu( E_n ) - \mu(E_j) ).
  \end{align*}
  Durch Subtraktion von $\mu(E_n)$ folgt die Behauptung. Dabei nutzen wir die Voraussetzung $\mu(E_n) < \infty$! \qedhere
 \end{enumerate}
\end{proof}

\begin{deno}
 Sei $(X, \mA, \mu)$ ein Maßraum. Wenn $E \in \mA$ und $\mu(E) = 0$, dann heißt $E$ eine \emph{Nullmenge}, oder eine \emph{$\mu$-Nullmenge}. Die Vereinigung von abzählbar vielen Nullmengen ist wieder eine Nullmenge\footnote{Das folgt aus der Subadditivität von $\mu$}.
 
 Gilt eine Eigenschaft für alle $x \in X$ bis auf eine Nullmenge, so sagt man, sie gilt \emph{fast überall} oder \emph{$\mu$-fast überall}.
 
 Ist $E$ eine Nullmenge $F \subset E$, so ist auch $F$ eine Nullmenge, vorausgesetzt $F$ ist messbar. Das ist im Allgemeinen nicht so\footnote{Einfaches Beispiel: $\mA = \{ \emptyset, X \}$, $X \ne \emptyset$, $\mu( \emptyset ) = \mu(X) = 0$; Dann ist $X$ eine Nullmenge, aber Teilmengen von $X$ gehören nicht zu $\mA$ und sind daher nicht messbar.}.
 
 Enthält $\mA$ alle Teilmengen von Nullmengen, so heißt $\mA$ \emph{$\mu$-vollständig}.
\end{deno}

\begin{thm}[Vervollständigung von Maßen]
 Sei $(X, \mA, \mu)$ ein Maßraum, $\mathcal{N} := \{ N \in \mA: \mu(N) = 0 \}$ die Menge der Nullmengen in $\mA$ und 
 \[ \obar{\mA} := \{ E \cup F : E \in \mA, F \subset N \text{ für ein } N \in \mathcal{N} \}. \]
 Dann ist $\obar{\mA}$ eine $\sigma$-Algebra und $\mu$ lässt sich eindeutig zu einem vollständigen Maß $\obar{\mu}$ auf $\obar{\mA}$ fortsetzen.
\end{thm}

\begin{proof}
 $\obar{\mA}$ ist eine $\sigma$-Algebra: Es ist klar, dass $\obar{\mA}$ abgeschlossen ist bezüglich abzählbarer Vereinigungen\footnote{Betrachte die Folgen $(E_j)_{j \in \nat} \subset \mA$ und $(F_j)_{j \in \nat}, F_j \subset N_j, \mu(N_j) = 0$. Es gilt $\bigcup E_j \in \mA$, weil $\mA$ eine $\sigma$-Algebra ist. Die abzählbare Vereinigung von Nullmengen ist ebenfalls wieder eine Nullmenge, also $(\bigcup E_j) \cup (\bigcup F_j) \in \obar{\mA}$.}.
 
 Noch zu zeigen: $A = E \cup F \in \obar{\mA} \Rightarrow A^C \in \obar{\mA}$ (wobei $F \subset N, \mu(N) = 0$). Wir dürfen annehmen, dass $E \cap N = \emptyset$ und damit $E \cap F = \emptyset$ (sonst ersetzen wir wir $F$ und $N$ durch $F \setminus E$ bzw. $N \setminus E$). Es gilt
 \begin{align*}
  E \cup F &= ( E \cup N ) \cap (N^C \cup F) \\
  (E \cup F)^C &= \underbrace{( E \cup N )^C}_{\in \mA} \cup \underbrace{(N \setminus F)}_{\in \mathcal{N}} \in \obar{\mA}. \qedhere
 \end{align*}
\end{proof}

\subsection*{Fortsetzung auf $\obar{\mA}$}
 Für $E \cup F \in \obar{\mA}$ sei $\obar{\mu}(E \cup F) := \mu(E)$. Wir müssen zeigen, dass diese Definition korrekt ist. Nehmen wir an, dass 
 \[ E_1 \cup F_1 = E_2 \cup F_2, \quad F_j \subset N_j \in \mathcal{N}. \]
 Dann $E_1 \subset E_2 \cup N_2$ $\Rightarrow$ $\mu(E_1) \le \mu(E_2) + \mu(N_2) = \mu(E_2)$ wegen der Monotonie und Subadditivität von $\mu$. Genauso sehen, wir dass $\mu(E_2) \le \mu(E_1)$ $\Rightarrow$ $\mu(E_1) = \mu(E_2)$.
 
 Es ist leicht zu sehen,  dass $\obar{\mu}$ ein vollständiges Maß ist und die Fortsetzung eindeutig ist (siehe Aufgabe 1.6.4).

\section{Aufgaben}
Siehe \verb+Aufgaben-1-6.pdf+.

\section{Äußere Maße}
Ziel des Abschnitts: Konstruktion von Maßen; $X \ne \emptyset$ beliebige Menge

\begin{defn}
 Ein \emph{äußeres Maß} auf $\pot(X)$ ist eine Abbildung $\mu^* : \pot(X) \to [0,\infty]$ mit den folgenden Eigenschaften:
 \begin{enumerate}[(i)]
  \item $\mu^*(\emptyset) = 0$,
  \item $\mu^*( A ) \le \mu^*(B)$, wenn $A \subset B \subset X$,
  \item $\mu^*( \bigcup_{j=1}^\infty A_j ) \le \sum_{j=1}^\infty A_j$, $A_j \in \pot(X)$.
 \end{enumerate}
\end{defn}

\begin{lem}
 Seien $\mE \subset \pot(X)$ und $\rho: \mE \to [0, \infty]$ so, dass $\emptyset, X \in \mE$ und $\rho(\emptyset) = 0$. Für $A \subset X$ definieren wir
 \[ \mu^*(A) := \inf \left\{ \sum_{j=1}^\infty \rho(E_j) : E_j \in \mE, A \subset \bigcup_{j=1}^\infty E_j \right\}. \]
 Dann ist $\mu^*$ ein äußeres Maß.
\end{lem}

\begin{exmp*}
 $X = \real$, $\mE$: Intervalle $(a,b)$, $\rho((a,b)) := b-a$ oder $X = \real^2$, $\mE$: Rechtecke $E = (a_1, b_1) \times (a_2, b_2)$, $\rho(E) := (b_1 - a_1) \cdot (b_2 - a_2)$.
\end{exmp*}

\begin{proof}
 Die Definition von $\mu^*$ ist sinnvoll, da $X \in \mE$. Es gilt: $\mu^*(\emptyset) = 0$ (man nehme $E_j := \emptyset$).
 
 1.7.1(ii) ist klar.
 
 Subadditivität: Seien $A_1, A_2, \ldots \in \pot(X)$, $A:= \bigcup_{j=1}^\infty A_j$ und $\eps > 0$. Für $j, k \in \nat$ wählen wir $E_j^k \in \pot(X)$, so dass\footnote{Wir können solche Mengen immer finden, weil $\mu^*$ als Infimum definiert ist.}
 \[ A_j \subset \bigcup_{k=1}^\infty E_j^k, \quad \text{und} \quad \sum_{k=1}^\infty \rho( E_j^k) \le \mu^* (A_j) + \frac{\eps}{2^j}. \]
 Es gilt $A \subset \bigcup_{j,k} E_j^k$ und
 \[ \sum_{j=1}^\infty \sum_{k=1}^\infty \rho( E_j^k ) \le \sum_{j=1}^\infty \mu^*(A_j) + \eps. \]
 Folglich $\mu^*(A) \le \sum_{j=1}^\infty \mu^*(A_j) + \eps$. Da $\eps > 0$ beliebig ist, folgt die Aussage.
\end{proof}

\begin{defn}
 Sei $\mu^*$ ein äußeres Maß auf $\pot(X)$. Eine Menge $A \subset X$ heißt \emph{$\mu^*$-messbar}, wenn
 \begin{enumerate}[(i)]
  \item $\mu^*(S) \ge \mu^*( S \cap A ) + \mu^*( S \setminus A )$ 
 \end{enumerate}
 gilt für alle $S \subset X$. Die Familie aller $\mu^*$-messbaren Mengen wird mit $\meas_{\mu^*}$ bezeichnet.
\end{defn}

\begin{rmrk*}
 Aus (i) folgt, dass
 \[ \mu^*(S) = \mu^* (S \cap A) + \mu^* ( S \setminus A ), \quad S \subset X \]
 wegen der Subadditivität 1.7.1(iii).
\end{rmrk*}

\begin{thm}[Caratheodory]
 Die Familie $\meas_{\mu^*}$ ist eine $\sigma$-Algebra und $\mu^*$ ist ein Maß auf $\meas_{\mu^*}$.
 
 Ist $\mu^*(A) = 0$, so gilt $A \in \meas_{\mu^*}$.
\end{thm}

\begin{proof}
 siehe \verb+Beweis-Satz-1-7-4.pdf+.
\end{proof}

\begin{defn}
 Ein \emph{Prämaß} auf einer Algebra $\mA$ ist eine Abbildung $\mu: \mA \to [0, \infty]$ mit 
 \begin{enumerate}[(i)]
  \item $\mu(\emptyset) = 0$.
  \item Für beliebige disjunkte Mengen $A_1, A_2, \ldots \in \mA$ mit $\bigcup_{j=1}^\infty A_j \in \mA$ gilt
   \[ \mu \left( \bigcup_{j=1}^\infty A_j \right) = \sum_{j=1}^\infty \mu( A_j ). \]
 \end{enumerate}
\end{defn}

\begin{rmrk*}
 $\mu$ ist monoton, also für $A, B \in \mA, A \subset B$ gilt $\mu(A) \le \mu(B)$. Das folgt aus $B = A \cup (B \setminus A)$.
\end{rmrk*}

Nach Lemma 1.7.2 wird durch
\[ \mu^*(A) := \inf \left\{ \bigcup_{j=0}^\infty \mu(E_j) : E_j \in \mA, A \subset \bigcup_{j=0}^\infty E_j \right \} \]
ein äußeres Maß  auf $\pot(X)$ definiert.

\begin{lem}
 Ist $\mu$ ein Prämaß auf einer Algebra $\mA$, so gilt
 \begin{enumerate}[(i)]
  \item $\mA \subset \meas_{\mu^*}$,
  \item $\mu^* |_\mA = \mu$ (das heißt $\mu^*$ ist die Fortsetzung von $\mu$).
 \end{enumerate}
\end{lem}

\begin{proof}
 (ii) Sei $E \in \mA$ und $E \subset \bigcup_{j=1}^\infty A_j$ wobei $A_j \in \mA$. Wir setzen für $n \in \nat$
 \[ B_n := E \cap \left( A_n \setminus \bigcup_{j = 1}^{n-1} A_j \right). \]
 Dann sind die $B_n$ disjunkte Elemente von $\mA$, $B_n \subset A_n$. Es gilt $\bigcup_{j=1}^\infty B_j = E$ wegen $E \subset \bigcup_{j=1}^\infty A_j$. Also folgt
 \[ \mu(E) = \sum_{j=1}^\infty \mu(B_j) \overset{\text{Mon.}}{\le} \sum_{j=1}^\infty \mu(A_j) \qRq \mu(E) \le \mu^*(E). \]
 Die Ungleichung $\mu^*(E) \le \mu(E)$ folgt aus $E \subset \bigcup_{j=1}^\infty A_j$, wobei $A_1 = E$ und $A_j = \emptyset$ für $j > 1$. $\mu^*(E) \le \mu(E)$ gilt jetzt, weil $\mu^*$ als Infimum definiert ist.
 
 Damit folgt $\mu^*(E) = \mu(E)$.
 
 (i) Seien $A \in \mA$, $S \subset X$ und $\eps > 0$. Wir wählen eine Folge $\{ B_j \}_{j=1}^\infty \subset \mA$ mit $S \subset \bigcup_{j=1}^\infty B_j$ und $\sum_{j=1}^\infty B_j \le \mu^*(S) + \eps$.
 \begin{align*}
  \mu^*(S) + \eps &\ge \sum_{j=1}^\infty \mu(B_j \cap A) + \sum_{j=1}^\infty \mu(B_j \cap A^C ) \\
  \intertext{(Das folgt aus der Additivität von $\mu$, weil $B_j \cap A$ und $B_j \cap A^C$ disjunkt sind.)}
  &\ge \mu^*( S \cap A ) + \mu^*( S \cap A^C )
 \end{align*}
 Weil $\eps$ beliebig war, ist
 \[ \mu^*(S) \ge \mu^*(S \cap A) + \mu^*(S \cap A^C), \]
 also folgt $A$ ist $\mu^*$-messbar.
\end{proof}

\begin{thm}
 Seien $\mA \subset \pot(X)$ eine Algebra, $\mu$ ein Prämaß auf $\mA$; $\meas := \sigma(\mA)$ und $\obar{\mu} := \mu|_\meas$. Dann gilt
 \begin{enumerate}[(i)]
  \item $\obar{\mu}$ ist ein Maß auf $\meas$ mit $\obar{\mu}|_\mA = \mu$.
  \item Ist $\nu$ ein Maß auf $\meas$ mit $\nu |_\mA = \mu$, so gilt $\nu(E) \le \obar{\mu}(E)$ für alle $E \in \meas$, wobei im Fall $\obar{\mu}(E) < \infty$ sogar Gleichheit gilt\footnote{Das heißt für Mengen mit endlichem Maß ist die Fortsetzung eindeutig.}.
  \item Ist $\mu$ $\sigma$-endlich, so ist $\obar{\mu}$ die \emph{einzige} Fortsetzung von $\mu$ zu einem Maß auf $\meas$.
 \end{enumerate}
\end{thm}

\begin{proof}
 \begin{enumerate}[(i)]
  \item Folgt aus Satz 1.7.4 (Caratheodory) und aus Lemma 1.7.6.
  \item Sei $E \in \meas$, $E \subset \bigcup_{j=1}^\infty A_j$, wobei $A_j \in \mA$. Dann gilt
  \[ \nu(E) \overset{\text{Subadd.}}{\le} \sum_{j=1}^\infty \nu(A_j) \overset{\text{Ann.}}{=} \sum_{j=1}^\infty \mu(A_j) \qRq \nu(E) \overset{\text{Def. von }\obar{\mu}}{\le} \obar{\mu}(E). \]
  Für $A := \bigcup_{j=1}^\infty A_j$ gilt
  \[ \nu(A) \overset{\text{Stet.}}{=} \lim_{n \to \infty} \nu \left( \bigcup_{j=1}^n A_j \right) = \lim_{n \to \infty} \mu \left( \bigcup_{j=1}^n A_j \right) \overset{\text{Stet.}}{=} \obar{\mu}(A), \]
  wobei ``Stet.'' für ``Stetigkeit von unten'' steht. 
  
  Im Falle $\obar{\mu}(E) < \infty$ können wir die $A_j$ so wählen, dass
  \[ \obar{\mu}(A) < \obar{\mu}(E) + \eps \qRq \obar{\mu}( A \setminus E ) < \eps \]
  und
  \begin{align*}
   \obar{\mu}(E) \le \obar{\mu}(A) = \nu(A) &= \nu(E) + \nu(A \setminus E) \\
   &\le \nu(E) + \obar{\mu}(A \setminus E) \\
   &\le \nu(E) + \eps.
  \end{align*}
  Weil $\eps$ beliebig, folgt $\obar{\mu}(E) = \nu(E)$.
  \item Sei $X = \bigcup_{j=1}^\infty A_j$, wobei $\mu(A_j) < \infty$. O.B.d.A. seien die $A_j$ disjunkt. Dann gilt für alle $E \in \meas$
  \[ \obar{\mu}(E) = \sum_{j=1}^\infty (E \cap A_j) \overset{\text{(ii)}}{=} \sum_{j=1}^\infty \nu(E \cap A_j) = \nu(E), \]
  das heißt $\obar{\mu} = \nu$ auf $\meas$. \qedhere
 \end{enumerate}
\end{proof}

Zusammenfassung:
\begin{itemize}
 \item Maß $\mu$ auf $\sigma$-Algebra
 \item Äußeres Maß $\mu^*$ auf $\pot(X)$ $\rightarrow$ Einschränkung auf die $\sigma$-Algebra von messbaren Mengen, dort Maß 
 \item Prämaß auf einer Algebra $\rightarrow$ Äußeres Maß $\rightarrow$ Maß auf $\sigma$-Algebra.
\end{itemize}

\section{Aufgaben}
Siehe \verb+Aufgaben-1-8.pdf+.

\section{Produkt von \texorpdfstring{$\sigma$}{Sigma}-Algebren}
Bezeichnungen in diesem Abschnitt:
\begin{itemize}
 \item $A \ne \emptyset$ eine Indexmenge,
 \item $\{ X_\alpha \}$, $\alpha \in A$ eine Familie von nichtleeren Mengen,
 \item $X := \prod_{\alpha \in A} X_\alpha = \{ f:A \to \bigcup_{\alpha \in A} X_\alpha$ mit $f( \alpha ) \in X_\alpha,$ für alle $\alpha \in A \}$.
 \item Falls $X_\alpha = Y$ für alle $\alpha$, dann ist $X = Y^A$ die Menge aller Funktionen von $A$ nach $Y$.
 \item Die Elemente von $X$ werden mit $X = (X_\alpha) = (X_\alpha)_{\alpha \in A} \in X$ bezeichnet.
 \item $\Pi_\alpha : X \to X_\alpha$ sind die \emph{Koordinatenabbildungen} $\Pi_\alpha(X) = X_\alpha$, zum Beispiel $A = \{ 1, 2 \}$, $X_1 = X_2 = \real$, $X = \real^2 = \real \times \real$, $\Pi_1(X) = X_1$.
 \item $\meas_\alpha$ eine $\sigma$-Algebra auf $X_\alpha$.
\end{itemize}

\begin{exmp*}
 $A = \{ 1,2 \}$, $X_1 = X_2 = \real$, $X_1 \times X_2 = \real^2$, $\meas_1, \meas_2 = \borel( \real )$.
\end{exmp*}

\begin{defn}
 Die \emph{Produkt-$\sigma$-Algebra} $\bigotimes_{\alpha \in A} \meas_\alpha$ auf $X$ ist die $\sigma$-Algebra, die durch
 \[ \{ \Pi_\alpha^{-1}( E_\alpha ) : E_\alpha \in \meas_\alpha, \alpha \in A \} \]
 erzeugt wird.
\end{defn}

\[ \Pi_\alpha^{-1} ( E_\alpha ) = \prod_{\beta \in A} E_\beta, \]
wobei $E_\beta = X_\alpha$, $\alpha \ne \beta$.
 
Im Falle $A= \{ 1, 2, \ldots, n \}$ schreiben wir $\bigotimes_1^n \meas_j$ oder $\meas_1 \otimes \cdots \otimes \meas_n$.

\begin{lem}
 Ist $A$ abzählbar, so wird $\bigotimes_{\alpha \in A} \meas_\alpha$ durch
 \[ \left\{ \prod_{\alpha \in A} E_\alpha : E_\alpha \in \meas_\alpha \right\} \]
 erzeugt.
\end{lem}

\begin{proof}
 Wenn $E_\alpha \in \meas_\alpha$, dann 
 \[ \Pi_\alpha^{-1}(E_\alpha) = \prod_{\beta \in A} E_\beta, \]
 wobei $E_\beta = X_\beta$, $\beta \ne \alpha$. Andererseits gilt
 \[ \prod_{\alpha \in A} E_\alpha = \bigcap_{\alpha \in A} \Pi^{-1}_\alpha (E_\alpha). \]
 Das ist ein abzählbarer Durchschnitt nach Voraussetzung für $A$. Die Aussage folgt deshalb aus Lemma 1.3.6. Wir können die Erzeuger der einen $\sigma$-Algebra durch die Erzeuger der anderen ausdrücken. Folglich müssen die erzeugten $\sigma$-Algebren identisch sein.
\end{proof}

\begin{lem}
 Sei $\mE_\alpha$ Erzeuger für $\meas_\alpha$, $\alpha \in A$. Dann ist
 \[ \mF_1 := \{ \Pi_\alpha^{-1} (E_\alpha) : E_\alpha \in \mE_\alpha, \alpha \in A \} \]
 Erzeuger für $\bigotimes_{\alpha \in A} \meas_\alpha$. Ist $A$ abzählbar und $X_\alpha \in \mE_\alpha$, so ist auch
 \[ \mF_2 := \left\{ \prod_{\alpha \in A} E_\alpha : E_\alpha \in \mE_\alpha \right\} \]
 Erzeuger für $\bigotimes_{\alpha \in A} \meas_\alpha$.
\end{lem}

\begin{proof}
 Offensichtlich gilt $\sigma(\mF_1) \subset \bigotimes_{\alpha \in A} \meas_\alpha$, da $\mE_\alpha \subset \meas_\alpha$. Andererseits ist 
 \[ \{ E \subset X_\alpha : \Pi_\alpha^{-1} (E) \in \sigma( \mF_1 ) \} \]
 eine $\sigma$-Algebra (Nachweis ist Übungsaufgabe), die $\mE_\alpha$ und damit auch $\meas_\alpha$ enthält. Das heißt $\Pi_\alpha^{-1}( E ) \in \sigma ( \mF_1 )$ für alle $E \in \meas_\alpha$, $\alpha \in A$, also
 \[ \bigotimes_{\alpha \in A} \meas_\alpha \subset \sigma ( \mF_1 ). \]
 
 Die zweite Aussage folgt aus der ersten genauso wie in Lemma 1.9.2.
\end{proof}

\begin{defn*}
 Seien $X_1, \ldots, X_n$ metrische Räume und $X = \prod_1^n X_j$. Für $x = (x_1, \ldots, x_n) \in X$, $y = (y_1, \ldots, y_n) \in X$ ist $x_j, y_j \in X_j$. Also existiert eine Metrik $d_j(x_j,y_j)$, die für jede Komponente eine andere sein kann, da $X$ ein Produkt metrischer Räume ist. Wir definieren die \emph{Produktmetrik}
 \[ d(x,y) := \max_{1 \le j \le n} d_j(x_j,y_j), \]
 also der größte Abstand von $x$ und $y$.
\end{defn*}
 
\begin{thm}
 Seien $X_1, \ldots, X_n$ metrische Räume und $X = \prod_1^n X_j$ der metrische Raum mit der Produktmetrik. Dann ist $\bigotimes_1^n \borel(X_j) \subset \borel(X)$. Sind die $X_j$ separabel\footnote{Das heißt es exisitiert eine höchstens abzählbare Teilmenge, die in diesen Räumen dicht liegt. Zum Beispiel ist $\real^n$ für alle $n \in \nat$ separabel, da $\rat^n$ abzählbar ist und dicht in $\real^n$ liegt. Allerdings ist $\real^n$ mit der diskreten Metrik nicht separabel!}, so gilt $\bigotimes_1^n \borel(X_j) = \borel(X)$.
\end{thm}

\begin{proof}
 Nach Lemma 1.9.3 wird $\bigotimes_1^n \borel(X_j)$ erzeugt durch die Mengen $\Pi_j^{-1}( U_j )$, $1 \le j \le n$, wobei $U_j \subset X_j$ offen ist. Da diese Mengen offen sind, folgt nach Lemma 1.3.6, dass $\bigotimes_1^n \borel(X_j) \subset \borel(X)$.
 
 Wir beweisen noch die zweite Aussage. Sei $D_j := \{ x_j^k \}_{k=1}^\infty$ eine dichte Teilmenge von $X_j$ und sei $\mE_j$ die (abzählbare) Familie aller Kugeln mit rationalem Radius und Mittelpunkt $x_j^k$ für ein $j$ und $k$. Dann ist jede offene Menge in $X_j$ eine abzählbare Vereinigung von Mengen aus $\mE_j$. Die Menge aller Punkte in $X$ mit Koordinaten aus $\bigcup D_j$ ist abzählbar und dicht in $X$ (Nachweis Übung). $\mE_j$ ist ein Erzeuger für $\borel(X_j)$ und nach Lemma 1.9.3 gilt $\bigotimes_1^n \borel(X_j) = \borel(X)$.
\end{proof}

\begin{folg}
 $\bigotimes_1^d \borel(\real) = \borel(\real^d)$.
\end{folg}


\end{document}