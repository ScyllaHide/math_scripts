\setcounter{secnumdepth}{1}
\section{Die Struktursätze und Zentralkollineationen}
\setcounter{secnumdepth}{0}

\begin{thm}[Koordinatierungssatz, 1. Hauptsatz der projektiven Geometrie]
 Sei $\proj$ ein desarguesscher projektiver Raum. Dann gibt es einen Schiefkörper $K$, einen $K$-Vektorraum $V$ und eine Kollineation\footnotemark $\alpha: \proj \to \proj(V)$ (das heißt $\proj \simeq \proj(V)$). 
\end{thm}
\footnotetext{Def. siehe Seite \pageref{def:kollineation}.}

Der Beweis des Satzes ist sehr lang und wird daher weggelassen.

\subsection{Darstellung von Kollineationen durch bijektive semilineare Abbildungen}
\begin{defn*}
 Seien $V, V'$ Vektorräume über $K$ bzw. $K'$ (Schiefkörper). Eine \emph{semilineare Abbildung} $\varphi: V \to V'$ bezüglich eines Körperisomorphismus $\alpha: K \to K'$ ist eine Abbildung, sodass für alle $x,y \in V$ gilt
 \[ \varphi(x+y) = \varphi(x) + \varphi(y) \]
 und für alle $x \in V$, $\lambda \in K$ gilt
 \[ \varphi( \lambda x ) = \alpha(\lambda) \varphi(x). \]
\end{defn*}

\begin{exmp*}
 $\varphi: \complex^3 \to \complex^3$ mit
 \[ \varphi\left( \begin{pmatrix} x_1 \\ x_2 \\ x_3 \end{pmatrix} \right) = \begin{pmatrix} \obar{x_1} \\ \obar{x_2} \\ \obar{x_3} \end{pmatrix} \]
 ist eine semilineare Abbildung bezüglich der komplexen Konjugation.
 \[ \varphi\left( \lambda \begin{pmatrix} x_1 \\ x_2 \\ x_3 \end{pmatrix} \right) = \obar{\lambda} \begin{pmatrix} \obar{x_1} \\ \obar{x_2} \\ \obar{x_3} \end{pmatrix} = \obar{\lambda} \varphi\left( \lambda \begin{pmatrix} x_1 \\ x_2 \\ x_3 \end{pmatrix} \right). \]
\end{exmp*}

\begin{thm}
 Seien $V, V'$ Vektorräume über Schiefkörpern $K$ bzw. $K'$ mt $\dim V = \dim V' \ge 3$. Sei $\varphi : V \to V'$ semilinear bezüglich eines Körperisomorphismus $\alpha: K \to K'$ und bijektiv, dann
 \begin{enumerate}[a)]
  \item $\obar{\varphi} : \proj(V) \to \proj(V')$ mit $\obar{\varphi}( [x] ) := [\varphi(x)]$ ist eine Kollineation\footnotemark, wenn $x, y$ linear unabhängig sind.
  \item Falls $\varphi': V \to V'$ ebenfalls semilinear bezüglich $\alpha': K \to K'$ und bijektiv ist mit $\obar{\varphi}' = \obar{\varphi}$, dann gibt es genau ein $b \in K' \setminus \{ 0 \}$, sodass $\varphi'(x) = b \cdot \varphi(x)$ für alle $x \in K$ und $\alpha'(\lambda) = b \alpha(\lambda) b^{-1}$.
 \end{enumerate}
\end{thm}
\footnotetext{Das heißt $\obar{\varphi}([x,y]) = [ \varphi(x), \varphi(y)]$.}

\begin{proof}
 \begin{enumerate}[a)]
  \item $\obar{\varphi}$ ist wohldefiniert:
   \[ \obar{\varphi}( [\lambda x] ) = [\alpha(\lambda) \varphi(x) ] = [\varphi(x)] \]
   für $\lambda \in K \setminus \{0\}$.
   \[ \begin{aligned}
      \varphi(Kx + Ky )
      &= \{ \varphi( \lambda x + \mu y) : \lambda, \mu \in K \} \\
      &= \{ \alpha( \lambda ) \varphi(x) + \alpha( \mu ) \varphi(y) \} \\
      &= K' \varphi(x)  + K' \varphi(y),
      \end{aligned} \]
   $\obar{\varphi}$ bildet also Geraden in $\proj(V)$ auf Geraden in $\proj(V')$ ab.
  \item Eindeutigkeit bis auf Vielfache: $\eta := \varphi^{-1} \circ \varphi_ V \to V$ ist eine bijektive semilineare Abbildung, bei der jeder Vektor auf ein Vielfaches ($\ne 0$) abgebildet wird.
  
  Seien $x,y$ linear unabhängig. Dann folgt, dass es $\lambda_1, \lambda_2, \lambda_3 \in K \setminus \{ 0 \}$ gibt mit
  \[ \eta(x) = \lambda_1 x, \quad \eta(y) = \lambda_2 y, \quad \eta(x+y) = \lambda_3 (x+y)  = \eta(x) + \eta(y) = \lambda_1 x + \lambda_2 y, \]
  also $\lambda_1 = \lambda_2 = \lambda_3$. Damit ist der Faktor für linear unabhängige Vektoren derselbe. Somit haben wir für alle $x,y \in V$ denselben Faktor, das zu $x,y$ einen linear unabhängigen Vektor $z$ gibt. Der Faktor $b$ ist eindeutig bestimmt, da für $x \ne 0$ gilt
  \[ \eta(x) = bx = b'x \qRq b = b'. \]
  Also ist $\eta = b \cdot \id$ und damit folgt $\varphi' = b \cdot \varphi$.
  
  Ferner ist
  \[ \varphi( \lambda x ) = \alpha'(\lambda) \varphi'(x) = \alpha'(\lambda) b \varphi(x) \]
  und
  \[ b \varphi(\lambda x) = b \alpha(\lambda) \varphi(x), \]
  daraus folgt für $x \ne 0$
  \[ \alpha'(\lambda) = b \alpha( \lambda ) b^{-1}. \qedhere \]
 \end{enumerate}
\end{proof}

$\obar{\varphi}$ injektiv: Falls $[x] \ne [y]$, also $x$ und $y$ linear unabhängig, folgt wegen der Bijektivität und Semilinearität von $\varphi$, dass $\varphi(x)$ und $\varphi(y)$ linear unabhängig sind. Damit gilt $[\varphi(x)] \ne [\varphi(y)]$.

\begin{rmrk*}
 Falls $\varphi_1 : V \to V'$ semilinear bezüglich $\alpha_1$ und $\varphi_2: V' \to V''$ semilinear bezüglich $\alpha_2$ ist, dann ist $\varphi_2 \circ \varphi_1: V \to V''$ semilinear bezüglich $\alpha_2 \circ \alpha_1$.
\end{rmrk*}

\begin{thm}[2. Hauptsatz der projektiven Geometrie]
 \textbf{Darstellung von Kollineationen durch semilineare Abbildungen.} \\
 Seien $V, V'$ Vektorräume über Schiefkörpern $K$ bzw. $K'$ mit $3 \le \dim V = \dim V' < \infty$. Sei $\psi: \proj(V) \to \proj(V')$ eine Kollineation. Dann gibt es eine eindeutige bijektive Abbildung $\varphi: V \to V'$, die semilinear ist bezüglich eines Körperisomorphismus $\alpha: K \to K'$, sodass $\obar{\varphi} = \psi$,
 \[ \obar{\varphi}([x]) = [\varphi(x)] = \psi( [x] ) \quad \forall [x] \in \proj(V). \]
\end{thm}

\subsection{Zentralkollineationen}
\begin{defn*}
 Eine Kollineation $\alpha: \proj \to \proj$ heißt eine \emph{Zentralkollineation}, falls es eine Hyperebene $H$ (\emph{Achse} von $\alpha$) gibt mit folgenden Eigenschaften:
 \begin{enumerate}
  \item Für alle $x \in H$ gilt $\alpha(x) = x$.
  \item Jede Gerade $g$ durch $z$ wir auf sich selbst abgebildet, das heißt $\alpha(g) = g$.
 \end{enumerate}
\end{defn*}

\begin{exmp*}[Zentralkollineation mit $z \notin H$]
 $V = \tilde{V} \times K$, $\begin{pmatrix} x \\ a \end{pmatrix} \in V$. $x \in \tilde{V}$, $a \in K$, $K$ Schiefkörper. Sei $H = [\tilde{V} \times \{ 0 \}]$, $z = \left[ \begin{pmatrix} 0 \\ 1 \end{pmatrix} \right]$, $0 \in \tilde{V}$. 
 
 Die Abbildung $\varphi : \tilde{V} \times K \to \tilde{V} \times K$ mit
 \[ \varphi\left( \begin{pmatrix} x \\ a \end{pmatrix} \right) := \begin{pmatrix} \lambda x \\ a \end{pmatrix} \]
 ist bijektiv und $\obar{\varphi}$ ist eine Zentralkollineation mit Achse $H$ und Zentrum $z$.
\end{exmp*}

\begin{exmp*}[Zentralkollineation mit $z \in H$]
 Bezeichnungen wie oben. Sei $H = [\tilde{V} \times \{ 0 \}]$, $z = \left[ \begin{pmatrix} t \\ 0 \end{pmatrix} \right]$, $t \ne 0 \in \tilde{V}$.
 
 Für $\lambda \in K \setminus \{ 0 \}$ beliebig gilt: Die Abbildung $\varphi: V \to V$ mit
 \[ \varphi\left( \begin{pmatrix} x \\ a \end{pmatrix} \right) := \begin{pmatrix} x + \lambda at \\ a \end{pmatrix} \]
 ist eine lineare, bijektive Abbildung und $[\varphi]$ ist eine Zentralkollineation mit Achse $H$ und Zentrum $z$.
 
 $\varphi( (x, 0)^T ) = ( x, 0 )^T$. Sei $g$ eine Gerade durch $[ (x, 0)^T ]$. Sie ist von der Form $g = K(t,0)^T + K(x,1)^T$ und
 \[ \obar{\varphi}(g) = \left[ K \begin{pmatrix} t \\ 0 \end{pmatrix} + K \begin{pmatrix} x+ \lambda t \\ 1 \end{pmatrix} \right] = \left[ K \begin{pmatrix} t \\ 0 \end{pmatrix} + K \begin{pmatrix} x \\ 1 \end{pmatrix} \right] = g. \]
\end{exmp*}

\textbf{Äquivalenz der Begriffe aus der linearen Algebra für $V$ und der Begriffe aus der projektiven Geometrie für $\proj(V)$.}
Sei $V$ ein Vektorraum über einem Schiefkörper $K$ mit $\dim V \ge 3$. Für $M \subseteq V$ bezeichne $[M] := \{ [x] : x \in M \setminus\{ 0 \} \} \subseteq \proj(V)$. Es gilt
\begin{enumerate}
 \item $U \subseteq V$ ist ein linearer Unterraum $\Leftrightarrow$ $0 \in U$ und $[U]$ ist ein Unterraum von $\proj(V)$.
 \item Für $M \subseteq V$ gilt $[ \operatorname{Lin} M ] = \angles{[M]}$, dabei ist $\operatorname{Lin} M$ die lineare Hülle von $M$ und $\angles{\cdot}$ ein Unterraum von $\proj(V)$.
 \item $M \subseteq V$ ist linear unabhängig $\Leftrightarrow$ $0 \notin M$ und $[M]$ ist in $\proj(V)$ unabhängig.
 \item $M \subseteq V$ ist eine Basis von $V$ $\Leftrightarrow$ $0 \notin M$, $[M]$ ist eine Basis von $\proj(V)$ und $[x] \ne [y]$ für alle $x,y \in M$ mit $x \ne y$.
 \item $\dim V = \dim \proj(V) + 1$.
\end{enumerate}

\begin{proof}
 Klar bzw. einfache Übung.
\end{proof}

\begin{rmrk*}
 Sei $\proj$ ein projektiver Raum und $H \subseteq P$ eine Hyperebene und $g$ eine Gerade. Dann gilt entweder $(g) \subseteq H$ oder $g$ schneidet $H$ in genau einem Punkt.
\end{rmrk*}

\begin{proof}
 Falls $(g) \nsubseteq H$ und $p \in (g) \setminus H$, dann ist nach 2.18
 \[ \bigcup \{ (pq) : q \in H \} = \angles{H,p} = \proj, \]
 da $H$ eine Hyperebene ist. 
 
 Sei $r \in (g) \setminus \{p\}$, dann gibt es $q \in H$ mit $r \in (pq)$, also $g = pq$.
\end{proof}

\begin{defn*}
 Eine Zentralkollineation mit dem Zentrum nicht auf der Achse heißt auch eine \emph{Homologie}. 
 
 Eine Zentralkollineation mit dem Zentrum auf der Achse heißt auch eine \emph{Elation}.
\end{defn*}

\subsection{Eigenschaften von Zentralkollineationen}
\begin{deno*}
 $\Gamma(H,z)$ bezeichne die Menge der Zentralkollineationen mit Achse $H$ und Zentrum $z$.
\end{deno*}

\begin{lem}
 $(\Gamma(H,z), \circ)$ ist eine Gruppe (eine Untergruppe der Kollineationen $\proj \to \proj$).
\end{lem}

\begin{proof}
 $\id \in \Gamma$ klar. Ebenso $\alpha, \beta \in \Gamma \Rightarrow \beta \circ \alpha \in \Gamma$ klar. Wenn $\alpha \in \Gamma$, dann ist auch $\alpha^{-1} \in \Gamma$, denn für $p \in H$ folgt 
 \[ p = \alpha^{-1} ( \alpha(p) ) \overset{p \in \text{Achse}}{=} \alpha^{-1}(p), \]
 für eine Gerade $g$ durch $z$ gilt
 \[ g = \alpha^{-1}( \alpha (g) ) = \alpha^{-1} (g). \qedhere \]
\end{proof}

\begin{lem}
 Seien $\alpha \in \Gamma(H,z)$, $p \in P$ mit $p \ne z$, $p' = \alpha(p)$. Dann gilt für jeden Punkt $x$ mit $x \in H$, $x \notin (pz)$:
 \[ x' := \alpha(x) = xz \cap p'f, \]
 wobei $f = px \cap H$.
\end{lem}

\begin{proof}
 Sei $x \in (xz)$, dann ist $\alpha(x) \in \alpha(xz) = xz$, da $z$ Zentrum. Sei $f := px \cap H$, also $\alpha(f) = f$, da $f \in H$, also 
 \[ \alpha(x) \in \alpha( xf ) = \alpha( pf ) = \alpha(p) \alpha(f) = p'f. \]
 Da $p \notin (xz)$, ist $f \notin (xz)$, also $xz \notin p'f$, also ist $\alpha(x) = xz \cap p'f$.
\end{proof}

\begin{folg}[Eindeutigkeit von Zentralkollineationen]
 Sei $\alpha \in \Gamma(H,z)$ und $\alpha \ne \id$. $p \in \proj \setminus H$, $p \ne Z$. Dann gilt
 \begin{enumerate}[a)]
  \item $\alpha(p) \ne p$.
  \item Für $\beta \in \Gamma(H,z)$ folgt aus $\beta(p) = \alpha(p)$, dass $\beta = \alpha$ ist.
 \end{enumerate}
\end{folg}

\begin{proof}
 \begin{enumerate}[a)]
  \item Angenommen $p' = \alpha(p) = p$. Für $x \in \proj \setminus H$, $x \notin (pz)$ gilt nach 3.4
  \[ \alpha(x) = zx \cap fp' = zx \cap fp = x, \]
  wobei $f = px \cap H$. Für $x \in (pz)$ folgt mit Hilfe eines $p_0 \in \proj \setminus (H \cup (pz))$, wegen $\alpha( p_0 ) = p_0$ (gerade gezeigt), dass $\alpha(x) = x$ ist. Also $\alpha = \id$.
  \item Klar mit 3.4, betrachte zunächst $\alpha(x) = \beta(x)$ für $x \notin (pz)$, damit folgt die Behauptung aber auch für $x \in (pz)$. \qedhere
 \end{enumerate}
\end{proof}

\begin{thm}[Existenz von Zentralkollineationen]
 Sei $\proj$ desarguessch, $H \subseteq P$ eine Hyperebene, $z, p, p'$ drei verschiedene kollineare Punkte mit $p, p' \notin H$.
 
 Dann gibt es genau eine Zentralkollineation $\alpha: \proj \to \proj$ mit Achse $H$, Zentrum $z$ ($\alpha \in \Gamma(H,z)$) und $\alpha(p) = p'$.
\end{thm}

\begin{proof}
 Die Eindeutigkeit folgt aus 3.6.
 
 Die Existenz folgt aus einer Konstruktion nach Lemma 3.5, dafür ist zu zeigen, dass die konstruierte Abbildung eine Kollineation definiert, dafür braucht man den Satz von Desargues.
\end{proof}

\subsection{Darstellung von Kollineationen}
Sei $V$ ein $K$-Vektorraum über einem Schiefkörper $K$. Bezeichnungen für einige Gruppen:
\begin{itemize}
 \item $GL(V) = \{ f: V \to V : f$ linear, bijektiv $\}$,
 \item $\Gamma L(V) = \{ f: V \to V : f$ semilinear, bijektiv $\}$,
 \item $P \Gamma L(V) = \{ \obar{f} : f \in \Gamma L(V) \} = \{ \obar{\varphi} : \proj(V) \to \proj(V) : \varphi$ Kollineation $\}$,
 \item $PGL(V) = \{ \obar{f} : f \in GL(V) \}$.
\end{itemize}

\begin{defn*}
 Eine Kollineation $\varphi: \proj(V) \to \proj(V)$ heißt \emph{projektiv}, falls $\varphi \in PGL(V)$, das heißt falls es $f \in GL(V)$ gibt mit $\obar{f} = \varphi$.
\end{defn*}

\begin{folg}
 Eine Kollineation $\varphi \in P\Gamma L(V)$, die eine Gerade $g$ punktweise fest lässt, ist in $PGL(V)$.
\end{folg}

\begin{proof}
 Sei $[x] \in (g)$, $f \in \Gamma L(V)$ mit $\obar{f} = \varphi$ und sei o.B.d.A. $f(x) = x$ (gegebenenfalls nach Multiplikation von $f$ mit $\tau \in K \setminus \{ 0 \}$). Sei $y$ linear unabhängig zu $x$ mit $[y] \in g$ und $\lambda \in K$.
 
 Dann ist 
 \[ [x + \lambda y] = \varphi ([x+\lambda y]) = [f(x+\lambda y)] = [ f(x) + f(\lambda y) ] = [x + f(\lambda y) ], \]
 also gibt es $\mu \in K \setminus \{0\}$, sodass $x + f(\lambda y) = \mu( x + \lambda y)$, wegen $x,y$ linear unabhängig folgt $\mu = 1$ und damit
 \[ f(\lambda y) = \lambda y = \alpha(y) f(y) = \alpha(\lambda) y, \]
 also $\alpha = \id$.
\end{proof}

\subsection{Normalteiler}
\textbf{Zur Erinnerung:}
\begin{thm*}[Abbildungssatz]
 Sei $f:A \to B$ eine Abbildung. Bezeichne $\sim_f$ die von $f$ induzierte Äquivalenzrelation auf $A$, das heißt $a_1 \sim_f a_2$ $:\Leftrightarrow$ $f(a_1) = f(a_2)$.
 
 Dann gibt es genau eine injektive Abbildung $\obar{f}: A/\sim_f \to B$, sodass $f = \obar{f} \circ \operatorname{nat}$.
\end{thm*}

\begin{thm*}[Homomorphiesatz]
 Seien $G,H$ Gruppen, $f:G \to H$ ein Gruppenhomomorphismus.
 
 Dann ist $\ker f := \{ x \in G : f(x) = e \}$ ein Normalteiler.
\end{thm*}

Ein \emph{Normalteiler} ist eine Untergruppe $N \subseteq H$, sodass $gN = Ng$ für alle $g \in G$, wobei $gN := \{ g \cdot x : x \in N \}$.

%%%%%%%%%%%%%%%%%%%

Eine Menge $G$ heißt bezüglich einer auf ihr definierten binären Operation $\cdot$ eine \emph{Gruppe}, falls
\[ \begin{aligned}
    \forall a,b,c \in G &: a \cdot (b \cdot c ) = (a \cdot b ) \cdot c, \\
    \exists e \in G \, \forall a \in G &: a \cdot e = e \cdot a, \\
    \forall b \in G \, \exists b^{-1} \in G &: b \cdot b^{-1} = e.
   \end{aligned} \]
   
Eine Teilmenge $U \subseteq G$ heißt bezüglich $\cdot$ \emph{Untergruppe}, falls das neutrale Element $e \in U$ und 
\[ \begin{aligned}
    \forall a,b \in U &: a \cdot b \in U, \\
    \forall a \in U &: a^{-1} \in U.
   \end{aligned} \]

Eine Untergruppe $N \subseteq G$ heißt ein \emph{Normalteiler}, falls für alle $a \in G$ die Beziehung 
\[ a^{-1} \cdot N \cdot a := \{ a^{-1} \cdot n \cdot a : n \in N \} = N \]
gilt. Äquivalent: $aN = Na$.

\textbf{Was nützt das?}
Es sei $N \subseteq G$ ein Normalteiler. Man betrachte die Menge $G / N = \{ a N : a \in G \}$.

\begin{rmrk*}
Die Mengen der Form $aN \subseteq G$ bilden eine disjunkte Zerlegung\footnote{Die Linksnebenklassen $aU$ bilden übrigens (mit dem gleichen Beweis) auch für jede Untergruppe $U \subseteq G$ eine disjunkte Zerlegung von $G$.} von $G$, denn gilt für $a,b \in G$ die Beziehung $aN \cap bN \ne \emptyset$, so folgt für $g \in aN \cap bN$ und gewisse $n_1, n_2 \in N$, dass $a n_1 = b n_2 = g$. Nun ist für $x \in G$
\[ x \in aN \qLRq a^{-1} x \in N \qLRq \underbrace{b^{-1} a}_{n_2 n_1^{-1} \in N} a^{-1} N \qLRq b^{-1} x \in N \qLRq x \in bN \]
und gleichzeitig ist
\[ \bigcup_{g \in G} g N = G, \]
denn für alle $a \in G$ gilt $a \in aN$.
\end{rmrk*}

Für $a \in G$ sei $[a] := aN = Na$. Auf $G / N$ definieren wir eine Multiplikation durch $[a] \cdot [b] := [a \cdot b]$. Diese ist wohldefiniert, denn für $x \in [a]$, $y \in [b]$ folgt $xN = aN$, $yN = bN$, also $aN \cdot bN = a bN N = ab N$ und $aN \cdot bN = xN \cdot yN = xy N$, also $ab N = xy N$ und $[a \cdot b] = [x \cdot y]$.

$G/N$ ist eine Gruppe, denn für $a,b \in G$ gilt
\[ \begin{aligned}
    ( [a] \cdot [b] ) \cdot [c] &= [a \cdot b ] \cdot [c] = [ (a \cdot b ) \cdot c ] \\
    &= [ a \cdot (b  \cdot c ) ] = [a] \cdot [b \cdot c ] = [a] \cdot ( [b] \cdot [c] ),
   \end{aligned} \]
\[ [e] \cdot [a] = [e \cdot a ] = [ a ] = [a \cdot e] = [a] \cdot [e],\]
\[ [a] \cdot [a^{-1}] = [a \cdot a^{-1}] = [e] = [a^{-1} \cdot a] = [a^{-1}] \cdot [a]. \]

Es seien $H,G$ Gruppen. Eine Abbildung $\varphi: H \to G$ heißt \emph{Homomorphismus}, falls für alle $h_1, h_2 \in H$ gilt: $\varphi( h_1 \cdot h_2 ) = \varphi( h_1 ) \cdot \varphi(h_2)$, wobei $h_1 \cdot h_2$ eine Multiplikation in $H$ und $\varphi( h_1 ) \cdot \varphi(h_2)$ eine Multiplikation in $G$ sind.

Es ist $\ker \varphi := \{ h \in H : \varphi(h) = e_G \}$ ($e_G$ ist das neutrale Element in $G$). $\ker \varphi$ ist ein Normalteiler, denn für $g \in G$, $x \in \ker \varphi$ ist
\[ \varphi( g^{-1} x g ) = \varphi( g )^{-1} \cdot \underbrace{\varphi(x)}_{e_G} \cdot \varphi( g ) =  \varphi( g )^{-1} \cdot \varphi( g ) = e_G. \]
Dabei haben wir benutzt, dass
\[ \varphi( e_H ) \varphi( e_H ) = \varphi( e_H e_H ) = \varphi( e_H ) \qRq \varphi( e_H ) = e_G \]
und
\[ \varphi( g^{-1} ) \varphi( g ) = \varphi( g^{-1} g ) = \varphi( e_H ) = e_G \qRq \varphi( g^{-1} ) = \varphi( g)^{-1}. \]

Es folgt damit $g^{-1} x g \in \ker \varphi$, das heißt $g^{-1} \ker \varphi g \subset \ker \varphi$ für alle $g \in G$, also auch $g^{-1} ( \ker \varphi ) g = \ker \varphi$. 

Es seien $H, G, Q$ Gruppen. Es heißt
\[ H \xrightarrow{\iota} G \xrightarrow{\pi} Q \]
eine \emph{kurze exakte Sequenz}, falls $\iota: H \to G$ ein injektiver Homomorphismus, $\pi: G \to Q$ ein surjektiver Homomorphismus ist und $\operatorname{im} \iota = \ker \pi$.

In diesem Fall ist $H$ als Gruppe isomorph zu $\iota(H) = \operatorname{im} \iota$. Weiterhin ist $\operatorname{im} \iota = \ker \pi $ ein Normalteiler\footnote{Das Bild eines Normalteilers unter einem Homomorphismus muss sonst kein Normalteiler sein.} von $G$. Also kann man $G / (\operatorname{im} \iota)$ betrachten und es gilt $Q \simeq G / \operatorname{im} \iota$. Dies folgt aus dem Homomorphiesatz.

Sei umgekehrt $N \subseteq G$ ein Normalteiler, $\iota_N: N \hookrightarrow G$ die Inklusion und $pi_N: G \to G/N$ die Projektion $\pi_N(a) = [a]$, $a \in G$. Dann ist
\[ N \overset{\iota_N}{\hookrightarrow} G \overset{\pi_N}{\hookrightarrow} G / N \]
eine kurze exakte Sequenz.
\begin{itemize}
 \item $\iota_n$: injektiv $\checkmark$
 \item $\pi_n$: surjektiv $\checkmark$, $[a] = \pi_N(a)$.
 \item $\iota_N(N) = \operatorname{im} \iota_N = \ker \pi_N = [e_G] = N \subseteq G.$ $\checkmark$
\end{itemize}

\begin{thm*}[Homomorphiesatz für Gruppen]
 Es sei $\varphi: G \to H$ ein Homomorphismus von Gruppen. Es sei $N \subseteq \ker \varphi$ ein Normalteiler von $G$. Dann existiert genau ein Homomorphismus $\tilde{\varphi}: G/N \to H$ derart, dass $\tilde{\varphi} \circ \pi_N = \varphi$ ist und es gilt: 
 \begin{enumerate}
  \item Falls $N = \ker \varphi$, so ist $\tilde{\varphi}$ injektiv.
  \item Falls $\varphi(G) = H$, so ist auch $\tilde{\varphi}$ surjektiv.
  \item Falls $N = \ker \varphi$ und $\varphi(G) = H$, so ist $\tilde{\varphi}$ ein Isomorphismus.
 \end{enumerate}
\end{thm*}

\begin{proof}
 Definiere $\tilde{\varphi} : G / N \to H$ durch $\tilde{\varphi}([a]) = \varphi(a)$. Das ist notwendig, da
 \[ \underbrace{\tilde{\varphi} \circ \pi_N}_{=\varphi} (a) = \tilde{\varphi}( [a] ) = \varphi(a) \]
 gelten muss. Dann ist $\tilde{\varphi}$ wohldefiniert.
 $\varphi(a) = \varphi(b)$, falls $[a] = [b]$, $\varphi( a^{-1} b ) = e_H$, weil $a^{-1} b \in N \subseteq \ker \varphi$.
\end{proof}

\begin{thm}[Homomorphiesatz]
 Sei $f: G  \to H$ ein Gruppenhomomorphismus. Dann gibt es genau einen Homomorphismus $\obar{f} : G / \ker f \to H$ mit $\obar{f} \circ \operatorname{nat} = f$. Ferner gilt dann für $\ubar{f}$, dass $\obar{f}$ injektiv ist. Folglich ist $G / \ker f \simeq \operatorname{im} f$.
\end{thm}

$U(n)$ unitäre Abbildung $f: \complex^n \to \complex^n$. $SU(n)$ spezielle unitäre Abbildungen $\{ f \in U(n) : \det f = 1 \}$.

$\det : U(n) \to (\complex \setminus \{ 0 \}, \cdot )$ ist ein Gruppenhomomorphismus. $SU(n) = \ker \det$ ist also ein Normalteiler.

Nach dem Homomorphiesatz ist $U(n) / SU(n) \simeq \im \det \simeq S^1 := \{ z \in \complex : |z| = 1 \}$.

Für $S \in U(n)$ ist $| \det S | = 1$, denn $S S^* = E$, also 
\[ \det( S S^* ) = 1 \qRq (\det S)(\det S^*) = (\det S)\obar{(\det S^*)} = |\det S |^2. \]

\[ \det A = \det \begin{pmatrix} a \\ & 1 \\ & & \ddots \\ & & & 1 \end{pmatrix} = a \]
und falls $|a| = 1$, so ist $\obar{a} = a^{-1}$, also $A \in U(n)$.

1 bezeichne die Gruppe mit einem Element. Wir definieren die \emph{kurze exakte Sequenz}
\[ 1 \to G_1 \xrightarrow{f_1} G_2 \xrightarrow{f_2} G_3 \to 1, \]
wobei $f_1$ injektiv, $\im f_1 = \ker f_2$, $f_2$ surjektiv sind.

\textbf{Bezeichnungen für einige Gruppen.}
Sei $K$ ein Schiefkörper.
\begin{itemize}
 \item $\dot{K} := ( K \setminus \{ 0 \}, \cdot)$ ist die \emph{multiplikative Gruppe},
 \item $Z := \{ a \in K : \forall x \in K : ax = xa \}$ ist das \emph{Zentrum} von $K$ (das ist ein Unterkörper),
 \item $\dot{Z} := ( Z \setminus \{ 0 \}, \cdot)$,
 \item $\operatorname{Aut}(K) := \{ f : K \to K : f$ ist ein Schiefkörperautomorphismus $\}$,
 \item $\operatorname{In}(K) := \{ \mu \in \operatorname{Aut} (K) : \exists a \in \dot{K} : \mu(x) = a x a^{-1}$ für alle $x \in K \}$.
\end{itemize}

Sei $V$ ein $K$-Vektorraum mit $\dim V \ge 3$.

Nach dem Satz über Darstellungen von Kollineationen durch semilineare Abbildungen und Determinanten von projektiven Kollineationen erhalten wir ein kommutatives Diagramm von Gruppen und Gruppenhomomorphismen mit kurzen exakten Zeilen und Spalten.

Wann ist $a \cdot \id$ linear? Bedingung $a(\lambda x) := \lambda \cdot (a x)$, also $a \lambda = \lambda a$ für alle $\lambda \in K$.

$\alpha$ sei der zu $f$ gehörige Körperautomorphismus.

Ein Diagramm für Abbildungen und Mengen heißt kommutativ, wenn die Kompositionen der Abbildungen mit demselben Start- und Zielpunkt im Diagramm übereinstimmen.
