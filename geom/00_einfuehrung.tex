\section{Einführung}
\subsection{Begriffe der Geometrie}
\begin{itemize}
 \item Felix Klein: Erlanger Programm (1872)
 \item Euklidischer Raum $\real^n$ (mit Standardskalarprodukt)
 \item Isometriegruppe
  \[ f(x) = Ax + t, A \in O(n), t \in \real^n. \]
 \item Kongruenz
  \begin{itemize} 
   \item Alle Geraden sind einander kongruent, ebenso alle Kreise vom selben Radius.
   \item Kongruente Objekte sind durch Invarianten verbunden.
  \end{itemize}
 \item In der Geometrie verwendete Begriffe: Längen, Flächeninhalte, Volumina, Winkel
 \item Winkel sind kongruent, wenn sie die selbe Größe haben.
 \item Kongruenzsätze
 \item Kreis, Invariante Radius
 \item Affine Geometrie, Gruppe: affine Gruppe des $\real^n$
  \[ \{ f: \real^n \to \real^n | f(x) = Ax + t, A \in \GL(n), A \in \realmat{n}{n} \text{ invertierbar}, t \in \real^n. \} \]
 \item Begriffe der affinen Geometrie:
  \begin{itemize}
   \item Geraden
   \item $k$-dimensionale affine Unterräume
   \item Ellipsen
   \item Mittelpunkte von Strecken 
    \[ A \left( \frac{x+y}{2} \right) + t = \frac{A(x) + t}{2} + \frac{A(y) + t}{2} \]
   \item Mittelpunkte von Ellipsen
  \end{itemize}
 \item \emph{Nicht} in der affinen Geometrie: 
  \begin{itemize}
   \item Längen, Winkel
   \item Kreise
   \item Winkelhalbierende
  \end{itemize}
 \item Euklidische Geometrie
 \item Flächeninhalt: Äquiaffine Abbildung
  \[ f(x) = Ax + t, A \in \GL(n), |\det A| = 1 \]
 \item ``Axiomatischer Zugang'' $\longleftrightarrow$ ``Modelle''
\end{itemize}

\subsection{Literatur}
Horst Knörrer: Geometrie (nur eingeschränkt hilfreich)

\subsection{Themen}
\begin{itemize}
 \item Isometrien des $\real^n$, eventuell normale Endomorphismen
 \item Projektive Geometrie
 \item Inversion (Spiegelung) an Sphären und die Möbiusgruppe
 \item Sphärische und hyperbolische (nicht-euklidische) Geometrie
 \item Quadriken
\end{itemize}

\subsection{Bezeichnungen und Konventionen}
\begin{itemize}
 \item Vektoren werden ohne Pfeile geschrieben.
 \item Skalare sind in der Regel griechische Buchstaben
 \item Matrizen werden durch Großbuchstaben bezeichnet.
 \item Wir betrachten oft den $\real^n$, $\complex^n$ mit 
  \begin{itemize}
   \item dem Standardskalarprodukt
    \[ \langle x, y \rangle := \sum_{i=1}^n x_i \bar{y}_i, \]
    wobei $x = (x_1, \ldots, x_n)^T, y = (y_1, \ldots, y_n)^T \in \real^n$ oder $\complex^n$, 
   \item der euklidischen bzw. unitären Norm für $\real^n$ bzw. $\complex^n$
    \[ \| x \| := \sqrt{ \langle x,x \rangle }. \]
  \end{itemize}
 \item Standardbasis des $K^n$ (Körper $K$):
  \[ e_i = \begin{pmatrix} 0 \\ \vdots \\ 0 \\ 1 \\ 0 \\ \vdots \\ 0 \end{pmatrix} \ldots i\text{-te Stelle} \quad (e_1, e_2, \ldots, e_n). \]
 \item Einheitsmatrix des $\real^n$: $E_n$ bzw. $E$, wenn $n$ aus dem Kontext klar ist.
\end{itemize}
