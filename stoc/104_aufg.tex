\section{Aufgaben}
\begin{aufg}
  Es sei $\Omega := [0, 1]$, $\mA$ die $\sigma$-Algebra der Borel-
  Mengen in $[0, 1]$ und $\pP$ das Lebesgue-Maß auf $[0, 1]$. Für jedes $n = 1, 2,
  \ldots$ sei
  \[ A_n := \left[ 0, \rez{2^n} \right] \cup
    \left[  \frac{2}{2^n}, \frac{3}{2^n} \right]
    \cup \cdots \cup
    \left[ \frac{2^n-2}{2^n}, \frac{2^n-1}{2^n} \right]. \]
  Dann ist die Folge $\{A_n\}$ dieser Ereignisse unabhängig.
\end{aufg}

\begin{proof}
  Es gilt
  \[ \bigcap_{i=1}^k A_i = \left[ 0,\rez{2^k} \right] \tag{$\circ$}. \]
  Das zeigt man zum Beispiel durch vollständige Induktion.

  Induktionsanfang:  Für $k=2$ gilt
  \[ A_1 \cap A_2 = \left[ 0, \rez{2} \right] \cap \left( \left[ 0, \rez{4}
      \right] \cup \left[ \rez{2}, \frac{3}{4} \right] \right) = \left[
      0,\rez{4} \right]. \]
  Induktionsschritt: Es gelte ($\circ$) für ein $k$, dann gilt
  \[ \bigcap_{i=1}^{k+1} A_i = \left[ 0,\rez{2^k} \right] \cap A_{k+1} =  \left[
      0,\rez{2^{k+1}} \right]. \]
  Aus ($\circ$) folgt die Unabhängigkeit der Folge. Für Unabhängigkeit muss
  \[ \pP \left( \bigcap_{i=1}^n A_i \right) = \prod_{i=1}^k \pP(A_i) \]
  gelten. Für jedes $A_n$ ist $\pP(A_n) = \rez{2}$. Also
  \[ \pP \left( \bigcap_{i=1}^n A_i \right) = \rez{2^n} = \left( \rez{2}
    \right)^k = \prod_{i=1}^k \pP(A_i). \qedhere \]
\end{proof}

\begin{aufg}
 Beweisen Sie Lemma \ref{lem:1_3_9}.
\end{aufg}

\begin{enumerate}[(i)]
\item Für $x \le y$ gilt $\mu((-\infty,x)) \le \mu((-\infty,y))$. Also ist $F(x)
  \le F(y)$.
\item Für jedes Maß $\mu$ gilt $\mu(A) \ge 0$ für $A \in \mA$, also ist auch
  $F(x) \ge 0$ für alle $x$.

  Damit folgt mit (i), dass $\lim_{x \to -\infty} F(x) = 0$.

  Weil $\mu$ ein $\pW$-Maß ist, gilt $\mu(\real) = 1$. Also gilt $F(x) \le 1$
  für alle $x$.

  Damit folgt aus (i), dass $\lim_{x \to \infty} F(x) = 1$.
\item Die Linksstetigkeit folgt aus Satz 1.1.4.(iv). Für jede steigende Folge $\{ x_n \}$
  mit $x_n \xrightarrow{n \to \infty} x$ gilt $(-\infty, x_{n-1}) \subset
  (-\infty, x_n)$. Damit ist
  \[ \mu \left( \bigcup_{n=1}^\infty (-\infty, x_n) \right) = \lim_{n \to
      \infty} \mu( (-\infty, x_n)  = \mu( (-\infty, x) ) \]
  und damit folgt $\lim_{x_n \uparrow x} F(x_n) = F(x)$.
\end{enumerate}

Jede reelle Funktion $F$ auf $\real$ mit den Eigenschaften (i)-(iii) ist die
Verteilungsfunktion einer Zufallsgröße.

\begin{proof}
  Sei $F: \real \to \real$ mit den Eigenschaften (i)-(iii). Dann existiert ein
  Prämaß $\mu_F$ auf der Algebra $\mA_H$ der rechtsoffenen Intervalle mit:
  \begin{align*}
    \mu_F( [a,b) ) &= F(b) - F(a), \\
    \mu_F((-\infty,b)) &= F(b).
  \end{align*}
  Dieses Prämaß kann zu einem Maß $\obar{\mu}_F$ auf $\borel(\real)$ fortgesetzt
  werden, dem Lebesgue-Stieltjes-Maß. Dann ist $\obar{\mu}_F$ ein $\pW$-Maß,
  denn
  \[ \obar{\mu}_F(\real) = \lim_{b \to \infty} F(b) = 1. \]
  Also ist $W := (\real, \borel(\real), \obar{\mu}_F)$ ein $\pW$-Raum. Sei $X :=
  \id$, dann ist $X$ eine Zufallsgröße auf $W$ und es gilt die zugehörige
  Verteilung:
  \[ \mu_X(B) = \obar{\mu}_F(X^{-1}(B)) = \obar{\mu}_F(B). \]
  Damit ist $\obar{\mu}_F$ die gesuchte Verteilungsfunktion.
\end{proof}

\begin{aufg}
  Beweisen Sie Lemma 1.3.10.
\end{aufg}
\begin{align*}
  \pP(a \le X < b )
  &= \pP((X<b)\setminus(X<a)) \tag{i} \\
  &= \pP(X<b) - \pP(X<a) \\
  &= F(b) - F(a). \\
  \pP(X = a)
  &= \lim_{n \to \infty} \pP \left( \left( X < a + \rez{n} \right) \setminus (X<a) \right) \tag{ii} \\
  &= \lim_{n \to \infty} \pP \left( \left( X < a + \rez{n} \right) \right) - \pP(x<a) \\
  &= F(a+0) - F(a). \\
  \pP(a < X < b)
  &= \lim_{n \to \infty} \pP \left( (X<b) \setminus \left( X < a + \rez{n} \right) \right) \tag{iii} \\
  &= \pP(X < b) - \lim_{n \to \infty} \pP \left( X < a + \rez{n} \right) \\
  &= F(b) - F(a+0) \\
  \pP(a \le X \le b)
  &= \lim_{n \to \infty} \pP \left( \left(X<b + \rez{n} \right) \setminus ( X < a ) \right) \tag{iv} \\
  &= \lim_{n \to \infty} \pP \left( X < b + \rez{n} \right) - \pP(X<a) \\
  &= F(b+0) - F(a) \\
  \pP(a < X \le b)
  &= \lim_{n \to \infty} \pP \left( \left(X<b + \rez{n} \right) \setminus \left(X < a + \rez{n} \right) \right) \tag{v} \\
  &= \lim_{n \to \infty} \pP \left( X < b + \rez{n} \right) -
    \lim_{n \to \infty} \pP \left( X < a + \rez{n} \right) \\
  &= F(b+0) - F(a+0)
\end{align*}
\[ F(x) = \pP(X < x) = \mu((-\infty,x)) = \int_{-\infty}^x p(t) \diffop
  t. \tag{vi} \]

\begin{aufg}
  Man berechne die Dichtefunktion der Zufallsgröße $X^2$, wobei $X \in N(0, 1)$.

  [Hinweis: Aufgabe 1.6.3.]
\end{aufg}

\begin{aufg}
  Beweisen Sie Satz 1.3.11 für den Fall, dass $p$ stetig ist.

  [Hinweis: Mittelwertsatz für das Riemann-Integral.]
\end{aufg}


%%% Local Variables:
%%% TeX-master: "master"
%%% End:
