\section{Wahrscheinlichkeitsräume}

Aufgabe der Wahrscheinlichkeitstheorie: Das Problem des \emph{Zufalls} mit einem
\emph{mathematischen Modell} zu erfassen.

Was ist Zufall? Was sind zufällige Ereignisse?

\textbf{Beispiele.}
\begin{enumerate}[a)]
\item Das Werfen eines Würfels. Ein Zufälliges Ereignis ist zum Beispiel das
  Auftreten einer \emph{geraden Augenzahl}.
\item Die Anzahl von vermittelten Telefongesprächen während einer bestimmten
  Stunde in einer Telefonzentrale.
\item Geburten (Junge oder Mädchen).
\end{enumerate}

Schon im 16. Jahrhundert betrachtete man Aufgaben rein
wahrscheinlichkeitstheoretischen Charakters. Die Lösung wurde meistens mit Hilfe
der Kombinatorik bestimmt.

Ein exaktes mathematisches Modell wurde erst zwischen 1909 und 1933
ausgearbeitet. Wichtige Beiträge lieferten zum Beispiel Borel, Wiener, Paley,
Zygmund, Lomnicki, Steinhaus und \emph{Kolmogorov}. Etwa in dieser Zeit wurden
auch das abstrakte Maß und Integral entwickelt.

\begin{prgp}[Modell für zufällige Ereignisse und
  Wahrscheinlichkeit]
\begin{enumerate}[a)]
\item \textbf{Ereignisse.}
  
  Ereignisalgebra (Ereignisse $A,B$ $\Rightarrow$ nicht $A$, $A$ oder $B$, $A$
  und $B$). 

  Modell: Algebra von Teilmengen einer Menge (Stone 1937).

  Eine Familie $\mA$ von Teilmengen einer Menge $\Omega$ heißt \emph{Algebra} ,
  falls
  \begin{enumerate}[a)]
  \item $A \in \mA$ $\Rightarrow$ $A^c \in \mA$
  \item $A, B \in \mA$ $\Rightarrow$ $A \cup B \in \mA$
  \end{enumerate}
  Meist: Erweiterung zu einer $\sigma$-Algebra $\sigma(\mA)$.
\item \textbf{Wahrscheinlichkeit.}

  Tritt ein zufälliges Ereignis $A$ in $n$ Versuchen $m$-mal ein ($0 \le m \le
  n$), so heißt
  \[ H_n(A) := m \]
  die \emph{absolute Häufigkeit} und
  \[ h_n(A) := \frac{m}{n} \]
  die \emph{relative Häufigkeit} von $A$ bezüglich dieser $n$ Versuche.

  Eigenschaften:
  \begin{enumerate}[(i)]
  \item $0 \le h_n(A) \le 1$
  \item $h_n($ Sicheres Ereignis $) = 1$, $h_n($ Unmögliches Ereignis $) = 0$
  \item $h_n (A \cup B) = h_n(A) + h_n(B)$, wenn $A$ und $B$ nicht gleichzeitig
    eintreten können.
  \end{enumerate}
  Die relative Häufigkeit zeigt eine \emph{Stabilität}, wenn $n$ groß ist.

  Beispiel: Wurf eines Geldstücks, $A$: Auftreten von ``Kopf''
  \begin{center}
    \begin{tabular}{l|l|l|l}
      & $n$ & $ H_n(A)$ & $ h_n(A)$ \\
      \hline
      Buffon (18. Jh.) & 4040 & 2048 & 0.5080 \\
      Pearson (20. Jh.) & 12000 & 6019 & 0.5016 \\
      Pearson & 24000 & 12012 & 0.5005
    \end{tabular}
  \end{center}

  Der ``Grenzwert'' $n \to \infty$ führt zu einem endlich additiven Maß auf $\mA$.
\end{enumerate}
\end{prgp}

\begin{defn}
  \begin{itemize}
  \item Es sei $\Omega \ne \emptyset$ eine Menge, $\mA$ eine $\sigma$-Algebra von
  Teilmengen von $\Omega$ und $\pP$ ein Maß auf $\mA$ mit $\pP(\Omega) = 1$. Das
  Tripel $(\Omega, \mA, \pP)$ heißt \emph{Wahrscheinlichkeitsraum} (kurz
  $\pW$-Raum). Die Elemente von $\mA$ heißen \emph{Ereignisse}, die Elemente von
  $\Omega$ heißen \emph{Ergebnisse, Stichproben, Realisierungen}\footnote{%
    Man findet auch  die Bezeichnung \emph{Elementarereignisse}, jedoch ist $\{w\}
    \notin \mA$ möglich, zum Beispiel für $\mA = \{ \emptyset, \Omega \}$. Also
    müssen Elementarereignisse keine Ereignisse sein.}.
  \item Das Maß $\pP$ heißt \emph{Wahrscheinlichkeitsmaß}.
  \item $\emptyset$ heißt \emph{unmögliches Ereignis}, $\Omega$ das \emph{sichere Ereignis}.
  \item $\obar{A} := \Omega \setminus A = A^c$
  \item Ereignisse $A$ mit $\pP(A) = 1$ bzw. $\pP(A) = 0$ heißen \emph{fast
      sicher} oder \emph{fast unmöglich}. Statt $\pP$-fast überall sagt man ``fast
    sicher'' oder ``mit Wahrscheinlichkeit 1''.
  \end{itemize}
\end{defn}

\textbf{Knobelaufgabe}.
Zwei Spieler $S_1, S_2$, drei Würfel $W_1, W_2, W_3$ mit den Augenzahlen
\begin{itemize}
\item $W_1: 5, 7, 8, 9, 10, 18$
\item $W_2: 2, 3, 4, 15, 16, 17$
\item $W_3: 1, 6, 11, 12, 13, 14$
\end{itemize}
Spiel: Zuerst wählt $S_1$ einen Würfel, dann $S_2$. Beide würfeln; wer die
größere Augenzahl hat, bekommt 1 EUR. Sie sind $S_1$. Welchen Würfel würden Sie
wählen?
\begin{align*}
  \pP( W_1 > W_2 ) &= \frac{21}{36} > \rez{2}, & W_1 \text{ ``besser'' als } W_2, \\
  \pP( W_2 > W_3 ) &= \frac{21}{36} > \rez{2}, & W_2 \text{ ``besser'' als } W_3, \\
  \pP( W_3 > W_1 ) &= \frac{21}{36} > \rez{2}, & W_3 \text{ ``besser'' als } W_1.
\end{align*}
Die beste Taktik ist also, den zweiten Spieler zuerst wählen zu lassen.

\textbf{Sprechweisen in der $\pW$-Theorie}.
Seien $A,B \in \mA$.
\begin{enumerate}[i)]
\item $A \subset B$: Aus $A$ folgt $B$.
\item $A \cap B = \emptyset$: $A$ und $B$ sind \emph{unvereinbar}.
\item $A \setminus B$: Es tritt $A$, aber nicht $B$ ein.
\item Seien $A_1, A_2, \ldots$ Ereignisse.
  \begin{itemize}
  \item $\bigcup_{n=1}^\infty A_n$: Mindestens ein $A_n$ tritt ein.
  \item $\bigcap_{n=1}^\infty A_n$: Alle $A_n$ treten ein.
  \item $\liminf_{n \to \infty} A_n := \bigcup_{n=1}^\infty \bigcap_{m=n}^\infty
    A_m$: Von einem Index ab treten alle $A_n$ ein.
  \item $\limsup_{n \to \infty} A_n := \bigcap_{n=1}^\infty \bigcup_{m=n}^\infty
    A_m$: Unendlich viele $A_n$ treten ein.
  \end{itemize}
\end{enumerate}

\begin{exmp}
  \begin{enumerate}[a)]
  \item Werfen eines Würfels mit den Augenzahlen $1, \ldots, 6$.
    \[ \Omega := \{ 1, 2, 3, 4, 5, 6 \}, \qquad \mA := \pot(\Omega)\footnote{%
      Das heißt, es gibt $2^6 = 64$ mögliche Ereignisse.}. \]
    \begin{itemize}
    \item Erscheinende Augenzahl ist $i$: $\{ i \} \in \mA$.
    \item Erscheinende Augenzahl ist gerade: $\{ 2, 4, 6 \}$.
    \item $\pP(A) = \frac{|A|}{6}$, $A \in \mA$.
    \item $\pP(\{i\}) = \rez{6}$, $A \in \mA$, wenn es sich um einen
      ``richtigen'' Würfel handelt.
    \end{itemize}
  \item Schießen auf eine Scheibe mit Radius 20 cm. Sie wird immer getroffen
    \[ \Omega = \{ (x,y) \in \real^2 : x^2 + y^2 \le 20^2 \}, \qquad A = \{ B
      \in \borel(\real^2) : subset \Omega \}. \]
    Nehmen wir an, dass die Treffer gleichmäßig verteilt sind, dann gilt
    \[ \pP( B ) := \frac{\lambda(B)}{\lambda(\Omega)}, \]
    wobei $\lambda$ das Lebesgue-Maß auf $\real^2$ bezeichnet.

    Es wäre auch möglich, $\mA :=$ Lebesgue-messbare Mengen $B \subset \Omega$
    zu definieren.
  \end{enumerate}
\end{exmp}

\begin{thm}
  Für jedes $\pW$-Maß $\pP$ gilt:
  \begin{enumerate}[(i)]
  \item $\pP( (\obar{A}) ) = 1 - \pP(A)$.
  \item $\pP( B \setminus A) = \pP(B) - \pP(A)$, wenn $A \subset B$.
  \item $\pP(A \cup B) = \pP(A) + \pP(B) - \pP(A \cap B)$.
  \item $A_1 \subset A_2 \subset \ldots$ $\Rightarrow$ $\pP( \bigcup_1^\infty
    A_n) = \lim_n \pP(A_n)$.
  \item $A_1 \supset A_2 \supset \ldots$ $\Rightarrow$ $\pP( \bigcap_1^\infty
    A_n) = \lim_n \pP(A_n)$.
  \end{enumerate}
\end{thm}

\begin{proof}
  Das ist bereits aus der Maßtheorie bekannt.
\end{proof}

\begin{defn}
  Zwei Ereignisse $A,B \in \mA$ heißen \emph{unabhängig} (bezüglich $\pP$), wenn
  \[ \pP(A \cap B) = \pP(A) \cdot \pP(B). \]
  Allgemeiner: Eine Familie von Ereignissen in $\mA$ heißt unabhängig (bezüglich
  $\pP$), wenn für jede endliche Teilmenge $\emptyset \ne \{i_1, \ldots, i_n\}
  \subset I$ gilt:
  \[ \pP \left( \bigcap_{i=1}^n A_{i_j} \right) = \prod_{j=1}^n \pP( A_{i_j} ). \]
\end{defn}

\begin{exmp}
  \begin{enumerate}[a)]
  \item Wir betrachten das \emph{zweimalige} Werfen eines Würfels.
    \[ \Omega = \{ (i,j) : 1 \le i,j \le 6 \}, \qquad \mA = \pot(\Omega), \qquad
      \pP( \{(i,j)\}) = \rez{36}. \]
    \begin{itemize}
    \item $A$: Beim ersten Wurf wurde eine Zahl $\le 3$ gewürfelt.
      \[ \pP(A) = \frac{18}{36} = \rez{2}. \]
    \item $B$: Beim zweiten Wurf wurde eine 6 gewürfelt.
      \[ \pP(B) = \frac{6}{36} = \rez{6}. \]
    \item $A \cap B$:
      \[ \pP( A \cap B ) = \frac{3}{36} = \pP(A) \cdot \pP(B), \]
      das heißt $A$ und $B$ sind unabhängig.
    \end{itemize}
  \item Bezeichnungen wie oben.
    \begin{itemize}
    \item $A_1$: Beim ersten Wurf ungerade.
    \item $A_2$: Beim zweiten Wurf ungerade.
    \item $A_3$: Die Summe der geworfenen Augen ist ungerade.
    \end{itemize}
    Dann sind je zwei dieser Ereignisse unabhängig\footnote{%
      Zum Beispiel $\pP(A_1 \cap A_3) = \pP(A_1) \cdot \pP(A_3) = \rez{2} \cdot
      \rez{2} = \rez{4}$ usw.}.
    Jedoch ist die Familie der Ereignisse \emph{nicht unabhängig}, denn
    \[ \pP( A_1 \cap A_2 \cap A_3 ) = 0, \qquad \pP(A_1) \cdot \pP(A_2) \cdot
      \pP(A_3) > 0.\]
    Aus der \emph{paarweisen} Unabhängigkeit kann also nicht auf Unabhängigkeit
    \emph{der Familie} geschlossen werden.
  \end{enumerate}
\end{exmp}

\begin{defn}
  Es sei $(\Omega, \mA, \pP)$ ein $\pW$-Raum und $\mA_1, \mA_2 \subset
  \mA$ Familien von Ereignissen. Dann heißen $\mA_1$ und $\mA_2$
  \emph{unabhängig}, wenn
  \[ \pP( A_1 \cap A_2 ) = \pP(A_1) \cdot \pP(A_2) \]
  für alle $A_1 \in \mA_1$, $A_2 \in \mA_2$.
\end{defn}

\begin{lem}[Approximationssatz]
  Es sei $(\Omega, \mA, \pP)$ ein $\pW$-Raum und $\mB$ die von einer Algebra
  $\borel_0$ von Ereignissen aus $\mA$ erzeugte $\sigma$-Algebra. Dann existiert
  für jedes $A \in \mB$ eine Folge $A_n \in \mB_0$ mit
  \begin{enumerate}[(i)]
  \item $\lim_{n \to \infty} \pP( A_n \Delta A ) = 0$.
  \item $\pP(A) = \lim_{n \to \infty} \pP(A_n)$.
  \end{enumerate}
\end{lem}

\begin{thm}
  Von unabhängigen Ereignisalgebren $\mA_0$ und $\mB_0$ erzeugte
  $\sigma$-Algebren $\mA$ und $\mB$ sind unabhängig.
\end{thm}

\begin{proof}
  Aufgabe, Hinweis: Lemma 1.1.8 benutzen.
\end{proof}

\begin{proof}[Beweis von Lemma 1.1.8]
  (i) ist äquivalent zu
  \[ \lim_{n \to \infty} \pP( A \setminus A_n) = \lim_{n \to \infty} \pP( A_n
    \setminus A ) = 0. \tag{1} \]
  Bezeichne $\mB^*$ die Gesamtheit aller Ereignisse $A \in \mB$, für die eine
  Folge $A_n \in \mB_0$ mit (i) bzw. (1) existiert.

  Zu zeigen: $\mB^* = \mB$. Es gilt $\mB_0 \subset \mB^* \subset \mB$. Wir
  zeigen, dass $\mB^*$ eine $\sigma$-Algebra ist\footnote{%
    Daraus folgt $\mB^* = \mB$, weil $\mB$ nach Definition die kleinste
    $\sigma$-Algebra ist, die $\mB_0$ enthält.}.

  Aufgabe: Zeige, dass $\mB^*$ eine Algebra ist. Benutze dabei
  \[ (A \cap B) \Delta (C \cap D) \subset (A \Delta C) \cup (B \Delta D). \]

  Sei nun $(B_j)_j \in \nat \subset \mB^*$ beliebig. Noch zu zeigen:
  \[ C := \bigcup_{j=1}^\infty  B_j \in \mB^*. \]
  Definiere $C_n := \bigcup_1^n B_j \in \mB^*$, da $\mB^*$ eine Algebra ist. Es
  existieren $A_n$ in $\mB_0$ mit
  \begin{align*}
    \pP( C_n \setminus A_n ) &< \rez{n}, & \pP(A_n \setminus C_n ) < \rez{n}. \tag{2}
  \end{align*}
  Behauptung:
  \[ \lim_{n \to \infty} \pP( C \setminus A_n ) = \lim_{n \to \infty} \pP( A_n
    \setminus C ) = 0, \]
  das heißt $C \in \mB^*$. Die Behauptung folgt aus
  \begin{align*}
    A_n \setminus C &\subset A_n \setminus C_n \\
    C \setminus A_n &\subset ( C\setminus C_n ) \cup (C_n \setminus A_n),
  \end{align*}
  denn
  \[ \pP( C \setminus A_n ) < \pP( C \setminus C_n ) + \rez{n}. \]
  Es gilt $\pP( C \setminus C_n ) \to 0$, wegen $\bigcap_n(C \setminus C_n) =
  \emptyset$ und der Stetigkeit von oben.

  (ii) folgt aus
  \[ \pP(A) = \pP( A_n \cap A ) + \pP( \obar{A}_n ) \cap A ) = \pP(A_n) -
    \underbrace{\pP( A_n \cap \obar{A} )}_{\to 0} +
    \underbrace{\pP( \obar{A}_n \cap A )}_{\to 0}. \qedhere \]
\end{proof}

%%% Local Variables:
%%% TeX-master: "master"
%%% End:
