\section{Satz von Gliwenko-Cantelli}
\begin{defn}
  Es seien $X_1, \ldots, X_n$ identisch verteilte, unabhängige Zufallsgrößen mit
  Verteilungsfunktion $F$ und $\omega \in \Omega$. Wir ordnen die Zahlen
  $X_1(\omega), \ldots, X_n(\omega)$ um:
  \[ X_1^*(\omega) \le X_2^*(\omega) \le \ldots \le X_n^*(\omega). \]

  Definiere die \emph{empirische Verteilungsfunktion}
  \[ F_n^\omega(x) =
    \begin{cases}
      0, & \text{falls } x \le X_1^*(\omega), \\
      \frac{k}{n}, & \text{falls } X_k^*(\omega) < x \le X_{k+1}^*(\omega), \\
      1, & \text{falls } x > X_n^*(\omega). \\
    \end{cases}
  \]
\end{defn}

\begin{thm}[Gliwenko-Cantelli]
  Es gilt
  \[ \lim_{n \to \infty} \sup_{x \in \real} | F_n^\omega(x) - F(x) | = 0 \]
  fast sicher.
\end{thm}

\begin{prgp}[Zusatzaufgabe]
  \begin{enumerate}[(a)]
  \item Man berechne $\pP[ F_n^\omega(x) = k/n ]$, $0 \le k \le n$ mit Hilfe von
    $F$.
  \item Man zeige:
    \begin{align*}
      \lim_{n \to \infty} F_n^\omega(x)
      &= F(x) \quad \text{fast sicher, für alle } x \in \real, \\
      \lim_{n \to \infty} F_n^\omega(x+0)
      &= F(x+0) \quad \text{fast sicher, für
        alle } x \in \real.
    \end{align*}
    Hinweis: $Y_n^* := \ind_{[X_n < x]}$, $Z_n^* := \ind_{[X_n \le x]}$.

    Dann gilt
    \begin{align*}
      F_n^\omega(x) &= \rez{n} \sum^n Y_i^* \to F(x) \text{ fast sicher,} \tag{$\ast$} \\
      F_n^\omega(x+0) &= \rez{n} \sum^n Z_i^* \to F(x+0) \text{ fast sicher.} \tag{$\ast\ast$}
    \end{align*}
  \item Man beweise Satz 2.9.2.

    Hinweis: Es bleibt nur noch zu zeigen, dass aus ($\ast$) und ($\ast\ast$)
    die gleichmäßige Konvergenz folgt. Dazu sei für jedes $k \in \nat$ und $1
    \le j < k$ die kleinste Zahl $x_{jk}$ mit $\frac{j}{k} \le F( x_{jk} + 0)$
    und es sei $x_{0k} := - \infty$, $x_{kk} := \infty$.

    Dann ist
    \[ F(x_{jk}) \le \frac{j}{k} \qRq F(x_{j+1,k}) - F(x_{jk} + 0) \le \rez{k}.
      \tag{1} \]
    Daher gilt für jedes $x \in (x_{jk}, x_{j+1,k})$:
    \begin{align*}
      F_n^\omega(x_{jk}) - F(x_{jk}+0) - \rez{k}
      &\overset{\text{(1)}}{\le} F_n^\omega (X_{jk} + 0) - F(x_{j+1,k}) \\
      &\le F_n^\omega(x_{jk}) - F(x) \\
      &\le F_n^\omega(x_{j+1,k}) -F(x_{jk} + 0) \\
      &\overset{\text{(1)}}{\le} F_n^\omega( x_{j+1,k} ) - F(x_{j+1,k}) + \rez{k}.
    \end{align*}
    Daraus folgt
    \begin{align*}
      \sup_{x \in \real} | F_n^\omega(x) - F(x) |
      & \le \max_{1 \le j < k} | F_n^\omega(x_{jk}) - F(x_{jk}) |
        + \max_{0 \le j < k} | F_n^\omega(x_{jk} + 0) - F(x_{jk} + 0) | + \rez{k} \\
      &\xrightarrow{\text{(ii)}} \rez{k} \text{ fast sicher für } n \to \infty.
    \end{align*}
  \end{enumerate}
\end{prgp}

Ab hier ist $F$ stetig.
\begin{thm}[Smirnow]
  Für $y > 0$ gilt
  \[\lim_{n \to \infty} \pP \left[ \sqrt{n}
      \sup_{x \in \real} (F_n^\omega(x) - F(x)) < y \right]
    = 1 - e^{-2y^2}. \]
\end{thm}

\begin{thm}[Kolmogorow]
  Es gilt
  \[ \lim_{n \to \infty} \pP \left[ \sqrt{n}
      \sup_{x \in \real} | F_n^\omega(x) - F(x) | < y \right]
    = \sum_{k = -\infty}^\infty (-1)^k 1 - e^{-2 k^2 y^2} =: K(y). \]
\end{thm}