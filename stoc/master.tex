\documentclass[
 a4paper,
 12pt,
 parskip=half
 ]{scrreprt}

\usepackage{../.tex/settings}

\usepackage{../.tex/mathpkgs}
\usepackage{../.tex/mathcmds}

\usepackage[numbers,with_chapter]{../.tex/fancy_thm}

\swapnumbers
\theoremstyle{plain}
%\newtheorem{thm}{Satz}[section] % reset theorem numbering for each chapter
%\newtheorem{lem}[thm]{Lemma}    

\theoremstyle{definition}
\newtheorem{defn}[thm]{Definition} 
\newtheorem{folg}[thm]{Folgerung} 
\newtheorem{rmrk}[thm]{Bemerkung} 
\newtheorem{deno}[thm]{Bezeichnungen}
\newtheorem{exmp}[thm]{Beispiel}
\newtheorem{aufg}[thm]{Aufgabe} 
\newtheorem{prgp}[thm]{} % Numbered paragraph

\newtheorem*{rmrk*}{Bemerkung}
\newtheorem*{exmp*}{Beispiel}
\newtheorem*{defn*}{Definition}
\newtheorem*{deno*}{Bezeichnungen}

\numberwithin{equation}{section}

\hypersetup{
  pdftitle={Stochastik},
  pdfauthor={Jonas Hippold},
  hidelinks
}

%opening
\title{Vorlesung\\Stochastik}
\subtitle{Sommersemester 2017}
\author{Vorlesung: Prof. Dr. Zoltán Sasvári\\Mitschrift: Jonas Hippold}

\begin{document}

\maketitle

\tableofcontents

\clearpage

\section*{Organisatorisches}
Aufgaben:\\
\url{http://www.math.tu-dresden.de/~sasvari/Download/BaStoch-Aufgaben/}

Abgabetermin für die Hausaufgaben in der Regel Mittwoch 13:00.

Prüfungsvorleistung: 50 \% der Aufgabenpunkte.

Schriftliche Prüfung: 90 Minuten

Formulierung von Definitionen, Sätzen; Beweise, Aufgaben ähnlich zu den
Übungsaufgaben.

Formelsammlung: Ein A4-Blatt, beidseitig beschrieben

Ausweichtermin für die Übung am 14. April ist der 12. April.

\subsection*{Literatur}
\begin{itemize}
\item H. Bauer: Wahrscheinlichkeitstheorie und Grundzüge der Maßtheorie
\item A. Rényi: Wahrscheinlichkeitsrechnung
\item K. D. Schmidt: Maß und Wahrscheinlichkeit
\end{itemize}

\chapter{Grundbegriffe}

\section{Wahrscheinlichkeitsräume}

Aufgabe der Wahrscheinlichkeitstheorie: Das Problem des \emph{Zufalls} mit einem
\emph{mathematischen Modell} zu erfassen.

Was ist Zufall? Was sind zufällige Ereignisse?

\textbf{Beispiele.}
\begin{enumerate}[a)]
\item Das Werfen eines Würfels. Ein Zufälliges Ereignis ist zum Beispiel das
  Auftreten einer \emph{geraden Augenzahl}.
\item Die Anzahl von vermittelten Telefongesprächen während einer bestimmten
  Stunde in einer Telefonzentrale.
\item Geburten (Junge oder Mädchen).
\end{enumerate}

Schon im 16. Jahrhundert betrachtete man Aufgaben rein
wahrscheinlichkeitstheoretischen Charakters. Die Lösung wurde meistens mit Hilfe
der Kombinatorik bestimmt.

Ein exaktes mathematisches Modell wurde erst zwischen 1909 und 1933
ausgearbeitet. Wichtige Beiträge lieferten zum Beispiel Borel, Wiener, Paley,
Zygmund, Lomnicki, Steinhaus und \emph{Kolmogorov}. Etwa in dieser Zeit wurden
auch das abstrakte Maß und Integral entwickelt.

\begin{prgp}[Modell für zufällige Ereignisse und
  Wahrscheinlichkeit]
\begin{enumerate}[a)]
\item \textbf{Ereignisse.}
  
  Ereignisalgebra (Ereignisse $A,B$ $\Rightarrow$ nicht $A$, $A$ oder $B$, $A$
  und $B$). 

  Modell: Algebra von Teilmengen einer Menge (Stone 1937).

  Eine Familie $\mA$ von Teilmengen einer Menge $\Omega$ heißt \emph{Algebra} ,
  falls
  \begin{enumerate}[a)]
  \item $A \in \mA$ $\Rightarrow$ $A^c \in \mA$
  \item $A, B \in \mA$ $\Rightarrow$ $A \cup B \in \mA$
  \end{enumerate}
  Meist: Erweiterung zu einer $\sigma$-Algebra $\sigma(\mA)$.
\item \textbf{Wahrscheinlichkeit.}

  Tritt ein zufälliges Ereignis $A$ in $n$ Versuchen $m$-mal ein ($0 \le m \le
  n$), so heißt
  \[ H_n(A) := m \]
  die \emph{absolute Häufigkeit} und
  \[ h_n(A) := \frac{m}{n} \]
  die \emph{relative Häufigkeit} von $A$ bezüglich dieser $n$ Versuche.

  Eigenschaften:
  \begin{enumerate}[(i)]
  \item $0 \le h_n(A) \le 1$
  \item $h_n($ Sicheres Ereignis $) = 1$, $h_n($ Unmögliches Ereignis $) = 0$
  \item $h_n (A \cup B) = h_n(A) + h_n(B)$, wenn $A$ und $B$ nicht gleichzeitig
    eintreten können.
  \end{enumerate}
  Die relative Häufigkeit zeigt eine \emph{Stabilität}, wenn $n$ groß ist.

  Beispiel: Wurf eines Geldstücks, $A$: Auftreten von ``Kopf''
  \begin{center}
    \begin{tabular}{l|l|l|l}
      & $n$ & $ H_n(A)$ & $ h_n(A)$ \\
      \hline
      Buffon (18. Jh.) & 4040 & 2048 & 0.5080 \\
      Pearson (20. Jh.) & 12000 & 6019 & 0.5016 \\
      Pearson & 24000 & 12012 & 0.5005
    \end{tabular}
  \end{center}

  Der ``Grenzwert'' $n \to \infty$ führt zu einem endlich additiven Maß auf $\mA$.
\end{enumerate}
\end{prgp}

\begin{defn}
  \begin{itemize}
  \item Es sei $\Omega \ne \emptyset$ eine Menge, $\mA$ eine $\sigma$-Algebra von
  Teilmengen von $\Omega$ und $\pP$ ein Maß auf $\mA$ mit $\pP(\Omega) = 1$. Das
  Tripel $(\Omega, \mA, \pP)$ heißt \emph{Wahrscheinlichkeitsraum} (kurz
  $\pW$-Raum). Die Elemente von $\mA$ heißen \emph{Ereignisse}, die Elemente von
  $\Omega$ heißen \emph{Ergebnisse, Stichproben, Realisierungen}\footnote{%
    Man findet auch  die Bezeichnung \emph{Elementarereignisse}, jedoch ist $\{w\}
    \notin \mA$ möglich, zum Beispiel für $\mA = \{ \emptyset, \Omega \}$. Also
    müssen Elementarereignisse keine Ereignisse sein.}.
  \item Das Maß $\pP$ heißt \emph{Wahrscheinlichkeitsmaß}.
  \item $\emptyset$ heißt \emph{unmögliches Ereignis}, $\Omega$ das \emph{sichere Ereignis}.
  \item $\obar{A} := \Omega \setminus A = A^c$
  \item Ereignisse $A$ mit $\pP(A) = 1$ bzw. $\pP(A) = 0$ heißen \emph{fast
      sicher} oder \emph{fast unmöglich}. Statt ``$\pP$-fast überall'' sagt man
    ``fast sicher'' oder ``mit Wahrscheinlichkeit 1''.
  \end{itemize}
\end{defn}

\textbf{Knobelaufgabe}.
Zwei Spieler $S_1, S_2$, drei Würfel $W_1, W_2, W_3$ mit den Augenzahlen
\begin{itemize}
\item $W_1: 5, 7, 8, 9, 10, 18$
\item $W_2: 2, 3, 4, 15, 16, 17$
\item $W_3: 1, 6, 11, 12, 13, 14$
\end{itemize}
Spiel: Zuerst wählt $S_1$ einen Würfel, dann $S_2$. Beide würfeln; wer die
größere Augenzahl hat, bekommt 1 EUR. Sie sind $S_1$. Welchen Würfel würden Sie
wählen?
\begin{align*}
  \pP( W_1 > W_2 ) &= \frac{21}{36} > \rez{2}, & W_1 \text{ ``besser'' als } W_2, \\
  \pP( W_2 > W_3 ) &= \frac{21}{36} > \rez{2}, & W_2 \text{ ``besser'' als } W_3, \\
  \pP( W_3 > W_1 ) &= \frac{21}{36} > \rez{2}, & W_3 \text{ ``besser'' als } W_1.
\end{align*}
Die beste Taktik ist also, den zweiten Spieler zuerst wählen zu lassen.

\textbf{Sprechweisen in der $\pW$-Theorie}.
Seien $A,B \in \mA$.
\begin{enumerate}[i)]
\item $A \subset B$: Aus $A$ folgt $B$.
\item $A \cap B = \emptyset$: $A$ und $B$ sind \emph{unvereinbar}.
\item $A \setminus B$: Es tritt $A$, aber nicht $B$ ein.
\item Seien $A_1, A_2, \ldots$ Ereignisse.
  \begin{itemize}
  \item $\bigcup_{n=1}^\infty A_n$: Mindestens ein $A_n$ tritt ein.
  \item $\bigcap_{n=1}^\infty A_n$: Alle $A_n$ treten ein.
  \item $\liminf_{n \to \infty} A_n := \bigcup_{n=1}^\infty \bigcap_{m=n}^\infty
    A_m$: Von einem Index ab treten alle $A_n$ ein.
  \item $\limsup_{n \to \infty} A_n := \bigcap_{n=1}^\infty \bigcup_{m=n}^\infty
    A_m$: Unendlich viele $A_n$ treten ein.
  \end{itemize}
\end{enumerate}

\begin{exmp}
  \begin{enumerate}[a)]
  \item Werfen eines Würfels mit den Augenzahlen $1, \ldots, 6$.
    \[ \Omega := \{ 1, 2, 3, 4, 5, 6 \}, \qquad \mA := \pot(\Omega)\footnote{%
      Das heißt, es gibt $2^6 = 64$ mögliche Ereignisse.}. \]
    \begin{itemize}
    \item Erscheinende Augenzahl ist $i$: $\{ i \} \in \mA$.
    \item Erscheinende Augenzahl ist gerade: $\{ 2, 4, 6 \}$.
    \item $\pP(A) = \frac{|A|}{6}$, $A \in \mA$.
    \item $\pP(\{i\}) = \rez{6}$, $A \in \mA$, wenn es sich um einen
      ``richtigen'' Würfel handelt.
    \end{itemize}
  \item Schießen auf eine Scheibe mit Radius 20 cm. Sie wird immer getroffen
    \[ \Omega = \{ (x,y) \in \real^2 : x^2 + y^2 \le 20^2 \}, \qquad
      \mA = \{ B \in \borel(\real^2) : B \subset \Omega \}. \]
    Nehmen wir an, dass die Treffer gleichmäßig verteilt sind, dann gilt
    \[ \pP( B ) := \frac{\lambda(B)}{\lambda(\Omega)}, \]
    wobei $\lambda$ das Lebesgue-Maß auf $\real^2$ bezeichnet.

    Es wäre auch möglich, $\mA :=$ Lebesgue-messbare Mengen $B \subset \Omega$
    zu definieren.
  \end{enumerate}
\end{exmp}

\begin{thm}
  Für jedes $\pW$-Maß $\pP$ gilt:
  \begin{enumerate}[(i)]
  \item $\pP( (\obar{A}) ) = 1 - \pP(A)$.
  \item $\pP( B \setminus A) = \pP(B) - \pP(A)$, wenn $A \subset B$.
  \item $\pP(A \cup B) = \pP(A) + \pP(B) - \pP(A \cap B)$.
  \item $A_1 \subset A_2 \subset \ldots$ $\Rightarrow$ $\pP( \bigcup_1^\infty
    A_n) = \lim_n \pP(A_n)$.
  \item $A_1 \supset A_2 \supset \ldots$ $\Rightarrow$ $\pP( \bigcap_1^\infty
    A_n) = \lim_n \pP(A_n)$.
  \end{enumerate}
\end{thm}

\begin{proof}
  Das ist bereits aus der Maßtheorie bekannt.
\end{proof}

\begin{defn}
  Zwei Ereignisse $A,B \in \mA$ heißen \emph{unabhängig} (bezüglich $\pP$), wenn
  \[ \pP(A \cap B) = \pP(A) \cdot \pP(B). \]
  Allgemeiner: Eine Familie von Ereignissen in $\mA$ heißt unabhängig (bezüglich
  $\pP$), wenn für jede endliche Teilmenge $\emptyset \ne \{i_1, \ldots, i_n\}
  \subset I$ gilt:
  \[ \pP \left( \bigcap_{i=1}^n A_{i_j} \right) = \prod_{j=1}^n \pP( A_{i_j} ). \]
\end{defn}

\begin{exmp}
  \begin{enumerate}[a)]
  \item Wir betrachten das \emph{zweimalige} Werfen eines Würfels.
    \[ \Omega = \{ (i,j) : 1 \le i,j \le 6 \}, \qquad \mA = \pot(\Omega), \qquad
      \pP( \{(i,j)\}) = \rez{36}. \]
    \begin{itemize}
    \item $A$: Beim ersten Wurf wurde eine Zahl $\le 3$ gewürfelt.
      \[ \pP(A) = \frac{18}{36} = \rez{2}. \]
    \item $B$: Beim zweiten Wurf wurde eine 6 gewürfelt.
      \[ \pP(B) = \frac{6}{36} = \rez{6}. \]
    \item $A \cap B$:
      \[ \pP( A \cap B ) = \frac{3}{36} = \pP(A) \cdot \pP(B), \]
      das heißt $A$ und $B$ sind unabhängig.
    \end{itemize}
  \item Bezeichnungen wie oben.
    \begin{itemize}
    \item $A_1$: Beim ersten Wurf ungerade.
    \item $A_2$: Beim zweiten Wurf ungerade.
    \item $A_3$: Die Summe der geworfenen Augen ist ungerade.
    \end{itemize}
    Dann sind je zwei dieser Ereignisse unabhängig\footnote{%
      Zum Beispiel $\pP(A_1 \cap A_3) = \pP(A_1) \cdot \pP(A_3) = \rez{2} \cdot
      \rez{2} = \rez{4}$ usw.}.
    Jedoch ist die Familie der Ereignisse \emph{nicht unabhängig}, denn
    \[ \pP( A_1 \cap A_2 \cap A_3 ) = 0, \qquad \pP(A_1) \cdot \pP(A_2) \cdot
      \pP(A_3) > 0.\]
    Aus der \emph{paarweisen} Unabhängigkeit kann also nicht auf Unabhängigkeit
    \emph{der Familie} geschlossen werden.
  \end{enumerate}
\end{exmp}

\begin{defn}
  Es sei $(\Omega, \mA, \pP)$ ein $\pW$-Raum und $\mA_1, \mA_2 \subset
  \mA$ Familien von Ereignissen. Dann heißen $\mA_1$ und $\mA_2$
  \emph{unabhängig}, wenn
  \[ \pP( A_1 \cap A_2 ) = \pP(A_1) \cdot \pP(A_2) \]
  für alle $A_1 \in \mA_1$, $A_2 \in \mA_2$.
\end{defn}

\begin{lem}[Approximationssatz]
  Es sei $(\Omega, \mA, \pP)$ ein $\pW$-Raum und $\mB$ die von einer Algebra
  $\borel_0$ von Ereignissen aus $\mA$ erzeugte $\sigma$-Algebra. Dann existiert
  für jedes $A \in \mB$ eine Folge $A_n \in \mB_0$ mit
  \begin{enumerate}[(i)]
  \item $\lim_{n \to \infty} \pP( A_n \Delta A ) = 0$.
  \item $\pP(A) = \lim_{n \to \infty} \pP(A_n)$.
  \end{enumerate}
\end{lem}

\begin{thm}
  Von unabhängigen Ereignisalgebren $\mA_0$ und $\mB_0$ erzeugte
  $\sigma$-Algebren $\mA$ und $\mB$ sind unabhängig.
\end{thm}

\begin{proof}
  Aufgabe, Hinweis: Lemma 1.1.8 benutzen.
\end{proof}

\begin{proof}[Beweis von Lemma 1.1.8]
  (i) ist äquivalent zu
  \[ \lim_{n \to \infty} \pP( A \setminus A_n) = \lim_{n \to \infty} \pP( A_n
    \setminus A ) = 0. \tag{1} \]
  Bezeichne $\mB^*$ die Gesamtheit aller Ereignisse $A \in \mB$, für die eine
  Folge $A_n \in \mB_0$ mit (i) bzw. (1) existiert.

  Zu zeigen: $\mB^* = \mB$. Es gilt $\mB_0 \subset \mB^* \subset \mB$. Wir
  zeigen, dass $\mB^*$ eine $\sigma$-Algebra ist\footnote{%
    Daraus folgt $\mB^* = \mB$, weil $\mB$ nach Definition die kleinste
    $\sigma$-Algebra ist, die $\mB_0$ enthält.}.

  Aufgabe: Zeige, dass $\mB^*$ eine Algebra ist. Benutze dabei
  \[ (A \cap B) \Delta (C \cap D) \subset (A \Delta C) \cup (B \Delta D). \]

  Sei nun $(B_j)_j \in \nat \subset \mB^*$ beliebig. Noch zu zeigen:
  \[ C := \bigcup_{j=1}^\infty  B_j \in \mB^*. \]
  Definiere $C_n := \bigcup_1^n B_j \in \mB^*$, da $\mB^*$ eine Algebra ist. Es
  existieren $A_n$ in $\mB_0$ mit
  \begin{align*}
    \pP( C_n \setminus A_n ) &< \rez{n}, & \pP(A_n \setminus C_n ) < \rez{n}. \tag{2}
  \end{align*}
  Behauptung:
  \[ \lim_{n \to \infty} \pP( C \setminus A_n ) = \lim_{n \to \infty} \pP( A_n
    \setminus C ) = 0, \]
  das heißt $C \in \mB^*$. Die Behauptung folgt aus
  \begin{align*}
    A_n \setminus C &\subset A_n \setminus C_n \\
    C \setminus A_n &\subset ( C\setminus C_n ) \cup (C_n \setminus A_n),
  \end{align*}
  denn
  \[ \pP( C \setminus A_n ) < \pP( C \setminus C_n ) + \rez{n}. \]
  Es gilt $\pP( C \setminus C_n ) \to 0$, wegen $\bigcap_n(C \setminus C_n) =
  \emptyset$ und der Stetigkeit von oben.

  (ii) folgt aus
  \[ \pP(A) = \pP( A_n \cap A ) + \pP( \obar{A}_n ) \cap A ) = \pP(A_n) -
    \underbrace{\pP( A_n \cap \obar{A} )}_{\to 0} +
    \underbrace{\pP( \obar{A}_n \cap A )}_{\to 0}. \qedhere \]
\end{proof}

%%% Local Variables:
%%% TeX-master: "master"
%%% End:


\clearpage

%%\section{Aufgaben}
%%Siehe \verb+Aufgaben-1-2.pdf+
\section{Aufgaben}
\begin{aufg}
  Zeigen Sie, dass die Familie $\mB^*$ aus Lemma 1.1.8 eine Algebra ist.
\end{aufg}

Wir rechnen die Eigenschaften der Algebra nach:
\begin{enumerate}[(i)]
\item $\Omega \in \mB^*$. Da $\mB$ eine Algebra ist, gilt $\Omega \in \mB$.
  Betrachte die Folge $(A_n)$, $A_i = \Omega$ für alle $i$. Dann gilt
  \[ \lim_{n \to \infty} \pP( A_n \Delta \Omega ) = \pP( \Omega \Delta \Omega)
    = 0.\]
\item $A \in \mB^* \Rightarrow \obar{A} \in \mB^*$. Sei $A \in \mB^*$, dann
  existiert eine Folge $(A_n) \subset \mB_0$, sodass
  \[ \lim_{n \to \infty} \pP( A_n \Delta A ) = 0, \qquad \lim_{n \to \infty}
    \pP( A_n \setminus A ) =  0, \pP( A_n \setminus A ) =  0. \]
  Daraus folgt
  \[ \lim_{n \to \infty} \pP( \obar{A} \setminus \obar{A}_n ) =
    \lim_{n \to \infty} \pP( \obar{A} \cap A_n ) =
    \lim_{n \to \infty} \pP( A_n \setminus A ) = 0 \]
  sowie
  \[ \lim_{n \to \infty} \pP( \obar{A}_n \setminus \obar{A} ) =
    \lim_{n \to \infty} \pP( \obar{A}_n \cap A ) =
    \lim_{n \to \infty} \pP( A \setminus A_n ) = 0, \]
  also gilt
  \[ \lim_{n \to \infty} \pP( \obar{A}_n \Delta \obar{A} ) = 0. \]
  Somit ist $\obar{A} \in \mB^*$.
\item Sei $(A_n)_{n=1}^N \subset \mB^*$. Die zugehörigen Folgen seien
  $(A_{n,i})$. Es gilt
  \begin{align*}
    \lim_{i \to \infty} \pP \left(
      \left( \bigcup_{n=1}^N A_n \right) \Delta
      \left( \bigcup_{n=1}^N A_{n,i} \right)
    \right)
    &=
    \lim_{i \to \infty} \pP \left(
      \left( \bigcap_{n=1}^N \obar{A}_n \right) \Delta
      \left( \bigcap_{n=1}^N \obar{A}_{n,i} \right)
      \right) \\
    \intertext{mit dem Hinweis $(A \cap B) \Delta (C \cap D) \subset (A \Delta B) \cup (C \Delta D)$:}
    &\le
    \lim_{i \to \infty} \pP \left( \bigcup_{n=1}^N
      \left( \obar{A}_n \Delta
      \obar{A}_{n,i} \right)
      \right) \\
    \intertext{und wegen der Subadditivität von Maßen:}
    &\le
    \lim_{i \to \infty} \sum_{n=1}^N \pP \left(
      \obar{A}_n \Delta
      \obar{A}_{n,i}
      \right) \\
    &=
     \sum_{n=1}^N \lim_{i \to \infty} \pP \left(
      \obar{A}_n \Delta
      \obar{A}_{n,i}
      \right) = 0.
  \end{align*}
  Also ist $\bigcup_1^N A_n \in \mB^*$ und damit ist $\mB^*$ eine Algebra.
\end{enumerate}

\begin{aufg}
  Beweisen Sie Satz 1.1.9.

  [Hinweis: Man benutze Lemma 1.1.8.]
\end{aufg}

Zwei Ereignisse $A,B$ sind unabhängig, wenn
\[ \pP(A \cap B) = \pP(A) \cdot \pP(B). \]

Zu zeigen: Die von unabhängigen Algebren $\mA_0, \mB_0$ erzeugten
$\sigma$-Algebren $\mA, \mB$ sind unabhängig.

\begin{proof}
  Für alle $A_0 \in \mA_0$, $B_0 \in \mB_0$ sind $A_0$ und $B_0$ unabhängig.

  Seien $A \in \mA$, $B \in \mB$. Wähle zwei Folgen $(A_n) \subset \mA_0$ und
  $(B_n) \subset \mB_0$ mit
  \[ \lim_{n \to \infty} \pP( A_n \Delta A) = 0, \qquad \lim_{n \to \infty} \pP(
    B_n \Delta B) = 0 \]
  wie in Lemma 1.1.8. Dann gilt auch
  \[ \lim_{n \to \infty} \pP( (A_n \cap B_n) \Delta (A \cap B)) = 0, \]
  weil
  \[ (A_n \cap B_n) \Delta (A \cap B) \subset (A_n \Delta A) \cup (B_n \Delta B). \]

  Nun folgt wegen Lemma 1.1.8(ii) und $\pP(A_n \cap B_n) = \pP(A_n) \cdot \pP(B_n)$:
  \[ \begin{aligned}
      \lim_{n \to \infty} \pP( (A_n \cap B_n) ) &= \pP ( (A \cap B ) ) \\
      \lim_{n \to \infty} \pP(A_n) \cdot \pP(B_n) &= \pP ( (A \cap B ) ) \\
      \pP(A) \cdot \pP(B) &= \pP ( (A \cap B ) )
    \end{aligned} \]
  und damit sind $A$ und $B$ unabhängig.
\end{proof}

\begin{aufg}[Poincaré]
  Man beweise die folgende Verallgemeinerung von Satz 1.1.4 (iii):
  \[ \pP( A_1 \cup \cdots \cup A_n ) = \sum_{k=1}^n (-1)^{k+1} \sum_{1 \le i_1 <
      \cdots < i_k \le n} \pP( A_{i_1} \cap A_{i_2} \cap \cdots \cap
    A_{i_k}). \]
  
  [Hinweis: Induktion nach n.]
\end{aufg}

\begin{proof}
  Induktionsanfang: Sei $n=2$, dann wissen wir aus Satz  1.1.4(iii):
  \[ \pP ( A_1 \cup A_2 ) = \pP( A_1 ) + \pP( A_2 ) - \pP( A_1 \cap A_2 ). \]
  Also gilt die Behauptung für $n=2$.

  Angenommen, die Behauptung gilt für ein $n$. Definiere das Ereignis $B := A_1
  \cup \cdots \cup A_n$. Dann ist
  \[ \pP( B \cup A_{n+1} ) = \pP( B ) + \pP( A_{n+1} ) - \pP( B \cap A_{n+1}
    ). \]
  Wegen
  \[ B \cap A_{n+1} = \left(\bigcup_{j=1}^n A_j\right) \cap A_{n+1} = \bigcap_{j=1}^n (A_i \cap
    A_{n+1}) \]
  folgt
  \[ \pP( B \cap A_{n+1} ) = \sum_{k=1}^n (-1)^{k+1} \sum_{1 \le i_1 < \cdots <
      i_k \le n} \pP( A_{i_1} \cap A_{i_2} \cap \cdots \cap A_{i_k} \cap A_{n+1}
    ) \]
  nach Induktionsvoraussetzung. Damit folgt nun
  \begin{align*}
    \pP( B \cup A_{n+1} ) =
    &\phantom{+} \sum_{k=1}^n (-1)^{k+1} \sum_{1 \le i_1 < \cdots < i_k \le n}
      \pP( A_{i_1} \cap A_{i_2} \cap \cdots \cap A_{i_k}) \\
    &+ \pP( A_{n+1} ) \\
    &- \sum_{k=1}^n (-1)^{k+1} \sum_{1 \le i_1 < \cdots < i_k \le n}
      \pP( A_{i_1} \cap A_{i_2} \cap \cdots \cap A_{i_k} \cap A_{n+1} ) \\
    \pP( B \cup A_{n+1} ) =
    &\phantom{+} \sum_{k=1}^{n+1} (-1)^{k+1}
      \sum_{1 \le i_1 < \cdots < i_k \le n+1} \pP( A_{i_1} \cap A_{i_2} \cap \cdots \cap A_{i_k})
  \end{align*}
  und die Behauptung gilt auch für $n+1$.
\end{proof}

\begin{aufg}
  Wie groß ist die Wahrscheinlichkeit, dass eine Ziehung beim Spiel ``6 aus 49''
  keine benachbarten Zahlen enthält? (Benachbart sind z.B. 2 und 3; 11 und 12).
\end{aufg}

Betrachte eine Lotterie mit den Zahlen 1 bis $n$, $\Omega_n := \{ 1, 2, \ldots,
n \}$. Werden $k$ Zahlen gezogen und der Größe nach geordnet, so ist die Menge
der gültigen Ereignisse
\[ \mA_k = \{ (x_1, x_2, \ldots, x_k) : x_1 < x_2 < \cdots < x_k, x_i \in \Omega
  \}. \]

Es gilt $|\mA_k| = \binom{n}{k}$. Für eine Lotterie 6 aus 49 gibt es also
$\binom{49}{6}$ mögliche Ereignisse.

Die Funktion $f: \Omega_{44} \to \Omega_{49}$ sei definiert durch
\[ f({x_1, \ldots, x_6}) = \{x_1, x_2 + 1, x_3 + 2, x_4 + 3, x_5 + 4, x_6 +5
  \}. \]
Diese Funktion ist offenbar bijektiv. Außerdem sind die Urbilder aller
Ziehungen, die keine aufeinander folgenden Zahlen enthalten, gültige Ereignisse
in der ``Urlotterie''. Also ist die Menge der gewünschten Ziehungen isomorph zur
Menge der gültigen Ereignisse $\mA_{44}$. Damit können wir die
Wahrscheinlichkeit bestimmen, dass eine Ziehung keine benachbarten Zahlen
enthält:
\[ \pP( \text{Keine benachbarten Zahlen} ) = \frac{\binom{44}{6}}{\binom{49}{6}}
  = \frac{7059052}{13983816} \approx 0.5048. \] 

%%% Local Variables:
%%% TeX-master: "master"
%%% End:


\clearpage

\section{Zufallsvariablen, Verteilungen}
\begin{defn}
  Sei $(\Omega, \mA, \pP)$ ein $\pW$-Raum und $(S, \mB)$ ein messbarer Raum.
  Eine \emph{Zufallsvariable} mit Werten in $S$ heißt jede $(\mA, \mB)$-messbare
  Abbildung $X: \Omega \to S$.

  In den Fällen $S = \real, \real^n, \complex, \complex^n$ ist $\mB$ im weiteren
  immer die $\sigma$-Algebra der Borelmengen. Für $S = \real$ heißt $X$
  \emph{Zufallsgröße} und für $S =\real^n$ \emph{Zufallsvektor}\footnote{%
  Für $\complex$ bzw. $\complex^n$ entsprechend \emph{komplexe Zufallsgröße} bzw.
  \emph{komplexer Zufallsvektor}.}.
\end{defn}

\begin{thm}
  \begin{enumerate}[(i)]
  \item Die Summen von Zufallsvektoren und Produkte von Zufallsgrößen sind
    wieder Zufallsvektoren bzw. -größen.
  \item Es seien $X$ eine Zufallsgröße und $f:\real \to \real$ eine
    Borel-messbare Funktion. Dann ist die Abbildung $f(X)$ eine Zufallsgröße.
  \item Sind $X_1, \ldots, X_n$ Zufallsgrößen und $f: \real^n \to \real$
    eine Borel-messbare Funktion, so ist $f(X_1, \ldots, X_n)$ eine
    Zufallsgröße. Weiterhin ist $Y := (X_1, \ldots, X_n)$ ein Zufallsvektor.
  \end{enumerate}
\end{thm}

\begin{proof}
  Das ist aus MINT bekannt.
\end{proof}

\begin{defn}
  Es sei $X: \Omega \to S$ eine Zufallsvariable. Bezeichnungen:
  \begin{align*}
    [ X \in B ] &:= \{ w \in \Omega : X(w) \in B \} = X^{-1}(B), \qquad B \in \mB \\
    \pP[ X \in B ] &:= P([X \in B]) =: \pP(X \in B)
  \end{align*}
  $[X \in B]$ ist das Ereignis ``$X$ liegt in $B$''. Setzt man
  \[ \mu_X(B) := \pP[ X \in B ], \qquad B \in \mB, \tag{1} \]
  so ist $\mu_X$ ein $\pW$-Maß auf $\mB$ \footnote{%
    Siehe Bildmaß in MINT.}. Dieses Maß heißt die \emph{Verteilung} von $X$
  (bezüglich $\pP$). Die Verteilung $\mu_X$ heißt \emph{stetig}, wenn für alle $s
  \in S$ gilt $\{s\} \in \mB$ und $\mu_X(\{s\}) = 0$.

  Sei nun $S = \real^n$. Die Verteilung $\mu_X$ heißt \emph{absolut
    stetig}\footnote{%
    Siehe MINT, Satz von Radon-Nikodym}, wenn eine Borel-messbare Funktion $p :
  \real^n  \to [0,\infty]$ existiert, so dass
  \[ \mu_X(A) = \int_A p \diffop \lambda, \qquad A \in \borel(\real^n). \]
  Dabei ist $\lambda$ das Lebesgue-Maß auf $\real^n$. Man nennt $p$ die
  \emph{Dichte}, \emph{Dichtefunktion} von $X$ oder $\mu_X$.
\end{defn}

\begin{rmrk*}
  Eine Borel-messbare Funktion $p : \real^n \to [0,\infty]$ ist genau dann die
  Dichte einer Verteilung, wenn sie $\lambda$-integrierbar ist und
  $\int_{\real^n} p \diffop \lambda = 1$.

  $(\Omega, \mA, \pP) := (\real^n, \borel(\real^n), p \diffop \lambda)$. Dann
  ist $p \diffop \lambda$ die Verteilung von $X$ und $p$ die Dichte.

  Absolut stetige Verteilungen sind stetig, die Umkehrung gilt im Allgemeinen
  nicht.
\end{rmrk*}

\begin{exmp}
  \begin{enumerate}[(a)]
  \item Für jedes $s \in S$ sei $\delta_s$ das durch die Einheitsmasse in $s$
    definierte $\pW$-Maß auf $\mB$ (Einpunktverteilung, Dirac-Maß). Eine
    Zufallsvariable besitzt eine Einpunktverteilung genau dann, wenn sie fast
    sicher konstant ($=s$) ist.
  \item Es sei $\{ s_n \}$ eine Folge in $S$ und $\{ p_n \}$ eine Folge
    nichtnegativer Zahlen mit $\sum_1^\infty p_k = 1$. Dann ist
    \[ \mu := \sum_{k=1}^\infty p_n \cdot \delta_{s_n} \]
    ein $\pW$-Maß. Jede solche Verteilung bzw. jede Zufallsvariable
    mit einer solchen Verteilung heißt \emph{diskret}.
  \item Es seien $n = 0, 1, \ldots$ und $0 \le p \le 1$. Definiere
    \[ B_n^p := \sum_{k=0}^n \binom{n}{k} \cdot p^k \cdot (1-p)^{n-k} \cdot
      \delta_k. \]
    Betrachte $(X, \mA, B_n^p)$; $X \supset \{1, \ldots, k\}$. Zum Beispiel $X =
    \real$, $\mA = \borel(\real)$. Dann ist
    \[ B_n^p(\real) = \sum_{k=0}^n \binom{n}{k} \cdot p^k \cdot (1-p)^{n-k} \cdot
      \delta_k(\real) = (p+(1-p))^n = 1. \]
    Also ist $B_n^p$ ein $\pW$-Maß auf $(\real, \borel(\real))$, die
    \emph{Binomialverteilung} mit den Parametern $n$ und $p$.

    Beispiel: wir betrachten eine Folge von Zufallsversuchen. In jedem der
    Versuche tritt ein gewisses Ereignis $A$ mit der Wahrscheinlichkeit $p$
    unabhängig von den Ausgängen der anderen Versuche ein.

    $X :=$ die Anzahl des Eintretens von $A$ in $n$ Versuchen.
    \[ \pP( X = k) = \binom{n}{k} \cdot p^k \cdot (1-p)^{n-k} \]
    für $k = 0, \ldots, n$.
  \item Die polynomiale Verteilung (Verallgemeinerung von (c)). Wir betrachten
    eine Folge von Versuchen. In jedem dieser Versuche treten gewisse
    \emph{unvereinbare} Ereignisse $A_1, \ldots, A_r$ mit den
    Wahrscheinlichkeiten $p_1, \ldots, p_r$; $p_1 + \cdots + p_r = 1$ unabhängig
    von den Ausgängen der anderen Versuche ein.

    $X_i :=$ die Anzahl des Eintretens von $A_i$ in $n$ Versuchen; $X_i$ ist
    binomialverteilt. $X := (X_1, \ldots, X_r)$ ist ein Zufallsvektor.

    $(x_1, \ldots, x_r) \in \real^r$ gehört zum Wertebereich von $X$
    $\Leftrightarrow$ $0 \le x_i \le n$, $x_i \in \integer$, $x_1 + \cdots + x_r
    = n$.
    \[ \pP( X_1 = x_1, \ldots, X_r = x_r) = \frac{n!}{x_1! \cdots x_r!} \cdot
      p_1^{x_1} \cdots p_r^{x_r}. \]
    Permutation mit Wiederholung.

    Eine solche Verteilung heißt \emph{Polynomialverteilung}.
  \item \emph{Poisson-Verteilung} mit dem Parameter $a$.
    \[ \pi_a := \sum_{k=0}^\infty e^{-a} \cdot \frac{a^k}{k!} \cdot \delta_k,
      \quad a \ge 0 (\text{auf } \borel(\real)). \]
    Wegen $e^a = \sum_0^\infty \frac{a^k}{k!}$ ist $\pi_a$ ein $\pW$-Maß.
  \item Die Funktion
    \[ g_{a,\sigma} = \rez{\sigma \cdot \sqrt{2 \pi}} \cdot
      e^{-\frac{(x \cdot a)^2}{2 \sigma^2}}, \quad x, a \in \real, \quad \sigma >
      0 \]
    ist eine Dichte.
    \begin{proof}
      Aus MINT ist bekannt:
      \[ \int_0^\infty e^{-x^2} \diffop x = \frac{\sqrt{\pi}}{2}. \]
      Es gilt
      \begin{align*}
        \rez{\sigma \sqrt{2 \pi}} \int_{-\infty}^\infty e^{- \frac{(x-a)^2}{2 \sigma^2}} \diffop x
        = \rez{\sqrt{\pi}} \int_{-\infty}^\infty e^{-u^2} \diffop u = \frac{2}{\sqrt{\pi}} \int_0^\infty e^{-u^2} \diffop u = 1
      \end{align*}
      mit der Substitution $u := \frac{x-a}{\sqrt{2} \sigma}$.
    \end{proof}
    das zugehörige $\pW$-Maß $\nu_{a,\sigma}$ heißt \emph{Normalverteilung} mit
    den Parametern $a$ und $\sigma$, auch \emph{Gauß-Verteilung}.

    Bezeichnung: $X \in N(a,\sigma)$.
  \item Die Funktion
    \[ p(x) = \rez{\pi \cdot (1+x^2)}, \quad x \in \real \]
    ist eine Dichte.
    \begin{proof}
      \[ \int_{-\infty}^\infty \rez{1+x^2} \diffop x = \arctan(x)
        |_{-\infty}^\infty = \frac{\pi}{2} -
        \left( - \frac{\pi}{2} \right) = \pi. \qedhere\]
    \end{proof}
    Die zugehörige Verteilung wird \emph{Cauchy-Verteilung} genannt.
  \item Es sei $B \subset \real^n$ eine Borel-Menge mit $\lambda(B) > 0$,
    \[ p_B(X) := \frac{\ind_B(x)}{\lambda(B)}, \quad x \in \real^n \]
    ist eine Dichte. \emph{Gleichverteilung} auf $B_i$.
  \item \emph{Exponentialverteilung} mit dem Parameter $\lambda > 0$. Sie ist
    gegeben durch die Dichte
    \[ p(x) = \begin{cases}
        \lambda \cdot e^{-\lambda x}, &x \ge 0, \\
        0, &x < 0.
      \end{cases} \]
  \item Die \emph{Gammaverteilung} mit den Parametern $a > 0$ und $\lambda > 0$.
    Dichte:
    \[ p(x) = \begin{cases}
        \frac{\lambda^a \cdot x^{a-1}}{\Gamma(a)} \cdot e^{-\lambda x},
        & x > 0, \\
        0, &x \le 0,
      \end{cases} \]
    wobei $\Gamma(a) = \int_0^\infty x^{a-1}e^{-x} \diffop x$.
  \end{enumerate}
\end{exmp}

\begin{rmrk*}
  \begin{enumerate}[(i)]
  \item $0 < \Gamma(a) < \infty$.
  \item $\Gamma(a+1) = a \cdot \Gamma(a)$.
  \item $\Gamma(n+1) = n!$ für alle $n \in \nat_0$.
  \item $\Gamma(1/2) = \sqrt{\pi}$.
  \end{enumerate}
\end{rmrk*}

\begin{thm}
  Jedes $\pW$-Maß auf $\real^n$ lässt sich zerlegen als
  \[ \mu = p_1 \cdot \mu_d + (1-p_1) \cdot \mu_c = p_1 \cdot \mu_d + p_2 \cdot
    \mu_{as} +  p_3 \cdot \mu_s. \]
  Hierbei sind $p_1, p_2, p_3 \ge 0$ mit $p_1 + p_2 + p_3 = 1$. $\mu_d, \mu_c,
  \mu_{as}, \mu_s$ sind $\pW$-Maße; $\mu_d$ ist diskret, $\mu_c$ ist stetig,
  $\mu_{as}$ ist absolut stetig, $\mu_s$ ist stetig und singulär im folgenden
  Sinne: Es gibt eine $\lambda$-Nullmenge $B \subset \real^n$ mit $\mu_s(B) =
  1$.

  Die Summanden der Zerlegung sind eindeutig.
\end{thm}

\begin{proof}
  Maßtheorie.
\end{proof}

\begin{defn}
  Sind $X_1, \ldots, X_n$ Zufallsgrößen, so ist $Y := (X_1, \ldots, X_n)$ ein
  Zufallsvektor (Satz 1.3.2). Die Verteilung von $Y$ auf $\real^n$ wird die
  \emph{gemeinsame Verteilung} der $X_1, \ldots, X_n$ genannt.
\end{defn}

\begin{thm}[Transformationssatz, Spezialfall]
  Für jede Borel-messbare Funktion $g: \real^n \to \real$, die nichtnegativ oder
  $\mu_Y$-integrierbar ist, gilt
  \[ \int_\Omega g(Y(\omega)) \diffop \pP(\omega) = \int_{\real^n} g(X_1,
    \ldots, X_n) \diffop \mu_Y (X_1, \ldots, X_n). \]
\end{thm}

\begin{proof}
  MINT.
\end{proof}

Die $\pW$-Maße auf $\real$  können mit Hilfe auf $\real$ definierter Funktionen beschrieben werden.

\begin{defn}
  Es sei $\mu$ ein $\pW$-Maß auf $\real$. Die Funktion
  \[ F(x) := \mu((-\infty, x)), \quad x \in \real \]
  heißt die \emph{Verteilungsfunktion} von $\mu$. Ist $\mu$ die Verteilung einer
  Zufallsgröße $X$, so nennt man $F$ auch die Verteilungsfunktion von $X$. Dann gilt:
  \[ F(x) = \pP[X<x], \quad x \in \real \]
  (wegen (1.3.3.1)).
\end{defn}

\begin{lem}
  Jede Verteilungsfunktion $F$ besitzt die folgenden Eigenschaften:
  \begin{enumerate}[(i)]
  \item $x \le y \Rightarrow F(x) \le F(y)$;
  \item $\lim_{x \to - \infty} F(x) = 0$, $\lim_{x \to \infty} F(x) = 1$;
  \item $F$ ist linksseitig stetig.
  \end{enumerate}
  Jede reelle Funktion $F$ auf $\real$ mit den Eigenschaften (i)-(iii) ist die
  Verteilungsfunktion einer Zufallsgröße.
  \label{lem:1_3_9}
\end{lem}

\begin{proof}
  Aufgabe. Hinweis: Für (ii) und (iii) benutze man (1.1.4) (iv-v).

  Hinweis für die letzte Aussage: $\Omega := (0, 1)$, $\mA :=$ Borel-Mengen in $(0, 1)$,
  $\pP :=$ Lebesgue-Maß auf $(0, 1)$.
\end{proof}

\begin{lem}
  Ist $F$ die Verteilungsfunktion der Zufallsgröße $X$, so gilt:
  \begin{enumerate}[(i)]
  \item $\pP (a \le X < b) = F(b) - F(a)$;
  \item $\pP (X = a) = F (a + 0) - F (a);$\footnotemark
  \item $\pP (a < X < b) = F(b) - F(a + 0)$;
  \item $\pP (a \le X \le b) = F (b + 0) - F(a)$;
  \item $\pP (a < X \le b) = F (b + 0) - F (a + 0)$, $(a, b \in R, a < b)$.
  \item Besitzt $X$ eine Dichte $p$, so gilt:
    \[ F(x) = \int_{-\infty}^x p(t) \diffop t, \quad x \in \real. \]
  \end{enumerate}
\end{lem}
\footnotetext{Aus (ii) folgt: $X$ besitzt eine stetige Verteilung genau dann,
  wenn $F$ stetig ist.}

\begin{proof}
  Aufgabe.
\end{proof}

\begin{thm}
 Die Zufallsgröße $X$ besitze eine Dichte $p$ und sei $F$ die
 Verteilungsfunktion von $X$. Dann ist $F$ $\lambda$-fast überall
 differenzierbar und
 \[ F' = p, \quad \lambda\text{-fast überall.} \]
\end{thm}

\begin{proof}
  Spezialfall, $p$ stetig: Aufgabe.
  
  Der allgemeine Fall ist aus der Maßtheorie bekannt, siehe MINT.
\end{proof}

%%% Local Variables:
%%% TeX-master: "master"
%%% End:

\clearpage

%\section{Aufgaben}
%Siehe \verb+Aufgaben-1-4.pdf+.
\section{Aufgaben}
\begin{aufg}
  Es sei $\Omega := [0, 1]$, $\mA$ die $\sigma$-Algebra der Borel-
  Mengen in $[0, 1]$ und $\pP$ das Lebesgue-Maß auf $[0, 1]$. Für jedes $n = 1, 2,
  \ldots$ sei
  \[ A_n := \left[ 0, \rez{2^n} \right] \cup
    \left[  \frac{2}{2^n}, \frac{3}{2^n} \right]
    \cup \cdots \cup
    \left[ \frac{2^n-2}{2^n}, \frac{2^n-1}{2^n} \right]. \]
  Dann ist die Folge $\{A_n\}$ dieser Ereignisse unabhängig.
\end{aufg}

\begin{proof}
  Es gilt
  \[ \bigcap_{i=1}^k A_i = \left[ 0,\rez{2^k} \right] \tag{$\circ$}. \]
  Das zeigt man zum Beispiel durch vollständige Induktion.

  Induktionsanfang:  Für $k=2$ gilt
  \[ A_1 \cap A_2 = \left[ 0, \rez{2} \right] \cap \left( \left[ 0, \rez{4}
      \right] \cup \left[ \rez{2}, \frac{3}{4} \right] \right) = \left[
      0,\rez{4} \right]. \]
  Induktionsschritt: Es gelte ($\circ$) für ein $k$, dann gilt
  \[ \bigcap_{i=1}^{k+1} A_i = \left[ 0,\rez{2^k} \right] \cap A_{k+1} =  \left[
      0,\rez{2^{k+1}} \right]. \]
  Aus ($\circ$) folgt die Unabhängigkeit der Folge. Für Unabhängigkeit muss
  \[ \pP \left( \bigcap_{i=1}^n A_i \right) = \prod_{i=1}^k \pP(A_i) \]
  gelten. Für jedes $A_n$ ist $\pP(A_n) = \rez{2}$. Also
  \[ \pP \left( \bigcap_{i=1}^n A_i \right) = \rez{2^n} = \left( \rez{2}
    \right)^k = \prod_{i=1}^k \pP(A_i). \qedhere \]
\end{proof}

\begin{aufg}
 Beweisen Sie Lemma \ref{lem:1_3_9}.
\end{aufg}

\begin{enumerate}[(i)]
\item Für $x \le y$ gilt $\mu((-\infty,x)) \le \mu((-\infty,y))$. Also ist $F(x)
  \le F(y)$.
\item Für jedes Maß $\mu$ gilt $\mu(A) \ge 0$ für $A \in \mA$, also ist auch
  $F(x) \ge 0$ für alle $x$.

  Damit folgt mit (i), dass $\lim_{x \to -\infty} F(x) = 0$.

  Weil $\mu$ ein $\pW$-Maß ist, gilt $\mu(\real) = 1$. Also gilt $F(x) \le 1$
  für alle $x$.

  Damit folgt aus (i), dass $\lim_{x \to \infty} F(x) = 1$.
\item Die Linksstetigkeit folgt aus Satz 1.1.4.(iv). Für jede steigende Folge $\{ x_n \}$
  mit $x_n \xrightarrow{n \to \infty} x$ gilt $(-\infty, x_{n-1}) \subset
  (-\infty, x_n)$. Damit ist
  \[ \mu \left( \bigcup_{n=1}^\infty (-\infty, x_n) \right) = \lim_{n \to
      \infty} \mu( (-\infty, x_n)  = \mu( (-\infty, x) ) \]
  und damit folgt $\lim_{x_n \uparrow x} F(x_n) = F(x)$.
\end{enumerate}

Jede reelle Funktion $F$ auf $\real$ mit den Eigenschaften (i)-(iii) ist die
Verteilungsfunktion einer Zufallsgröße.

\begin{proof}
  Sei $F: \real \to \real$ mit den Eigenschaften (i)-(iii). Dann existiert ein
  Prämaß $\mu_F$ auf der Algebra $\mA_H$ der rechtsoffenen Intervalle mit:
  \begin{align*}
    \mu_F( [a,b) ) &= F(b) - F(a), \\
    \mu_F((-\infty,b)) &= F(b).
  \end{align*}
  Dieses Prämaß kann zu einem Maß $\obar{\mu}_F$ auf $\borel(\real)$ fortgesetzt
  werden, dem Lebesgue-Stieltjes-Maß. Dann ist $\obar{\mu}_F$ ein $\pW$-Maß,
  denn
  \[ \obar{\mu}_F(\real) = \lim_{b \to \infty} F(b) = 1. \]
  Also ist $W := (\real, \borel(\real), \obar{\mu}_F)$ ein $\pW$-Raum. Sei $X :=
  \id$, dann ist $X$ eine Zufallsgröße auf $W$ und es gilt die zugehörige
  Verteilung:
  \[ \mu_X(B) = \obar{\mu}_F(X^{-1}(B)) = \obar{\mu}_F(B). \]
  Damit ist $\obar{\mu}_F$ die gesuchte Verteilungsfunktion.
\end{proof}

\begin{aufg}
  Beweisen Sie Lemma 1.3.10.
\end{aufg}
\begin{align*}
  \pP(a \le X < b )
  &= \pP((X<b)\setminus(X<a)) \tag{i} \\
  &= \pP(X<b) - \pP(X<a) \\
  &= F(b) - F(a). \\
  \pP(X = a)
  &= \lim_{n \to \infty} \pP \left( \left( X < a + \rez{n} \right) \setminus (X<a) \right) \tag{ii} \\
  &= \lim_{n \to \infty} \pP \left( \left( X < a + \rez{n} \right) \right) - \pP(x<a) \\
  &= F(a+0) - F(a). \\
  \pP(a < X < b)
  &= \lim_{n \to \infty} \pP \left( (X<b) \setminus \left( X < a + \rez{n} \right) \right) \tag{iii} \\
  &= \pP(X < b) - \lim_{n \to \infty} \pP \left( X < a + \rez{n} \right) \\
  &= F(b) - F(a+0) \\
  \pP(a \le X \le b)
  &= \lim_{n \to \infty} \pP \left( \left(X<b + \rez{n} \right) \setminus ( X < a ) \right) \tag{iv} \\
  &= \lim_{n \to \infty} \pP \left( X < b + \rez{n} \right) - \pP(X<a) \\
  &= F(b+0) - F(a) \\
  \pP(a < X \le b)
  &= \lim_{n \to \infty} \pP \left( \left(X<b + \rez{n} \right) \setminus \left(X < a + \rez{n} \right) \right) \tag{v} \\
  &= \lim_{n \to \infty} \pP \left( X < b + \rez{n} \right) -
    \lim_{n \to \infty} \pP \left( X < a + \rez{n} \right) \\
  &= F(b+0) - F(a+0)
\end{align*}
\[ F(x) = \pP(X < x) = \mu((-\infty,x)) = \int_{-\infty}^x p(t) \diffop
  t. \tag{vi} \]

\begin{aufg}
  Man berechne die Dichtefunktion der Zufallsgröße $X^2$, wobei $X \in N(0, 1)$.

  [Hinweis: Aufgabe 1.6.3.]
\end{aufg}

\begin{aufg}
  Beweisen Sie Satz 1.3.11 für den Fall, dass $p$ stetig ist.

  [Hinweis: Mittelwertsatz für das Riemann-Integral.]
\end{aufg}


%%% Local Variables:
%%% TeX-master: "master"
%%% End:


\clearpage

\section{Unabhängige Zufallsvariablen}
\begin{defn}
  Zwei Zufallsvariablen $X_1 : \Omega \to S_1$, $X_2: \Omega \to S_2$ heißen
  \emph{unabhängig}, wenn die Ereignisse $[X_1 \in B_1]$ und $[X_2 \in B_2]$ für
  beliebige $B_i \in \borel_i$ ($i=1,2$) unabhängig sind, das heißt
  \[ \pP[ X_1 \in B_1, X_2 \in B_2] = \pP[ X_1 \in B_1] \cdot \pP[X_2 \in
    B_2]. \]
  Allgemeiner: Eine Familie $\{X_i\}$ von Zufallsvariablen $X_i : \Omega \to
  S_i$ heißt unabhängig, wenn für jede nichtleere endliche Teilmenge $\{i_1,
  \ldots, i_n\}$ von $I$ und beliebige $B_{i_1} \in \borel_{i_1}, \ldots,
  B_{i_n} \in \borel_{i_n}$ die Ereignisse $[X_{ij} \in B_{ij}]$, $j  = 1,
  \ldots, n$ unabhängig sind.
  \[ \pP[ X_{i_1} \in B_{i_1}, \ldots, X_{i_n} \in B_{i_n}] = \pP[ X_{i_1} \in B_{i_1}]
    \cdots \pP[X_{i_n} \in B_{i_n}]. \]
\end{defn}

\begin{thm}
  Es seien $X_1, \ldots, X_m$ Zufallsgrößen. Die folgenden Aussagen sind
  äquivalent:
  \begin{enumerate}[(i)]
  \item $X_1, \ldots, X_m$ sind unabhängig.
  \item Ihre gemeinsame Verteilung $\mu_Y = \mu_{(X_1, \ldots, X_n)}$ ist das
    Produkt ihrer einzelnen Verteilungen $\mu_{X_i}$:
    \[ \mu_Y = \mu_{X_1} \times \cdots \times \mu_{X_n}. \]
  \item $\pP[X_1 < t_1, \ldots, X_n < t_n] = \pP[X < t_1] \cdots \pP[X_n < t_n]$
    für beliebige $t_i \in \real$, $i = 1, \ldots, n$.
  \end{enumerate}
  Die Äquivalenz von (i) und (ii) gilt auch für Zufallsvariablen.
\end{thm}

\begin{proof}
  Aus MINT ist bekannt, dass die Beziehungen
  \begin{align*}
    \mu_Y
    &= \mu_{X_1} \times \cdots \times \mu_{X_n},
      \tag{1} \\
    \mu_Y( B_1 \times \cdots \times B_n)
    &= \mu_{X_1}(B_1) \cdots \mu_{X_n}(B_n),
      \tag{2} \\
    \intertext{für beliebige $B_j \in \borel(\real)$,}
    \mu_Y( B_1 \times \cdots \times B_n)
    &= \mu_{X_1}(B_1) \cdots \mu_{X_n}(B_n),
      \tag{3}
  \end{align*}
  für beliebige $B_j$ der Form $B_j = (-\infty, t_j)$ äquivalent sind.
  
  Da $\pP[ (X_1, \ldots, X_n) \in B_1 \times \cdots \times B_n] = \pP[X_1 \in
  B_1, \ldots, X_n \in B_n]$ und
  \[ \mu_{X_1}(B_1) \cdots \mu_{X_n}(B_n) = \pP[X_1 \in B_1] \cdots \pP[X_n \in
    B_n], \]
  ist es leicht zu sehen, dass (2) $\sim$ (i), (1) $\sim$ (ii) und (3) $\sim$
  (iii).
\end{proof}

\begin{thm}
  Es seien $X_1, \ldots, X_n$ unabhängige Zufallsgrößen und $g: \real^k \to
  \real^l$, $1 \le k \le n$ eine Borel-messbare Funktion. Dann sind die
  Zufallsvektoren $g(X_1, \ldots, X_k)$, $X_{k+1}, \ldots, X_n$ unabhängig.
\end{thm}

\begin{proof}
  Es seien $B \subset \real^k$, $B_{k+1} \subset \real, \ldots, B_n \subset
  \real$ beliebige Borel-Mengen.
  \begin{align*}
    &\pP [ g(X_1, \ldots, X_k) \in B, X_{k+1} \in B_{k+1}, \ldots, X_n \in B_n ] \\
    =\, &\pP [ (X_1, \ldots, X_k) \in g^{-1}(B), X_{k+1} \in B_{k+1}, \ldots, X_n \in B_n ] \\
    =\, &\pP [ (X_1, \ldots, X_n) \in g^{-1}(B) \times B_{k+1} \times \cdots \times B_n ] \\
    =\, &\mu_{(X_1, \ldots, X_n)} [ g^{-1}(B) \times B_{k+1} \times \cdots \times B_n ] \\
    \overset{(1.5.2)}{=}\, &\mu_{X_1} \times \cdots \times \mu_{X_n} [ g^{-1}(B) \times B_{k+1} \times \cdots \times B_n ] \\
    =\, &\mu_{X_1} \times \cdots \times \mu_{X_k} (g^{-1}(B)) \cdot \mu_{X_{k+1}}(B_{k+1}) \cdot \mu_{X_n}(B_n) \\
    \overset{(1.5.2)}{=}\, &\mu_{(X_1, \ldots, X_n)} (g^{-1}(B)) \cdot \mu_{X_{k+1}}(B_{k+1}) \cdot \mu_{X_n}(B_n) \\
    =\, &\pP [ (X_1, \ldots, X_k) \in g^{-1}(B)] \cdot \pP[X_{k+1} \in B_{k+1}] \cdots \pP[X_n \in B_n] \\
    =\, &\pP [ g(X_1, \ldots, X_k) \in B] \cdot \pP[X_{k+1} \in B_{k+1}] \cdots \pP[X_n \in B_n]. \qedhere
  \end{align*}
\end{proof}

\begin{folg}
  Sind $\{ X, Y, Z \}$ unabhängig, dann sind auch $\{ X+Y, Z\}$, $\{X \cdot Y, Z
  \}$ unabhängig.
\end{folg}

\begin{rmrk}
  In der Analysis werden die folgenden Resultate bewiesen: Es seien $\mu$ und
  $\nu$ $\pW$-Maße (oder beliebige endliche Maße) auf $\real^n$. Durch
  \[ \mu \ast \nu(B) = \int_{\real^n} \mu( B-y ) \diffop \nu(y), \quad B \in
    \borel(\real^n,) \]
  wobei $B-y := \{ b-y : b \in B \}$, wird ein $\pW$-Maß (endliches Maß) $\mu
  \ast \nu$ definiert. Es wird als \emph{Faltung} von $\mu$ und $\nu$
  bezeichnet.

  \textbf{Beispiel.} Seien $\mu = \delta_x$, $\nu = \delta_y$, $x,y \in \real^n$, $\delta_x \ast
  \delta_y = \delta_{x+y}$.

  Es gilt
  \begin{enumerate}[(i)]
  \item $\mu \ast \nu = \nu \ast \mu$.
  \item $\mu \ast (\nu_1 + \nu_2) = (\mu \ast \nu_1) + (\mu \ast \nu_2)$.
  \item Wenn $\mu$ eine Dichte $p$ besitzt, dann hat auch $\mu \ast \nu$ eine
    Dichte $h$ und
    \[ h(x) = \int_{\real^n} p(x-y) \diffop \nu(y), \quad x \in \real^n. \]
  \item Wenn auch $\nu$ eine Dichte $q$ besitzt, dann gilt:
    \[ h(x) = \int_{\real^n} p(x-y) \cdot q(y) \diffop y. \]
    Dies wird als \emph{Faltung von $p$ und $q$} bezeichnet. Ist $p$ oder $q$
    stetig, so ist auch $h$ stetig.
  \item Es gilt
    \[ \int_{\real^n} f(t) \diffop (\mu \ast \nu)(t) = \int_{\real^n}
      \int_{\real^n} f(t+s) \diffop \mu(t) \diffop \nu(s), \quad f \in
      \intf^1(\mu \ast \nu). \]
  \end{enumerate}
\end{rmrk}

\clearpage

\begin{thm}
  Sind $X$ und $Y$ unabhängige $d$-dimensionale Zufallsgrößen, so gilt
  \[ \mu_{X + Y} = \mu_X \ast \mu_Y. \]
\end{thm}

\begin{proof}
  Seien $Z := (X,Y)$, $\varphi(X,Y) := x+y$, $x,y \in \real^d$. Für jede
  Borel-Menge $B \subset \real^d$ gilt:
  \begin{align*}
    \mu_{X+Y}(B)
    &= \pP[X+Y \in B] = \pP[\varphi(Z) \in B]  \\
    &= \pP[ Z \in \varphi^{-1}(B) ] = \mu_Z( \varphi^{-1}(B) ) \\
    &= \int_{\real^d \times \real^d} \ind_{\varphi^{-1}(B)}(x,y) \diffop \mu_Z(x,y) \\
    &= \int_{\real^d \times \real^d} \ind_{B}(x+y) \diffop \mu_Z(x,y) \\
    &= \int_{\real^d} \int_{\real^d} \ind_{B}(x+y) \diffop \mu_X(x) \diffop \mu_Y(y) \\
    &= \int_{\real^d} \int_{\real^d} \ind_{B - y}(x) \diffop \mu_X(x) \diffop \mu_Y(y) \\
    &= \int_{\real^d} \mu_X(B - y) \diffop \mu_Y(y) \\
    &= (\mu_X \ast \mu_Y)(B). \qedhere
  \end{align*}
\end{proof}

Alternative Definition der Faltung: Bildmaß des Produktmaßes $\mu \times \nu$
bezüglich $\varphi$.

\begin{rmrk}
  Die Umkehrung gilt nicht.
\end{rmrk}

\begin{exmp}
  Es seien $X$ und $Y$ unabhängige und gammaverteilte Zufallsgrößen mit den
  Dichtefunktionen
  \begin{align*}
    p_X(x) &= \frac{\lambda^a \cdot x^{a-1}}{\Gamma(a)} \cdot e^{-\lambda x} &
    p_Y(x) &= \frac{\lambda^b \cdot x^{b-1}}{\Gamma(b)} \cdot e^{-\lambda x},
  \end{align*}
  wobei $\lambda, a, b, x > 0$.

  Gesucht ist die Dichte $p_{X+Y}$.
  \begin{align*}
    p_{X+Y}(x)
    &= \int_\real p_X(x-y) p_Y(y) \diffop y \\
    &= \frac{\lambda^{a+b}}{\Gamma(a) \cdot \Gamma(b)} \cdot e^{-\lambda x} \cdot \int_0^x (x-y)^{a-1} \cdot y^{b-1} \diffop y \\
    \intertext{Substitution: $y = t \cdot x$, $\diffop y = x \cdot \diffop t$.}
    &= \frac{\lambda^{a+b}}{\Gamma(a) \cdot \Gamma(b)} \cdot e^{-\lambda x} \cdot \int_0^x (1-t)^{a-1} \cdot t^{b-1} \diffop t.
  \end{align*}
  Also ist $p_{X+Y} = c \cdot p_{a+b,\lambda}$, wobei $c$ eine Konstante ist.
  Für $c=1$ ist $X+Y$ die Gammaverteilung mit den Parametern $a+b$ und
  $\lambda$.

  \textbf{Folgerung.} Wir definieren die \emph{Beta-Funktion} $B$
  \[ B(a,b) := \int_0^1 (1-t)^{a-1} \cdot t^{b-1} \diffop t = \frac{\Gamma(a)
      \cdot \Gamma(b)}{\Gamma(a+b)}, \quad a,b > 0. \]
\end{exmp}

\begin{prgp}[Die $\chi^2$-Verteilung]
  Seien $X_1, \ldots, X_n$ unabhängige, $N(0,1)$-verteilte Zufallsgrößen. Die
  Verteilung der Zufallsgröße $Y_n := X_1^2 + \ldots + X_n^2$ heißt
  \emph{$\chi_n^2$-Verteilung mit dem Freiheitsgrad $n$}. Die Verteilung von
  $\sqrt{Y_n}$ heißt \emph{$\chi_n$-Verteilung mit dem Freiheitsgrad $n$}.

  \textbf{Behauptung.} Für die Dichte $h_n$ von $Y_n$ gilt\footnote{%
    Es ergibt sich eine Gammaverteilung mit $\lambda = \rez{2}$, $a = \frac{n}{2}$.
  }:
  \[ h_n(x) = \frac{x^{n/2 - 1} \cdot e^{-x/2}}{\lambda^{n/2} \cdot \Gamma( n/2
      )}, \quad x > 0. \]

  \begin{proof}
    Induktion bezüglich $n$. Sei $n = 1$, zu zeigen:
    \[ h_1(x) = \frac{x^{-1/2}}{\sqrt{2 \pi}} \cdot e^{-x/2}, \quad x > 0. \]
    Dies ist einfach zu erkennen (siehe Aufgabe 1.4.4).

    Induktionsschritt:
    \begin{align*}
      h_{n+1}
      &= \int_\real h_n(x-y) h_1(y) \diffop y \\
      &= \rez{2^{n/2} \cdot \Gamma( n/2 ) \cdot \sqrt{2 \pi}} \cdot
        \int_0^x (x-y)^{n/2-1} \cdot e^{-(x-y)/2} \cdot e^{-x/2} \cdot y^{-1/2} \diffop y \\
      \intertext{Substitution: $y = t \cdot x$, $\diffop y = x \diffop t$}
      &= \frac{x^{(n+1)/2-1}}{\sqrt{2\pi} \cdot 2^{n/2} \cdot \Gamma(n/2)} \cdot e^{-x/2} \cdot
        \underbrace{ \int_0^1 (1-t)^{-1/2} \cdot t^{n/2-1} \diffop t}_{=B(1/2, n/2)} \\
      \intertext{Es gilt $B(1/2, n/2) = \frac{\Gamma(1/2) \cdot
      \Gamma(n/2)}{\Gamma((n+1)/2)}$, wobei $\Gamma(1/2) = \sqrt{2}$}
      &= \frac{x^{(n+1)/2-1}}{2^{(n+1)/2} \cdot \Gamma((n+1)/2)} \cdot e^{-x/2}, \quad x > 0. \qedhere
    \end{align*}
  \end{proof}
\end{prgp}

\begin{rmrk*}
  \begin{enumerate}[a)]
  \item $h_2(x) = \rez{2} \cdot e^{-x/2}$ ist die Exponentialverteilung.
  \item Die Dichte $g_n$ von $\sqrt{Y_n}$:
    \[ g_n(x) = \frac{x^{n-1}}{2^{n/2-1} \cdot \Gamma(n/2)} \cdot e^{-x^2/2},
      \quad x > 0. \]
    Beweisidee: Die Verteilungsfunktion
    \[ F(x) = \pP( \sqrt{Y_n} < x )) = \pP(Y_n < x^2) = G(x^2) \]
    ableiten nach $x$, ... ($G$ ist die Verteilungsfunktion $Y_n$).
  \end{enumerate}
\end{rmrk*}

Seien $X,Y$ unabhängig mit den Dichten $p$ bzw. $q$, dann besitzt $X+Y$ die
Dichte
\[ (p \ast q)(x) = \int_{\real^n} p(x-y) q(y) \diffop y. \]
Das folgt aus 1.5.6 und 1.5.5.iv.

\begin{rmrk*}
  Wenn $X$ \emph{oder} $Y$ eine Dichte besitzt, dann auch $X+Y$ (siehe
  1.5.5.iii).
\end{rmrk*}

\begin{prgp}
  Seien $X_1, \ldots, X_n$ Zufallsgrößen, $\mu$ ihre gemeinsame Verteilung (auf
  $\real^n$) und $h: \real^n \to \real$ Borel-messbar, $X := (X_1, \ldots,
  X_n)$.

  Die Verteilungsfunktion der Zufallsgröße $Z := h(X_1, \ldots, X_n)$ ist
  \[ \pP[ Z < t ] = \mu( \{ x : h(x) < t \} ) = \int_{h(x)<t} 1 \diffop \mu(x).
    \tag{$\ast$} \]

  \textbf{Spezialfall.} $Z = X \cdot Y$; $X$ und $Y$ seien unabhängig mit Dichte
  $f$ bzw. $g$.

  \textbf{Aufgabe.} $X \cdot Y$ besitzt eine Dichte $p$ mit
  \[ p(x) = \int_{-\infty}^\infty \rez{|y|} \cdot g(y) \cdot f(x/y) \diffop y. \]
  Hinweise: Setze ($\ast$) fort.

  Analog: Dichte von $X/Y$, falls $Y \ne 0$ fast sicher:
  \[ q(x) = \int_{-\infty}^\infty |y| \cdot g(y) \cdot f(x \cdot y) \diffop
    y. \]
\end{prgp}

\begin{thm}
  Für eine beliebige Familie $\{ \mu_i \}_{i \in I}$ von Verteilungen auf
  $\real$ existiert ein $\pW$-Raum $(\Omega, \mA, \pP)$ und darauf definierte
  Zufallsgrößen $X_i$, die unabhängig sind und die gegebenen Verteilungen
  $\mu_i$ besitzen.
\end{thm}

\begin{proof}
  Für endliches $I = \{1, \ldots, n\}$. Sei $\Omega := \real^n$, $\mA :=
  \borel(\real^n)$, $\pP := \mu_1 \times \cdots \times \mu_n$, $X_i(X_1, \ldots,
  X_n) := X_i$. Dann ist $\mu_i$ die Verteilung von $X_i$ und $X_1, \ldots, X_n$
  sind unabhängig.

  Der allgemeine Fall wird genauso bewiesen, wir haben jedoch beliebige Produkte
  von Maßen nicht behandelt.
\end{proof}

\begin{exmp}
  Es seien $X_1, X_2, \ldots$ unabhängige Zufallsgrößen mit $\pP[X=0] = \pP[X=1]
  = \rez{2}$ und
  \[ S_r = \sum_{n=1}^\infty r^n \cdot X_n, \quad 0 < r < 1 \]
  Der Wertebereich ist damit $\left[ 0, \frac{r}{1-r} \right]$.

  \begin{enumerate}[(i)]
  \item $S_{1/2}$ ist auf dem Intervall $[0,1]$ gleichmäßig verteilt.
  \item Für $r = 2^{-1/k}$, $k = 1, \ldots$ ist $S_r$ absolut stetig.
  \item Für $0 < r < \rez{2}$ ist $S_r$ \emph{stetig} und \emph{singulär}.
  \end{enumerate}
  Lösung für (iii): Wir zeigen, dass der Wertebereich $W(S_r)$ von $S_r$ das
  Lebesgue-Maß 0 hat. $\lambda^*$ ist das äußere Lebesgue-Maß.
  \[ W(S_r) = W(0 + r \cdot S_r) \cup W(r + r \cdot S_r), \]
  also
  \[ S_r = \underbrace{r \cdot X_1}_{0 \text{ oder } r} +
    \underbrace{r \cdot \sum_{n=1}^\infty r^n \cdot X_{n+1}}_{\text{verteilt wie }
      S_r} \]
  Dieser Bereich hat das Maß
  \begin{align*}
    \lambda^*(W(S_r))
    &\le \lambda^*(W(0 + r \cdot S_r)) + \lambda^*(W(r + r \cdot S_r)) \\
    &= r \cdot \lambda^*(W(S_r)) + r \cdot \lambda^*(W(S_r)) \\
    &= \underbrace{2 r}_{<1} \cdot \lambda^*(W(S_r)).
  \end{align*}
  Also ist $\lambda^*(W(S_r)) = 0$.
\end{exmp}


\section{Aufgaben}
Siehe \verb+Aufgaben-1-6-Teil-1.pdf+ und \verb+Aufgaben-1-6-Teil-2.pdf+.

\section{Numerische Charakteristika von Zufallsgrößen}
\begin{defn}
  Es sei $X$ eine $\pP$-integrierbare oder nichtnegative Zufallsgröße. Dann
  heißt
  \[ \pE(X) := \int_\Omega X \diffop \pP \]
  der \emph{Erwartungswert} von $X$. Das gilt analog für integrierbare, komplexe
  Zufallsgrößen. Es gilt:
  \begin{enumerate}[(i)]
  \item $\pE(aX + bY) = a \cdot \pE(X) + b \cdot \pE(Y)$ für alle $a,b \in \real$,
  \item Ist $X = c$ fast sicher, so gilt $\pE(X) = c$,
  \item Aus $a \le X \le b$ fast sicher folgt $a \le \pE(X) \le b$.
  \item $| \pE(X) |  \le \pE(|X|)$,
  \item Aus $X \ge 0$ fast sicher und $\pE(X) = 0$ folgt $X = 0$ fast sicher.
  \end{enumerate}
\end{defn}

\begin{itemize}
\item Mit $g(x) := x$ in (1.3.7):
  \[ \pE( X ) = \int_\real x \diffop \mu_x (x) \tag{1} \]
\item Ist $X$ diskret mit den Werten $x_1, \ldots$ und den zugehörigen
  Wahrscheinlichkeiten $p_1, \ldots$, so gilt
  \[ \pE( X ) = \sum_{j} x_j p_j. \tag{2} \]
\item Wenn $X$ eine Dichte $p$ besitzt, dann ist
  \[ \pE( X ) = \int_\real x \cdot p(x) \diffop x. \tag{3} \]
\end{itemize}

\begin{thm}
  Es seien $X$ und $Y$ unabhängige, integrierbare Zufallsgrößen. Dann ist $X
  \cdot Y$ integrierbar und es gilt
  \[ \pE(X \cdot Y) = \pE( X ) \cdot \pE( Y ). \]
\end{thm}

\begin{proof}
  Bezeichne $\mu_X$ bzw. $\mu_Y$ die Verteilungsfunktion von $X$ bzw. $Y$ und
  sei $\nu$ die gemeinsame Verteilung von $X$ und $Y$ (auf $\real^2$). Dann
  gilt mit dem Satz von Fubini (f):
  \begin{align*}
    \pE(X) \cdot \pE(Y)
    & = \int_\real x \diffop \mu_X(x) \cdot \int_\real y \diffop \mu_Y(y) \\
    & \overset{(\text{f})}{=}
      \int_{\real^2} x \cdot y \diffop (\mu_X \times \mu_Y)(x, y) \\
    & \overset{(1.5.2)}{=}
      \int_{\real^2} x \cdot y \diffop \nu(x,y) \\
    & \overset{(1.3.7)}{=} \pE( X \cdot Y ). \qedhere
  \end{align*}
\end{proof}

\begin{rmrk}
  \begin{enumerate}[a)]
  \item $\pE( X \cdot Y) = \pE(X) \cdot \pE(y)$ ist möglich, auch wenn $X$ und
    $Y$ abhängig sind. Es sei zum Beispiel $X \in L^3(P)$ symmetrisch zum
    Punkt 0 verteilt\footnote{%
      Das heißt $-X$ hat die selbe Verteilung wie $+X$, äquivalent $\mu_X(B) =
      \mu_X(-B)$ für alle $B \in \borel(\real)$ oder $F_X(t) = 1 -F_X(-t).$}.
    Wir setzen $Y := X^2$. Dann ist auch $X^3$ symmetrisch. Es gilt $\pE(X) =
    0$, $\pE(X \cdot Y ) = \pE(X^3) = 0 = \pE(X) \cdot \pE(Y)$.

    Zwei Zufallsgrößen heißen \emph{unkorreliert}, wenn
    \[ \pE( X \cdot Y ) = \pE(X) \cdot \pE(Y) \]
    gilt.
  \item Satz 1.7.2 gilt auch für komplexe Zufallsgrößen (Übungsaufgabe).
  \end{enumerate}
\end{rmrk}

\begin{exmp}
  \begin{enumerate}[(a)]
  \item Es sei $X$ eine binomialverteilte Zufallsgröße mit den Parametern $n$
    und $p$.
    \begin{align*}
      \pE(X)
      &= \sum_{k=0}^n k \cdot \binom{n}{k} \cdot p^k (1-p)^{n-k} \\
      &= \sum_{k=0}^n k \cdot \frac{n!}{k! (n-k)!} \cdot p^k (1-p)^{n-k} = \, ?
    \end{align*}
    Einfacher: Es seien $X_1, \ldots, X_n$ unabhängige Zufallsgrößen\footnote{%
      Binomialverteilt mit $n=1$.}
    mit $\pP(X_j = 1) = p$, $\pP(X_j = 0) = 1-p$. Dann ist $\tilde{X} := X_1 +
    \cdot + X_n$ binomialverteilt mit den Parametern $n$ und $p$.
    \begin{align*}
      \pE(X)
      &= \pE( \tilde{X} ) = \pE( X_1 + \cdots + X_n ) = \pE(X_1) + \cdots + \pE(X_n) \\
      &= n \cdot \pE(X_1) = n \cdot (0 \cdot (1-p) + 1 \cdot p ) = n \cdot p.
    \end{align*}
  \item Sei $X$ gleichmäßig verteilt im Intervall $[a,b]$.
    \[ \pE(X) \int_{-\infty}^\infty x \cdot p_X(x) \diffop x = \int_a^b x \cdot
      \rez{b-a} \diffop x = \left. \frac{x^2}{2(b-a)} \right|_{x=a}^b =
      \frac{b^2 - a^2}{2(b-a)} = \frac{a+b}{2}. \]
  \item Sei $X$ poissonverteilt mit
    \[ \pP( X = k ) = \frac{a^k \cdot e^{-a}}{k!}, \quad a > 0, \quad k = 0, 1,
      2, \ldots \]
    Dann ist
    \begin{align*}
      \pE( X ) & = \sum_{k=0}^\infty k \cdot \frac{a^k e^{-a}}{k!}
                 = \sum_{k=1}^\infty \frac{a^k \cdot e^{-a}}{(k-1)!} \\
               & = e^{-a} \cdot a  \cdot \sum_{k=1}^\infty \frac{a^{k-1}}{(k-1)!}
                 = e^{-a} \cdot a \cdot e^a = a.
    \end{align*}
  \item Sei $X$ normalverteilt mit der Dichte
    \[ g_{a,\sigma}(x) = \rez{\sigma \sqrt{2 \pi}} \cdot e^{-(x-a)^2 / 2
        \sigma^2}. \]
    Dann ist
    \begin{align*}
      \pE(X)
      & = \int_{-\infty}^\infty x \cdot \rez{\sigma \sqrt{2 \pi}}
        \cdot e^{-(x-a)^2 / 2 \sigma^2} \diffop x \\
      \intertext{Substitution $z = \frac{x-a}{\sigma}$, $\diffop x = \sigma \diffop z$}
      & = \rez{\sqrt{2 \pi}} \cdot \int_{-\infty}^\infty (\sigma z + a)
        \cdot e^{-z^2 / 2} \diffop z \\
      & = \rez{\sqrt{2 \pi}} \cdot
        \left[ \underbrace{%
        \int_{-\infty}^\infty \sigma z \cdot e^{-z^2 / 2} \diffop z
        }_{\text{ungerade}}
        + \underbrace{%
        a \cdot \int_{-\infty}^\infty e^{-z^2 / 2} \diffop z
        }_{= \sqrt{2 \pi} } \right]
        = a.
    \end{align*}
  \end{enumerate}
\end{exmp}

\begin{defn}
  Als \emph{Streuung} (\emph{Varianz}, Dispersion) einer Zufallsgröße $X \in L^2
  ( \pP )$  bezeichnet man die Größe
  \[ \sigma^2 := \var(X) := D^2(X) := \pE( (X-\pE(X))^2 )
    = \int_\Omega (X -\pE(X))^2 \diffop \pP. \tag{1} \]
  Die positive Quadratwurzel $\sigma$ heißt \emph{Standardabweichung}.
\end{defn}

\begin{rmrk*}
  Aus $X \in L^2( \pP )$ folgt $X \in L^1( \pP )$. Warum?
  \[ |X(w) \le
    \begin{cases}
      |X(w)|^2, &\text{wenn } |X(w)| \ge 1, \\
      1, &\text{sonst.}
    \end{cases} \]
  Wählt man $g(x) := (x - \pE(X))^2$ in Satz 1.3.7, so folgt
  \[ D^2(X) = \int_{-\infty}^\infty (x - \pE(X))^2 \diffop \mu_X (x). \tag{2} \]
\end{rmrk*}

\begin{thm}
  Für $X,Y \in L^2(\pP)$ gilt:
  \begin{enumerate}[(i)]
  \item $D^2(X) = \pE( X^2 ) - \pE( X )^2$,
  \item $D^2(aX + b) = a^2 \cdot D^2(X)$,
  \item $D^2(X) = 0$ $\Leftrightarrow$ $X$ ist konstant fast sicher,
  \item $D^2(X + Y) = D^2(X) + D^2(Y) + 2 \pE \big[ (X - \pE(X)) \cdot (Y -
    \pE(Y)) \big]$,
  \item Sind $X$ und $Y$ unkorreliert, so gilt $D^2(X+Y) = D^2(X) + D^2(Y)$.
  \end{enumerate}
\end{thm}

\begin{proof}
  Einfach.
\end{proof}

\begin{exmp}
  \begin{enumerate}[(a)]
  \item Sei $X$ binomialverteilt mit $n$ und $p$. Dann ist
    \[ D^2(X) = \sum_{k=0}^n (k-n \cdot p)^2 \cdot \binom{n}{k} p^k
      (1-p)^{n-k}. \]
    Seien $X_1, \ldots, X_n$ unabhängig mit $\pP(X=1) = p$, $\pP(X=0) = 1-p$,
    dann ist $\tilde{X} := X_1 + \cdots + X_n$ binomialverteilt mit $n$, $p$ und
    es gilt
    \begin{align*}
      D^2(X)
      & \overset{(\text{v})}{=} D^2(\tilde{X}) = D^2(X_1) + \cdots + D^2(X_n) \\
      & \overset{(\text{i})}{=} n \cdot D^2(X_1) = n \cdot [ \pE(X_1^2) - \pE(X_1)^2 ]
        = n \cdot (p - p^2) = n \cdot p \cdot (1-p).
    \end{align*}
  \item Sei $X$ gleichmäßig verteilt in $[a,b]$. Dann ist
    \[ D^2(X) = \pE(X^2) - \pE(X)^2 = \pE(X^2) - \left(  \frac{a+b}{2}
      \right)^2. \]
    Wir berechnen
    \[ \begin{aligned}
        \pE(X^2)
        & = \int_{-\infty}^\infty x^2 \diffop \mu_X(x) = \int_a^b x^2
        \rez{b-a} \diffop x = \left. \frac{x^3}{3(b-a)} \right|_a^b \\
        & = \frac{b^3 - a^3}{3(b-a)} = \frac{(b-a)(b^2 + ab + a^2)}{3(b-a)} =
        \frac{b^2 + ab + a^2}{3}.
      \end{aligned} \]
    Damit folgt
    \[ D^2(X) = \frac{b^2 + ab + a^2}{3} - \left( \frac{a+b}{2} \right)^2
      = \frac{(b-a)^2}{12}. \]
  \item Sei $X \sim N(a,\sigma)$. Bekannt: $\pE(X) = a$.
    \begin{align*}
      D^2(X)
      &= \rez{\sigma \sqrt{2 \pi}} \int_{-\infty}^\infty (x-a)^2 \cdot e^{-(x-a)^2 / 2
        \sigma^2} \diffop x \\
      \intertext{Substitution: $z = \frac{x-a}{\sigma}$, $\diffop x = \sigma \diffop z$}
      &= \frac{\sigma^2}{\sqrt{2 \pi}} \int_{-\infty}^\infty z^2 \cdot e^{-z^2 / 2} \diffop z \\
      \intertext{Partielle Integration: $\int fg' = fg - \int f'g$, $f = z$, $f'=1$, $g' = z \cdot e^{-z^2 / 2}$, $g = - e^{-z^2 / 2}$}
      &= \frac{\sigma^2}{\sqrt{2 \pi}}
        \underbrace{\left[ -z \cdot e^{-z^2 / 2} \right]_{-\infty}^\infty}_{=0} +
        \underbrace{\int_{-\infty}^\infty e^{-z^2 / 2} \diffop z}_{=\sqrt{2 \pi}} = \sigma^2.
    \end{align*}
  \end{enumerate}
\end{exmp}

\begin{defn}
  Als \emph{Kovarianz} der Zufallsgrößen $X,Y \in L^2(\pP)$ bezeichnet man die
  Größe
  \[ \cov(X,Y) = \pE[ (X- \pE(X)) \cdot (Y - \pE(Y)) ] = \pE(XY) - \pE(X) \cdot
    \pE(Y). \tag{1} \]
  Aus (1) folgt: $X$, $Y$ sind unkorreliert $\Leftrightarrow$ $\cov(X,Y) = 0$.

  \emph{Korrelationskoeffizient}:
  \[ \corr( X, Y ) := \frac{\cov(X,Y)}{D(X) \cdot D(Y)}, \]
  falls $D(X) \cdot D(Y) \ne 0$.

  Es gilt: $-1 \le \corr(X,Y) \le 1$.
  \begin{proof}
    Ungleichung von Cauchy für Integrale:
    \[ \left| \int f g \right|^2 \le \int |f|^2 \cdot \int |g|^2, \]
    aus der linearen Algebra bekannt:
    \[ |\angles{x,y}|^2 \le \angles{x,x} \cdot \angles{y,y}. \qedhere \]
  \end{proof}
\end{defn}

\begin{rmrk*}
  \begin{enumerate}[(a)]
  \item $\corr (X,X) = 1$, $\corr(X,-X) = -1$.
  \item $D^2(X_1 + \cdots X_n) = \sum_1^n D^2(X_i) + 2 \cdot \sum_{i<j}
      \cov(X_i, X_j)$, das folgt aus (1.7.6.iv).
  \end{enumerate}
\end{rmrk*}

\begin{thm}
  Sei $D(X) \cdot D(Y) = 0$. Dann gilt
  \[ |\corr(X,Y)| = 1 \qLRq \exists a,b \in \real, a \ne 0 : Y = a \cdot X + b
    \text{ fast sicher.} \]
\end{thm}

\begin{proof}
  Wir zeigen nur ``$\Rightarrow$'', die andere Richtung ist einfach.

  O.B.d.A. sei $\corr(X,Y) = 1.$ Definiere
  \begin{align*}
    X' &:= \frac{X - \pE(X)}{D(X)}, & Y' &:= \frac{Y-\pE(Y)}{D(Y)}.
  \end{align*}
  Dann gilt
  \begin{align*}
    \pE(X') &= \pE(Y') = 0, & \pE(X'^2) &= \pE(Y'^2) = 1.
  \end{align*}
  und
  \[ 1 = \corr(X,Y) = \pE( X' \cdot Y' ) \qRq \pE( (X' - Y')^2 ) = 
    \pE(X'^2) - 2 \pE(X' \cdot Y') + \pE(Y'^2) = 0. \]
  $(X'-Y')^2$ ist eine nichtnegative Zufallsgröße und ihr Erwartungswert ist 0.
  Also ist $X' = Y'$ fast sicher. Damit folgt
  \[ Y = \pE(Y) + D(Y) \cdot \frac{X - \pE(X)}{D(X)}. \qedhere \]
\end{proof}

\begin{defn}
  Als \emph{Moment $k$-ter Ordnung} einer Zufallsgröße $X \in L^k(\pP)$
  bezeichnet man
  \[ M_k(X) := \pE(X^k) = \int_{-\infty}^\infty x^k \diffop \mu_X(x), \quad k =
    0, 1, \ldots \]
  \emph{Absolutes Moment:}
  \[ A_k(X) := \pE(|X^k|) = \int_{-\infty}^\infty |X|^k \diffop \mu_X(x). \]
\end{defn}

\begin{rmrk*}
  \begin{enumerate}[(a)]
  \item Aus der Existenz vom $M_k$ folgt die Existenz von $M_l$ für alle $l \le
    k$ wegen $|X|^l \le |X|^k + \ind_{[-1,1]}(x)$, $x \in \real$.
  \item Sei $n > k \ge 1$ und nehmen wir an, dass $A_n$ existiert. Dann folgt
    mit der Hölderschen Ungleichung (h)
    \[ A_k = \int_{-\infty}^\infty |X|^k \cdot 1 \diffop \mu(x)
      \overset{(\text{h})}{\le}
      \left( \int_{-\infty}^\infty |X|^{p \cdot k} \diffop \mu(x)
      \right)^{1/p}, \quad p > 1. \]
    Mit $p = \frac{n}{k}$ gilt
    \[ A_k^{1/k} \le A_n^{1/n}. \]
  \end{enumerate}
\end{rmrk*}

\begin{exmp}
  Momente und absolute Momente der Normalverteilung mit $a = 0$. Für ungerade
  $k$ ist $M_k = 0$, für gerade $k$ gilt $M_k = A_k$. Wir substituieren wieder
  $x := \frac{z-a}{\sigma}$ und nutzen die Geradheit des Integranden:
  \begin{align*}
    A_k
    & = \sqrt{\frac{2}{\pi}} \sigma^k \int_0^\infty x^k \cdot e^{-x^2 / 2}
      \diffop x \\
  \intertext{Weitere Substitution $x^2 = 2z$, $\diff{z}{x} = x$, also $\diffop x
    = \rez{x} \diffop z$}
    & = \sqrt{\frac{2}{\pi}} \cdot \sigma^k \cdot 2^{(k-1)/2} \cdot
      \underbrace{ \int_0^\infty z^{(k-1)/2} \cdot e^{-z} \diffop z}_{=\Gamma((k+1)/2)} \\
    & = \rez{\sqrt{\pi}} (\sqrt{2} \cdot \sigma)^k \cdot \Gamma \left( \frac{k+1}{2} \right).
  \end{align*}
\end{exmp}

\begin{rmrk*}
  Die Normalverteilung ist die \emph{einzige} Verteilung, die diese Momente
  besitzt. Eine Verteilung ist aber im Allgemeinen nicht eindeutig durch ihre
  Momente bestimmt.
\end{rmrk*}

\begin{thm}
  Sei $X$ eine Zufallsgröße mit Verteilungsfunktion $F$. Existiert $\pE(X)$, so
  gilt:
  \begin{align*}
    \lim_{x \to \infty} x \cdot ( 1 - F(x) ) &= 0, & \lim_{x \to -\infty} x \cdot F(x) &= 0,
  \end{align*}
  \[ \pE(X) = \int_0^\infty (1-F(y)) \diffop y - \int_{-\infty}^0 F(y) \diffop
    y. \]
\end{thm}

\begin{proof}
  Weil $\pE(X)$ existiert, gilt
  \[ \pE(|X|) = \int_{-\infty}^\infty |x| \diffop F(x) < \infty, \]
  also
  \[ 0 \le \lim_{x \to \infty} x \cdot
    \underbrace{(1-F(x))}_{\text{Maß von } [x, \infty]}
    \le \lim_{x \to \infty} \int_x^\infty
    \underbrace{\underbrace{y}_{\ge x} \diffop F(y)}_{\text{endl. Maß}}
    \overset{\text{(So)}}{=} 0,
  \]
  wobei (So) für die Stetigkeit von oben steht.

  Der zweite Limes wird analog bewiesen.

  Zur Erinnerung, partielle Integration:
  \[ \int_0^x f(y) \diffop g(y) = f(y) \cdot g(y) \Big|_0^x - \int_0^x g(y)
    \diffop f(y). \]
  Damit können wir zeigen:
  \begin{align*}
    \int_{-x}^0 y \diffop F(y) &= x \cdot F(-x) - \int_{-x}^0 F(y) \diffop y, \\
    \int_0^x y \diffop F(y) &= -x \cdot(1-F(x)) + \int_0^x (1-F(y)) \diffop y, \\
                               &= -x \cdot(1-F(x)) + x - \int_0^x F(y) \diffop y.
  \end{align*}
  Die Behauptung folgt nun durch Addition und Grenzwertbildung $x \to \infty$.
\end{proof}

\begin{prgp}[Zufallsvektoren]
  Sei $X = (X_1, \ldots, X_d)$ ein $d$-dimensionaler Zufallsvektor, reell oder
  komplex, sodass $X_j \in L^1(\pP)$ für alle $j$. Dann wird der
  \emph{Erwartungsvektor} von $X$ definiert durch
  \[ \pE X := ( \pE X_1, \ldots, \pE X_d ). \]
  Sei $X_j \in L^2(\pP)$ für alle $j$ und $m_j := \pE X_j$. Die Matrix $\cov(X)
  := (c_{jk})^d$, wobei
  \[ c_{jk} = \pE [ (X_j - m_j) \cdot \obar{(X_k - m_k)} = \pE[X_j \obar{X_k}] -
    m_j \cdot \obar{m_k} ], \]
  heißt die \emph{Kovarianz-Matrix} von $X$. Wenn die $X_j$ paarweise unabhängig
  sind, dann ist $\cov(X)$ eine Diagonalmatrix. Wegen
  \[ \sum_{j,k = 1}^d c_{jk} \cdot z_j \obar{z_k} \overset{\footnotemark}{=}
    \pE \left[ \left| \sum_{j=1}^d z_j \cdot (X_j - m_j) \right|^2 \right] \ge
    0,
    \quad
    z_j, z_k \in \complex,
  \]
  \footnotetext{{Es gilt $|w|^2 = w \cdot \obar{w}$ für $w \in \complex$.}}
  ist die Kovarianz-Matrix \emph{positiv semidefinit}.

  Sind die Elemente der Matrix $Y = (Y_{ik})$ integrierbare Zufallsgrößen, so
  schreiben wir $\pE Y := (\pE Y_{ik})$.

  Mit dieser Bezeichnung lässt sich die Kovarianz-Matrix schreiben als
  \[ \cov X = \pE[ (X - \pE X) \cdot \obar{(X-\pE X)}^T]. \]

  \textbf{Vereinbarung.} In Ausdrücken, die Matrizenoperationen enthalten,
  werden Elemente von $\real^d$ und $\complex^d$ als Spaltenvektoren betrachtet.

  Sei $A$ eine $d \times d$-Matrix, dann gilt
  \[ \cov( A \cdot X ) = A \cdot \cov(X) \cdot A^*,\]
  wobei $A^* := \obar{A^T}$.
\end{prgp}

Sei $\alpha := (\alpha_1, \ldots, \alpha_d) \in \nat_0^d$ und sei $\mu$ die
Verteilung eines $d$-dimensionalen Zufallsvektors $X$.

Ist die Funktion $X \to X^\alpha = X_1^{\alpha_1} \cdots X_d^{\alpha_d}$, $X =
(X_1, \ldots, X_d) \in \real^d$ integrierbar bezüglich $\mu$, so heißt
\[ M_\alpha = M_\alpha(\mu) = M_\alpha(X) = \int_{\real^d} X^\alpha \diffop
  \mu(x) \]
das \emph{Moment der Ordnung $\alpha$} (von $\mu$ oder $X$).
\[ A_\alpha = A_\alpha(\mu) = A_\alpha(X) = \int_{\real^d} |X^\alpha| \diffop
  \mu(x) \]
heißt das \emph{absolute Moment der Ordnung $\alpha$}.


\section{Aufgaben}
Siehe \verb+Aufgaben-1-8.pdf+.

\section{Aufgaben}
Siehe \verb+Aufgaben-1-9.pdf+.

\section{Bedingte Wahrscheinlichkeit}
\begin{defn}
  Hat bei $N$ Versuchen das Ereignis $B$ genau $n$-mal stattgefunden und ist bei
  diesen $n$ Versuchen $k$-mal (zusammen mit $B$) auch das Ereignis $A$
  eingetreten, so wird der Quotient
  \[ h_{A|B} = \frac{k}{n} \]
  die \emph{bedingte relative Häufigkeit} des Ereignisses $A$ \emph{unter der
    Bedingung $B$} genannt.

  Es gilt
  \[ h_{A|B} = \frac{k}{n} = \frac{\frac{k}{N}}{\frac{n}{N}} = \frac{h_{A \cap
        B}}{h_B}, \]
  wenn $h_B \ne 0$.
\end{defn}

\begin{defn}
  Es sei $B \in \mA$ ein Ereignis mit $\pP(B) > 0$. Für $A \in \mA$ nennt man
  die Zahl
  \[ \pP( A | B ) := \frac{\pP(A \cap B)}{\pP(B)} \]
  die \emph{bedingte Wahrscheinlichkeit} von $A$ \emph{unter der Bedingung $B$}.

  Wenn $A$ und $B$ unabhängig sind, gilt $P(A|B) = P(A)$ $\Leftrightarrow$
  \[ \pP( A \cap B ) = \pP(A) \cdot \pP(B). \]

  Ist $\pP(A) > 0$, so gilt
  \[ \pP(B|A) = \frac{\pP(A|B) \cdot \pP(B)}{\pP(A)}. \]
\end{defn}

\begin{thm}
  Es seien $B_1, B_2, \ldots$ unvereinbare Ereignisse mit $\pP(B) > 0$ und
  $\bigcup_n B_n = \Omega$. Dann gilt
  \begin{enumerate}[(i)]
  \item $\pP(A) = \sum_n \pP(A \cap B_n)$ \hfill
    {\footnotesize (da $A = \bigcup_n (A \cap B_n)$)}
  \item $\pP(A) = \sum_n \pP( A | B_n ) \cdot \pP(B)$ \hfill
    {\footnotesize (folgt aus (i))}
  \item Bayes'sche Formel
    \[ \pP(B_k | A) = \frac{\pP(A|B_k \cdot \pP(B_k)}{\sum_n \pP(A | B_n) \cdot
        \pP(B_n)}, \]
    $\pP(A) > 0$, $k= 1, 2, \ldots$
  \end{enumerate}
\end{thm}

\begin{defn}
  Es sei $X$ eine Zufallsgröße und $B$ ein Ereignis mit positiver
  Wahrscheinlichkeit. Unter der \emph{bedingten Verteilungsfunktion} bezüglich
  der Bedingung $B$ verstehen wir die Funktion
  \[ F( t | B ) := \pP( X < t | B ), \quad t \in \real. \]
  Ist $B = \Omega$, so erhält man die gewöhnliche Verteilungsfunktion.
\end{defn}

\begin{exmp}
  Beste Wahl. Betrachte eine Autobahn mit $n$ Tankstellen, die Benzinpreise sind vorher
  unbekannt.

  Ziel: An der billigsten Tankstelle zu tanken.

  Strategie: An $s-1$ ($s=2,3, \ldots$) Tankstellen vorbeifahren, den
  niedrigsten Preis notieren. Wenn danach eine billigere Tankstelle kommt,
  tanken.
  \[ p(s) := \pP[ \text{Billigste Tankstelle gewählt}] \to \max. \]
  Es gilt
  \begin{align*}
    p(s) &= \sum_{k=s}^n \pP[ \text{Die $k$-te ist die billigste und wird
           gewählt.} ] \\
         &= \sum_{k=s}^n \pP[ \text{$k$-te ist die billigste.} ] \cdot
           \pP[ \text{$k$-te wird gewählt} | \text{$k$-te ist die billigste.}] \\
    &= \rez{n} \sum_{k=s}^n \frac{s-1}{k-1} \to \max \text{ bezüglich $s$}
  \end{align*}
  Nun erhalten wir
  \begin{align*}
    p(s) &= \frac{s-1}{n} \cdot \sum_{k=s-1}^{n-1} \rez{k} \\
         &= \frac{s-1}{n} \cdot \left[ \sum_{k=1}^{n-1} \rez{k} - \sum_{k=1}^{s-2} \rez{k} \right] \\
         &= \frac{s-1}{n} \cdot [ \ln(n-1) - \ln(s-2)] \\
         &\approx \frac{s}{n} \cdot \ln \frac{n}{s},
  \end{align*}
  wenn $n, s$ groß sind. Ableiten nach $s$ liefert ein Maximum, wenn $s =
  \frac{n}{e}$, dann ist
  \[ p(s) \approx \rez{e} \approx 0,367. \]
  Das heißt, man muss an etwa 36 \% der Tankstellen vorbei fahren.
\end{exmp}

\chapter{Gesetze}
\section{0-1-Gesetze}
\begin{lem}
  Für jede Zahl $p \in [0,1)$ gilt
  \[ \ln(1-p) \le -p. \]
\end{lem}

\begin{proof}
  Mittelwertsatz der Differentialrechnung
  \begin{align*}
    - \ln (1-p) &= \ln(1) - \ln(1-p) \\
                &= \left( 1 - (1-p) \cdot \rez{\eta} \right) \\
                &= \frac{p}{\eta}.
  \end{align*}
  Es gilt $1-p < \eta < 1$, $\ln(x) \le x -1$. Aus $\rez{\eta} \ge p$ folgt die
  Behauptung.
\end{proof}

\begin{lem}
  Für jede Folge $\{p_n\}$ reeller Zahlen mit $0 \le p_n \le 1$ gilt:
  \[ \sum_{k=1}^\infty p_k = \infty \qRq \lim_{n \to \infty} \prod_{k=1}^n
    (1-p_k) = 0. \tag{i} \]
\end{lem}

\begin{proof}
  O.B.d.A. $p_n < 1$, anderenfalls ist die Behauptung trivial.
  \[ \ln \prod_{k=1}^n (1-p_k) = \sum_{k=1}^n \ln (1-p_k)
    \overset{\text{(2.1.1)}}{\le} - \sum_{k=1}^n p_k \]
  Wir bilden $\exp(\cdot)$:
  \[ \prod_{k=1}^n (1-p_k) = \exp \left( - \sum_{k=1}^n p_k \right)
    \xrightarrow{n \to \infty} 0 \]
  wegen der Annahme.
\end{proof}

\clearpage

\begin{lem}
  Für jede Folge $\{A_n\}$ von unabhängigen Ereignissen gilt:
  \[ \pP \underbrace{\left[ \limsup_{n \to \infty} A_n \right]}_{=: A} = 1 -
    \lim_{n \to \infty} \lim_{N \to \infty} \prod_{m=n}^N (1-\pP(A_m)). \]
\end{lem}

\begin{proof}
  Betrachte 
  \[ A = \bigcap_{n=1}^\infty \bigcup_{m=n}^\infty A_m,\quad \obar{A} =
    \bigcup_{n=1}^\infty \bigcap_{m=n}^\infty \obar{A_m}. \]
  Der Beweis erfolgt durch Berechnung von $\pP$ auf beiden Seiten und durch
  Ausnutzung der Stetigkeit von $\pP$ und der Unabhängigkeit der $A_m$.
\end{proof}

\begin{thm}[Lemma von Borel-Cantelli]
  Sei $\{A_n\}$ eine Folge von Ereignissen und $A := \limsup_n A_n$. Dann gilt
  \begin{enumerate}[(i)]
  \item $\sum_1^\infty \pP(A_n) < 0$ $\Rightarrow$ $\pP(A) = 0$.
  \item Ist die Folge $\{A_n\}$ unabhängig, so gilt
    \[ \begin{aligned}
        \sum_{n=1}^\infty \pP(A_n) &< \infty & &\Leftrightarrow & \pP(A) &= 0, \\
        \sum_{n=1}^\infty \pP(A_n) &= \infty & &\Leftrightarrow & \pP(A) &= 1.
      \end{aligned} \]
  \end{enumerate}
\end{thm}

\begin{proof}
  (i): $A \subset \bigcup_{m=n}^\infty A_m$ $\Rightarrow$ $\pP(A) \le
  \sum_{m=n}^\infty \pP(A_m)$ für jedes $n=1,2,\ldots$, woraus (i) folgt.
  \[ \sum_1^\infty \cdot = \sum_1^{n-1} \cdot + \underbrace{\sum_n^\infty
      \cdot}_{\to 0} \]
  (ii): Sei $\{ A_n \}$ unabhängig und $\sum_1^\infty \pP(A_n) = \infty$. Nach
  Lemma 2.1.3 ist
  \[ \pP(A) = 1 - \lim_{n \to \infty} \lim_{N \to \infty} \prod_{m=n}^N (1 -
    \pP(A_m) ). \]
  Aus Lemma 2.1.2 folgt
  \[ \lim_{N \to \infty} \prod_{m=n}^N (1 - \pP(A_m)) = 0 \]
  für jedes $n$, daraus folgt $\pP(A) = 1$.
\end{proof}

\begin{folg}[0-1-Gesetz von Borel]
  Existiert in einer Folge $\{ A_n \}$ von Ereignissen eine \emph{unabhängige}
  Teilfolge $\{A_{n_k}\}$ mit
  \[ \sum_{k=1}^\infty \pP(A_{n_k}) = \infty, \]
  so ist $\pP( \limsup A_n ) = 1$.
\end{folg}

\begin{proof}
  Satz 2.1.4 für $\{A_{n_k}\}$ und $\limsup_k A_{n_k} \subset \limsup_n A_n $.
\end{proof}

\begin{exmp}
  \begin{enumerate}[(a)]
  \item Eine Münze werde unendlich oft hintereinander geworfen. Man gebe die
    Wahrscheinlichkeit dafür an, dass unendlich oft zweimal hintereinander Kopf
    geworfen wird.

    Lösung: $A_n :=$ Kopf im $n$-ten und $n+1$-ten Wurf. Es gilt $\pP(A_n) =
    \rez{4}$ für alle $n$. Die $A_n$ sind nicht unabhängig:
    \[ \pP( A_n \cap A_{n+1}) = \rez{8} \ne \rez{16} = \pP(A_n) \cdot
      \pP(A_{n+1}). \]
    Allerdings ist $\{A_{2n}\}$ unabhängig und $\sum_1^\infty \pP(A_{2n}) =
    \infty$, also ist $\pP(\limsup A_n) = 1$.

    $\Omega = \{0,1\}^\infty$; $\pP_0(\{0\}) = \pP_0(\{1\}) =\rez{2}$.

    $N$ Versuche: $\{0,1\}^N$, $P_0 \times \cdots \times P_O$, oder über
    Zufallsgrößen $X_n$, unabhängig, mit $\pP(X_n=0) = \pP(X_n=1) = \rez{2}$ und
    Existenzsatz.
  \item Eine Folge $\{t_n\}$ reeller Zahlen nennen wir eine \emph{Unterfolge}
    (Oberfolge) für eine Folge von Zufallsgrößen $\{X_n\}$, wenn die Ereignisse
    $[X_n > t_n]$ ($[X_n \le t_n]$) mit Wahrscheinlichkeit 1 unendlich oft
    eintreten. Sind die $X_k$ unabhängig, so ist jede Folge$\{t_n\}$ entweder
    Ober- oder Unterfolge für $\{X_n\}$. Das folgt aus Satz 2.1.4 und aus der
    Tatsache, dass die Reihen $\sum_1^\infty \pP(X_n > t_n)$ und $\sum_1^\infty
    \pP(X_n \le t_n) = \sum_1^\infty (1-\pP(X_n > t_n))$ nicht gleichzeitig
    konvergieren können.

    \textbf{Bemerkung.} Ist es möglich, dass eine Folge gleichzeitig Ober- und
    Unterfolge ist? $X_n := 0$, $t_n = (-1)^n$.
  \end{enumerate}
\end{exmp}

\begin{lem}
  \begin{enumerate}[(i)]
  \item Es sei $\{a_n\}$ eine Folge reeller Zahlen mit $a_1 \ge a_2 \ge \ldots
    \ge 0$. Dann gilt:
    \[ \sum_{n=1}^\infty a_n < \infty \qLRq \sum_{k=1}^\infty 2^k \cdot a_{2^k}
      < \infty. \]
  \item $\sum_1^\infty \rez{n^a} < \infty$ für $a > 1$.
  \item $\sum_1^\infty \rez{(n \cdot \log_b n)^a} = \infty$ für $0 < a <
    1$, $b>1$.
  \end{enumerate}
\end{lem}  

\begin{proof}
  Aufgabe, Hinweis für (i):
  \[ S_n := a_1 + \ldots + a_n; \quad t_k = a_1 + 2 a_2 + \ldots + 2^k
    a_{2^k}. \]
  Dann gilt: $S_n \le t_k$ wenn $n < 2^k$ und $2 \cdot S_n \ge t_k$, wenn $n > 2^k$.
\end{proof}

\clearpage

\begin{lem}
  Es sei $a, p \in (0,1)$, $\log := \log_{1/p}$ und $k_n := \lfloor n \cdot \log
  n \rfloor$. Dann ist $\{ k_n \}_2^\infty$ eine streng monoton wachsende Folge
  von positiven ganzen Zahlen mit
  \begin{enumerate}[(i)]
  \item $k_n + \lfloor a \cdot \log k_n \rfloor + 1 < k_{n+1}$, wenn $n$
    hinreichend groß ist.
  \item $\sum_2^\infty \rez{k_n^a} = \infty$.
  \end{enumerate}
\end{lem}  

\begin{proof}
  (i): Aus $l \le n \cdot \log n$ folgt 
  \[ \lfloor a \cdot \log k_n \rfloor 
    \le a \cdot \log( n \cdot \log n ) 
    = a \cdot (\log n + \log \log n), \]
  also
  \begin{align*}
    k_n + \lfloor a \cdot \log k_n \rfloor
    &\le n \cdot \log n + a \cdot (\log n + \log \log n) \\
    &= (n+1) \cdot \log n - (1-a) \cdot \log n + a \cdot \log \log n \\
    &\le k_{n+1} + 1 - (1-a) \cdot \log n + a \cdot \log \log n \\
    &< k_{n+1} - 1
  \end{align*}  
  für großes $n$, wenn $\log n > \rez{1-a} \cdot ( 1 + a \cdot \log \log n)$.

  (ii) folgt aus (2.1.7.iii).
\end{proof}

\begin{defn}
  Es sei $\{ X_n \}$ eine Folge unabhängiger Zufallsgrößen mit $\pP [X_n = 1] =
  p$, $\pP[X_n = 0] = 1-p$, $0 < p < 1$.

  Für $n = 1, 2, \ldots$ definieren wir die Zufallsgröße $N_n$ durch
  \[ N_n(w) := \begin{cases}
    0, & \text{falls } X_n(w) = 0, \\
    j, & \text{falls } X_n(w) := \ldots = X_{n+j-1}(w) = 1 \text{ und }
    X_{n+j}(w) = 0.
  \end{cases} \]
\end{defn}

\begin{prgp}[Aufgabe]
  \begin{enumerate}[(a)]
  \item Zu zeigen: $\pP[N_n = j] = p^j \cdot (1-p)$; $\pP[N_n \ge j] = p^j$ und
    $\pE(N_n) = \frac{p}{1-p}$ ($N_n$ ist geometrisch verteilt).
  \item Ist $n+j \le m$, so sind die Ereignisse $[N_n \ge j]$ und $[N_m \ge
    k]$ unabhängig, $k \ge 0$ beliebig.
  \end{enumerate}
\end{prgp}

\begin{thm}
  Mit den Bezeichnungen von 2.1.9 gilt
  \[ \pP \left[ \limsup_{n \to \infty} \frac{N}{\log n} = 1  \right] = 1,
    \tag{1} \]
  wobei $\log := \log_{1/p}$.
\end{thm}

\begin{proof}
  Für $a > 1$ gilt:
  \[ \pP [N_n > a \cdot \log n] \le \pP( N_n \ge \lfloor a \cdot \log n
    \rfloor) \overset{\text{(2.1.10.a)}}{=} p^{\lfloor a \cdot \log n \rfloor} \le p^{a
      \cdot \log n - 1} = \rez{p \cdot n^a}. \]
  Mit Borel-Cantelli für (2.1.7.iii) folgt $\pP( N > a \cdot \log n, \text{
    unendlich oft}) = 0$, also 
  \[ \pP \left[  \limsup_{n \to \infty} \frac{N}{\log n} \le a \right] = 1 \]
  für alle $a > 1$. Daraus folgt
  \[ \pP \left[ \limsup_{n \to \infty} \frac{N}{\log n} \le 1 \right] = 1.
    \tag{2} \]

  Hinweis zu dieser Folgerung: Sei $X$ eine beliebige Zufallsgröße, $\pP [X \le
  a] = 1$ für alle $a > 1$, also ist $\pP( X \le 1 = 1)$, denn es gilt
  \[ [ X \le 1 ] = \bigcap_{n=1]}^\infty \left[ X \le 1 + \rez{n} \right] \]

  \textbf{Bemerkung.} Die Ereignisse $[N_n > a \cdot \log n]$ sind nicht
  unabhängig.

  Seien jetzt $a \in (0,1)$ und $k_n$ aus (2.1.8). Dann sind
  \[ A_n := [ N_k \ge \lfloor a \cdot \log k_n \rfloor + 1 ], \quad n \ge n_0 \]
  unabhängig, wenn $n_0$ hinreichend groß ist\footnote{%
    Das folgt aus (2.1.10.b) und (2.1.8.i).}.
  Weiterhin gilt:
  \[ \pP(A_n) = p^{\lfloor a \cdot \log k_n \rfloor + 1} \ge p^{a \cdot \log k_n
      + 1} = \frac{p}{k_n^a}. \]
  Mit (2.1.8.ii) folgt
  \[ \sum_{n=n_0}^\infty \pP(A_n) = \infty \]
  und damit können wir mit Borel-Cantelli feststellen, dass
  \[ \begin{aligned}
      &\pP [ N_n \ge a \cdot \log n, \text{ unendlich oft}] \ge \\
      &\pP N_{k_n} \ge a \cdot \log k_n, \text{ unendlich oft}] \ge \\
      &\pP N_{k_n} \ge \lfloor a \cdot \log k_n \rfloor + 1 , \text{ unendlich
        oft}] = 1
    \end{aligned} \]
  für alle $a \in (0,1)$, also
  \[ \pP \left[ \limsup_{n \to \infty} \frac{N}{\log n} \ge 1 \right] = 1.
    \tag{3} \]
  Aus (2) und (3) folgt (1).
\end{proof}

\clearpage

\begin{thm}
  Es sei $\{X_n\}$ eine Folge unabhängiger Zufallsgrößen.
  \[ \lim_{n \to \infty} X_n = 0 \text{ fast sicher} \qLRq \sum_{n=1}^\infty
    \pP\left[ |X_n| \ge \rez{k} \right] < \infty \]
  für $k = 1, 2, \ldots$.
\end{thm}

\begin{defn}
  Es sei $\{ X_n \}$ eine Folge von Zufallsgrößen. Ein Ereignis $A \in \mA$
  heißt \emph{terminales Ereignis}, wenn $A \in \sigma( X_n, X_{n+1}, \ldots)$,
  $n=1,2,\ldots$. Hier bezeichnet $\sigma(X_n, X_{n+1}, \ldots)$ die
  $\sigma$-Algebra, die durch die Ereignisse $[X_j \in B]$, $j \ge n$, $B \in
  \borel(\real)$ erzeugt wird.

  Analog: $\sigma(X_1, \ldots, X_n)$.

  Zum Beispiel ist
  \[ A_r = \limsup [X_k > r] = \bigcap_{n=1}^\infty \bigcup_{k=1}^\infty
    [X_k > r], r \in \real \]
  ein terminales Ereignis.
\end{defn}

\begin{lem}
  Seien $X_1, X_2, \ldots$ unabhängige Zufallsgrößen.
  \begin{enumerate}[(i)]
  \item Ist $A \in \sigma( X_{n+1}, X_{n+2}, \ldots )$ für ein $n$, so ist $A$
    unabhängig von der $\sigma$-Algebra $\sigma(X_1, \ldots, X_n)$.
  \item Ist $A$ unabhängig von $\sigma( X_1, \ldots, X_n)$ für alle $n \ge 1$,
    so ist $A$ unabhängig von $\sigma( X_1, X_2, \ldots )$.
  \end{enumerate}
\end{lem}

\begin{proof}
  Übungsaufgabe. Hinweis: Satz 1.1.9 benutzen.
\end{proof}

\begin{thm}[0-1-Gesetz von Kolmogorov]
  Sind $X_1, X_2, \ldots$ unabhängig und $A$ ein terminales Ereignis, so gilt
  $P(A) = 0$ oder $P(A) = 1$.
\end{thm}

\begin{proof}
  Sei $A \in \sigma(X_{n+1}, \ldots)$, für alle $n \ge 0$, dann ist $A$
  unabhängig von $\sigma(X_1, \ldots, X_n)$ für alle $n \ge 1$. Damit ist $A$
  unabhängig von $\sigma( X_1, X_2, \ldots)$. $A$ ist aber in dieser
  $\sigma$-Algebra enthalten. Damit ist $A$ unabhängig von $A$ und es muss
  gelten
  \[ \pP(A) = \pP(A \cap A) = \pP(A) \cdot \pP(A), \]
  das ist nur der Fall für $\pP(A) = 0$ oder $\pP(A) = 1$.
\end{proof}

\begin{exmp*}
  $X_1, X_2, \ldots$ seien unabhängig. Die Folge $\{ X_n \}$ konvergiert oder
  divergiert fast sicher.
  \[ \{ w : X_n(w) \text{ konvergiert} \} =
    \bigcap_{k=1}^\infty \bigcup_{n_0=1}^\infty \bigcap_{n=n_0}^\infty
    \bigcap_{m=n_0}^\infty
    \left\{ w : | X_n(w) - X_m(w) | < \rez{k} \right\} \]
  ist ein terminales Ereignis, da $| X_n(w) - X_m(w) |$ eine Cauchy-Folge ist.
\end{exmp*}

\begin{proof}[Beweis von Satz 2.1.12]
  Wir betrachten das Ereignis
  \[ A := \{ w : X_n(w) \to 0 \} =
    \bigcap_{k=1}^\infty \bigcup_{n_0=1}^\infty \bigcap_{n=n_0}^\infty
    \left\{ w : | X_n(w) | < \rez{k} \right\} \in \mA. \]
  Wegen Borel-Cantelli (bc) gilt
  \begin{align*}
    \sum_{n_0 = 1}^\infty \pP\left[ |X_{n_0}| \ge \rez{k} \right] < \infty
    &\quad \overset{\text{(bc)}}{\Leftrightarrow} \quad
    \pP \left( \bigcap_{n_0=1}^\infty \bigcup_{n=n_0}^\infty \left[ |X_n| \ge
        \rez{k} \right]  \right) = 0 \\
    &\qLRq 
    \pP \left( \bigcup_{n_0=1}^\infty \bigcap_{n=n_0}^\infty \left[ |X_n| <
        \rez{k} \right]  \right) = 1 \\
    &\qLRq \pP(A) = 1.
    \qedhere
  \end{align*}
\end{proof}

\begin{rmrk*}
  Die Unabhängigkeit ist wichtig, Beispiel: Sei $\Omega = [0,1]$, $\pP = $
  Lebesgue-Maß, $X_n = \ind_{(0,a_n)}$ wobei $0 < a_n < 1$, $a_n \to 0$. Dann
  gilt $\lim_{n \to \infty} X_n(w) = 0$ für alle $w \in \Omega$ und
  \[ \pP \left[  X_n \ge \rez{k} \right] = a_n ( = \lambda( (0,a_n)) ) \]
  \[ \sum_{n=1}^\infty a_n = \infty \]
  zum Beispiel für $a_n = \rez{n}$, $< \infty$ z.B. für $a_n = \rez{n^2}$.
\end{rmrk*}

\begin{defn}
  $\real^\infty := \real \times \real \times \cdots$.

  $\borel^\infty :=$ die $\sigma$-Algebra, die durch Mengen der Form
  \[ B = B_n \times \real \times \real \times \cdots \subset \real^\infty
    \tag{1} \]
  erzeugt wird, wobei $B_n \subset \real^n$ eine beliebige Borel-Menge ist.
  Diese Mengen bilden eine Algebra.

  Eine Menge $B \subset \real^\infty$ heißt \emph{permutierbar}, wenn $t = (t_1,
  t_2, \ldots) \in B$ $\Rightarrow$ $\tau(t) := (t_{\tau(1)}, t_{\tau(2)},
  \ldots) \in B$ für eine beliebige \emph{endliche}\footnote{%
    Das heißt $\tau(j) = j$ bis auf endlich viele $j$.}
  Permutation $\tau$ von $\{ 1, 2, 3, \ldots \}$.

  Es sei nun $X_1, X_2, \ldots$ eine Folge von Zufallsgrößen und $Y := (X_1,
  X_2, \ldots )$. Die Abbildung $Y: \Omega \to \real^\infty$ ist
  $\borel^\infty$-messbar, das heißt $Y$ ist eine $\real^\infty$-wertige
  Zufallsvariable.

  Zu zeigen: $[Y \in B] \in \mA$, $B \in \borel^\infty$ der Form (1). Es gilt:
  $[Y \in B] = [(X_1, \ldots, X_n) \in B_n] \in \mA$.

  Ist die Menge $B \in \borel^\infty$ permutierbar, so heißt auch das Ereignis
  $A := [ Y \in B]$ \emph{permutierbar}.

  \textbf{Beispiel.} Das Ereignis $[X_n \to 0]$ ist permutierbar.
\end{defn}

\clearpage

\begin{thm}[0-1-Gesetz von Hewitt-Savage]
  Sind die Zufallsgrößen $X_1, X_2, \ldots$ unabhängig und \emph{identisch
    verteilt} und ist $A$ ein permutierbares Ereignis, dann gilt entweder
  $\pP(A) = 0$ oder $\pP(A) = 1$.
\end{thm}

\begin{prgp}[Aufgabe]
  Mit den Bezeichnungen von 2.1.16, $X_1, X_2, \ldots$ unabhängig und identisch
  verteilt. Bezeichne $\mu$ die Verteilung\footnote{%
    Das heißt $\pP[Y \in B] = \mu(B), B \in \borel^\infty$}
  von $Y$. Für $D_n, D \in \borel^\infty$ ($n=1,2,\ldots$) schreiben wir $D_n
  \to D$, wenn
  \[ \lim_{n \to \infty} \mu( D_n \Delta D) = 0.\]
  Dann gilt:
  \begin{enumerate}[(i)]
  \item $D_n \to D$ $\Rightarrow$ $\mu(D) = \lim_{n \to \infty}\mu(D_n)$,
  \item $D_n \to D$, $C_n \to D$ $\Rightarrow$ $D_n \cap C_n \to D$,
  \item Ist $\tau: \real^\infty \to \real^\infty$ eine injektive Abbildung, so
    ist $\tau(C \Delta D) = \tau(C) \Delta \tau(D)$.
  \item Ist $\tau$ eine endliche Permutation in $\real^\infty$, so gilt
    \[ \mu( \tau(B) ) = \mu(B) \]
    für $B \in \borel^\infty$.
  \end{enumerate}
  \textbf{Hinweis.} (i) und (ii): 1.1.8 anwenden.

  (iv): $\mu \circ \tau$ ist ein $\pW$-Maß, das mit $\mu$ für Mengen der Form
  2.1.16.1 übereinstimmt $\Rightarrow$ Sie stimmen auch auf $\borel^\infty$
  überein.
\end{prgp}

\begin{prgp}[Beweis von Satz 2.1.17]
  Sei $A = [Y \in B]$ ein permutierbares Ereignis, wobei $B \in \borel^\infty$
  permutierbar ist.

  Für $(X_1, X_2, \ldots) \in \real^\infty$ und $n=1, 2, \ldots$ setzen wir:
  \[ \tau_n( X_1, X_2, \ldots ) := ( X_{n+1}, \ldots, X_{2n}, X_1, \ldots X_n,
    X_{2n+1}, \ldots ). \]
  $\tau_n$ ist eine endliche Permutation.

  Nach dem Approximationssatz existiert eine Folge $B_k \in \borel(\real^{k_n})$
  mit
  \[ \boxed{ D_n := B_{k_n} \times \real \times \cdots \to B } \tag{1} \]
  
  \textbf{Behauptung.} $\tau_{k_n} (D_n) \to \tau_{k_n} (B) = B$
  \begin{proof}
    \[ \mu( \tau_{k_n}( D_n ) \Delta \tau_{k_n}(B)) =
      \overset{\text{2.1.18.iii}}{=}
      \mu( \tau_k( D_n \Delta B)  )
      \overset{\text{2.1.18.iv}}{=}
      \mu( D_n \Delta B ) \to 0.
    \]
    mit 2.1.18.ii folgt
    \[ D_n \cap \tau_{k_n} (D_n) = \boxed{ B_{k_n} \times B_{k_n} \times \real
        \times \real \times \cdots \to B} \tag{2} \]
    Damit gilt
    \begin{align*}
      \pP(A) = \pP( Y \in B ) = \mu(B) = \lim_{n \to \infty} \mu( D_n)
             = \lim_{n \to \infty} \pP [ (X_1, \ldots, X_{k_n}) \in B_{k_n} ]
    \end{align*}
    und
    \begin{align*}
      \pP(A) = \pP( Y \in B )
      &= \lim_{n \to \infty} \pP [ (X_1, \ldots, X_{2k_n}) \in B_{k_n} \times B_{k_n} ] \\
      &= \lim_{n \to \infty} \pP [ (X_1, \ldots, X_{k_n}) \in B_{k_n} ].
    \end{align*}
    Also gilt $\pP(A) = \pP(A)^2$. Das ist nur möglich, wenn $\pP(A) = 0$ oder
    $\pP(A) = 1$.
  \end{proof}
\end{prgp}

\section{Aufgaben}
Siehe \verb+Aufgaben-2-2-Teil-1.pdf+ und \verb+Aufgaben-2-2-Teil-2.pdf+.

\section{Einige Ungleichungen}
\begin{thm}[Ungleichung von Hájek-Rènyj]
  Gegeben seien $n$ unabhängige Zufallsgrößen $X_1, \ldots, X_n \in
  \intf^2(\pP)$ und $n$ positive Zahlen $r_1 \ge r_2  \ge \ldots \ge r_n$. Wir
  setzen
  \[ S_i := \sum_{k=1}^i (X_k - \pE(X_k)), \quad i = 1, \ldots, n. \]
  Dann gilt für jedes $m = 1, \ldots, n$ und jedes $\eps > 0$:
  \[ \pP \left[ \max_{m \le i \le n} r_i | S_i | \ge \eps \right] \le
    \rez{\eps^2} \left( r_m^2 \cdot \sum_{j=1}^m D^2(X_j) + \sum_{j=m+1}^n r_j^2
      D^2( X_j) \right) \tag{1} \]
  Es gilt $\sum_{j=m+1}^n r_j^2 D^2( X_j) := 0$  für $m=n$.
\end{thm}

Zwei Spezialfälle:

\begin{folg}[Ungleichung von Kolmogorov]
  \[ \pP \left[  \max_{1 \le i \le n} |S_i| \ge \eps \right] <
    \rez{\eps^2} \sum_{i=1}^n D^2(X_i).  \]
\end{folg}

\begin{proof}
  $m=1$ und $r_1 = \ldots = r_n = 1$ in 2.3.1.
\end{proof}

\begin{folg}[Ungleichung von Tschebischew]
  \[ \pP [ | X - \pE(X) | \ge \eps ] \le \frac{D^2(X)}{\eps^2}.  \]
\end{folg}

\begin{proof}
  $m = n = 1$, $r_1 = 1$, $X = X_1$ in 2.3.1.
\end{proof}

\textbf{Knobelaufgabe.} Für festes $n \ge 2$ ist das Maximum bzw. Minimum des
Ausdrucks
\[ D = \pP \left( \bigcap_{j=1}^\infty A_j \right)
  - \prod_{j=1}^\infty \pP( A_j )\]
gesucht, wobei die $A_j$ beliebige Ereignisse sind.

\begin{proof}[Beweis von 2.3.1]
  O.B.d.A. sei $\pE(X_i) = 0$, $i = 1, \ldots, n$.  Dann ist
  \[ S_i = X_1 + \ldots + X_i. \]
  Seien $A := $ das Ereignis auf der linken Seite von (1) und $B_i := [ r_i
  \cdot |S_i| \ge \eps]$, $i = m, \ldots, n$. Dann ist
  \[ A = \bigcup_{i = m}^n B_i. \]
  Außerdem seien
  \begin{align*}
    A_m &:= B_m \\
    A_{m+1} &:= \obar{B}_m \cap B_{m+1} \\
    A_{m+2} &:= \obar{B}_m \cap \obar{B}_{m+1} \cap B_{m+2} \\
         &\vdots \\
    A_n &:= \obar{B}_m \cap \ldots \cap \obar{B}_{n-1} \cap B_n.
  \end{align*}
  Die Ereignisse $A_m, \ldots, A_n$ sind paarweise unvereinbar. Es gilt
  $A_i \subset B_i$ und $A = \bigcup_{i=m}^n A_i$. Also folgt
  \[ \pP(A) = \sum_{i=m}^n \pP(A_i). \]

  Sei weiterhin
  \[ Z := \sum_{i=m}^n (r_i^2 - r_{i+1}^2) s_i^2, \]
  wobei $r_{n+1} := 0$. Es gilt
  \[ \pE(Z) = D^2(S_i) = D^2(X_1) + \ldots D^2(X_i), \]
  damit folgt
  \[ \pE(Z) = r_m^2 \cdot \sum_{j=1}^m D^2(X_j) + \sum_{j=m+1}^n r_j^2
    D^2(X_j). \]
  Also ist $\frac{\pE(Z)}{\eps^2}$ die rechte Seite von (1)\footnote{%
    Es gilt
    \[ \sum_{i=m}^n \sum_{j=1}^i \ldots = \sum_{j=1}^m \sum_{i=m}^n \ldots +
      \sum_{j=m+1}^n \sum_{i=j}^n \ldots \]
  }.

  Wir setzen  $Y_i := \ind_{A_i}$, $i = m, \ldots, n$.

  \textbf{Behauptung.}
  \[ \pE( Y_i S_j^2 ) \ge \frac{\eps^2}{r_i^2} \cdot \pP(A_i) \tag{2} \]
  für $m \le i \le j \le n$.

  \textit{Beweis.}
  \[ U_{ij} := S_j - S_i = \sum_{k=i+1}^j X_k, \]
  für $m \le i \le j \le n$. Dann ist $S_j^2 = (S_i + U_{ij})^2$. Also
  \[ \pE(Y_i S_j^2) = \pE(Y_i S_i^2) + \pE(Y_i U_{ij}) + 2 \cdot \pE(Y_i S_i
    U_{ij}). \tag{3} \]

  \begin{prgp}[Aufgabe]
    Die Zufallsgrößen $Y_i S_i$ und $U_{ij}$ sind unabhängig. Hinweis: 2.1.14
    benutzen.
  \end{prgp}

  Aus der Unabhängigkeit folgt
  \[ \pE(Y_i S_i U_{ij}) = \pE(Y_i S_i) \cdot \underbrace{\pE(U_{ij})}_{=0} =
    0. \]

  In (3) eingesetzt
  \[ \pE( Y_i S_j^2 ) \ge \pE(Y_i S_i^2) = \int_{A_i} S_i^2 \diffop \pP \ge
    \frac{\eps^2}{r_i^2} \cdot \pP(A_i), \]
  da $A_i \subset B_i$. Damit ist (2) bewiesen.

  Aus $Z \ge 0$ und $1 \ge \ind_A = \sum_{i=m}^n Y_i$ folgt
  \[ Z \ge \sum_{i=m}^n Y_i Z. \]
  Wegen $r_j^2 - r_{j+1}^2 \ge 0$, (2) und (1) erhalten wir
  \begin{align*}
    \pE(Z) &\ge \pE \left( \sum_{i=m}^n Y_i Z \right) \\
           &= \sum_{i=m}^n \sum_{j=i}^n (r_j^2 - r_{j+1}^2) \pE( Y_i S_j^2 ) \\
           &\ge \sum_{i=m}^n \sum_{j=i}^n (r_j^2 - r_{j+1}^2) \pE( Y_i S_i^2 ) \\
           &\ge \sum_{i=m}^n \sum_{j=i}^n (r_j^2 - r_{j+1}^2) \frac{\eps^2}{r_i^2} \cdot \pP(A_i) \\
           &= \sum_{i=m}^n \frac{\eps^2}{r_i^2} \cdot \pP(A_i) \cdot
             \underbrace{\sum_{j=i}^n (r_j^2 - r_{j+1}^2)}_{r_i^2}
             = \eps^2 \pP(A). \qedhere
  \end{align*}
\end{proof}

\clearpage

\begin{thm}
  Für jede stetige Funktion $f : [0,1] \to \real$ gilt:
  \[ f(x) = \lim_{n \to \infty} \sum_{k=0}^n f \left( \frac{k}{n} \right)
    \binom{n}{k} x^k (1-x)^{n-k}, \quad x \in [0,1], \]
  wobei die Konvergenz gleichmäßig ist.
\end{thm}

\begin{folg}[Weierstrass]
  Jede stetige Funktion $f : [0,1] \to \real$ lässt sich gleichmäßig durch
  Polynome approximieren.
\end{folg}

\begin{proof}[Beweis von 2.3.5]
  Sei $Y_n^x$ eine Zufallsgröße, sodass $n \cdot Y_n^x$ binomialverteilt ist mit
  den Parametern $n$ und $x \in [0,1]$. Es gilt
  \[ \pE(Y_n^x) = x, \qquad D^2(Y_n^x) = \frac{x(1-x)}{n} \]
  und
  \[ \pE(f(Y_n^x)) = \sum_{k=0}^n f \left( \frac{k}{n} \right) \binom{n}{k} x^k
    (1-x)^{n-k}. \]
  Deshalb brauchen wir nur zu zeigen, dass
  \[ \lim_{n \to \infty} \pE( f(Y_n^x) - f(x) ) = 0 \]
  gleichmäßig auf $[0,1]$.

  Aus der Tschebischew-Ungleichung folgt
  \[ \pP [ |Y_n^x - x| > n^{-1/4} ] \le n^{1/2} \cdot D^2(Y_n^x) =
    \frac{x(1-x)}{\sqrt{n}} \le \rez{\sqrt{n}}. \tag{1} \]
  Es sei $K$ eine Konstante mit $|f| \le K$ und bezeichne $A$ das Ereignis auf
  der linken Seite von (1). Dann ist $\pP(A) \le \rez{\sqrt{n}}$ und
  \begin{align*}
    \pE( |f(Y_n^x) - f(x)|) 
    &= \int_\Omega | f(Y_n^x) - f(x) | \diffop \pP(w) \\
    &= \int_A \ldots + \int_{\Omega \setminus A} \ldots \\
    &\ge \frac{2K}{\sqrt{n}} + \sup_{|y-z| \le n^{-1/4}} |f(y) - f(z)|
      \to 0
  \end{align*}
  für $n \to \infty$, da $f$ gleichmäßig stetig\footnote{%
    $\forall \eps \exists \delta : | f(x) - f(y) | < \eps$, wenn $|x-y| < \delta$.
  } ist.
\end{proof}

\begin{defn}
  Es sei $X$ eine Zufallsgröße. Eine reelle Zahl $m = m(X)$ heißt \emph{Median}
  (Zentralwert) von $X$, wenn $\pP[X \le m] \ge \rez{2}$ und $\pP[X \ge m] \ge
  \rez{2}$ gelten\footnote{%
    Im Allgemeinen ist der Median nicht eindeutig, zum Beispiel $\pP[X=1] =
    \pP[X=-1] = \rez{2}$. Jede Zahl $m \in [-1,1]$ ist Median.}.

  Eine Zufallsgröße heißt \emph{symmetrisch}, wenn für alle $t \in \real$
  \[ \pP[X < t] = \pP[x > -t]. \]
\end{defn}

\begin{rmrk*}
  $X$ ist genau dann symmetrisch, wenn $X$ und $-X$ die selbe Verteilung
  besitzen, das heißt $\mu_x(B) = \mu_X(-B)$ für jede beliebige Borelmenge $B
  \in \borel(\real)$.

  Es gilt $\pP[X < t] = \pP[X > -t]$ für alle $t \in \real$ $\Leftrightarrow$
  $\mu_X(B) = \mu_X(-B)$.  Wegen des Eindeutigkeitssatzes und $\borel(\real) =
  \sigma( \{(-\infty,a) : a  \in \real\})$ reicht es zu zeigen, dass
  \[ \mu_X((-\infty,a)) = \mu_X((-a,\infty)). \]

  Ist $X$ symmetrisch, dann ist 0 ein Median von $X$.
  \begin{align*}
    1 &= \pP[X \le 0] + \pP[X \ge 0] = 2 \pP[X \le 0] = 2 \pP[X \ge 0] \\
      &\Leftrightarrow \pP[X \le 0] \ge \rez{2}, \quad \pP[X \ge 0] \ge \rez{2}
  \end{align*}

  Die Summe von unabhängigen, symmetrischen Verteilungen ist wieder symmetrisch.
\end{rmrk*}

\begin{thm}[Ungleichung von Levy]
  Es seien $X_1, \ldots, X_n$ unabhängige Zufallsgrößen und $S_j = \sum_{i=1}^j
  X_i$. Für jedes $\eps > 0$ gilt:
  \begin{enumerate}[(i)]
  \item $\pP[\max_j (S_j - m(S_j - S_n)) \ge \eps] \le 2 \pP[ S_n \ge \eps]$,
  \item $\pP[\max_j (S_j - m(S_j - S_n)) \ge \eps] \le 2 \pP[ |S_n| \ge \eps]$,
  \end{enumerate}
  wobei $m(Y)$ einen beliebigen Median von $Y$ bezeichnet.
\end{thm}

\begin{proof}
  (i): Sei $S_0 := 0$, $\eps > 0$ beliebig. Es bezeichne
  \[ T(\omega) := \min_j \{ S_j(\omega) - m(S_j - S_n) \ge \eps \}, \]
  falls eine solche Zahl $j$ existiert, anderenfalls $T(\omega) = n + 1$.

  Dann sind zunächst $\{ T = j \}_{j=1, \ldots, n}$ disjunkt und
  \begin{align*}
    \{1 \le T \le n\}
    &= \{ \max_j (S_j - m(S_j - S_n)) \ge \eps \} \\
    &= \{ \omega \in \Omega : \exists j \in \{1, \ldots, n\} : S_J - m(S_j - S_n) \ge \eps  \}.
  \end{align*}
  Definiere $B_j := \{ m(S_j - S_n) \ge S_j - S_n\}$, $1 \le j \le n$. Dann gilt
  \begin{itemize}
  \item $\pP(B_j) \ge \rez{2}$,
  \item $\{ T = j \} \in \sigma(X_1, \ldots, X_j)$,
  \item $B_j \in \sigma(X_{j+1}, \ldots, X_n)$.
  \end{itemize}
  Nach 2.1.14(i) sind $\{T=j\}$ und $B_j$ unabhängig und
  \[ \bigcup_{j=1}^n (B_j \cap \{T=j\}) \subseteq \{S_n \ge \eps\}. \]
  
  Es existiert $j \in \{1, \ldots, n\}$, so dass $S_j(\omega) - m(S_j - S_n) \ge
  \eps$ und $S_i(\omega) - m(S_i - S_n) < \eps$ für alle $i < j$ und $m(S_j -
  S_n) \ge S_j - S_n$, also gilt
  \[ S_n \ge S_j - m(S_j - S_n) \ge \eps. \]

  Es folgt mit der Disjunktheit der $\{T=j\}$ (d):
  \begin{align*}
    \pP[S_n \ge \eps]
    &\ge \pP\left[ \bigcup_{i=1}^n B_j \cap \{T=j\} \right] \\
    &\overset{\text{(d)}}{=}
      \sum_{j=1}^n \pP[B_j \cap \{T=j\}] \\
    &= \sum_{j=1}^n \pP(B_j) \cdot \pP(\{T=j\}) \\
    &= \rez{2} \sum_{j=1}^n \pP(\{T=j\})\\
    &= \rez{2} \pP \left[ \bigcup_{j=1}^n \{T=j\} \right] = \rez{2} \pP [1 \le T \le n].
  \end{align*}

  (ii). Eventuell Übungsaufgabe. Hinweis: Schreibt man in (i) $-X_i$ anstelle
  von $X_j $ und benutzt man $m(-Y) = -m(Y)$, folgt (ii).
\end{proof}

\begin{folg}
  Es seien $X_1, \ldots, X_n$ unabhängige, symmetrische Zufallsgrößen und $S_j =
  \sum_{i=1}^j X_i$. Für jedes $\eps > 0$ gilt dann:
  \begin{enumerate}[(i)]
  \item $\pP[\max_j S_j \ge \eps] \le 2 \pP[S_n \ge \eps]$,
  \item $\pP[\max_j |S_j| \ge \eps] \le 2 \pP[|S_n| \ge \eps]$.
  \end{enumerate}
\end{folg}

\begin{proof}
  Das folgt aus 2.3.8 und der Symmetrie der $X_j$ bzw. $S_j$.
\end{proof}

\begin{thm}[Jensensche Ungleichung]
  Es seien $I \subseteq \real$ ein Intervall $f: I \to \real$ eine konvexe
  Funktion und $X: \Omega \to I$ eine integrierbare Zufallsgröße\footnotemark.
  Dann gilt 
  \[ \pE( f(X)^- ) < \infty \quad \text{und} \quad
    f(\pE(X)) \le \pE(f(X)) := \pE(f(X)^+) - \pE(f(X)^-).  \]
  Ist $f$ im Punkt $\pE(X)$ streng konvex, so gilt Gleichheit genau dann, wenn
  $X$ fast sicher konstant ist.
\end{thm}
\footnotetext{%
$X$ besitzt einen Erwartungswert, wenn $\min \{\pE X^+, \pE X^-\} < \infty$,
bzw. $X$ besitzt einen endlichen Erwartungswert, wenn $\max\{\pE X^+, \pE X^-\}
< \infty$.}

\begin{proof}
  \begin{enumerate}[i)]
  \item Sei $h(t) := f(\pE(X)) + a \cdot (t - \pE(X))$ eine Gerade\footnote{%
      Tangente an der Funktion $f$}
    durch den Punkt $(\pE X, f(\pE X))$ mit $h(t) \le f(t)$, $t \in I$.

    Da $h(X)$ integrierbar ist (weil $X$ integrierbar) und wegen $f < 0$
    $\Rightarrow$ $h < 0$ ($f(x)$) erhalten wir
    \[ \pE( f(X)^-) \le \pE(h(X)^-) < \infty. \]

    Somit ist $\pE(f(X))$ definiert und
  \item $\pE(f(X)) \ge \pE(h(X)) = f(\pE(X))$,
  \item $f(\pE(X)) = \pE(f(X))$ $\Leftrightarrow$ $X = \pE(X)$ fast sicher
    (bezogen auf $\pP$).
  \end{enumerate}
  Angenommen $X \ne \pE(X)$, dann ist wegen der strikten Konvexität
  \[ f(X) - h(X) > 0 \qRq \text{Widerspruch!} \]
  Also gilt
  \[ \pE(f(X)) = f(\pE(X)) = \pE(h(x)). \qedhere \]
\end{proof}

\begin{defn}
  Sei $X$ eine Zufallsvariable mit Werten in $S = \{s_1, \ldots, s_n\}$ und
  $p(s) = \pP[X=s]$. Unter der \emph{Entropie} von $X$ versteht man die Zahl
  \[ H(X) = \pE(-\log_2(p(X))) = - \sum_{s \in S} p(s) \log_2 p(s), \]
  wobei $0 \cdot \log_2 0 = 0$.
\end{defn}

\section{Aufgaben}
Siehe \verb+Aufgaben-2-4.pdf+

\section{Konvergenz von Zufallsgrößen}
\begin{defn}
  Eine Folge $(X_n)$ von Zufallsgrößen heißt
  \begin{enumerate}[a)]
  \item \emph{Fast sicher konvergent} gegen die Zufallsgröße $X$, wenn
    \[ \pP \left[ \lim_{n \to \infty} X_n = X \right] = 1. \]
    $\left[ \lim_{n \to \infty} X_n = X \right] = \{ \omega \in \Omega : \lim_n
    X_n(\omega) = X(\omega) \}.$
  \item \emph{Konvergent in Wahrscheinlichkeit} gegen die Zufallsgröße $X$, wenn
    \[ \lim_{n \to \infty} \pP \left[ |X_n-X| < \eps \right] = 0 , \quad \forall
      \eps > 0. \]
  \item \emph{Konvergent in Verteilung} gegen die Zufallsgröße $X$, wenn für die
    Folge der Verteilungsfunktion $F_n$ von $X$ bzw. $F$ von $X$ gilt:
    \[ \lim_{n \to \infty} F_n(x) = F(x) \]
    für jeden Stetigkeitspunkt von $F$.
  \end{enumerate}
\end{defn}

\begin{rmrk}
  \begin{enumerate}[(a)]
  \item $[w : \lim_n X_n(w) = X(w)]$ ist ein Ereignis, Beweis wie in (2.1.12).
  \item Konvergiert eine Folge $\{X_n\}$ gegen $X$ fast sicher oder in
    Wahrscheinlichkeit und gleichzeitig auch gegen $Y$, so ist $X=Y$ fast
    sicher.

    Für die fast sichere Konvergenz ist das trivial, für die Konvergenz in
    Wahrscheinlichkeit folgt sie aus
    \begin{align*}
      \pP [ |X-Y| > \eps | ]
      &\le \pP \left[ |X_n - X| \ge \frac{\eps}{2} \right]
        + \pP \left[ |X_n - Y| \ge \frac{\eps}{2} \right].
    \end{align*}
    Dabei wurde benutzt, dass
    \[ [|X-Y| > \eps ] \subseteq
      \left[ |X_n-X| > \frac{\eps}{2} \right] \cup
      \left[ |X_n-Y| > \frac{\eps}{2} \right].
    \]
    Das folgt aus der Dreiecksungleichung. Es ergibt sich
    \[ \pP [ |X-Y| > \eps ] = 0, \quad \text{für alle } \eps > 0,  \]
    also ist $X=Y$ fast sicher, da
    \[ [X=Y] = \bigcap_{k=1}^\infty \left[ |X-Y| \le \rez{k} \right]. \]
  \end{enumerate}
  Für die Konvergenz  in Verteilung gilt das nicht, es gibt verschiedene
  Zufallsgrößen mit der selben Verteilung.

  Einfaches Beispiel: $\Omega = \{1,2\}$, $\mA = \pot(\omega)$, $\pP(\{1\}) =
  \pP(\{2\})= \rez{2}$,
  \begin{align*}
    X(1) &= 1, & X(2) &= 0, \\
    Y(1) &= 0, & Y(2) &= 1.
  \end{align*}
  Dann gilt $\pP[X=1] = \pP[Y=1] = \rez{2}$. Nun definiere $X_n := X$, es gilt
  $X_n \to X$ und $X_n \to Y$ in Verteilung, aber $\pP[X=Y]=0$.
\end{rmrk}

\begin{thm}
  $X_n \to X$ fast sicher $\Rightarrow$ $X_n \to X$ in Wahrscheinlichkeit
  $\Rightarrow$ $X_n \to X$ in Verteilung.
\end{thm}

\begin{proof}
  Nehmen wir an, dass $X_n \to X$ fast sicher. Es sei $\eps > 0$ beliebig,
  \[ A:= \{ w : \lim X_n(w) = X(w) \}, \quad
    B_n := \{ w : |X_m(w) - X_n(w) | \le \eps, m \ge n \}. \]
  Dann gilt $\pP(A) = 1$, $B_1 \subset B_2 \subset \ldots$ und
  \[ \bigcup_{n=1}^\infty B_n \supset A \qRq \lim \pP(B_n) \ge \pP(A) = 1
    \qRq \lim \pP(B_n) = 1. \]
  Definiere
  \[ A_n := [ |X_n - X| \le \eps], \quad A_n \supset B_n, \]
  also folgt $\pP(A_n) \to 1$ und damit $X_n \to X$ in Wahrscheinlichkeit.

  Nehmen wir jetzt an, dass $X_n \to X$ in Wahrscheinlichkeit, $A_n$ wie oben.
  Dann gilt $\pP(A_n) \to 1$.
  \[ F_n(t) := \pP[X_n < t] = \pP([X_n < t] \cap A_n) + \pP([x_n < t] \cap
    \obar{A}_n), \tag{1} \]
  also
  \[ F_n(t) \le \pP([X_n < t] \cap A_n) + \pP( \obar{A}_n)  \le
    \pP([X< t + \eps ]) + \pP(\obar{A}_n)
    \xrightarrow{n \to \infty} F(t + \eps). \]
  Also gilt
  \[ \limsup_{n \to \infty} F_n(t) \le F(t+\eps). \]
  Andererseits ist wegen (1) und $\pP(A_n) \to 1$:
  \[ F_n(t) \ge \pP( [X_n < t] \cap A_n) \ge \pP([X < t - \eps] \cap A_n)
    \xrightarrow{n \to \infty} \pP[X < t - \eps] = F(t - \eps). \]
  Also gilt
  \[ \liminf_{n \to \infty} F_n(t) \ge F(t-\eps). \]
  Wir erhalten
  \[ F(t-\eps) \le \liminf F_n(t) \le \limsup F_n(t) \le F(t + \eps), \]
  für alle $\eps > 0$. Ist $F$ stetig in $t$, so folgt (mit $\eps \to 0$):
  \[ F(t) = \lim_{n \to \infty} F_n(t). \qedhere \]
\end{proof}

\begin{exmp}
  \begin{enumerate}[(a)]
  \item $\Omega := [0,1]$, $\pP = \lambda$, $\mA :=$ alle Borel-Mengen $\subset
    [0,1]$. $X_n^i$ sei die Indikatorfunktion der Menge $\left[  \frac{i-1}{n},
      \frac{i}{n} \right]$, $i = 1, \ldots, n$.

    Dann konvergiert die Folge $X_1^1, X_2^1, X_2^2, X_3^1, X_3^2, X_3^3,
    \ldots$ in Wahrscheinlichkeit gegen 0, aber diese Folge ist in keinem Punkt
    konvergent.
  \item $(\Omega, \mA, \pP)$ wie oben,
    \[ X_1, X_3, X_5, \ldots := \ind_{[0,\rez{2}]}, \quad X_2, X_4, X_6,
      \ldots := \ind_{[\rez{2}, 1]} \]
    sowie
    \[ F(x) := F_n(x) =
      \begin{cases}
        0, & x \le 0 \\
        \rez{2}, & 0 < x \le 1, \\
        1, & 1 < x.
      \end{cases}
    \]
    Dann gilt $X_n \to X_1$ und $X_n \to X_2$ in Verteilung, aber $\{X_n\}$ ist
    nicht konvergent in Wahrscheinlichkeit.
  \end{enumerate}
\end{exmp}

\clearpage

\begin{thm}
  Für eine Folge $\{ X_n \}$ von Zufallsgrößen sind die folgenden Aussagen
  äquivalent:
  \begin{enumerate}[(i)]
  \item $\{X_n\}$ konvergiert in Wahrscheinlichkeit gegen eine Zufallsgröße
    $X$,
  \item $\lim_{n \to \infty}  \sup_{m \ge n} \pP [ |X_m - X_n | > \eps ] = 0$
    für alle $\eps > 0$,
  \item Es gibt eine Zufallsgröße $X$, so dass eine beliebige Teilfolge von
    $\{X_n\}$ eine Teilfolge besitzt, die \emph{fast sicher} gegen $X$
    konvergiert.
  \end{enumerate}
\end{thm}

Beweis im Netz.

\begin{thm}
  Für eine Folge $\{X_n\}$ von Zufallsgrößen sind die folgenden Aussagen
  äquivalent:
  \begin{enumerate}[(i)]
  \item $\{X_n\}$  konvergiert fast sicher gegen eine Zufallsgröße $X$.
  \item Es gilt
    \[ \pP\left[ |X_n - X| > \rez{k}, \text{unendlich oft} \right] = 0. \]
  \item Es gilt
    \begin{align*}
      \lim_{n \to \infty} \pP \left[ \sup_{m \ge n} |X_m - X| > \eps \right]
      &= 0, \quad \eps > 0.
    \end{align*}
  \item Es gilt
    \begin{align*}
      \lim_{n \to \infty} \pP \left[ \sup_{m \ge n} |X_m - X_n| > \eps \right]
      &= 0, \quad \eps > 0.
    \end{align*}
  \end{enumerate}
\end{thm}

Beweis im Netz.

\begin{thm}[Levy]
  Für eine Folge $\{X_n\}$ von \emph{unabhängigen} Zufallsgrößen konvergiert die
  Folge $S_n = X_1 + \cdots + X_n$ genau dann fast sicher, wenn sie in
  Wahrscheinlichkeit konvergiert.
\end{thm}

\begin{proof}
  ``$\Rightarrow$'': Bereits bewiesen.

  ``$\Leftarrow$'' Nehmen wir an, dass $S_n$ in Wahrscheinlichkeit konvergiert.
  Sei $S_{h,n} := S_n - S_h$, $1 \le h \le n$.

  Wegen Satz 2.5.5.(ii) existiert für jedes $\eps \in (0, \rez{2})$ ein $h_0$,
  sodass
  \[ \pP [ |S_{h,n}| \ge \eps ] < \eps, \quad n \ge h \ge h_0. \tag{1} \]
  Es gilt\footnote{%
    Aus $\pP[|X| \ge c] < \rez{2}$ folgt für den Median $|m(X)| < c$.}
  $|m(S_{h,n})| < \eps$.

  Für $k > h \ge h_0$ gilt
  \begin{align*}
    \pP \left[ \max_{h < n \le k} |S_{h,n}| \ge 2 \cdot \eps \right]
    &\le \pP \Big[ \max_{h < n \le k} |S_{h,n} - m(\underbrace{S_{h,n} -
      S_{h,k}}_{= S_{n,k}}) | \ge \eps \Big] \\
    &\overset{\text{(L)}}{\le} 2 \cdot \pP[ |S_{h,k}| \ge \eps ] \overset{\text{(1)}}{=} 2 \cdot \eps,
  \end{align*}
  wobei (L) die Ungleichung von Levy 2.3.8.(ii) bezeichnet. Also folgt
  \[ \pP \left[ \sup_{n > h} |S_{h,n}| \ge 2 \cdot \eps  \right] \le 2 \cdot
    \eps, \quad h \ge h_0\]
  und nach Satz 2.5.6 konvergiert $S_n$ fast sicher gegen eine Zufallsgröße $S$.
\end{proof}

Im restlichen Teil dieses Abschnitts ist $X_1, X_2, \ldots$ eine Folge
unabhängiger Zufallsgrößen und $S_n = X_1 + \ldots + X_n$.

\begin{thm}
  Wenn $\sum_{n=1}^\infty D^2(X_n) < \infty$ und $\pE(X_n)=0$, $n \in \nat$,
  dann konvergiert die Reihe $\sum_{n=1}^\infty X_n$ fast sicher.
\end{thm}

\begin{proof}
  Für $n < m$ gilt
  \[ \pE(S_m - S_n) = \pE( X_{n+1} + \ldots + X_m) = 0, \]
  also
  \begin{align*}
    \pP[ |S_m - S_n| \ge \eps ]
    &\overset{\text{(T)}}{\le} \rez{\eps^2} D^2(S_m - S_n) \\
    &= \rez{\eps^2} \sum_{i=n + 1}^m D^2(X_i) \xrightarrow{m,n \to \infty} 0,
  \end{align*}
  wobei (T) für die Ungleichung von Tschebischew 2.3.3 steht. Der Grenzwert 0
  folgt aus der Konvergenz der Reihe $\sum_1^\infty D^2(X_i)$.

  Mit Satz 2.5.5 folgt, dass $S_n$ in Wahrscheinlichkeit konvergiert und mit
  Satz 2.5.7 folgt die fast sichere Konvergenz.
\end{proof}

\begin{thm}[Jegorow]
  Es seien $Y, Y_n$ Zufallsgrößen ($n \in \nat$), $A \in \mA$, $\pP(A) > 0$ und
  für $\omega \in A$ gelte
  \[ \lim_{n \to \infty} Y_n(\omega) = Y(\omega). \]
  Dann existiert für jedes $\eps > 0$ eine Menge $F \in \mA$, sodass für $F
  \subset A$
  \[ \pP(F) > \pP(A) - \eps \]
  und die Folge $\{Y_n\}$ konvergiert auf $F$ \emph{gleichmäßig} gegen $Y$.
\end{thm}

\begin{proof}
  Das ist bereits aus der Maßtheorie bekannt.
\end{proof}

\begin{folg}
  Wenn $S_n \to S$ fast sicher und $|X_n| \le c$, $n \in \nat$ für ein $c \in
  \real$, dann existiert ein $d > 0$, sodass die Wahrscheinlichkeit von
  \[ E := \bigcap_{n=1}^\infty \{ \omega : |S_n(\omega)| \le d \}\]
  positiv ist.
\end{folg}

\begin{proof}
  Es existiert $k \in \real$, so dass $\pP[|S| \le k] > 0$. Sei $A := [ |S| \le
  k ]$.

  Wegen Satz 2.5.9 existiert $F \subset A$, so dass $\pP(F) > 0$. Weil $S_n \to
  S$ \emph{gleichmäßig} auf $F$, existiert ein $n_0$, so dass
  \[ |S_n(\omega) - S(\omega)| \le 1, \quad n \ge n_o, \quad \omega \in F. \]
  Also folgt
  \[ |S_n(\omega)| \le |S(\omega)| + 1 \le k + 1, \quad n \ge n_0, \quad \omega
    \in F.\]
  Andererseits ist
  \[ |S_n| = | X_1 + \ldots + X_n| \le n \cdot c < n_0 \cdot c, \quad n <
    n_0. \]
  Also gilt
  \[ |S_n(\omega)| \le d := \max\{ k+1, n_0 \cdot c\} \]
  für alle $n$, $\omega \in F$.
\end{proof}

\begin{thm}
  Gilt $|X_n| \le c$ für ein $c \in \real$, so konvergiert die Reihe
  $\sum_1^\infty X_n$ genau dann fast sicher, wenn
  \begin{enumerate}[(i)]
  \item $\sum_1^\infty \pE(X_n) < \infty$,
  \item $\sum_1^\infty D^2(X_n) < \infty$.
  \end{enumerate}
\end{thm}

\begin{proof}
  Nehmen wir an, dass (i) und (ii) gelten und sei
  \[ Y_n := X_n - \pE(X_n). \]
  Dann ist
  \[ \pE( Y_n ) = 0, \quad D^2(Y_n) = D^2(X_n). \]
  Nach Satz 2.5.8 konvergiert $\sum_1^\infty Y_n$ fast sicher und damit auch
  $\sum_1^\infty X_n$ wegen (i), weil $X_n = Y_n - \pE(X_n)$.

  Wir zeigen noch die andere Richtung: Nehmen wir an, dass $\sum_1^\infty X_n$
  fast sicher konvergiert und zeigen, dass (i) und (ii) gelten.

  \textbf{Spezialfall.} $\pE(X_n) = 0$ (dann ist (i) erfüllt). Sei $S_0 := 0$,
  $E$ und $d$ wie in 2.5.10,
  \[ E_n := \bigcap_{i=0}^n [|S_i| \le d], \quad n = 0, 1, 2, \ldots \]
  Dann gilt $E_0 \subset E_1 \subset \ldots$ und $E = \bigcap_{n=0}^\infty E_n$.
  Sei $F_n := E_{n-1} \setminus E_n$ und  $a_n := \int_{E_n} S_n^2 \diffop \pP$,
  $n = 0, 1, 2, \ldots$
  \begin{align*}
    a_n - a_{n-1}
    &= \left( \int_{E_{n-1}} S_n^2 \diffop \pP - \int_{F_n} S_n^2 \diffop \pP \right)
      - \int_{E_{n-1}} S_{n-1}^2 \diffop \pP \\
    &= \int_{E_{n-1}} X_n^2 \diffop \pP + 2 \cdot \int_{E_{n-1}} X_n S_{n-1} \diffop \pP
      - \int_{F_n} S_n^2 \diffop \pP,
  \end{align*}
  da $S_n^2 = (S_{n-1} + X_n)^2 = S_{n-1}^2 + 2 \cdot X_n \cdot S_{n-1} +
  X_n^2$.

  Es gilt
  \[ \int_{E_{n-1}} X_n^2 \diffop \pP = \pE( \ind_{E_{n-1}} \cdot X_n^2) = \pE(
    \ind_{E_{n-1}}) \cdot \pE(X_n^2) = \pP( E_{n-1} ) \cdot D^2(X_n). \]
  Der Erwartungswert darf getrennt werden, weil die $X_i$ unabhängig sind.

  Es gilt
  \begin{align*}
    \int_{E_{n-1}} X_n \cdot S_{n-1} \diffop \pP
    &= \pE( X_n \cdot ( \ind_{E_{n-1}} \cdot S_{n-1})) \\
    &= \pE( X_n ) \cdot \pE( \ind_{E_{n-1}} \cdot S_{n-1}) = 0
  \end{align*}
  sowie
  \[ \int_{F_n} S_n^2 \diffop \pP \le (d+c)^2 \pP(F_n), \]
  da $|S_n| = |S_{n+1} + X_n| \le d+c$ auf $F_n$.

  Wir erhalten
  \[a_n - a_{n-1} \ge \pP(E) \cdot D^2(X_n) - (d+c)^2 \cdot \pP(F_n). \]

  Summieren von $n=1$ bis $k$:
  \[ a_k \ge \pP(E) \cdot \sum_{n=1}^k D^2 X_n - (d+c)^2. \]
  Andererseits ist $a_k \le \pP(E_k) \cdot d^2 \le d^2$, also
  \[ \underbrace{\pP(E)}_{> 0} \cdot \sum_{n=1}^k D^2 (X_n) \le d^2 + (d+c)^2. \]
  $\pP(E) > 0$ folgt mit dem Satz von Jegorow.
  
  Nach Grenzwertbildung folgt (ii).

  Damit ist der Spezialfall $\pE(X_n) = 0$ bewiesen.

  Allgemeiner Fall: Wir betrachten den $\pW$-Raum $(\Omega \times \Omega, \pP
  \times \pP, \mA \times \mA)$ und die Zufallsgrößen\footnote{%
  $\varphi(\omega_1, \omega_2) := \omega_1$, $\varphi: \Omega \times\Omega \to
  \Omega$ ist messbar bezüglich $\mA$ und $\mA \times \mA$.}
  \[ Z_n( \omega_1, \omega_2) := X_n(\omega_1) - X_n(\omega_2), \quad \omega_1,
    \omega_2 \in \Omega. \]
  Es gilt
  \[ \pE(Z_n) =
    \int_{\Omega \times \Omega} X_n(\omega_1) \diffop (\pP \times \pP) -
    \int_{\Omega \times \Omega} X_n(\omega_2) \diffop (\pP \times \pP). \]
  Wir wenden den Satz von Fubini auf das erste Integral an und es folgt
  \[ \int_\Omega
    \underbrace{\int_\Omega X_n(\omega_1) \diffop \pP(\omega_1)}_{\pE(X_n)}
    \diffop \pP(\omega_2) = \pE(X_n). \]
  Für das zweite Integral verfahren wir genauso und erhalten
  \[ \pE(Z_n) = 0. \]

  \[ D^2(Z_n) = \pE(Z_n^2) = \int_{\Omega \times \Omega} [X_n(\omega_1) -
    X_n(\omega_2)]^2 \diffop \pP \times \pP(\omega_1, \omega_2). \]
  Wir wenden wieder Fubini an und erhalten
  \[ D^2(Z_n) = 2 (\pE(X_n)^2 - \pE(X_n)^2)= 2 \cdot D^2(X_n). \]
  $|Z_n| \le 2 \cdot c$.

  Sei $E \subset \Omega$ das Ereignis, gegen das die Reihe $\sum X_n$
  konvergiert, dann ist
  \[ \pP(E) = 1 \qRq \pP \times \pP(E \times E) = \pP(E) \cdot \pP(E) = 1. \]
  Die Reihe $\sum Z_n$ konvertiert auf $E \times E$, also konvergiert sie fast
  sicher (bezogen auf das Produktmaß $\pP \times \pP$). Daraus folgt der
  Spezialfall (ii).

  Setzen wir wieder $Y_n := X_n - \pE(X_n)$. Dann ist $\pE(Y_n) = 0$,
  \[ \sum_{n=1}^\infty D^2(Y_n) = \sum_{n=1}^\infty D^2(X_n) < \infty. \]
  Mit 2.5.8 folgt, dass $\sum Y_n$ fast sicher konvergiert. Aus $\pE(X_N) = X_n
  - Y_n$ folgt nun (i).
\end{proof}

\begin{defn}
  Zwei Folgen $\{Y_n\}$ und $\{Z_n\}$ von Zufallsgrößen heißen
  \emph{äquivalent}, wenn
  \[ \sum_{n=1}^\infty \pP [Y_n \ne Z_n] < \infty \]

  Wenn $\{Y_n\}$ und $\{Z_n\}$ äquivalent sind, dann gilt:
  \[ \sum Y_n \text{ konvergiert fast sicher} \qLRq \sum Z_n \text{ konvergiert
      fast sicher}. \]

  Begründung: Aus Borel-Cantelli folgt $\pP[Y_n \ne Z_n, $ unendlich oft $] =
  0$.
\end{defn}

\clearpage

\begin{thm}[Dreireihensatz von Kolmogorow]
  Die Reihe $\sum_1^\infty X_n$ konvergiert genau dann fast sicher, wenn
  \begin{enumerate}[(i)]
  \item $\sum_1^\infty \pP[|X_n| > 1] < \infty$,
  \item $\sum_1^\infty \pE(X'_n) < \infty$
  \item $\sum_1^\infty D^2(X'_n) < \infty$
  \end{enumerate}
  wobei $X'_n(\omega) = X_n(\omega)$, wenn $|X_n(\omega)| \le 1$ und
  $X'_n(\omega) = 0$ sonst.
\end{thm}

\begin{rmrk}
  Die Konstante 1 kann hier durch eine beliebige Zahl $c > 0$ ersetzt werden.
\end{rmrk}

\begin{proof}
  Seien (i) bis (iii) erfüllt. Aus (ii) und (iii) folgt mit 2.5.11, dass $\sum
  X'_n$ fast sicher konvergiert. Aus (i) folgt, dass die Folgen $\{X_n\}$ und
  $\{X'_n\}$ äquivalent sind, also konvergiert $\sum X_n$ fast sicher.

  Nehmen wir jetzt an, dass $\sum X_n$ fast sicher konvergiert. Dann gilt $X_n
  \to 0$ fast sicher, also
  \[ \pP[ |X_n| > 1, \text{ unendlich oft}] = 0 \]
  und mit Borel-Cantelli folgt (i).

  Daraus ergibt sich die Äquivalenz von $\{X_n\}$ und $\{X'_n\}$, also
  konvergiert $\sum X'_n$ fast sicher und mit 2.5.11 folgen (ii) und (iii).
\end{proof}

\section{Aufgaben}
Siehe \verb+Aufgaben-2-6.pdf+

\section{Das starke und das schwache Gesetz der großen Zahlen}
\begin{defn}
  Eine Folge $\{ X_n \}$ integrierbarer Zufallsgrößen heißt dem \emph{starken
    Gesetz der großen Zahlen} genügend, wenn
  \[ \lim_{n \to \infty} \rez{n} \cdot \sum_{i=1}^n [X_i - \pE(X_i)] = 0\]
  fast sicher.

  Sie heißt dem \emph{schwachen Gesetz der großen Zahlen} genügend, wenn
  \[ \lim_{n \to \infty} \rez{n} \cdot \sum_{i=1}^n [X_i - \pE(X_i)] = 0\]
  in Wahrscheinlichkeit.
\end{defn}

\begin{lem}[Kronecker]
  Sei $\{a_n\}$ eine Folge reeller Zahlen und $\{t_n\}$ eine monoton wachsende,
  gegen $\infty$ strebende Folge positiver Zahlen derart, dass die Reihe
  $\sum_{k=1}^\infty \frac{a_k}{t_k}$ konvergiert. Dann gilt
  \[ \lim_{n \to \infty} \rez{n} \sum_{k=1}^n a_k = 0. \]
\end{lem}

Beweis im Netz.

\begin{exmp*}
  $a_k = \rez{k}$, $t_k = k$.
  \[ \sum_{k=1}^\infty \frac{a_k}{t_k} =\sum_{k=1}^\infty \rez{k^2} =
    \frac{\pi^2}{6}. \]
  (Bewiesen von Euler);
  \[ \lim_{n \to \infty} \rez{n} \sum_{k=1}^n \rez{k} = 0 \]
\end{exmp*}

\begin{thm}[Kolmogorow]
  Es sei $X_1, X_2, \ldots \in \intf^2(\pP)$ eine Folge unabhängiger
  Zufallsgrößen. Gilt dann
  \[ \sum_1^\infty \frac{D^2(X_n)}{n^2} < \infty, \tag{1} \]
  so genügt die Folge $\{X_n\}$ dem starken Gesetz der großen Zahlen.
\end{thm}

\begin{proof}
  Betrachte $Y_n := \rez{n} \sum_{i=1}^n (X_i - \pE(X_i))$. Dann gilt
  \begin{align*}
    \pP \left[ \sup_{m \le i \le n} |Y_i| \ge \eps \right]
    &\overset{\text{2.3.1}}{\le} \rez{\eps^2}
      \left[ \rez{m^2} \sum_{j=1}^m D^2(X_j) + \sum_{j=m+1}^n \frac{D^2(X_j)}{j^2} \right] \\
    \intertext{Grenzwertbildung $n \to \infty$} \\
    \pP \left[ \sup_{m \le i} |Y_i| \ge \eps \right]
    &\le \rez{\eps^2}
      \left[ \rez{m^2} \sum_{j=1}^m D^2(X_j) + \sum_{j=m+1}^\infty \frac{D^2(X_j)}{j^2} \right].
  \end{align*}
  Wir bilden auch noch den Grenzwrt $m \to \infty$ und erhalten
  \[ \lim_{m \to \infty} \pP \left[ \sup_{i \ge m} |Y_i| \ge \eps \right]
    \overset{\text{2.7.2 und Vorr.}}{=} 0 \]
  für alle $\eps > 0$. Also folgt mit 2.5.6, dass $Y_n \to 0$ fast sicher.
\end{proof}

\begin{exmp}
  Wir betrachten eine Folge von Versuchen. In jedem dieser Versuche tritt ein
  gewisses Ereignis $A$ mit Wahrscheinlichkeit $p$ unabhängig von den Ausgängen
  der anderen Versuche ein.
  \[ X_k := \begin{cases} 1,
      & \text{wenn $A$ im $k$-ten Versuch eingetreten  ist,}
      \\ 0, & \text{sonst.}
    \end{cases} \]
  Dann ist $\{ X_k \}$ eine Folge von unabhängigen Zufallsgrößen. $\pP[X_k =
  1] = p$, $\pP[X_k = 0] = p - 1$, $\pE X_k = p$, $D^2(X_k) = p(1-p)$.

  $S_n := X_1 + \ldots + X_n$ ist die Anzahl des Eintretens von $A$ in den
  ersten $n$ Versuchen. $S_n$ ist binomialverteilt, $\pE(S_n) = n \cdot p$, $D^2
  (S_n) = n \cdot p(1-p)$.

  $\frac{S_n}{n}$ ist die relative Häufigkeit von $A$ in $n$ Versuchen. Aus 2.7.
  folgt:
\end{exmp}

\begin{thm}[Borel]
  Es gilt
  \[ \lim_{n \to \infty} \frac{S_n}{n} = p \]
  fast sicher.
\end{thm}

\begin{rmrk*}
  Tschebischew:
  \[ \pP \left[ \left| \frac{S_n}{n} - p \right| \ge \eps \right] \le
    \frac{pq}{n \eps^2}. \]
  Zum Beispiel $p = q = \rez{2}$, $\eps = \rez{20}$:
  \[ \pP \left[ \left| \frac{S_n}{n} - \rez{2} \right| \ge \rez{20} \right] \le
    100 \tag{1} \]
  wenn $n \ge 10000$.

  Renyi:
  \[ \pP \left[ \left| \frac{S_n}{n} - p \right| \ge \eps \right] \le 2 \cdot
    \exp \left( - \frac{n \eps^2}{2 pq \left(1 + \frac{\eps}{2pq} \right)^2} \right).
    \tag{2} \]

  Aus (2) folgt: (1) gilt, wenn $n \ge 1283$.
\end{rmrk*}

\begin{thm}[Etemadi]
  Jede Folge paarweise unabhängiger, integrierbarer und identisch verteilter
  Zufallsgrößen genügt dem starken Gesetz der großen Zahlen.
\end{thm}

\begin{thm}[Khinchin]
  Gilt für eine Folge paarweise unkorrelierte Zufallsgrößen $X_n \in
  \intf^2(\pP)$
  \[ \lim_{n \to \infty} \rez{n^2} \sum_{i=1}^n D^2(X_i) = 0, \]
  so genügt die Folge dem schwachen Gesetz der großen Zahlen.
\end{thm}

\begin{proof}
  Betrachte
  \[ S_n := \sum_{k=1}^n (X_k - \pE(X_k)), \quad D^2(S_n) = \sum_{k=1}^n
    D^2(X_k). \]
  Es gilt
  \[ \pP \left[ \frac{S_n}{n} \ge \eps \right]
    \overset{\text{(T)}} \le \frac{D^2 \left(  \frac{S_n}{n} \right)}{\eps^2}
    = \rez{\eps^2 n^2}  \sum_{i = 1}^n D^2(X_i) \xrightarrow{n \to \infty} 0, \]
  wobei (T) für die Ungleichung von Tschebischew steht.
\end{proof}

\section{Aufgaben}
Siehe \verb+Aufgaben-2-8.pdf+

\section{Satz von Gliwenko-Cantelli}
\begin{defn}
  Es seien $X_1, \ldots, X_n$ identisch verteilte, unabhängige Zufallsgrößen mit
  Verteilungsfunktion $F$ und $\omega \in \Omega$. Wir ordnen die Zahlen
  $X_1(\omega), \ldots, X_n(\omega)$ um:
  \[ X_1^*(\omega) \le X_2^*(\omega) \le \ldots \le X_n^*(\omega). \]

  Definiere die \emph{empirische Verteilungsfunktion}
  \[ F_n^\omega(x) =
    \begin{cases}
      0, & \text{falls } x \le X_1^*(\omega), \\
      \frac{k}{n}, & \text{falls } X_k^*(\omega) < x \le X_{k+1}^*(\omega), \\
      1, & \text{falls } x > X_n^*(\omega). \\
    \end{cases}
  \]
\end{defn}

\begin{thm}[Gliwenko-Cantelli]
  Es gilt
  \[ \lim_{n \to \infty} \sup_{x \in \real} | F_n^\omega(x) - F(x) | = 0 \]
  fast sicher.
\end{thm}

\begin{prgp}[Zusatzaufgabe]
  \begin{enumerate}[(a)]
  \item Man berechne $\pP[ F_n^\omega(x) = k/n ]$, $0 \le k \le n$ mit Hilfe von
    $F$.
  \item Man zeige:
    \begin{align*}
      \lim_{n \to \infty} F_n^\omega(x)
      &= F(x) \quad \text{fast sicher, für alle } x \in \real, \\
      \lim_{n \to \infty} F_n^\omega(x+0)
      &= F(x+0) \quad \text{fast sicher, für
        alle } x \in \real.
    \end{align*}
    Hinweis: $Y_n^* := \ind_{[X_n < x]}$, $Z_n^* := \ind_{[X_n \le x]}$.

    Dann gilt
    \begin{align*}
      F_n^\omega(x) &= \rez{n} \sum^n Y_i^* \to F(x) \text{ fast sicher,} \tag{$\ast$} \\
      F_n^\omega(x+0) &= \rez{n} \sum^n Z_i^* \to F(x+0) \text{ fast sicher.} \tag{$\ast\ast$}
    \end{align*}
  \item Man beweise Satz 2.9.2.

    Hinweis: Es bleibt nur noch zu zeigen, dass aus ($\ast$) und ($\ast\ast$)
    die gleichmäßige Konvergenz folgt. Dazu sei für jedes $k \in \nat$ und $1
    \le j < k$ die kleinste Zahl $x_{jk}$ mit $\frac{j}{k} \le F( x_{jk} + 0)$
    und es sei $x_{0k} := - \infty$, $x_{kk} := \infty$.

    Dann ist
    \[ F(x_{jk}) \le \frac{j}{k} \qRq F(x_{j+1,k}) - F(x_{jk} + 0) \le \rez{k}.
      \tag{1} \]
    Daher gilt für jedes $x \in (x_{jk}, x_{j+1,k})$:
    \begin{align*}
      F_n^\omega(x_{jk}) - F(x_{jk}+0) - \rez{k}
      &\overset{\text{(1)}}{\le} F_n^\omega (X_{jk} + 0) - F(x_{j+1,k}) \\
      &\le F_n^\omega(x_{jk}) - F(x) \\
      &\le F_n^\omega(x_{j+1,k}) -F(x_{jk} + 0) \\
      &\overset{\text{(1)}}{\le} F_n^\omega( x_{j+1,k} ) - F(x_{j+1,k}) + \rez{k}.
    \end{align*}
    Daraus folgt
    \begin{align*}
      \sup_{x \in \real} | F_n^\omega(x) - F(x) |
      & \le \max_{1 \le j < k} | F_n^\omega(x_{jk}) - F(x_{jk}) |
        + \max_{0 \le j < k} | F_n^\omega(x_{jk} + 0) - F(x_{jk} + 0) | + \rez{k} \\
      &\xrightarrow{\text{(ii)}} \rez{k} \text{ fast sicher für } n \to \infty.
    \end{align*}
  \end{enumerate}
\end{prgp}

Ab hier ist $F$ stetig.
\begin{thm}[Smirnow]
  Für $y > 0$ gilt
  \[\lim_{n \to \infty} \pP \left[ \sqrt{n}
      \sup_{x \in \real} (F_n^\omega(x) - F(x)) < y \right]
    = 1 - e^{-2y^2}. \]
\end{thm}

\begin{thm}[Kolmogorow]
  Es gilt
  \[ \lim_{n \to \infty} \pP \left[ \sqrt{n}
      \sup_{x \in \real} | F_n^\omega(x) - F(x) | < y \right]
    = \sum_{k = -\infty}^\infty (-1)^k 1 - e^{-2 k^2 y^2} =: K(y). \]
\end{thm}

\section{Bedingte Erwartung}
In diesem Abschnitt: $(\Omega, \mA, \pP)$ ein $\pW$-Raum.

Ist $\mB \subset \mA$ eine $\sigma$-Algebra, so bezeichnet $\pP|_\mB$ die
Einschränkung ovn $\pP$ auf $\mB$.

\begin{prgp}[Aufgabe]
  Sei $\mB \subset \mA$ eine $\sigma$-Algebra. Eine $\mB$-messbare Zufallsgröße
  $X$ (reell oder komplex) ist genau dann $\pP$-integrierbar, wenn sie
  $\pP|_\mB$-integrierbar ist. Im Fall der Integrierbarkeit gilt
  \[ \int_B X \diffop \pP = \int_B X \diffop \pP|_\mB, \quad B \in \mB. \]
\end{prgp}

\begin{thm}
  Sei $\mB \subset \mA$ eine $\sigma$-Algebra und $X$ eine $\pP$-integrierbare
  Zufallsgröße. Dann existiert eine $\mB$-messbare $\pP$-integrierbare
  Zufallsgröße $Z$ mit
  \[ \int_B X \diffop \pP = \int_B Z \diffop \pP|_\mB,  \quad B \in \mB.
    \tag{1} \]
  Diese Zufallsgröße ist bis auf $\pP$-fast sichere Gleichheit eindeutig
  bestimmt.
\end{thm}

\begin{proof}
  Wir definieren das Maß $\mu$ auf $\mB$ durch
  \[ \mu( B) := \int_B X \diffop \pP, \quad B \in \mB. \]
  Ist $B$ eine $\pP|_\mB$-Nullmenge, so ist sie auch eine $\pP$-Nullmenge und
  folglich ist $\mu(B) = 0$. Also ist $\mu$ absolut stetig bezüglich $\pP|_\mB$.

  Die Existenz und Eindeutigkeit von $Z$ folgt aus dem Satz von Radon-Nikodim
  (und aus (2.10.1)).
\end{proof}

\begin{defn}
  Die Zufallsgröße in 2.10.2 heißt \emph{bedingte Erwartung} von $X$ \emph{unter
    der Bedingung} oder \emph{bezüglich} $\mB$ und wird mit $\pE(X|\mB)$ oder
  $\pE_\mB(X)$ oder $\pE^\mB(X)$ bezeichnet.

  Falls $\mB = \sigma( Y_1, \ldots, Y_n)$, wobei die $Y_j$ Zufallsvariablen
  sind, schreiben wir auch $\pE(X|_{Y_1, \ldots, Y_n})$.

  Ist $X = \ind_A$, $A \in \mA$, so heißt $\pE(X|\mB)$ die \emph{bedingte
    Wahrscheinlichkeit} von $A$ unter der Bedingung oder bezüglich $\mB$ und
  wird mit $\pP(A|\mB)$ oder $\pP_\mB(A)$ oder $\pP^\mB(A)$ bezeichnet.
\end{defn}

\begin{rmrk}
  \begin{enumerate}[(a)]
  \item Zwei Extremfälle für $\mB$: $\mB = \mA$ und $\mB = \{ \emptyset, \Omega
    \}$.

    Wenn $\mB = \mA$: $\pE_\mB X = X$.

    Wenn $\mB = \{ \emptyset, \Omega \}$: $\pE_\mB X = \pE X$.
  \item Extremfall für $X$: $X$ ist $\mB$-messbar, dann ist $\pE_\mB X = X$.
  \item Extremfall in (1) mit $B = \Omega$: Die Zufallsgröße $X$ und $pE_\mB(X)$
    haben denselben Erwartungswert, $\pE X = \pE \pE_\mB X$.
  \item $X = Y$ fast sicher $\Rightarrow$ $\pE(X|\mB) = \pE(Y|\mB)$ fast sicher.
  \end{enumerate}
\end{rmrk}

\begin{prgp}[Aufgabe]
  \begin{enumerate}[(a)]
  \item Wird $\mB$ von disjunkten Mengen $B_1, B_2, \ldots \in \mA$ mit
    $\bigcup_n B_n = \Omega$ erzeugt, so ist jede $\mB$-messbare Zufallsgröße
    konstant auf den Mengen $B_n$. Beschreiben Sie die $\sigma$-Algebra $\mB$.
  \item Sei $\mB$ wie in (a), dann gilt (fast sicher):
    \[ \pP(A|\mB) = \pE( \ind_A | \mB ) = \sum_{n : \pP(B_n) > 0} \pP(A | B_n)
      \cdot \ind_{B_n} \]
    und
    \[ \pE(X|\mB) = \sum_{} \pE_{B_n} X \cdot \ind_{B_n},\]
    wobei $X$ eine integrierbare Zufallsgröße ist und
    \[ \pE_{B_n}(X) = \rez{\pP(B_n)} \int_{B_n} X \diffop \pP. \]

    Spezialfall: $B_1 = B, B_2 = B^C$.
    
    Hinweis: Verwenden Sie (a).
  \item  Für jedes Ereignis $A \in \mA$ gilt
    \[ \int_B \pP( A | \mB ) \diffop \pP = \pP(A \cap B), \quad B \in \mB. \]
    Hinweis:
    \[ \int_B \ind_A \diffop \pP = \pP(A \cap B), \]
    siehe 2.10.2.1.
  \end{enumerate}
\end{prgp}

\begin{thm}
  Sei $\mB \subset \mA$ eine $\sigma$-Algebra, sowie $X$ und $Y$ integrierbare
  Zufallsgrößen auf $(\Omega, \mA, \pP)$. Die folgenden Eigenschaften bestehen
  fast sicher:
  \begin{enumerate}[(i)]
  \item $\pE_\mB( aX + bY ) = a \cdot \pE_\mB(X) + b \cdot \pE_\mB(Y)$, $a,b \in
    \real$.
  \item Wenn $X \ge Y$, dann auch $\pE_\mB(X) \ge \pE_\mB(Y)$; speziell gilt
    $\pE_\mB(x) \ge 0$, wenn $X \ge 0$.
  \item Ist $X$ $\mB$-messbar, so gilt $\pE_\mB(X) = X$; speziell gilt
    \[ \pE_\mB( a \cdot \ind_\Omega) = a \cdot \ind_\Omega \quad \text{und}
      \quad \pE_\mB[ \pE_\mB(X) ] = \pE_\mB(X). \]
  \item $|\pE_\mB(X)| \le \pE_\mB(|X|)$ und $\|\pE_\mB(X)\|_1 \le \|X\|_1.\footnotemark$
  \end{enumerate}
\end{thm}
\footnotetext{Das heißt, die Abbildung $X \to \pE_\mB(X)$ ist ein linearer
  Operator von $L^1(\pP)$ in $L^1(\pP)$. Dieser Operator ist nichtnegativ,
  kontraktiv und idempotent, die $\mB$-messbaren Zufallsgrößen sind Fixpunkte.}

\begin{proof}
  (i) bis (iii) folgen sofort aus der Definition des bedingten Erwartungswertes.

  Kontraktivität: Linear und nichtnegativ.
  \[ - |X| \le X \le |X| \qRq |\pE_\mB(X)| \le \pE(|X|) \]
  Aus der Definition von $\pE_\mB$ (D) folgt
  \begin{align*}
  \|X\|_1 &= \int_\Omega |X| \diffop \pP \overset{\text{(D)}}{=} \int_\Omega
            \pE_\mB(|X|) \diffop \pP \\
          &\ge \int_\Omega |\pE_\mB(X)| \diffop \pP|_\mB = \| \pE_\mB(X) \|_1. \qedhere
  \end{align*}
\end{proof}

\begin{exmp*}
  Sei $(\Omega, \mA, \pP) := ([0,1], \borel([0,1]), \lambda)$, $X(\omega) :=
  \omega$, $\omega \in [0,1]$.
  \begin{itemize}
  \item $\mB_1 := \{ \emptyset, \Omega \}$
  \item $\mB_2 := \{ \emptyset, [0,1/2], (1/2,1], [0,1] \}$
  \end{itemize}
\end{exmp*}

\begin{thm}
  Seien $\mB, \mB_1, \mB_2$ Teil-$\sigma$-Algebren von $\mA$, sowie $X$, $X_n$
  und $Y$ integrierbare Zufallsgrößen sowie $Z$ eine beschränkte Zufallsgröße
  auf $(\Omega, \mA, \pP)$. Die folgenden Eigenschaften bestehen fast sicher:
  \begin{enumerate}[(i)]
  \item Ist $Z$ messbar bezüglich $\mB$, so gilt
    \[ \pE_\mB( X \cdot Z ) = \pE_\mB(X) \cdot Z, \]
    speziell
    \[ \pE_\mB( X \cdot \pE_\mB(Y)) = \pE_\mB(X) \cdot \pE_\mB(Y), \]
    falls $\pE_\mB(Y)$ beschränkt ist.
  \item Ist $\mB_1 \subset \mB_2$, so gilt
    \[ \pE_{\mB_1}( \pE_{\mB_2} (X) ) = \pE_{\mB_2}( \pE_{\mB_1}(X) ) =
      \pE_{\mB_1}(X). \]
  \item Sind $B$ und $\sigma(X)$ unabhängig, so gilt
    \[ \pE_\mB(X) = \pE(X). \]
  \item Gilt $X_n \to X$ und $|X_n| \le Y$ für alle $n$, dann folgt
    \[ \pE_\mB(X_n) \to \pE_\mB(X). \]
  \end{enumerate}
\end{thm}

\begin{proof}
  (i): Es gilt für $B \in \mB$:
  \[ \int_B \pE_\mB(X \cdot Z) \diffop \pP|_\mB = \int_B X \cdot Z \diffop \pP
    = \int_B Z \cdot \pE_\mB(X) \diffop \pP|_\mB. \]
  Spezialfall: $Z = \ind_A$, $A \in \mB$. Die zweite Gleichung hat die Form
  \[ \int_{A \cap B} X \diffop \pP = \int_{A \cap B} \pE_\mB(X) \diffop
    \pP|_\mB \]
  und sie ist wegen $A \cap B \in \mB$ richtig.

  Der allgemeine Fall folgt durch Approximation von $Z$ durch Treppenfunktionen.

  (ii): Die Gleichung $\pE_{\mB_2}( \pE_{\mB_1}(X) ) = \pE_{\mB_1}(X)$ folgt aus
  (2.10.6.iii), da $\pE_{\mB_1}(X)$ messbar ist bezüglich $\mB_2$. Für $B \in
  \mB_1 \subset \mB_2$ gilt:
  \[ \int_B \pE_{\mB_1}(\pE_{\mB_2}(X)) \diffop \pP =
    \int_B \pE_{\mB_2}(X) \diffop \pP = \int_B X \diffop \pP = \int_B
    \pE_{\mB_1}(X) \diffop \pP. \]
  Der erste und der letzte Integrand sind $\mB_1$-messbar, ihre Integrale sind
  für alle $B \in \mB_1$ gleich, also sind sie fast sicher gleich.

  (iii): Annahme, $X$ und $\ind_B$ sind unabhängige Zufallsgrößen, dann gilt
  \begin{align*}
    \int_B \pE(X) \diffop \pP
    &= \pE(X) \cdot \pP(B) = \pE(X) \cdot \pE(\ind_B) \\
    &= \pE(X \cdot \ind_B) = \int_B X \diffop \pP = \int_B \pE_\mB(X) \diffop \pP,
  \end{align*}
  also sind die Integranden gleich.

  (iv): Nach (2.10.6.iv) ist
  \[ |\pE_\mB(X_n) - \pE_\mB(X)| \le \pE_\mB(|X_n - X|) \le \pE_\mB(Y_n), \]
  wobei $Y_n := \sup_{k \ge n} |X_n - X|$.

  Wir zeigen, dass die rechte Seite
  $\xrightarrow{n \to \infty} 0$. Es gilt $0 \le Y_n \le 2Y$ und $Y_n\to 0$. Die
  Folge $\{ Y_n \}$ und damit auch $\{\pE_\mB(Y_n)\}$ sind monoton fallend. Also
  \[ 0 \le \int \lim_n \pE_\mB(Y_n) \diffop \pP = \lim_n \int \pE_\mB(Y_n)
    \diffop \pP = \lim_n \int Y_n \diffop \pP = \int \lim_n Y_n \diffop \pP =
    0. \qedhere \]
\end{proof}

\begin{thm}[Faktorisierungslemma]
  Sei $T: \Omega \to \Omega'$ eine Abbildung in einem Messraum $\Omega', \mA')$
  und sei $X : \Omega \to \realext$ eine Zufallsgröße. Dann ist $X$ genau
  $\sigma(T)$-messbar, wenn es eine Zufallsgröße $Y : \Omega' \to \realext$ gibt
  mit $X = Y(T)$.
\end{thm}

\begin{proof}
  Siehe Bauer, 1990, S. 71.
\end{proof}

\begin{folg}
  Für beliebige Zufallsgrößen $X, Y_1, \ldots, Y_n$ existiert eine
  Borel-messbare Funktion $g : \real^n \to \real$, sodass
  \[ \pE( X | Y_1, \ldots, Y_n) = g(Y_1, \ldots, Y_n) \]
  fast sicher.
\end{folg}

\clearpage

\begin{thm}
  Es seien $X$ eine integrierbare Zufallsgröße, $Y$ eine beliebige Zufallsgröße
  und $g$ die nach (2.10.9) existierende, Borel-messbare Funktion mit
  \[ \pE( X | Y )(\omega) = g(Y(\omega)) \quad \text{fast sicher.} \tag{1} \]
  Dann ist $g$ bezüglich der Verteilung $\mu_Y$ von $Y$ integrierbar und genügt
  der Gleichung
  \[ \int_B g \diffop \mu_Y = \int_{[Y \in B]} X \diffop \pP \tag{2} \]
  für jede Borel-Menge $B \subset \real$. Sie ist hierdurch $\mu_Y$-fast sicher
  bestimmt.

  Ist umgekehrt $g$ eine reelle, Borel-messbare, $\mu_Y$-integrierbare Funktion
  mit (2), so gilt (1).
\end{thm}

\begin{proof}
  Wegen (1) ist die Funktion $g(Y)$ integrierbar. Nach dem Transformationssatz
  gilt
  \[ \int_B g \diffop \mu_Y = \int_{[Y \in B]} g(Y) \diffop \pP, \quad B \in
    \borel(\real). \tag{$\circ$} \]
  Aus (1) und $[Y \in B] \in \sigma(Y)$ folgt, dass
  \[ \int_{[Y \in B]} g(Y) \diffop \pP = \int_{[Y \in B]} X \diffop \pP \]
  und damit (2).

  Die Eindeutigkeit folgt daraus, dass die Dichtefunktion bezüglich $\mu_Y$ des
  Maßes $\nu := g \diffop \mu_Y$ $\mu_Y$-fast sicher bestimmt.

  Gilt nun (2) mit eine Funktion $g$, so folgt aus ($\circ$), dass
  \[ \int_{[Y \in B]} X \diffop \pP = \int_{[Y \in B]} g(Y) \diffop \pP \]
  und damit (1).
\end{proof}

\begin{prgp}[Aufgabe]
  Für die Umkehrung im vorhergehenden Satz genügt es zu fordern, dass
  \[ \int_{-\infty}^x g \diffop \mu_Y = \int_[Y \le X] X \diffop \pP, \quad x
    \in \real. \]
\end{prgp}

\begin{defn}
  Seien $X$, $Y$ und $g$ wie in (2.10.10). Für jedes $y \in \real$ heißt $g(y)$
  die \emph{bedingte Erwartung von $X$} unter der Hypothese, dass $Y$ gleich $y$
  ist, in Zeichen
  \[ \pE(X | Y = y) := g(y), \quad y \in \real. \]
\end{defn}

\begin{rmrk*}
  $\pE(X|Y)$ ist eine Zufallsgröße, $\pE(X|Y=y)$ ist eine Zahl.
\end{rmrk*}

Aus der Definition folgt:
\[ \pE(X|Y) = \pE(X | Y = Y(\omega)) \quad (\text{fast sicher}). \]

In einem für viele Anwendungen wichtigen Spezialfall kann man die Funktion $g$
und damit $\pE(X|Y = y)$ und $\pE(X|Y)$ relativ einfach berechnen.

\begin{thm}
  Es seien $X$ und $Y$ Zufallsgrößen, deren gemeinsame Verteilung $\mu_{X,Y}$
  eine Dichte $p$ besitzt. Ferner sei $X$ integrierbar und für alle $y \in
  \real$ sei die Dichte von $Y$
  \[ p_Y (y) := \int_{-\infty}^\infty p(x,y) \diffop x > 0. \]

  Dann gilt für die Funktion $g(y) = \pE(X|Y=y)$
  \[ \pE(X|Y=y) = \rez{p_Y(y)} \int_{-\infty}^\infty x \cdot p(x,y) \diffop x
    \tag{1} \]
  für $\mu_Y$-fast alle $y \in \real$.

  Ferner gilt
  \[ \pE(X|Y)(\omega) = \rez{p_Y(Y(\omega))} \cdot \int_{-\infty}^\infty x
    \cdot p(x,Y(\omega)) \diffop x \tag{2}\]
  $p$-fast sicher.
\end{thm}

\begin{rmrk*}
  $p(x|y) = \frac{p(x,y)}{p_Y(y)}$ heißt die \emph{bedingte Dichtefunktion unter
    der Bedingung $Y=y$}. Aus (1) folgt:
  \[ \pE( X | Y = y ) = \int_{-\infty}^\infty x \cdot p( x | y ) \diffop x \]
  $\mu_Y$-fast sicher.
\end{rmrk*}


\section{Aufgaben}
Siehe \verb+Aufgaben-2-11.pdf+.

\section{Martingale}
\begin{defn}
  Sei $I \ne \emptyset$ eine Menge (Indexmenge) versehen mit einer
  Ordnungsrelation\footnote{%
    Ordnungsrelationen sind reflexiv, antisymmetrisch und transitiv.}
  ``$\le$''.

  Eine Familie $(\mF_t)_{t \in I}$ von Unter-$\sigma$-Algebren von $\mA$ heißt
  eine \emph{Filtration} (Filterung), wenn $\mF_s \subset \mF_t$ für alle $s$,
  $t$ mit $s \le t$,
  \[ \mF_\infty := \sigma \left( \bigcup_{t \in I} \mF_t \right). \]

  \begin{rmrk*}
    Ist $(\mF_j)_{j \in \nat_0}$ eine Filtration, sodass $F_i \ne F_j$ für alle $i
    \ne j$, so ist $\bigcup_{j \in \nat_0} \mF_j$ \emph{keine} $\sigma$-Algebra.
    Der Beweis dieser Tatsache ist nicht ganz einfach.
  \end{rmrk*}

  $(\Omega, \mA, (\mF_t)_{t \in I}, \pP)$ heißt \emph{gefilterter $\pW$-Raum}.

  Eine Familie $(X_t)_{t \in I}$ von Zufallsvektoren heißt an die Filterung
  \emph{angepasst} oder \emph{adaptiert}, wenn $X_t$ für alle $t$ messbar ist
  bezüglich $\mF_t$.
\end{defn}

Im Weiteren ist hauptsächlich $I = \nat_0$ mit der üblichen Ordnung.

\textbf{Interpretation.}
Die Elemente von $I$ sind Zeitpunkte und die $\sigma$-Algebren $\mF_t$ sind
verfügbare Information. $A \in \mF_t$ bedeutet in diesem Zusammenhang, dass zum
Zeitpunkt $t$ die Frage ``Ist $\omega \in A$'' eindeutig mit ``ja'' oder
``nein'' beantwortet werden kann.

Filterung aufsteigend: Eine einmal erlangte Information geht nicht wieder
verloren.

\begin{exmp}
  \begin{enumerate}[a)]
  \item Sei $(X_t)_{t \in I}$ eine Familie von Zufallsvektoren
    (\emph{stochastischer Prozess}), so wird
    \[ \mF_t := \sigma( X_s : s \le t ) \]
    als \emph{natürliche Filterung} des Prozesses bezeichnet.
  \item Durch $\mF_t := \mA$, $t \in I$ wird die Filterung der
    \emph{vollständigen Information} definiert.
  \end{enumerate}
\end{exmp}

\begin{defn}
  Sei $(\Omega, \mA, (\mF_t)_{t \in I}, \pP)$ ein gefilterter $\pW$-Raum. Eine
  adaptierte Familie $(X_t)_{t \in I}$ von integrierbaren Zufallsvektoren heißt
  \emph{Martingal}, wenn $\pE(X_t | \mF_t) = X_s$ für $s,t \in I$ und $s \le t$.

  $\ge$: \emph{Submartingal} \\
  $\le$: \emph{Supermartingal}
\end{defn}

\begin{rmrk}
  Aus $\pE \pE( X_t | \mF_s ) = \pE X_t$ folgt, dass für Submartingale
  (Supermartingale) $\pE X_s$ monoton wächst (fällt); für Martingale ist $\pE
  X_s$ konstant.
\end{rmrk}

\begin{prgp}[Aufgabe] Mit den obigen Bezeichnungen und $I = \nat_0$ zeigen Sie:
  $(X_t)_{t \in \nat_0}$ ist genau dann ein Martingal, wenn $\pE(X_{n+1} |
  \mF_n) = X_n$, $n \in \nat_0$.

  Hinweis: (2.10.7)(ii)
\end{prgp}

\begin{exmp}
  \begin{enumerate}[a)]
  \item Sei $X_n = a_n$, wobei $\{ a_n \}$ eine monoton wachsende (fallende)
    Folge reeller Zahlen ist. Dann ist $(X_n)_{n \in \nat_0}$ ein Submartingal
    (Supermartingal).
  \item Sei $(X_n)_{n \in \nat}$ eine Folge unabhängiger, identisch verteilter,
    integrierbarer Zufallsvektoren und
    \[ S_n := X_1 + \ldots + X_n \]
    eine natürliche Filtration. Dann sind $X_{n+1}$ und $S_1, \ldots, S_n$
    unabhängig und:
    \begin{align*}
      \pE( S_{n+1} | S_1, \ldots, S_n)
      &= \pE( S_n + X_{n+1} | S_1, \ldots, S_n ) \\
      &= \pE( S_n | S_1, \ldots, S_n ) + \pE( X_{n+1} | S_1, \ldots, S_n ) \\
      &= S_n + \pE( X_{n+1} )
    \end{align*}
    Also ist $\{S_n\}$ ein Martingal (Sub-, Supermartingal), wenn $\pE
    X_n = 0$ ($\ge 0$, $\le 0$).

    Interpretation: \\
    $X_n$: Gewinn im $n$-ten Spiel \\
    $S_n$: Gesamtgewinn nach $n$ Spielen \\
    $\mF_n$: Information über die ersten $n$ Spiele \\
    $\pE(S_m | S_n)$, $m > n$: Voraussage über künftigen Gewinn \\
    $\pE(X_n) = 0$: Faires Spiel

  \item Sei $(X_n)$ eine Folge von i.i.d. Zufallsgrößen, $X_n \ge 0$ mit
    $\pE(X_n) = 1$. $P_n := X_1 \cdot \ldots \cdot X_n$ $\Rightarrow$ $\{P_n\}$
    ist ein Martingal bezüglich der natürlichen Filtration.

    $P_n$ ist integrierbar und
    \begin{align*}
      \pE( P_{n+1} | P_1, \ldots, P_n)
      &= \pE( X_{n+1} P_n | P_1, \ldots, P_n) = P_n \cdot \pE(X_{n+1} | P_1, \ldots, P_n) \\
      &= P_n \cdot \pE(X_{n+1}) = P_n.
    \end{align*}
    Allgemeiner: $\pE(X_n)$ beliebig $\Rightarrow$ Sub- oder Supermartingal.
  \item Betrachte Population aus (2.11.8)
    \[ \pE(X_{n+1} | X_1, \ldots, X_n ) = \mu X_n \]
    Dann ist $(\mu^{-n} X_n)$ ein Martingal bezüglich der natürlichen
    Filtration.
    \[ \pE( \mu^{-(n+1)} X_{n+1} | X_1, \ldots, X_n ) = \mu^{-n} X_n. \]
  \item $(\mF_n)$ sei eine Filtration und $X$ sei eine integrierbare
    Zufallsgröße. Dann ist $\mM_n := \pE( X | \mF_n )$ ein Martingal, da
    \begin{align*}
      \pE(\mM_{n+1} | \mF_n) = \pE_{\mF_n} \pE_{\mF_{n+1}} X = \pE_{\mF_n} X = \mM_n.
    \end{align*}
    Adaptierbarkeit und Interpretierbarkeit folgen aus der Definition der
    bedingten Erwartung.
  \end{enumerate}
\end{exmp}

\begin{prgp}[Aufgabe]
  \begin{enumerate}[a)]
  \item Wald'sche Martingale. Sei $(Y_n)$ eine i.i.d. Folge, $S_n := Y_1 +
    \ldots + Y_n$. Gilt für ein $t \in \real \setminus \{ 0 \}$
    \[ \mM(t) := \pE e^{t Y_1} < \infty, \]
    dann bildet $X_n := \rez{\mM_n(t)} e^{t S_n}$ ein Martingal.
  \item $Y_n$ und $S_n$ wie oben; $\pP(Y_n = 1) = \pP(Y_n = -1) = \rez{2}$, dann
    ist $X_n = S_n^2 - n$ ein Martingal.
  \end{enumerate}
\end{prgp}

\begin{prgp}[Aufgabe]
  \begin{enumerate}[i)]
  \item $(X_t)$ ist ein Martingal $\Leftrightarrow$ $(X_t)$ ist ein Sub- bzw.
    Supermartingal.
  \item $(X_t)$ ist ein Submartingal $\Leftrightarrow$ $(-X_t)$ ist ein
    Supermartingal.
  \item $(X_t)$, $(Y_t)$ sind Martingale $\Rightarrow$ $(aX_t + bY_t)$ ist ein
    Martingal.
  \item $(X_t)$, $(Y_t)$ sind Sub- oder Supermartingale $\Rightarrow$ $(aX_t +
    bY_t)$ ist ein Sub- oder Supermartingal.
  \item $(X_t)$ ist (Sub-) Martingal und $f$ eine (monoton wachsende) konvexe
    Funktion, so dass $f(X_t)$ integrierbar ist $\Rightarrow$ $f(X_t)$ ist ein
    Sub-Martingal.
  \end{enumerate}
\end{prgp}


\section{Charakteristische Funktion}
\begin{prgp}
  Unter der \emph{charakteristischen Funktion} einer Zufallsgröße $X$ versteht
  man die Funktion
  \[ f_X(t) := \pE( e^{itX}) = \pE( \cos(tX) + i \cdot \sin( tX ) ), \quad t
    \in \real. \]
  Ist $\mu$ die Verteilung von $X$, so gilt wegen (1.2.10)
  \[ f(t) = \int_\real e^{ity} \diffop \mu(y) =
    \int_\real \cos (ty) \diffop \mu(y)
    + i \cdot \int_\real \sin(ty) \diffop \mu(y). \]
  Man nennt $f$ auch die \emph{charakteristische Funktion} von $\mu$.
\end{prgp}

\begin{thm}
  Jede Verteilung ist durch ihre charakteristische Funktion eindeutig bestimmt.
\end{thm}

\begin{thm}
  Sind $X$ und $Y$ unabhängige Zufallsgrößen, so ist
  \[ f_{X+Y} = f_X \cdot f_Y, \]
  das heißt das Produkt von charakteristischen Funktionen ist wieder eine
  charakteristische Funktion.
\end{thm}

\begin{proof}
  Einfach. ($e^{itX}$ und $e^{itY}$ sind unabhängig).
\end{proof}

\begin{thm}
  Sei $Y = aX + b$, $a,b \in \real$, dann ist
  \[ f_Y(t) = e^{ibt} f_X(at). \]
\end{thm}

\begin{proof}
  Es gilt
  \begin{align*}
    f_Y(t)
    &= \pE(e^{itY)} = \pE( e^{it(aX+b)}) \\
    &= \pE( e^{itaX} \cdot \pE( e^{itb} ) )
      = e^{itb} \cdot f_X(at). \qedhere
  \end{align*}
\end{proof}

\begin{thm}
  Für jede charakteristische Funktion $f$ gilt:
  \begin{enumerate}[i)]
  \item $f(0) = 1$,
  \item $f(-t) = \obar{f(t)}$,
  \item $|f(t)| \le 1$,
  \item $f$ ist gleichmäßig stetig.
  \end{enumerate}
\end{thm}

\begin{lem}
  Die Ungleichung
  \[ \left| e^{iz} - 1 - iz - \ldots - \frac{(iz)^n}{n!} \right| \le
    \begin{cases}
      \frac{|z|^{n+1}}{(n+1)!}, &\text{wenn } \Im z \ge 0, \\
      \frac{|z|^{n+1}}{(n+1)!} e^{|z|}, &\text{wenn } \Im z < 0
    \end{cases} \]
  gilt für alle $n = 0, 1, 2, \ldots$ und $z \in \complex$.
\end{lem}

\begin{thm}
  Sei $f$ die charakteristische Funktion einer Zufallsgröße $X$ und für ein $k
  \ge 1$ existiere das Moment $\mM_k$. Dann ist $f$ $k$-mal differenzierbar. Die
  $k$-te Ableitung $f^{(k)}$ ist beschränkt, gleichmäßig stetig und
  \begin{enumerate}[i)]
  \item $f^{(k)}(t) = i^k \int_\real x^k e^{itx} \diffop \mu(x)$, \hfill ($t \in
    \real$)
  \item $f^{(k)}(0) = i^k \mM_k$,
  \item es gilt
    \begin{align*}
      f(t)
      &= \sum_{n=0}^k f^{(n)}(0) \frac{t^n}{n!} + \Theta(t) \frac{t^k}{k!} \\
      &= \sum_{n=0}^k \mM_n \frac{(it)^n}{n!} + \Theta(t) \frac{t^k}{k!},
    \end{align*}
    wobei $\Theta : \real \to \complex$ eine stetige Funktion mit $\Theta(0) =
    0$ und
    \[ | \Theta(t) | \le \sup_{0 \le \eta \le 1} | f^{(k)} (\eta t) - f^{(k)}(0)
      |, \quad t \in \real. \]
  \end{enumerate}
\end{thm}

\begin{thm}
  Seien $X, X_1, X_2, \ldots$ Zufallsgrößen und $f, f_1, f_2, \ldots$ die
  zugehörigen charakteristischen Funktionen. Dann ist
  \[ \lim_{ n \to \infty} f_n(t) = f(t), \quad t \in \real
    \qLRq
    \lim_{n \to \infty} X_n = X \text{ in Verteilung.} \]
\end{thm}

\subsection*{Beispiele für charakteristische Funktionen}
\begin{enumerate}[a)]
\item Sei $X \sim N(0,1)$, dann ist $f_X(t) = e^{-\rez{2}t^2}$, $t \in \real$.
  \[ f_X(t) = \pE(e^{itX}) = \rez{\sqrt{2 \pi}} \int_{-\infty}^\infty e^{itx}
    \cdot e^{x^2/2} \diffop x \]
  Betrachte
  \begin{align*}
    f'_X(t) \overset{2.13.7}{=} i \pE(X e^{itX} )
    &= \rez{\sqrt{2 \pi}} \int_{-\infty}^\infty i x e^{itx} \cdot e^{-x^2/2} \diffop x \\
    \intertext{Partielle Integration: $u = -i \cdot e^{itx}$, $v' = -x \cdot e^{x^2/2}$}
    &= \rez{\sqrt{2 \pi}} \left( \big[ -i e^{-tx} e^{-x^2/2} \big]_{-\infty}^\infty
      - \int_{-\infty}^\infty e^{itx} \cdot t \cdot e^{-x^2/2} \diffop x \right) \\
    &= -t \rez{\sqrt{2 \pi}} \int_{-\infty}^\infty e^{itx} e^{-x^2/2} \diffop x \\
    f'_X(t) &= -t f_X(t).
  \end{align*}
  TdV:
  \[ \diff{f}{t} = - t f \qRq \int \frac{\diffop f}{f} = \int -t \diffop t \qRq
    \ln f = \frac{-t^2}{2} \qRq f_X(t) = e^{-t^2/2}. \]
\item Sei $X \sim \operatorname{Bin}(n,p)$, $p \in (0,1)$.
  \begin{align*}
    f_X(t) = \pE( e^{itX} )
    &= \sum_{k=0}^n e^{itk} \binom{n}{k} p^k (1-p)^{n-k} \\
    &= \sum_{k=0}^n (e^{it} \cdot p)^k (1-p)^{n-k} \binom{n}{k} \\
    &= (e^{it} \cdot p) + (1-p))^n
  \end{align*}
\item Sei $X \sim \operatorname{Poi}(\lambda)$, $\lambda > 0$.
  \begin{align*}
    f_X(t) = \pE( e^{itX} )
    &= \sum_{k=0}^\infty e^{itk} \cdot \pP[ X = k ]
      = \sum_{k=0}^\infty e^{itk} \frac{\lambda^k}{k!} e^{-\lambda} \\
    &= e^{-\lambda} \sum_{k=0}^\infty \frac{(\lambda e^{it})^k}{k!}
      = e^{-\lambda} \cdot e^{\lambda e^{it}} \\
    &= e^{\lambda( e^{it} - 1)}.
  \end{align*}
  Wenn $X \sim \operatorname{Poi}(\lambda_1)$, $Y \sim
  \operatorname{Poi}(\lambda_2)$ unabhängig:
  \begin{align*}
    f_{X+Y}(t) = f_X(t) \cdot f_Y(t)
    &= e^{\lambda_1 ( e^{it} - 1)} \cdot e^{\lambda_2 ( e^{it} - 1)} \\
    &= e^{(\lambda_1 + \lambda_2) ( e^{it} - 1)}.
  \end{align*}
  Also ist $X + Y \sim \operatorname{Poi}(\lambda_1 + \lambda_2)$.
\end{enumerate}


\addtocounter{section}{2}

\section{Der Zentrale Grenzwertsatz}
\begin{deno*}
  $\Phi$ bezeichnet im Weiteren die Verteilungsfunktion der standartisierten
  Normalverteilung, das heißt
  \[ \Phi(x) = \rez{\sqrt{2 \pi}} \int_{- \infty}^x e^{-y^2/2} \diffop y, \quad
    x \in \real. \]

  Sind $X_1, \ldots, X_n \in L^2(\pP)$ i.i.d. Zufallsgrößen mit $D(X_j) =:
  \sigma > 0$, so schreiben wir
  \[ S_n := \rez{\sigma \sqrt{n}} \sum_{j=1}^n (X_j - \pE X_j). \]
  Dann gilt: $\pE S_n = 0$, $D^2(S_n) = 1$.\footnote{%
    Es gilt
    \[ D^2 \left( \sum_{j=1}^n \frac{X_j - \pE X_j}{\sigma \sqrt{n}}  \right)
      = \sum_{j=1}^n D^2 \left( \frac{X_j - \pE X_j}{\sigma \sqrt{n}} \right)
      = \frac{n}{\sigma^2 n} D^2( X_j - \pE X_j) = 1. \]
  }
\end{deno*}

\begin{lem}
  Es sei $(r_n)$ eine Folge reeller Zahlen mit $\lim_{n \to \infty} r_n = 0$.
  Dann gilt
  \[ \lim_{n \to \infty} \left( 1 + \frac{r+r_n}{n} \right)^n = e^r, \quad r
    \in \real. \]
\end{lem}

\begin{proof}
  Aufgabe. Hinweis: Mittelwertsatz der Differentialrechnung.
\end{proof}

\begin{thm}[Zentraler Grenzwertsatz nach Moavre-Laplace]
  Für jede unabhängige Folge $(X_n) \subseteq L^2(\pP)$ von i.i.d. Zufallsgrößen
  mit Streuung $\sigma^2 > 0$ gilt:
  \[ \lim_{n \to \infty} \pP( S_n < x ) = \Phi(x), \quad x \in \real. \]
\end{thm}
Das heißt
\[ \rez{\sigma \sqrt{n}} \sum_{i = 1}^n (X_i - \pE X_i) \xrightarrow{n \to
    \infty} N(0,1) \text{ in Verteilung.} \]

\begin{proof}
  Wir können annehmen, dass $\pE X_j = 0$ (Idee: $X_n \xrightarrow{\text{Vert.}}
  X$ $\Leftrightarrow$ $f_{X_n}(t) \to f_X(t)$ für alle $t \in \real$).
  \begin{align*}
    f_{S_n}(t)
    &\overset{\phantom{2.13.3}}{=} f_{\sum_{j=1}^n X_j / \sigma \sqrt{n}} (t)
    & \text{$X, Y$ unabhängig $\Rightarrow$ $f_{X+Y} = f_X f_Y$} \\
    &\overset{2.13.3}{=} (f_{X_j / \sigma \sqrt{n}} (t))^n
    & \text{$f_{aX + b}(t) = e^{itb} \cdot f_X(at)$} \\
    &\overset{2.13.4}{=} \left(f_{X_j} \left( \frac{t}{\sigma \sqrt{n}} \right) \right)^n.
  \end{align*}
  Nach 2.13.8 ist somit zu zeigen, dass
  \[ \lim_{n \to \infty} \left(f_{X_j} \left( \frac{t}{\sigma \sqrt{n}} \right)
    \right)^n = e^{t^2/2}, \quad t \in \real. \]
  Aus $X_j \in L^2(\pP)$ und 2.13.7 (Taylor) folgt:
  \[ f_{X_j}(t) = 1 - \sigma^2 \frac{t^2}{2} + R_2(t), \tag{1} \]
  wobei $R_2$ eine stetige Funktion ist mit
  \[ \lim_{t \to 0} \frac{R_2(t)}{t^2} = 0. \]

  Es gilt
  \[ \lim_{n \to \infty} n R_2 \left( \frac{t}{\sigma \sqrt{n}} \right) =
    \frac{t^2}{\sigma^2} \lim_{n \to \infty} \frac{R_2(t / \sigma \sqrt{n})}{(t
      / \sigma \sqrt{n})^2} = 0. \tag{2} \]

  Aus (1), (2) und Lemma 2.16.1 folgt
  \[ \lim_{n \to \infty}
    \left(f_{X_j} \left( \frac{t}{\sigma \sqrt{n}} \right) \right)^n
    \overset{(1)}{=} \lim_{n \to \infty}
    \left(  1 - \rez{2} \frac{t^2}{n} + \frac{n R_2( t / \sigma \sqrt{n})}{n}
    \right)^n
    \overset{2.16.1}{=} e^{-t^2/2}. \qedhere
  \]
\end{proof}

\textbf{Verallgemeinerungen:} Lindeberg, Ljapunov, Feller.

\end{document}