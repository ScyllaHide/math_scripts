\documentclass[
 a4paper,
 12pt,
 parskip=half
 ]{scrreprt}
 \setcounter{secnumdepth}{3}
 
\usepackage{../.tex/settings}

\usepackage{../.tex/mathpkgs}
\usepackage{../.tex/mathcmds}

\usepackage{stackengine}
\newcommand\groupequation[1]{%
  \setbox0=\hbox{$\displaystyle#1$}%
  \setbox2=\hbox{\,(\theequation)}%
  \stackengine{0pt}{\copy0}{%
    \makebox[\linewidth]{\hfill$\left.\rule{0pt}{\ht0}\right\}$\kern\wd2}}
    {O}{c}{F}{T}{L}
}

\usepackage[numbers_per_chapter]{fancy_thm}

\theoremstyle{plain}
\newtheorem*{thm*}{Satz}
%\newtheorem{flg}[thm]{Folgerung}

\theoremstyle{definition}
\newtheorem{defn}{Definition}
\newtheorem{prgp}{}
\newtheorem{rmrk}{Bemerkung}
\newtheorem{exmp}{Beispiel}

\numberwithin{rmrk}{chapter}
\numberwithin{defn}{chapter}
\numberwithin{exmp}{chapter}

\numberwithin{prgp}{subsection}

\numberwithin{thm}{chapter}

\newtheorem*{rmrk*}{Bemerkung}
%\newtheorem*{exmp*}{Beispiel}
\newtheorem*{defn*}{Definition}
%\newtheorem*{deno*}{Bezeichnungen}

\numberwithin{equation}{chapter}

\hypersetup{
  pdftitle={OPTINUM},
  pdfauthor={Jonas Hippold},
  hidelinks
}

\DeclareMathOperator*{\KA}{KA}
\DeclareMathOperator*{\FT}{FT}
\DeclareMathOperator*{\GP}{GP}

%opening
\title{Vorlesung\\Optimierung und Numerik}
\subtitle{Wintersemester 2017}
\author{Vorlesung: Dr. rer. nat. Guntram Scheithauer\\Mitschrift: Jonas Hippold}

\begin{document}

\maketitle

\tableofcontents

\clearpage

\section{Einleitung}
\subsection{Aufgabenstellung und Grundbegriffe}
Es seien $G \subset \real^n$ und $f: G \to \real$ gegeben. Optimierungsproblem:
\begin{equation}
  f(x) \to \min \quad \text{bei } x \in G.
\end{equation}
$f$ ... Zielfunktion \\
$G$ ... Zulässiger Bereich \\
$x \in G$ ... Zulässiger Punkt, zulässiges Element, zulässige Lösung

$x^* \in G$  heißt \emph{potimal} oder \emph{optimale Lösung} oder
\emph{Lösung}, falls
\begin{equation}
  f(x^*) \le f(x) \quad \text{für alle } x \in G.
\end{equation}
$f^* = f_{\min} := f(x^*)$ ... Optimalwert

Falls $G = \real^n$, so heißt (1.1) \emph{freies} oder \emph{unrestringiertes}
Optimierungsproblem (OP).

(1.1) ist ein \emph{diskretes} OP, falls $G$ eine
\emph{diskrete} Menge ist, zum Beispiel $G = \integer^n$.

(1.1) ist ein \emph{stetiges} oder \emph{kontinuierliches} OP, falls alle
Variablen ``stetig'' sind. Sonst gemischt-ganzzahliges Problem.

(1.1) ist ein \emph{lineares} OP, falls $f(x) = c^T x$ und $G$ durch lineare
Restklassen gegeben ist:
\begin{equation}
  G = \{ x \in \real^n : g_i(x) \le 0, i \in I, h_j(x) = 0, j \in J \},
\end{equation}
wobei $g_i, h_j$ für alle $i,j$ affin linear sind. In diesem Fall kann (1.1) als
\[ c^T x \to \min \quad \text{bei } x \in G:= \{ x \in \real^n, Ax = a, Bx \le b
  \} \]
geschrieben werden.

\begin{defn}
  Betrachtet werden die beiden OP
  \begin{align*}
    f(x) &\to \min \quad \text{bei } x \in D \cap E, \tag{P} \\
    g(x) &\to \min \quad \text{bei } x \in E. \tag{Q}
  \end{align*}
  (Q) heißt \emph{Relaxation} zu (P), falls $g(x) \le f(x)$ auf $D \cap E$. Der
  Optimalwert von (Q) kann als (untere) Schranke bzw. Schrankenwert für (P)
  bezeichnet werden.
\end{defn}

\begin{thm}
  Sei $\obar{x}$ Lösung von (Q) und gelte $\obar{x} \in D$ sowie $f(\obar{x}) =
  g(\obar{x})$. Dann ist $\obar{x}$ Lösung von (P).
\end{thm}

\begin{proof}
  Übung.
\end{proof}

\begin{defn}
  Seien (Q1) und (Q2) Relaxationen zu (P). (Q1) heißt \emph{stärker} (strenger)
  als (Q2), wenn die Schranke von (Q1) größer oder gleich der Schranke von (Q2)
  für \emph{jede} Instanz von (P) ist.
\end{defn}

\subsection{Beispiele zur kontinuierlichen Optimierung}
\subsubsection{Transportoptimierung}
Lineare Optimierung

Es seien Erzeuger $i \in I := \{1, \ldots, n\}$ und Verbraucher $j \in J := \{1,
\ldots, n\}$ gegeben. Weiterhin seien die Kosten $c_{ij}$ für den Transport
\emph{einer} Einheit von $i$ nach $j$ sowie der Vorrat $a_i > 0$ und der
Verbrauch $b_j > 0$ für alle $i \in I$ und $j \in J$ bekannt. Wie muss der
Transport organisiert werden, damit die Gesamtkosten minimal sind?

Variable $x_{ij} \ge 0$ ... Transportmenge von $i$ nach $j$.
\[ \sum_{i \in I} \sum_{j \in J} c_{ij} \cdot x_{ij} \to \min \quad \text{bei }
  \sum_{j \in J} x_{ij} = a_i, \quad i \in I. \]

\subsubsection{Kürzeste euklidische Entfernung eines Punktes zu einer Menge}
Nichtlineare Optimierung

Gegeben sind ein Punkt $\tilde{x}$ und Menge $G \subset \real^n$ mit $\tilde{x}
\notin G$.
\[ f(x) := \rez{2} \| \tilde{x} - x \|_2^2 \to \min \quad \text{bei } x \in
G. \]
Falls $G$ konvex ist, dann ist es der Spezialfall der \emph{konvexen
  Optimierung}.

\subsubsection{Tschebyscheff-Approximation}
Semi-infinite Optimierung

Seien $f: \real \to \real$ und $S: \real^{n+1} \to \real$ stetig, zum Beispiel
\[ S(y, x_1, \ldots, x_n) = \sum_{i=1}^n x_i s_i(y), \]
wobei $s_i$ ein Ansatz ist für
\[ \max_{y \in [a,b]} | f(y) - S(y, x_1, \ldots, x_n) | \to \min_{x \in
    \real^n}. \]
Umformulierung:
\[ \tilde{f}(x,\lambda) := \lambda \to \min \quad \text{bei } - \lambda \le
  f(y) - S(y, x_1, \ldots, x_n) \le \lambda \text{ für alle } y \in [a,b]. \]
Das ist ein Beispiel mit endlich vielen Variablen und unendlich vielen
Restriktionen.

\subsection{Beispiele zur diskreten Optimierung}
\subsubsection{Rucksackproblem}
Gegeben sind ein Rucksack mit Volumen $b$, Teile mit Volumen $a_i$ und Bewertung
$c_i$, $i \in I$.

Voraussetzung: $a_i, c_i, b \in \integer_{>0}$, $0 < a_i \le b$ für alle $i \in
I = \{1, \ldots, n\}$.

\paragraph{0/1-Rucksackproblem}
Entscheidungsvariable $x_i \in \boole := \{0,1\}$.
\[ \sum_{i \in I} c_i x_i \to \max \quad \text{bei } \sum_{i \in I} a_i x_i \le
  b, x_i \in \boole. \]
Alternative Formulierung:
\[ \sum_{i \in \tilde{I}} \to \max \quad \text{bei } \sum_{i \in \tilde{I}} a_i
  \le b, \tilde{I} \subset I. \]

\paragraph{Klassisches (Standard-)Rucksackproblem}
\[ \sum_{i \in I} c_i x_i \to \max \quad \text{bei } \sum_{i \in I} a_i x_i \le
  b, x_i \in \integer_+ \text{ für alle } i. \]
Hier bezeichnet $x_i$ die Anzahl, wie oft Teil $i$ mitgenommen wird.

\subsubsection{Das eindimensionale Zuschnittproblem}
Cutting Stock Problem, CSP

Aus möglichst wenig Ausgangsmaterial der Länge $L$ sind $b_i$ Teile der Länge
$l_i$, $i \in I := \{ 1, \ldots, n \}$ zuzuschneiden.

Zuschnittvariante $a_j = (a_{1j}, \ldots, a_{nj})^T \in \integer_+^n$ mit
$\sum_{i \in I} l_i a_{ij} \le L$ für alle $j \in J$.

\paragraph{Kantorovich-Modell}
Variablen: $a_{ij}$ Anzahl, wie oft Teil $i$ aus ZV $j$ erhalten wird, $y_i
\in \boole$ Entscheidungsvariable.
\[ \begin{aligned}
    z &= \sum_{j \in J} y_j \to \min \quad \text{bei } &
    &\sum_{i \in I} l_i \cdot a_{ij} \le L \cdot y_j, & \forall j \in J, \\
    & &
    &\sum_{j \in J} a_{ij} = b_i, & i \in I, \\
    & &
    &a_{ij} \in \integer_+, y_j \in \boole, & \forall i,j.
  \end{aligned} \]
Der Optimalwert $z_{LP}^*$ der stetigen (bzw. LP) Relaxation ist gleich der
Materialschranke $= \rez{L} \sum_{i \in I} l_i b_i$. Bei LP-Relaxation gilt
\[ a_{ij} \ge 0, \quad 0 \le y_i \le 1, \quad \forall i, j. \]

\paragraph{Gilmore/Gomory-Modell (1961)}
Annahme: \emph{Alle} ZV sind bekannt, das heißt die $a_{ij}$ sind Koeffizienten.

Variablen: $x_j \in \integer_+$ ... Anzahl, wie oft Variante $j$ genommen wird.
\[ \begin{aligned}
    z &= \sim_{i \in J} x_j \to \min \quad \text{bei } &
    \sum_{j \in J} a_{ij} x_j = b_i, & \forall i \in I \\
    & &
    x_j &\ge 0, \text{ ganzzahlig}, & \forall j \in J.
  \end{aligned} \]

Vermutung: $z^* - z_{LP} < 2$ für alle Instanzen.

Bemerkung: Das größte bekannte Gap ist 1.

\subsubsection{Facility Location Problem}
Ein großer Dienstleistungsbetrieb möchte neue Filialen aufbauen, um den Kunden
$k \in K$ mit Leistungen zu dienen. Aus der Menge $S := \{1, \ldots, n\}$ der
möglichen Standorte $s$ ist eine Auswahl zu treffen, an denen die Filialen
aufgebaut werden.

Ziel ist, die Gesamtkosten, das heißt die Kosten zum Aufbau
und während des Betriebs, zu minimieren.

$d_{ks} \in \real_+$ ... Kosten, um Kunde $k$ von Standort $s$ aus zu bedienen,
\\
$c_s \in \real_+$ ... einmalige Kosten zum Einrichten der Filiale $s$, \\
$x_j \in \boole$ ... Entscheidungsvariable, \\
$y_{ks} \in [0,1]$ ... Teil des Bedarfs von $k$, der von $s$ erledigt wird.
\[ \begin{aligned}
    z &= \sum_{s \in S} c_s x_s + \sum_{k \in K} \sum_{s \in S} d_{ks} y_{ks} \to
    \min \quad \text{bei } &
    &\sum_{s \in S} y_{ks} = 1, & k \in K, \\
    & &
    &y_{ks} - x_s \le 0, & \forall k, s, \\
    & &
    &y_{ks} \ge 0, x_s \in \boole, & \forall k, s.
  \end{aligned}
\]

\subsubsection{Quadratic Alignment Problem}
In einem Gebäude müssen $m$ Personen auf $n$ Räume verteilt werden. Person $i$
muss Person $j$ persönlich treffen und zwar $c_{ij}$-mal täglich ($c_{ij} \ge
0$). Büro $k$ hat von Büro $l$ die Entfernung $d_{kl} \ge 0$.

Wird Person $i$ dem Raum $k$ und Person $j$ dem Raum $l$ zugeordnet, so muss $i$
die Entfernung $2 d_{kl} c_{ij}$ zurücklegen.

\paragraph{Permutationsmodell}
\[ f(x) = \sum_{i=1}^n \sum_{j=1}^n 2 c_{ij} d_{\pi(i) \pi(j)} \to \min \quad
  \text{bei } \pi \in \Pi(1, \ldots, n). \]

\paragraph{Modellierung mit bodeschen Variablen}
\[ x_{ik} := \begin{cases}
    1, &\text{falls Person $i$ in Zimmer $k$}, \\
    0, &\text{sonst.}
  \end{cases} \]
\[ z = 2 \cdot \sum_{i \in I} \sum_{j \in J} \sum_{k \in K} \sum_{l \in L} c_{ij}
  x_{ik} \cdot x_{jl} \to \min \]
mit den Zuordnungsbedingungen
\[ \begin{aligned}
    \sum_{k = 1}^n x_{ik} &= 1 & \text{für alle } i, \\
    \sum_{i = 1}^n x_{ik} &= 1 & \text{für alle } k
  \end{aligned} \]
bei $x_{ik} \in \boole$.

\subsubsection{Ganzzahlige lineare Optimierung}
\[ z = c^T x \to \min \quad \text{bei} \quad Ax \le b, x \ge 0, x \in
  \integer. \]


\chapter{Grundlagen}
\section{Existenz von Lösungen}
Wir betrachten die Aufgabe
\begin{equation}
  f(x) \to \min \subjto x \in G,
\end{equation}
wobei folgende Voraussetzungen erfüllt seien:
\begin{itemize}
\item $G \subset X$ kompakt, $X$ Banachraum, zumeist $X = \real^n$,
\item $G \ne \emptyset$, das heißt $G$ ist nicht leer,
\item $f: G \to \real$ ist stetig.
\end{itemize}

\begin{thm}
  Unter obigen Voraussetzungen existiert ein $\obar{x} \in G$ mit
  \[ f^* := f( \obar{x} ) \le f(x) \quad \text{für alle } x \in G. \]
\end{thm}

\begin{proof}
  Wir wählen eine Folge $\{f_k\}_{k \in \nat}\} \subset \real$ mit $f_k > f^*$
  für alle $k \in \nat$ und
  \[ \lim_{k \to \infty} f_k = f^* = \inf_{x \in G} f(x). \]
  Entsprechend der Definition vom Infimum existiert für jedes $k$ ein
  $x_k \in G$ mit $f(x_k) \le f_k$, $k \in \nat$. Wegen der Kompaktheit von $G$
  besitzt die Folge $\{x_k\}$ eine in $G$ konvergente Teilfolge $\{ \tilde{x}_k \}
  \subseteq \{ x_k \}$ mit
  \[ \lim_{k \to \infty} \tilde{x}_k = \obar{x} \in G. \]
  Die Stetigkeit von $f$ liefert
  \[ \lim_{k \to \infty} f(\tilde{x}_k) = f(\obar{x}) = f^*. \qedhere \]
\end{proof}

\begin{rmrk}
  Die Voraussetzung der Stetigkeit von $f$ kann abgeschwächt werden. $f$ muss
  dann nur \emph{unterhalbstetig} (bzw. halbstetig nach unten) sein, das heißt
  es gilt
  \[ f( \obar{x} ) \le \lim_{x \to \obar{x}} f(x) \quad \text{für alle } x \in
    G. \]
\end{rmrk}

\begin{exmp}
  \begin{enumerate}[(1)]
  \item Satz 2.1 ist anwendbar, $G$ kompakt, der Limes existiert:
    \[ f(x_1, x_2) = 2 x_1 - 3 x_2 \to \min \subjto x_1^2 + x_2^2 \le 1. \]
  \item Satz 2.1 nicht anwendbar, $G$ unbeschränkt, $f^* = - \infty$:
    \[ f(x_1, x_2) = 2 x_1 - 3 x_2 \to \min \subjto x_1^2 + x_2^2 \ge 1. \]
  \item Satz 2.1 nicht anwendbar, $G$ unbeschränkt, kein Minimum, $f^* = 0$:
    \[ f(x_1, x_2) = \rez{x_1} \to \min \subjto x_2 \le \rez{x_1}, x_1 \ge 1,
      x_2 \ge 0. \]
  \item Satz 2.1 nicht anwendbar, $G$ unbeschränkt, Minimum existiert, $f^* =
    -1$:
    \[ f(x_1, x_2) = -\rez{x_1} \to \min \subjto x_2 \le \rez{x_1}, x_1 \ge 1,
      x_2 \ge 0. \]
  \end{enumerate}
\end{exmp}

\begin{defn}
  $\obar{x}$ heißt \emph{lokale Lösung} von (2.1), falls $\obar{x} \in G$ gilt
  und ein $\rho > 0$ existiert mit
  \[ f( \obar{x} ) \le f(x) \quad \text{für alle } x \in G \cap
    B_\rho(\obar{x}), \]
  wobei $B_\rho(\obar{x}) := \{ x \in \real^n : \| x - \obar{x} \| < \rho \}$.
\end{defn}

\begin{rmrk}
  Jede globale Lösung ist auch lokale Lösung. Die Umkehrung gilt im Allgemeinen
  nicht.
\end{rmrk}

\begin{defn}
  \begin{enumerate}[(1)]
  \item $G \subset X$ ist \emph{konvex}, falls für alle $x,y \in G$ gilt:
    \[ [x,y] := \{ x(\lambda) \in X: x(\lambda) = (1-\lambda)x + \lambda y = x +
      \lambda(y-x), \lambda \in [0,1] \} \subset G. \]
  \item Sei $G$ konvex. $f: G \to \real$ heißt \emph{konvex}, falls
    \[ f( x + \lambda(y-x) ) \le f(x) + \lambda( f(y) - f(x) ) \]
    für alle $x,y \in G$, $0 \le \lambda \le 1$ gilt.
  \item Sei $G$ konvex, $f: G \to \real$ heißt \emph{streng konvex}, falls
    \[ f( x + \lambda(y-x) ) < f(x) + \lambda( f(y) - f(x) ) \]
    für alle $x,y \in G$, $0 < \lambda < 1$ gilt.
  \end{enumerate}
\end{defn}

\begin{thm}
  Sei $G \subset X$ eine konvexe Menge und $f: G \to \real$ eine konvexe
  Funktion.
  \begin{enumerate}[(1)]
  \item Dann ist ein lokales Minimum gleichzeitig globale Lösung von (2.1).
  \item Falls $f$ streng konvex ist, dann existiert höchstens eine Lösung.
  \end{enumerate}
\end{thm}

\begin{proof}
  Zu (1): Sei $\tilde{x}$ ein lokales Minimum. Wir nehmen an, dass $\bar{x}$
  existiert mit $f(\bar{x}) < f(\tilde{x})$. Mit der Konvexität von $f$ folgt
  \[ \begin{aligned}
      f( x(\lambda) ) = f( \tilde{x} + \lambda(\bar{x} - \tilde{x}) )
      &\le f( \tilde{x} ) + \lambda(f(\tilde{x}) - f(\obar{x})) \\
      &= \underbrace{\lambda f( \bar{x} )}_{< \lambda f(\tilde{x})}
      + (1-\lambda) f( \tilde{x} )
      &< f(\tilde{x})
    \end{aligned} \]
  für alle $\lambda \in (0,1)$. Widerspruch zur Optimalität von $\tilde{x}$.

  Zu (2): Die Annahme von zwei globalen Lösungen $x \ne y \in G$, das heißt
  $f(x) = f(y) = f^*$, ergibt wegen
  \[ f(x + \lambda(y-x)) < f(x) + \lambda( f(y) - f(x) ) = f(x) \]
  einen Widerspruch zur Optimalität von $x$ und $y$.
\end{proof}

\begin{aus}
  Sei $G$ gegeben durch
  \[ G = \{ x \in \real^n : g_i(x) \le 0, i \in I, h_j(x) = 0, j \in J \}, \]
  dann gilt: Falls alle $g_i$, $i \in I$, konvex sind und alle $h_j$, $j \in J$,
  affin linear sind, dann ist $G$ konvex.
\end{aus}

\begin{proof}
  Übung.
\end{proof}

\section{Notwendige Optimalitätsbedingungen}
\begin{defn}
  Eine Menge $K \subset X$ heißt \emph{Kegel}, falls gilt:
  \[ x \in K \qRq \lambda x \in K \text{ für alle } \lambda \ge 0. \]
  Ein Kegel $K$ ist ein \emph{konvexer Kegel}, falls $K$ eine konvexe Menge ist
  bzw. falls gilt:
  \[ x,y \in K \qRq x + y \in K. \]
  Der \emph{Kegel der zulässigen Richtungen} $Z{\obar{x}}$ (zu $\obar{x} \in G$)
  ist definiert durch
  \[ Z(\obar{x}) := \{ d \in X : \exists \obar{t} = \obar{t}(\obar{x},d) > 0,
    \text{ sodass } \obar{x} + t \cdot d \in G \forall t \in [0,\obar{t}] \}. \]
\end{defn}

\begin{aus}[Notwendiges Optimalitätskriterium]
  Ist $f$ stetig differenzierbar und $\obar{x}$ ein lokales Minimum, dann gilt:
  \begin{equation}
    \nabla f(\obar{x})^\top d \ge 0 \text{ für alle } d \in Z(\obar{x}),
  \end{equation}
  Ist $G$ konvex, dann kann (2.2) wie folgt geschrieben werden:
  \begin{equation}
    \nabla f(\obar{x})^\top d \ge 0 \text{ für alle } x \in G.
  \end{equation}
\end{aus}

\begin{proof}
  Wir betrachten die Hilfsfunktion $\chi(t) := f( \obar{x} + td)$. Für ein lokales
  Minimum gilt dann
  \[ \chi(t) \ge \chi(0) \text{ für alle } t \in [0,\obar{t}]
    \qRq
    \liminf_{t \to 0} \frac{\chi(t) - \chi(0)}{t} \ge 0 \]
  und damit folgt die Gültigkeit von (2.2).

  Für konvexe Mengen gilt speziell $x - \obar{x} \in Z(\obar{x})$ für alle $x
  \in G$ und damit folgt (2.3).
\end{proof}

\begin{rmrk}
  Ein Punkt, für den (2.2) erfüllt ist, heißt \emph{stationärer Punkt}.
\end{rmrk}

\begin{rmrk} %2.4
  Bei der \emph{freien Minimierung}, das heißt $G = X$, ergibt sich wegen $Z(x)
  = X$ für alle $x \in G$:
  \[ \obar{x} \text{ ist lokale Lösung} \qRq \nabla f(\obar{x}) = 0. \]
  Für konvexe Optimierungsprobleme gilt auch die Umkehrung.
\end{rmrk}

\begin{aus}[Hinreichendes Optimalitätskriterium 1. Ordnung]
  Es seien $G \subset \real^n$ und $f: G \to \real$ konvex. Falls ein $\bar{x}
  \in G$ existiert, welches der notwendigen Bedingung (2.3) genügt, dann ist
  $\bar{x}$ ein globales Minimum von (2.1).
\end{aus}

\begin{proof}
  Wenn $f$ konvex und stetig differenzierbar ist, so gilt (s. Übung)
  \[ f(x) \ge f( \bar{x} ) + \nabla f( \bar{x} )^\top (x - \bar{x}) \]
  für alle $x \in G$.

  Wegen (2.3) folgt unmittelbar die Optimalität und wegen Satz 2.2 die globale
  Optimalität von $\bar{x}$.
\end{proof}

Im Fall polyedrischer Mengen $G \subset \real^n$ kann die notwendige
Optimalitätsbedingung (2.2) präzisiert werden, da dann $Z(\bar{x})$ eine
einfachere Struktur besitzt.

\begin{defn} %2.4
  $G \subset \real^n$ heißt \emph{polyedrisch}, falls eine Darstellung
  \[ G = \{ x \in \real^n : Ax \le b \} \]
  existiert, wobei $A \in \realmat{m}{n}$ und $b \in \real^m$. Hierbei gilt:
  \[ Ax \le b \qLRq a^\top x = \sum_{j=1}^n a_{ij} x_j \le b_i \text{ für alle } i
    \in I = \{ 1, \ldots, m \}. \]
  $a_i$ ist die $i$-te Zeile von $A$.
\end{defn}

\begin{rmrk} %2.5
  $G$ ist konvex und abgeschlossen, aber im Allgemeinen nicht beschränkt.
  Impliziert können Gleichungen enthalten sein.
\end{rmrk}

\begin{defn} %2.5
  Für $x \in G$ ist \emph{Indexmenge der aktiven Restriktion} definiert durch
  \[ I_0(x) := \{ i \in I : a_i^\top x = b_i \}. \]
  Aus $d \in Z(x)$ ergibt sich
  \[ a_i^\top (x + t d) \le b_i \qLRq t a_i^\top  d \le b_i - a_i^\top x \quad
    \text{für alle} i \in I, \text{ für alle } t \in [0, \bar{t}] \]
  und damit die Bedingung
  \[ a_i^\top d \le 0 \quad \text{für alle } i \in I_0(x). \]
  Die \emph{inaktiven} Restriktionen ergeben keine weitere Einschränkung an
  $Z(x)$. Somit ist die Größe $\bar{t}$ wohldefiniert und positiv:
  \begin{equation} %2.4
    \bar{t} := \min \left\{ \frac{b_i - a_i^\top x}{a_i^\top d} : i \in I(x,d)
    \right\},
  \end{equation}
  wobei $I(x,d) := \{ i \in I: a_i^\top d > 0 \}$. Damit kann $Z(x)$ für beliebige
  $x \in G$ angegeben werden in der Form
  \begin{equation} %2.5
    d \in Z(x) \qLRq a_i^\top d \le 0 \quad \text{für alle } i \in I_0(x).
  \end{equation}
\end{defn}

\begin{rmrk} %2.6
  Falls $I(x,d) = \emptyset$ gilt, so setzen wir $\bar{t} := \infty$.
\end{rmrk}

%%\[ f(x) \to \min \subjto  x \in G = \{ x \in \real^n : Ax \le b \}. \]
%% $a_i \in \real^n$, $a_i^\top =$ $i$-te Zeile von $A$, $I = \{1, \ldots, m\}$.
\begin{flg} %2.6
  Sei $G$ polyedrisch, das heißt $G = \{ x \in \real^n : Ax \le b\}$. Ist
  $\obar{x}$ eine lokale Lösung von (2.1), so gilt
  \begin{equation} %2.6
    \nabla f( \obar{x} )^\top d \ge 0 \quad \text{für alle } d \in \real^n
  \end{equation}
  mit $a_i^\top d \le 0$ für alle $i \in I_0(\obar{x})$.

  Ist $f$ zusätzlich konvex, dann impliziert (2.6) die globale Optimalität von
  $\obar{x}$.
\end{flg}

\begin{defn}
  Ein Kegel $K$ heißt \emph{polyedrisch}, wenn er sich in der Form
  \begin{equation} %2.7
    K = \{ x \in \real^n : x = \sum_{j=1}^m u_j \cdot a_j, u_j \ge 0, j = 1,
    \ldots, m \}
  \end{equation}
  darstellen lässt, wobei die $a_j \in \real^n$ gegeben sind. Eine alternative
  Darstellung ist
  \begin{equation} %2.8
    K = \{ x \in \real^n : b_j^\top x \le 0, j = 1, \ldots, m \},
  \end{equation}
  wobei die $b_j$ bekannt sind.
\end{defn}

\begin{lem} %2.7
  Polyedrische Kegel der Form (2.7) sind nichtleer, konvex und abgeschlossen.
  Analoges gilt für (2.8).
\end{lem}

\begin{proof}
  Grossmann/Terno
\end{proof}

\begin{exmp}[Orthoprojektion auf eine konvexe Menge]
  Sei $G \subset \real^n$, $G \ne \emptyset$ konvex, abgeschlossen, $q \notin
  G$,
  \[ f(x) := \rez{2} \| x - \|_2^2 \to \min \subjto x \in G. \]
  $f$ ist streng konvex. Dann sind die \emph{Niveaumengen}
  \[ N_f(c) := \{ x \in \real^n : f(x) \le c \} \]
  kompakt für $c \ge 0$. Nach Satz 2.2 folgt die Existenz einer eindeutigen
  Lösung $\obar{x} \in G$, das heißt
  \[ f( \obar{x} ) = \rez{2} \| \obar{x} - q \|_2^2 \le \| x - q \| \]
  für alle $x \in G$ (globales Minimum).

  Die Anwendung von (2.3) ergibt
  \[ \nabla f( \obar{x} )^\top (x-\obar{x}) = - (q-\obar{x})^\top(x-\obar{x}) \ge 0
    \quad \text{für alle } x \in G. \]

  Spezialfall: Ist $G$ affin lineare Menge (z.B. Unterraum)
  \[ \nabla(\obar{x})^\top (x-\obar{x}) = - (q-\obar{x})^\top(x-\obar{x}) = 0. \]
  Ist $G$ ein Unterraum: $(q-\obar{x})^\top x = 0$ bzw. $q-\obar{x} \perp x$ für
  alle $x \in G$.

  \paragraph{Geometrische Interpretation}
  (für beliebiges, abgeschlossenes, konvexes $G$)
  Mit $s := q - \obar{x}, \quad s_0 := s^\top x^\top$ gilt
  \[ s^\top x \le s_0 \quad \text{für alle } x \in G, \]
  \emph{aber} $s^\top q > s_0$.
\end{exmp}

\section{Lemma von Farkas}\label{sect:kkt-bed}
\begin{lem}[Farkas] \label{lem:farkas} %2.8
  Es seien $A \in \realmat{m}{n}$ und $a \in \real^n$. Von den Systemen
  \begin{align*}
    Az &\le 0, & a^\top z &> 0, \tag{1} \\
    A^\top u &= a, & u &\ge 0 \tag{2}
  \end{align*}
  ist \emph{genau} eines lösbar. Dabei sind $A z \le 0$ und $u > 0$
  komponentenweise zu verstehen.
\end{lem}

\begin{proof}
  Seien (1) und (2) gleichzeitig lösbar und $z,u$ die zugehörigen Lösungen. Dann
  gilt
  \[ 0 \ge u^\top A z = (A^\top u)^\top z = a^\top z > 0. \]
  Widerspruch!

  Wir zeigen noch: Die Unlösbarkeit von (1) impliziert die Lösbarkeit von (2).
  Sei (1) nicht lösbar, das heißt
  \[ a \notin K := \{ x = A^\top u : u \in \real^m_+ \}. \]
  Wir betrachten die Optimierungsaufgabe
  \[ f(x) := \rez{2} \| a - x \|_2^2 = \rez{2} (a-x)^\top (a-x) \to \min \subjto x
    \in K. \]
  Dann existiert eine (globale) Lsöung $\obar{x} \in K$ mit
  \[ \nabla f(\obar{x})^\top \obar{x} = 0 \tag{3}\]
  und
  \[ \nabla f(\obar{x})^\top x \ge 0 \]
  für alle $x \in K$.

  Nun zeigen wir, dass $z := a - \obar{x}$ das System (1) löst. Es gilt
  $\nabla f( \obar{x} ) = -z$ und damit
  \[ 0 = \nabla f(\obar{x})^\top \obar{x} = -(a-\obar{x})^\top(\obar{x}-a+a)
    \qRq a^\top z = z^\top z > 0. \]
  Weiter gilt $\obar{x} \in K$ genau dann, wenn ein $u \in \real^m_+$ existiert
  mit $\obar{x} = A^\top u$. Aus (4) folgt
  \[ -(a-\obar{x})^\top A^\top u \ge 0 \quad \text{für alle } u \in \real^m_+. \]
  Also gilt
  \[ (Ax)^\top u \le 0 \text{ für alle } u \in \real^m_+ \qRq Ax \le 0. \qedhere \]
\end{proof}

Damit können die notwendigen Optimalitätskriterien für den Spezialfall affin
linearer Ungleichungen insgesamt zu folgendem System zusammengefasst werden:
\begin{equation} %2.9
  \text{(KKT)} \left\{ \quad \begin{aligned}
      \nabla f(x) + \sum_{i =1}^m u_i a_i = 0, & \\
      u_i \ge 0, \quad a_i^\top x \le b_i, & &i = 1, \ldots, m, \\
      u^\top(Ax - b) = 0, 
    \end{aligned} \right.
\end{equation}

\begin{rmrk*}
  \begin{enumerate}[(1)]
  \item KKT steht für Karush-Kuhn-Tucker.
  \item Die $u$-Variablen werden als \emph{Lagrange-Multiplikatoren} bezeichnet.
  \item Gibt es neben den Ungleichungs- auch Gleichungsrestriktionen $a_i^\top x =
    b_i$, $i = m+1, \ldots, \obar{m}$ ($\obar{m} > m$), dann erhält man das
    folgende KKT-System:
    \begin{equation} \label{eq:kkt-bed} %2.10 
      \text{(KKT)} \left\{ \quad \begin{aligned}
          \nabla f(x) + \sum_{i =1}^m u_i a_i + \sum_{i=m+1}^{\obar{m}} u_i a_i =
          0, & \\
          u_i \ge 0, \quad a_i^\top x - b_i \le 0, & &i = 1, \ldots, m, \\
          u_i(a_i^\top x - b_i) = 0, & &i = 1, \ldots, m, \\
          a_i^\top x - b_i = 0, & &i = m + 1, \ldots, \obar{m}.
        \end{aligned} \right.
    \end{equation}
  \end{enumerate}
\end{rmrk*}

\section{KKT-Bedingungen für nichtlineare Optimierungsprobleme}
Wir betrachten
\begin{equation} %2.11
  f(x) \to \min \subjto x \in G,
\end{equation}
wobei $G$ gegeben ist durch
\[ G := \{ x \in \real^n : g_i(x) \le 0, i \in I, h_j(x) = 0, j \in J \} \]
und alle Funktionen $f$, $g_i$, $h_j$  stetig differenzierbar sind.

Falls $\obar{x} \in G$ ein lokales Minimum ist, dann ist zumindest das
notwendige Optimalitätskriterium (2.2) erfüllt. Dies ist aber im Allgemeinen
ungeeignet für eine praktische Nutzung, da $Z(\obar{x})$ im Allgemeinen nicht
explizit beschreibbar ist.

Daher ersetzt man die nichtlinearen Bedingungen durch deren
\emph{Linearisierung} (Tangente, Tangentialebene).
\[ \begin{aligned}
    g_i(x) &\approx g_i(\obar{x}) + \nabla g_i(\obar{x})^\top (x-\obar{x}),
    & i \in I, \\
    h_j(x) &\approx h_j(\obar{x}) + \nabla h_j(\obar{x})^\top (x-\obar{x}),
    & j \in J.
  \end{aligned} \]

Anstelle von $Z(\obar{x})$ wird der \emph{Linearisierungskegel}
\[ L( \obar{x} ) := \{ d \in \real^n : \nabla g_i(\obar{x})^\top d \ge 0, i \in
  I_0(\obar{x}), \nabla h_j(\obar{x})^\top d = 0, j \in J \} \]
verwendet.

\clearpage

\begin{thm} %2.9
  Seien $f,g_i$ ($i \in I$) stetig differenzierbar. ISt $\obar{x}$ (lokale)
  Lösung von $f(x) \to  \min$ bei $x \in G$ und
  \[ G = \{ x \in \real^n : g_i(x) \le 0, i \in I \} \]
  und gilt zusätzlich die Regularitätsbedingung (constraint qualification)
  \[ \tag{CQ} \operatorname{cl}(Z(\obar{x})) = \obar{Z(\obar{x})} =
    L(\obar{x}), \]
  dann ist (2.2) äquivalent zur Existenz einer Lösung von
  \begin{equation} %2.12
    \text{(KKT)} \left\{ \quad \begin{aligned} 
        \nabla f(x) + \sum_{i =1}^m u_i \nabla g_i(\obar{x}) = 0, & \\
        u_i \ge 0, \quad a_i^\top x \le b_i, & &i \in I, \\
        g_i(\obar{x}) \le 0, & &i \in I, \\
        u_i g_i(x) = 0, & &i \in I.
      \end{aligned} \qquad \right.
  \end{equation}
\end{thm}


\chapter{Lineare Optimierung}
Wir betrachten nun das Problem
\begin{equation}
  z = c^\top x \to \min \subjto x \in G := \{x \in \real^n : Ax = b, x \ge 0 \}
\end{equation}
mit $A \in \realmat{m}{n}$, $b \in \real^m_+$. Außerdem seien $\rang(A) =
\operatorname{rg}(A) = m$ und $m < n$.

\begin{rmrk}
  \begin{enumerate}[(1)]
  \item $G$ ist polyedrisch.
  \item Alle endlich-dimensionalen Optimierungsprobleme lassen sich in der
    \emph{Standardform} (3.1) schreiben (Übung).
  \end{enumerate}
\end{rmrk}

\section{Basislösungen und Ecken}
Sei $I = \{1, \ldots, n \}$. Da $\rang(A) = m$ gilt, existiert eine Indexmenge
$I_B \subset I$ mit $|I_B| = m$ derart, dass die Spalten $A^i$, $i \in I_B$
linear unabhängig sind. $I_B$ wird \emph{Basis-Indexmenge} genannt. Mit $I_N :=
I \setminus I_B$ (Nichtbasis) definieren wir
\[ A_B := (A^i)_{i \in I_B}, \qquad A_N := (A^i)_{i \in I_N}. \]
Dann kann (3.1) geschrieben werden in der Form
\begin{equation}\groupequation{
  \begin{aligned}
    z = c_B^\top x_B + c_N^\top x_N = \sum_{i \in I_B} c_i x_i + \sum_{i \in I_N} c_i
    x_i \to \min \subjto \\
    A_B x_B + A_N x_N = \sum_{i \in I_B} A^i x_i + \sum_{i \in I_N} A^i x_i = b, \\
    x_B \ge 0, x_N \ge 0.
  \end{aligned}
}\end{equation}
bzw. nach Auflösen nach $x_B$:
\begin{equation}\groupequation{
  \begin{aligned}
    z = (c_N^\top - c_B^\top A_B^{-1} A_N) x_N + c_B^\top A_B^{-1} b \to \min
    \subjto \\
    x_B = -A_B^{-1} A_N x_N + A_B^{-1} b, \\
    x_N \ge 0, (x_B \ge 0).
  \end{aligned}
}\end{equation}
$x_B$ heißt \emph{Basisvariable} (``abhängige Variable''). $x_N$ heißt
\emph{Nichtbasisvariable} (``unabhängige Variable'').

\begin{defn} %3.1
  Der Punkt (Vektor)
  \[ x \leftrightarrow \pmat{ x_B \\ x_N } = \pmat{A_B^{-1} b \\ 0 } \]
  heißt \emph{Basislösung} zu $I_B$. Gilt zusätzlich $A_b^{-1} b \ge 0$, dann
  heißt $x$ \emph{zulässige Basislösung}.
\end{defn}

\begin{defn} %3.2
  Ein Punkt $x \in G$ heißt \emph{Ecke} von $G$, falls aus $x = \rez{2}(y+z)$
  mit $y,z \in G$ stets $x = y = z$ folgt.
\end{defn}

\clearpage

\begin{thm} %3.1
  Gilt $\rang(A) = m$, dann ist jede zulässige Basislösung eine Ecke von $G$.
  Umgekehrt gibt es zu jeder Ecke von $G$ mindestens eine Basislösung.
\end{thm}

\begin{thm} %3.2
  Sei $G \ne \emptyset$. Dann besitzt $G$
  \begin{enumerate}[(1)]
  \item mindestens eine Ecke,
  \item höchstens endlich viele Ecken.
  \end{enumerate}
\end{thm}

\begin{thm} %3.3
  Ist (3.1) lösbar, dann gibt es eine Ecke von $G$, die (3.1) löst.
\end{thm}

\begin{aus}[Optimalitätskriterium] %3.4
  Gilt für die Basislösung $x \leftrightarrow \pmat{x_B \\ x_N } =
  \pmat{A_B^{-1} b \\ 0 }$ die Bedingung $A_B^{-1} b \ge 0$ und
  \[ c_N^\top - c_B^\top A_B^{-1} A_N \ge 0, \]
  das heißt
  \[ c_i - (c_B^\top A_B^{-1})A^i \ge 0, \quad i \in I_N, \]
  dann ist $x$ Lösung von (3.1).
\end{aus}

\begin{proof}
  Sei $x$ zulässige Basislösung. Wir zeigen zunächst
  \[ Z(x) \subseteq \{ d \in \real^n : Ad = 0, d_N \ge 0 \}. \]
  Sei $d \in Z(x)$, dann existiert $\obar{t} \ge 0$ mit $0 \le t \le \obar{t}$
  \[ A(x+td) = Ax + t \cdot Ad = b, \]
  also gilt $Ad = 0$.

  Wegen $x_N \ge 0$ ergibt sich aus $x + td \overset{!}{\ge} 0$ und $x_N \ge 0$
  die Bedingung $d_N \ge 0$. Insbesondere gilt
  \[ A_B d_B + A_N d_N = 0 \qRq d_B = - A_B^{-1} A_N d_N \text{ für alle } d
    \in Z(x). \]
  Damit folgt unter Berücksichtigung von (3.3)
  \[ \nabla f(x)^\top d = c^\top d = ( c_N^\top - c_B^\top A_B^{-1} A_N) d_N \ge 0 \text{
      für alle} d \in Z(x). \]
  Also genügt $x$ den notwendigen Optimalitätsbedingungen (2.2). Wegen Aussage
  2.2 folgt dann, dass $x$ Lösung von (3.1) ist.
\end{proof}

\section{Das primale Simplexverfahren}
Das primale Simplexverfahren durchläuft zwei Phasen, falls erforderlich. \\
\emph{Phase 1} besteht in der Ermittlung einer ersten zulässigen Basislösung
(Ecke), \\
\emph{Phase 2} in der Bestimmung einer optimalen Ecke.

\subsection{Phase 2}
Wir betrachten zunächst den Fall, dass eine erste zulässige  Basislösung bekannt
sei. Zur Vereinfachung schreiben wir (3.3) in Form des \emph{Simplex-Tableaus}.
\begin{equation}
  \begin{array}{r|cc}
    \mathrm{ST}_0 & x_N & 1 \\
    \hline
    x_B = & P & p \\
    \hline
    z = & q^\top & q_0
  \end{array}
\end{equation}
mit $P = -A_B^{-1} A_N$, $p = A_B^{-1} b$, $q^\top = c_N^\top - c_B^\top A_B^{-1} A_N$,
$q_0 = c_B^\top A_B^{-1} b$.

O.B.d.A. nehmen wir (zunächst) an, dass $x_B = (x_1, \ldots, x_m)^\top$ und $x_N
= (x_{m+1}, \ldots, x_n)^\top$. Die zu $\mathrm{ST}_0$ gehörige Basislösung ist
somit $x \leftrightarrow \pmat{x_B \\ x_N} = \pmat{p \\ 0}$.

Frage: Wenn $x$ nicht optimal ist, wie kann eine bessere zulässige Lösung
gefunden werden?

Antwort: Wahl einer zulässigen Richtung, das heißt $d \in Z(x)$ mit maximaler
Schrittweite, die eine Verkleinerung des Zielfunktionals ergibt.

Nach Aussage 3.4 ist $x$ optimal, falls $q \ge 0$ gilt. Sei nun $q_\tau < 0$ für
ein $\tau \in I_N$. Wir setzen $x_\tau := t$ und verfolgen die zulässige
Richtung
\begin{equation}
  d_i := \begin{cases}
    P_{i\tau}, &i \in I_B \\
    1, &i = \tau \\
    0, &i \in I_N \setminus \{ \tau \}.
  \end{cases}
\end{equation}
Wegen der Bedingung $x(t) := x + td \ge 0$ folgt $t \ge 0$ und wegen (2.4)
\[ t \le \bar{t} := \min \left\{ \frac{-p_i}{P_{i\tau}} : P_{i\tau} < 0, i \in
    I_B \right\}, \]
\[ i \in I_B : [x_B] = p_i + P_{i\tau} t \overset{!}{\ge} 0 \]
bzw. $\bar{t} := + \infty$, falls $P_{i\tau} \ge 0$ für alle $i \in I_B$.

\begin{aus}
  Im Fall $\bar{t} = \infty$ besitzt (3.1) keine Lösung, da der
  Zielfunktionswert nach unten unbeschränkt ist.
\end{aus}

\begin{proof}
  Wegen $q_\tau < 0$ gilt
  \[ z = q^\top x_N + q_0 = \underbrace{q_\tau}_{<0} \cdot t + q_0 \xrightarrow{t
      \to \infty} -\infty \]
  und $x(t) \in G$ für $t \to \infty$.
\end{proof}

\begin{rmrk}
  Die beiden Fälle
  \begin{enumerate}[(1)]
  \item $q_i \ge 0$ für \emph{alle} $i \in I_N$ und
  \item es existiert ein $\tau \in I_N$ mit $q_\tau < 0$ $P_{i\tau} \ge 0$ für
    alle $i \in I_B$
  \end{enumerate}
  werden \emph{entscheidbar} genannt.

  Im sogenannten \emph{nicht entscheidbaren} Fall, das heißt
  \[ (\exists \tau \in I_N : q_\tau < 0)
    \wedge
    \left( \exists \sigma \in I_B : \bar{t} = - \frac{p_\tau}{P_{\sigma\tau}} =
      \min \left\{ -\frac{p_i}{P_{i\tau}} : P_{i\tau} < 0, i \in I_B \right\}
    \right) \]
  ergibt die (maximale) Schrittweite $\bar{t}$ den Punkt
  $\bar{x} := x + \bar{t} d \in G$ mit
  \[ f(\bar{x}) = q^\top \bar{x}_N + q_0 = f(x) + \bar{t} q_\tau = q_0 +
    \underbrace{\bar{t}}_{\ge 0} \cdot
    \underbrace{q_\tau}_{\le 0}
    \le q_0. \]
\end{rmrk}

\begin{thm}
  $\bar{x}$ ist eine Ecke von $G$ mit der Basis-Indexmenge
  \begin{equation}
    \bar{I}_B = I_B(\bar{x}) = (I_B \setminus \{ \sigma \} ) \cup \{ \tau
    \}, \qquad \bar{I}_N := (I_n \setminus \{ \tau \}) \cup \{ \sigma \}.
  \end{equation}
\end{thm}

Um zu zeigen, dass die Matrix $\bar{A}_B = (A^j)_{j \in \bar{I}_B}$ regulär ist,
benutzen wir das
\begin{lem}[Sherman/Morrison]
  Es seien $B \in \realmat{m}{n}$ regulär und $u,v \in \real^m$. Die Matrix
  $\bar{B} := B + uv^\top$ ist genau dann regulär, wenn $1 + v^\top B^{-1} u \ne 0$
  und es gilt
  \begin{equation}
    \bar{B}^{-1} := B^{-1} + \frac{B^{-1} u v^\top B^{-1}}{1 + v^\top B^{-1} u}.
  \end{equation}
\end{lem}

\begin{proof}
  Übung.
\end{proof}

\begin{proof}[Beweis zu Satz 3.6]
  Der ``Austausch'' der Spalten $A^\sigma$ und $A^\tau$ kann durch ein
  dyadisches Produkt $uv^\top$ beschrieben werden:
  \[ \bar{A}_B = A_B + uv^\top \quad \text{mit} \quad u = A^\tau - A^\sigma, v =
    e^\sigma. \]
  Dabei ist $e^\sigma$ der $\sigma$-te Einheitsvektor.

  Wegen
  \[ \begin{aligned}
      1 + v^\top A_B^{-1} u
      &= 1 + (e^\sigma)^\top (A_B^{-1}) (A^\sigma - A^\tau) \\
      &= 1 + (e^\sigma)^\top A_B^{-1} A^\tau - 1 \\
      &= - P_{\sigma \tau} \ne 0
    \end{aligned} \]
  folgt mit dem Lemma von Sherman/Morrison die Regularität von $\bar{A}_B$.
\end{proof}

\begin{exmp}
  Betrachtet werde
  \[ \begin{aligned} z = x_1 + x_2 \to \max \subjto x_1 + 2 x_2 &\le 6, \\
      4x_1 + x_2 &\le 10, \\
      x_1, x_2 \ge 0.
    \end{aligned}
  \]
  Schlupfvariable: $x_3, x_4 \ge 0$.
  \[ x_1 + 2x_2 + x_3 = 6, \quad 4x_1 + x_2 + x_4 = 10. \]
  Umwandlung zu Minimierungsaufgabe:
  \[ -z = -x_1 - x_2 \to \min. \]
  Wahl
  \begin{align*}
    I_B &= \{3,4\}, & I_N &= \{1,2\}, & A_B &= \pmat{ 1 & 0 \\ 0 & 1 }, & \ldots
    \\
    c_B^\top &= (0,0), & c_N^\top &= (-1,-1), & x_B &= (x_3, x_4)^\top
  \end{align*}
  \begin{center}
    \begin{tabular}{r|ccc}
      $\mathrm{ST}_0$ & $x_1$ & $x_2$ & 1 \\
      \hline
      $x_3$ & $-1$ & $-2$ & $6$ \\
      $x_4$ & $-4$ & $-1$ & $10$ \\
      \hline
      $-z$ & $-1$ & $-1$ & $0$
    \end{tabular}
  \end{center}
  Mit $\tau = 1$ erhält man $\sigma = 4$, $\bar{t} = \frac{5}{2}$.
  \[ \bar{x} = x + td = \pmat{ 0 \\ 0 \\ 6 \\ 10 } + \frac{5}{2}
    \pmat{1 \\ 0 \\ -1 \\ -4} =\pmat{ 5/2 \\ 0 \\ 7/2 \\ 0 } \quad \text{(Ecke
      von $G$)} \]
  \[ Z(\bar{x}) = -1 \cdot \bar{t} + 0 = -\frac{5}{2} < 0. \]
\end{exmp}

Der Austausch von $x_\tau$ mit $x_\sigma$ im Simplexverfahren kann durch die
sogenannten \emph{Austauschregeln} erfolgen:
\begin{center}
  \begin{tabular}{r|cc}
    $\mathrm{ST}_0$ & $x_N$ & 1 \\
    \hline
    $x_B$ & $P$ & $p$ \\
    \hline
    $z$ & $q^\top$ & $q_0$
  \end{tabular}
  \hspace{1cm}
  $\Rightarrow$
  \hspace{1cm}
  \begin{tabular}{r|cc}
    $\mathrm{ST}_1$ & $\bar{x}_N$ & 1 \\
    \hline
    $\bar{x}_B$ & $\bar{P}$ & $\bar{p}$ \\
    \hline
    $z$ & $\bar{q}^\top$ & $\bar{q}_0$
  \end{tabular}  
\end{center}
mit $\bar{I}_B := (I_B \setminus \{ \sigma \} ) \cup \{ \tau \}$, $\bar{I}_N :=
(I_N \setminus \{ \tau \} ) \cup \{ \sigma \}$.

Austauschregeln:
\begin{align*}
  \bar{P}_{\sigma \tau}
  &:= \rez{P_{\sigma \tau}} \\
  \bar{P}_{\sigma j}
  &:= - \frac{P_{\sigma j}}{P_{\sigma \tau}}, \quad
    j \in I_N \setminus \{ \tau \},
  & \bar{p}_\sigma
  &:= - \frac{p_\sigma}{P_{\sigma \tau}}, \\
  \bar{P}_{i \tau}
  &:= \frac{P_{i \tau}}{P_{\sigma \tau}}, \quad
    i \in I_B \setminus \{ \sigma \},
  & \bar{q}_\tau
  &:= \frac{q_\tau}{P_{\sigma \tau}}, \\
  \bar{P}_{ij}
  &:= P_{ij} - \frac{P_{\sigma j}}{P_{\sigma \tau}} \cdot P_{i \tau}, \quad
    i \in I_B \setminus \{\sigma\}, \quad j \in I_N \setminus \{ \tau \}, \\
  \bar{q}_j
  &:= q_j - \frac{P_{\sigma j}}{P_{\sigma \tau}} q_\tau, \quad
    j \in I_N \setminus \{ \tau \}, \\
  \bar{q}_0
  &:= q_0 - \frac{ p_\sigma }{ P_{\sigma \tau} \cdot q_\tau}.
\end{align*}

\begin{exmp}
  Wie in 3.1 \\[1.5em]
  \begin{tabular}{r|rrr}
    $\mathrm{ST}_0$ & $x_1$ & $x_2$ & 1 \\
    \hline
    $x_3$ & $-1$ & $-2$ & $6$ \\
    $x_4$ & $-4$ & $-1$ & $10$ \\
    \hline
    $-z$ & $-1$ & $-1$ & $0$ \\
    \hline
    $k$ & $-1/2$ & & $3$
  \end{tabular}
  \hspace{1cm}
  $\tau = 2$, $\bar{t} = 3$ \\[1.5em]
  \begin{tabular}{r|rrr}
    $\mathrm{ST}_1$ & $x_1$ & $x_3$ & 1 \\
    \hline
    $x_2$ & $1/2$ & $-1/2$ & $3$ \\
    $x_4$ & $-7/2$ & $1/2$ & $7$ \\
    \hline
    $-z$ & $-1/2$ & $1/2$ & $-3$ \\
    \hline
    $k$ & & $1/7$ & $2$
  \end{tabular} \\[1.5em]
  \begin{tabular}{r|rrr}
    $\mathrm{ST}_2$ & $x_4$ & $x_3$ & 1 \\
    \hline
    $x_2$ & & & $2$ \\
    $x_1$ & & & $2$ \\
    \hline
    $-z$ & $1/7$ & $3/7$ & $-4$
  \end{tabular}
  \hspace{1cm}
  $q = \pmat{1/7 \\ 3/7}$.

  $\mathrm{ST}_2$ ist optimal, da $q \ge 0$. (Eine) Lösung: $x^* = (2,2,0,0)^\top$.
  $-z_{\min} = -4$, $z_{\max} = 4$. $x^*$ ist eindeutige Lösung, da $g_j > 0$
  für alle $j \in I_N = \{3,4\}$.
\end{exmp}

\subsection{Phase 1}
Wir betrachten das Problem
\begin{equation} %% 3.8
  z = c^\top x \to \min \subjto Ax = b, x \in \real^n_+.
\end{equation}
Falls nicht einfach möglich, kann durch folgendes Hilfsproblem eine erste
zulässige Basislösung gefunden werden, falls eine existiert.
\begin{equation} %% 3.9
  h = e^\top y \to \min \subjto y + Ax = b, x \in \real^n_+, y \in \real^n_+
\end{equation}
mit $e =(1, \ldots, 1)^\top \in \real^m$.

Die erste Basislösung zu (3.9) ist
\begin{equation}
  \begin{array}{r|cc}
    & x & 1 \\
    \hline
    y = & -A & b \\
    h = & -e^\top A & e^\top b
  \end{array}
\end{equation}

\begin{thm}
  Das Ausgangsproblem (3.8) besitzt genau dann eine zulässige Lösung, wenn
  $h_{\min} = 0$ den Optimalwert von (3.9) bildet.
\end{thm}

\begin{proof}
  Nach Definition der Hilfszielfunktion gilt $h_{\min} = 0$ genau dann, wenn
  $y=0$. Besitzt (3.8) eine zulässige Lösung $\bar{x}$, dann ist $\pmat{\bar{x}
    \\ \bar{y}}$ mit $\bar{y} = 0$ zulässig für (3.9). Wegen $0 \le h \le e^\top
  \bar{y} = 0$ ist $\pmat{\bar{x} \\ \bar{y}}$ optimal mit $h_{\min} = 0$.

  Hat man umgekehrt $h_{\min} = 0$,  so gilt $\bar{y} = 0$ für jede (optimale)
  Lösung von (3.9). Aus der Zulässigkeit von $\pmat{\bar{x} \\ \bar{y}}$ folgt
  die Zulässigkeit von $\bar{x}$ für (3.8).
\end{proof}

\subsection{Das revidierte Simplexverfahren}
Bei der in Abschnitt 3.2.1 vorgestellen Vorgehensweise wird je Simplexschritt
die Nichtbasismatrix $A_N$, die das Format $n \times (n-m)$, transformiert,
während im Abschnitt 3.1 eine $m \times m$-Matrix aufdatiert wird. Ist $n \gg
m$, kann letzteres effizienter sein.

$A^j$ bezeichne wieder die $j$-te Spalte von $A$.

\paragraph{Revidiertes Simplexverfahren}
\begin{enumerate}[(1)]
  \setcounter{enumi}{-1}
\item Ermittle eine erste zulässige Basislösung $x_b = p$, $x_N = 0$ mit $p =
  (A_B)^{-1} b$ und Indexmengen $I_B$, $I_N$ sowie $A_B = (A^j)_{j \in I_B}$.
\item Optimalitätstest: Berechne entsprechend der Aussage 3.4
  \[ \bar{c} := \min_{j \in I_N} \bar{c}_j := \min_{j \in I_N} c_j - d^\top A^j, \]
  wobei $d^\top := c_B^\top A_B^{-1}$.

  Gilt $\bar{c} \ge 0$, dann ist $\pmat{x_B \\ x_N} = \pmat{p \\ 0}$ Lösung von
  (3.1). Anderenfalls sei $\bar{c} = \bar{c}_\tau$ ($\tau \in I_N$).
\item Berechne die transformierte Spalte von $A^\tau$: $\bar{a} := -(A_B^{-1})
  A^\tau$. Falls  $\bar{a}_i \ge 0$ für alle $i \in I_B$, dann hat (3.1) keine
  Lösung ($z$ ist nach unten unbeschränkt). Bestimme $\sigma$ gemäß
  \[ \frac{ p_\sigma }{ \bar{a}_\sigma } := \min \left\{ \frac{p_i}{\bar{a_i}} :
      \bar{a_i} < 0, i \in I_B \right\}. \]
\item Ersetze die Spalte $A^\sigma$ durch $A^\tau$ in $A_B$, aktualisiere $I_B$,
  $I_N$ und berechne das neue $p$ (bzw. neue $A_B^{-1}$) und gehe zu (1).
\end{enumerate}

\begin{rmrk}
  In ``guten'' Implementierungen werden lineare Gleichungssysteme gelöst.
  \[ A_B p = b, \quad A_B^\top d = c_B, \quad A_B  \bar{a} = -A^\tau \]
  mit $LU$-Zerlegung und entsprechender Aufdatierung.
\end{rmrk}

\begin{rmrk}
  Der Schritt (1) kann zu Aufwandseinsparungen führen, wenn $A$ eine
  \emph{sparse} Matrix ist (wie zum Beispiel beim Transportproblem) oder eine
  Struktur besitzt, die die Anwendung der sogenannten \emph{Spaltengenerierung}
  erlaubt.
\end{rmrk}

\subsection{Spaltengenerierung}
Wir betrachten die \emph{$LP$-Relaxation} des Gilmore/Gomory-Modells beim
1-dimensionalen CSP\footnote{cutting stock problem, S. \pageref{sect:csp}}: Aus
möglichst wenig Material der Länge $L$ sind $b_i$ Teile der Länge $l_i$, $i \in
I = \{1, \ldots, m\}$ zuzuschneiden.

Zuschnittvariante
\[ a^j = (a_{1j}, \ldots, a_{nj})^\top \in \integer^m_+ \quad \text{mit} \quad
  \sum_{i \in I} l_i a_{ij} \le L, \]
$x_j \in \integer_0$ ... Häufigkeit, wie oft Zuschnittvariante $a^j$ in der
Lösung verwendet wird.

GG-Modell der LP-Relaxation:
\[ \begin{aligned}
    z = \sum_{j \in J} x_j \to \min \subjto
    \sum_{j \in J} a_{ij} x_j &= b_i  \quad \forall i \in I, \\
    x_j &\ge 0 \quad \forall j \in J.
  \end{aligned}
\]
Hier: $c = e = (1, \ldots, 1) \in \real^{|J|}$, $J$ ist die Indexmenge aller
zulässigen Zuschnittvarianten. $A = (A^1, \ldots, A^{|J|}) \in
\realmat{m}{|J|}$, wobei die $A^j$ in $A$ nur für $\sum_{i \in I} l_i A_{ij} \le
L$. 

Da $(A_B)^{-1} A_j$ für $j \in J$ einen Einheitsvektor ergibt, gilt stets
\[ \bar{c} = \min_{j \in J_N} (c_j - d^\top A^j) < 0 \qLRq 
  \min_{j \in J} c_j - d^\top A^j < 0. \]

Schritt (1) im revidierten Simplexverfahren ist nun: Bestimme
\[ \begin{aligned}
    \min_{j \in J_N} \{ c_j - d^\top A^j \} 
    &= \min_{j \in J_N} \{ 1 - d^\top A^j \} \\
    &= 1 - \max_{j \in J_N} d^\top A^j \\
    &= 1 - \max_{j \in J} d^\top A^j, & \text{falls } d^\top A^j > 1.
  \end{aligned} \]

Im Fall des 1-dimensionalen CSP gilt damit mit $l = (l_1, \ldots, l_m)$
\[ \max_{j \in J} d^\top A^j = \max \{ d^\top a : l^\top a \le L, a \in
  \integer_0^m \}, \]
das heißt es ist ein lineares Rucksackproblem zu lösen.

\section{Das duale Simplexverfahren}
Nach  Aussage 3.4 ist ein Simplextableau optimal, wenn $p \ge 0$ und $q \ge 0$
gilt. Nach Konstruktion gilt beim primalen Simplexverfahren stets $p \ge 0$.
Beim dualen Simplexverfahren soll nun stets $q \ge 0$ gelten.

Sei nun ein Tableau mit $q \ge 0$ und $p \ngeq 0$ gegeben, das heißt es gibt ein
$\sigma \in I_B$ mit $p_\sigma < 0$ (formal richtige Schreibweise:
$[x_B]_\sigma, [x_N]_\tau$). Die zu $\ST_\sigma$ gehörige Basislösung ist
somit \emph{nicht} zulässig.

Unter Beibehaltung von $q \ge 0$ für jedes erzeugte Simplextableau soll eine
Basislösung gefunden werden, sofern eine existiert. Entsprechend der
Austauschregeln ergeben sich die Bedingungen
\[ \tilde{q} := q_j - \frac{p_{\sigma j}}{p_{\sigma \tau}} q_\tau \ge 0, \quad
  \text{für alle } j \in I_N \setminus {\tau}, \]
\[ \tilde{q}_\tau := \frac{q_\tau}{P_{\sigma \tau}} \overset{!}{\ge} 0, \qquad
  \tilde{p}_\sigma := - \frac{p_\sigma}{p_{\sigma \tau}} \overset{!}{\ge} 0. \]
Wegen $p_\sigma < 0$ ist somit $p_{\sigma \tau} > 0$ (Pivot) zu wählen mit
\[ \frac{q_{\tau}}{p_{\sigma \tau}} = \min \left\{ \frac{q_j}{p_{\sigma j}} :
    p_{\sigma j} > 0, j \in I_N \right\}. \]

\begin{rmrk}
  Im Unterschied zum primalen SV, bei dem die Folge der Zielfunktionswerte nicht
  wachsend (fallend) ist, ist diese beim dualen ZV nichtfallend (wachsend).
\end{rmrk}

\begin{rmrk}
  Falls eine zulässige Lösung gefunden wird, das heißt $p \ge 0$ wird erreicht,
  dann ist diese optimal.
\end{rmrk}

\begin{exmp}
  Betrachte
  \[ \begin{aligned}
      z = 6 x_1 + 5 x_2 + 12 x_3 + 8 x_4 + 9 x_5 \to \min \subjto \\
      x_1 + x_3 + x_4 + x_5 &\ge 300, \\
      x_2 + 2 x_3 + x_4 &\ge 400, \\
      \forall i : x_i &\ge 0.
  \end{aligned}
\]
\begin{align*}
    &\begin{array}{r|cccccc}
      \ST_1 & x_1 & x_2 & x_3 & x_4 & x_5 & 1 \\
      \hline
      x_6 = & 1 & 0 & 1 & 1 & 1 & -300 \\
      x_7 = & 0 & \boxed{1} & 2 & 1 & 0 & -400 \\
      \hline
      z =   & 6 & 5 & 12 & 8 & 9 & 0 \\
      \hline
      (x_z =) k & 0 & \ast & -2 & -1 & 0 & 400 
    \end{array}
    &\quad &\leftarrow \sigma = 2 \\
    &\begin{array}{r|cccccc}
      \ST_2 & x_1 & x_7 & x_3 & x_4 & x_5 & 1 \\
      \hline
      x_6 = & 1 & 0 & \boxed{1} & 1 & 1 & -300 \\
      x_2 = & 0 & 1 & -2 & -1 & 0 & 400 \\
      \hline
      z = 6 & 5 & 2 & 3 & 9 & 2000 \\
      \hline
      (x_3 = ) k & -1 & 0 & \ast & -1 & -1 & 300
    \end{array}
    &\quad &\leftarrow \sigma = 3 \\
    &\begin{array}{r|cccccc}
       \ST_3 & x_1 & x_7 & x_6 & x_4 & x_5 & 1 \\
       \hline
       x_3 = & -1 & 0 & 1 & -1 & -1 & 300 \\
       x_2 = & 2 & -2 & -2 & \boxed{1} & 2 & -200 \\
       \hline
       z = & 4 & 5 & 2 & 1 & 7 & 2600 \\
       \hline
      (x_4 = ) k & -2 & 2 & 2 & \ast & -2 & 200 \\
    \end{array}
    &\quad &\leftarrow \sigma = 4
\end{align*}
\[ \begin{array}{r|cccccc}
      \ST_4 & x_1 & x_7 & x_6 & x_2 & x_5 & 1 \\
      \hline
      x_3 = \\
      x_2 = \\
      \hline
      z = & 2 & 7 & 4 & 1 & 5 & 2800
    \end{array}
\]
$ST_4$ ist optimal.
\[ x^* = (0, 0, 100, 200, 0)^\top,  \qquad x_6 = x_7 = 0, \qquad z_{\min} =
  2800. \]
$x_6 = x_7 = 0$ $\rightsquigarrow$ Die Ungleichungsbedingungen werden mit
Gleichheit erfüllt!
\end{exmp}

\section{Dualität}
\subsection{Herleitung dualer Optimierungsaufgaben}
Wir betrachten die Optimierungsaufgabe
\addtocounter{equation}{3}
\begin{equation} %% 3.13
  \begin{aligned} z = c^\top x \to \min \subjto Ax \ge b, x \ge 0, \\
    I =\{ 1, \ldots, m \}, J = \{1, \ldots, n \}, A \in \realmat{m}{n}.
  \end{aligned}
  \tag{P, \theequation}
\end{equation}

\begin{thm}[Charakterisierungssatz]
  Ein $\bar{x} \in \real^n$ ist genau dann Lösung von (P), wenn ein $\bar{u}$
  existiert, so dass insgesamt das System erfüllt ist:
  \begin{align*}
    A \bar{x} - \bar{b} &\ge 0, & \bar{x} &\ge 0, \tag{1} \\
    A^\top \bar{u} - c &\le 0, & \bar{u} &\ge 0, \tag{2} \\
    \bar{u}^\top (A \bar{x} - b) &= 0, & \bar{x}^\top (A^\top \bar{u} - c) &= 0. \tag{3}
  \end{align*}
\end{thm}

\begin{proof}
  Wegen (1) ist $\bar{x}$ zulässig. Weiterhin muss $\bar{x}$ die notwendigen
  Optimalitätsbedingungen (\ref{eq:kkt-bed}) aus Abschnitt \ref{sect:kkt-bed}
  erfüllen, falls es Lösung von (P) ist.

  Es seien $I_0(\bar{x}) := \{ i \in I : [Ax - b]_i = 0 \}$, $J_0(\bar{x}) = \{
  j \in J : \bar{x}_j = 0 \}$. Damit kann der Kegel der zulässigen Richtungen
  $Z(\bar{x})$ wie folgt beschrieben werden:
  \[ Z(\bar{x}) = \{ d \in \real^n : d^\top a_i \ge 0, i \in I_0(\bar{x}),
    d^\top e^j \ge 0, j \in J_0(\bar{x}) \}, \]
  wobei $a_i \in \real^n$ die $i$-te Zeile von $A$ bezeichnet.

  Falls $\bar{x}$ Lösung von (P) ist, dann existiert \emph{kein} $d \in
  Z(\bar{x})$ mit $d^T c < 0$. Mit dem Farkas-Lemma (S. \pageref{lem:farkas})
  ist dies äquivalent zu
  \[ \exists u_i \ge 0, \quad i \in I_0 (\bar{x}), \qquad
    \bar{v}_j, \quad j \in J_0(\bar{x}) : \qquad 
    \sum_{i \in I_0(\bar{x})} \bar{u}_i a_i + \sum_{j \in J_0(\bar{x})} \bar{v}_j
    e^j = c \]
  Durch ``Auffüllen mit 0'' erhält man die äquivalente Formulierung
  \[ \exists \bar{u} \in \real^n_*, \quad
    \bar{v} \in \real^n_+: \qquad
    A^\top u + v = c. \tag{4} \]
  \[ \bar{u}_i [A \bar{x} - b ]_i = 0, \quad i \in I, \qquad
    \bar{v}_j \bar{x}_j = 0, \quad j \in J. \tag{5} \]
  Weitere Umformungen ergeben
  \[ (4) \text{ und } \bar{v} \ge 0 \qLRq
    (-v) = A^\top u - c \ge 0, \]
  also (2). Wegen $\bar{x} \ge 0$, $\bar{u} \ge 0$:
  \[ (5) \qLRq \bar{u}^\top (A \bar{x} - b) = 0, \qquad
    \bar{x}^\top (A^\top \bar{u} - c) = 0, \]
  also (3).

  Da nur Äquivalenz-Transformationen benutzt wurden, folgt die Gültigkeit des
  Satzes.
\end{proof}

\begin{defn}
  Das Problem
  \[ w = b^\top u \to \max \subjto A^\top u \le c, \quad u
    \ge 0 \tag{D} \]
  heißt \emph{duale Optimierungsaufgabe} zu (P).
\end{defn}

\emph{Begründung:} Die Anwendung von Satz 3.9 ergibt wieder das Problem (P).
Dazu schreibe (D) als Minimum-Aufgabe
\[ -w = (-b)^\top u \to \min \subjto (-A)^\top u \ge -c, \quad u \in
  \real^m_+. \]
Damit hat es die Form von (P) in Satz 3.9.

Einsetzen in (1) bis (3):
\[ \begin{aligned}
    (1) &\Rightarrow & (-A)^\top u - (-c) &\ge 0, & u &\ge 0 \\
    (2) &\Rightarrow & ((-A)^\top)^\top v - (-b) &\le 0 & v &\ge 0 \\
    (3) &\Rightarrow & v^\top ((-A)^\top u - (-c)) &= 0, & u^\top
    ((-A)^\top)^\top v - (-b) &= 0.
  \end{aligned} \]
Umformung ergibt
\[ \begin{aligned}
    (1) &\Rightarrow & A^\top u - c &\le 0, & u &\ge 0 \\
    (2) &\Rightarrow & A v - b) &\ge 0 & v &\ge 0 \\
    (3) &\Rightarrow & v^\top (A^\top u - c) &= 0, & u^\top
    (Ax - b) &= 0. 
  \end{aligned} \]
(D) liefert also das \emph{gleiche} System (1)-(3) wie (P).

\begin{thm}[Schwache Dualität]
  Gegeben seien die Optimierungsaufgaben
  \begin{align*}
    c^\top x &\to \min \subjto & Ax \ge b, x \ge 0, \tag{P} \\
    b^\top u &\to \max \subjto & A^\top \le c, u \ge 0. \tag{D}
  \end{align*}
  Sei $x$ zulässig für (P) und $u$ zulässig für (D). Dann gilt $b^\top u \le
  c^\top x$.
\end{thm}

\clearpage

\begin{thm}[Starke Dualität]
  Das lineare Optimierungsproblem (P) ist genau dann lösbar, wenn das zugehörige
  duale Problem (D) lösbar ist. Für Lösungen $\bar{x}$ von (P) und $\bar{u}$ von
  (D) gilt dann
  \[ \boxed{b^\top u = c^\top x }, \]
  also Gleichheit der Optimalwerte.
\end{thm}

\begin{proof}
  Der erste Teil des Satzes folgt aus Satz 3.10. Aus (3) folgt die Gleichheit
  der Optimalwerte.
  \[ \bar{u}^\top ( A \bar{x} - b ) = 0, \qquad \bar{x}^\top (A^\top \bar{u} -
    c) = 0, \qquad \bar{u}^\top b = \bar{u}^\top A \bar{x} = c^\top \bar{x}.
    \qedhere \]
\end{proof}

\begin{flg}
  Es gilt:
  \[ \text{(P) lösbar} \qLRq
    \text{(Q) lösbar} \qLRq
    \exists x \ge 0, u \ge 0 : Ax \ge b, A^\top u \le c. \]
\end{flg}

\subsection{Gleichzeitiges Lösen von (P) und (Q)}
\begin{align*}
  z_P &= c^\top x \to \min & &\text{bei} & Ax \ge b, \quad x \ge 0 \tag{P} \\
  z_D &= -b^\top u \to \max & &\text{bei} & -A^\top u \le c, \quad u
        \ge 0 \tag{D}
\end{align*}
Einführen von Schlupfvariablen:
\begin{align*}
  z_P &= c^\top x \to \min & &\text{bei} & Ax + s = b, \quad x \ge 0, \quad s \ge 0 \tag{P} \\
  z_D &= b^\top (-u) \to \max & &\text{bei} & A^\top (-u) + v = c, \quad u
        \ge 0, \quad v \ge 0 \tag{D}
\end{align*}
Es gilt $Ax \le b$ $\Leftrightarrow$ $-Ax \ge -b$.
\[ \begin{array}{r|cc}
     (\mathrm{P}) & x & 1 \\
     \hline
     s = & -A & b \\
     z_P = & c^\top & 0
   \end{array}
   \qquad
   \begin{array}{r|cc}
     (\mathrm{D}) & -u & 1 \\
     \hline
     v = & -A^\top & c \\
     z_D = & b^\top & 0
   \end{array}
\]
Beide Schemata sind ``zueinander transponiert''. Damit gilt: primales Schema
für (P) $\hat{=}$ duales Schema für (D).

\begin{exmp}[Fortsetzung von Beispiel 3.3]
  \begin{align*}
    z &= c^\top x \to \min & &\text{bei} & Ax \ge b, \quad x \ge 0 \tag{P} \\
    z &= b^\top u \to \max & &\text{bei} & A^\top u \le c, \quad u
                                              \ge 0 \tag{D}
  \end{align*}
  $m=2$, $n=5$.
  \[
    \begin{array}{r|ccc}
      & u_1 & u_2 & 1 \\
      \hline
      v_1 = & \boxed{-1} & 0 & 6 \\
      v_2 = & 0 & 1 & 5 \\
      v_3 = & -1 & -2 & 12 \\
      v_4 = & -1 & -1 & 8 \\
      v_5 = & -10 & 0 & 9 \\
      \hline
      z_D = & -300 & -400 & 0 \\
      \hline
      k & \ast & 0 & 6
    \end{array}
    \qquad
    \begin{array}{r|ccc}
      & v_1 & u_2 & 1 \\
      \hline
      u_1 = & -1 & 0 & 6 \\
      v_2 = & 0 & -1 & 5 \\
      v_3 = & 1 & -2 & 6 \\
      v_4 = & 1 & \boxed{-1} & 2 \\
      v_5 = & 1 & 0 & 3 \\
      \hline
      z_D = & 300 & -400 & -1800
    \end{array}
  \]
  $u_2 \leftrightarrow v_4$, $v_1 \leftrightarrow v_3$ $\Rightarrow$
  \[
    \begin{array}{r|ccc}
      & v_3 & v_4 & 1 \\
      \hline
      u_1 = & & & 4 \\
      v_2 = & & & 1 \\
      v_1 = & & & 2 \\
      u_2 = & & & 4 \\
      v_5 = & & & 5 \\
      \hline
      z_D = & 100 & 200 & -2800
    \end{array}
  \]
  %% Marvin nochmal fragen
  Lösung:
  \begin{align*}
    u_1 &= 4 & u_1 &= 4 \\
    u_2 &= 4 & u_2 &= 4 \\
    v &= (2, 1, 0, 0, 5)^\top & v &= (2, 1, 0, 0, 5)^\top
  \end{align*}
\end{exmp}

\subsection{Zusammenhang Dualität - Sattelpunkte der Lagrange-Funktion}
Zugehörig zum Optimierungsproblem
\[ f(x) \to \min \subjto g_i (x) \le 0, \quad i \in I = \{1, \ldots, m \} \]
wird die \emph{Lagrange-Funktion}
\[ L(x,u) := f(x) + u^\top g(x) = f(x) + \sum_{i \in I} u_i g_i(x), \quad x \in
  \real^n, u \in \real^m_+ \]
definiert. Für das lineare Optimierungsproblem (P) in (3.13), das heißt für
\[ z = c^\top x \to \min \subjto Ax \ge b, \quad x \ge 0 \]
erhält man
\addtocounter{equation}{3}
\begin{equation}
  L(x,u) = c^\top x + u^\top (b-Ax) = b^\top u + x^\top (c - A^\top u), \quad
  x \ge 0, u \ge 0.
\end{equation}

\begin{defn}
  Ein Punkt $(\bar{x}, \bar{u}) \in \real^n_+ \times \real^m_+$ heißt
  \emph{Sattelpunkt} von $L$, falls die folgende Bedingung erfüllt ist:
  \[ L( \bar{x}, u ) \le L(\bar{x},\bar{u}) \le L(x, \bar{u} )\]
  für alle $x \in \real^n_+$, $u \in \real^m_+$.
\end{defn}

\begin{thm}[Sattelpunktstheorem]
  Ein Punkt $(\bar{x}, \bar{u}) \in \real^n_+ \times \real^m_+$ ist genau dann
  ein Sattelpunkt von $L$, wenn $\bar{x}$ und $\bar{u}$ Lösungen von (P) bzw.
  (D) sind.
\end{thm}

\begin{proof}
  Unter Verwendung von (3.17)\footnote{$L$ ist linear und damit konvex.} erhält
  man 
  \begin{align*}
    &
    & L( \bar{x}, \bar{u} ) &\le L(x, \bar{u})
    & \forall x \in \real^n_+ \\
    &\Leftrightarrow
    & \partial_x L( \bar{x}, \bar{u})^\top (x-\bar{x}) &\ge 0
    & \forall x \in \real^n_+ \\
    &\Leftrightarrow
    & (c - A^\top \bar{u})^\top (x-\bar{x}) &\ge 0
    & \forall x \in \real^n_+ \\
    \intertext{Speziell: $x^1 = 2 \bar{x}, x^2 = \rez{2} \bar{x}$,
    $\bar{x} \ge 0$ $\Rightarrow$ $x^1 \ge 0, x^2 \ge 0$}
    &\Leftrightarrow
    & (c - A^\top \bar{u})^\top \bar{x} &= 0, \quad c - A^\top \bar{u} \ge 0.
  \end{align*}
  Analog für den zweiten Teil:
  \begin{align*}
    &
    & L( \bar{x}, \bar{u} ) &\le L(\bar{x}, u)
    & \forall u \in \real^m_+ \\
    &\Leftrightarrow
    & \partial_u L( \bar{x}, \bar{u})^\top (u-\bar{u}) &\le 0
    & \forall u \in \real^m_+ \\
    &\Leftrightarrow
    & (b - A \bar{x})^\top (u-\bar{u}) &\le 0
    & \forall u \in \real^m_+ \\
    \intertext{Speziell: $u^1 = 2 \bar{u}, u^2 = \rez{2} \bar{u}$,
    $\bar{u} \ge 0$ $\Rightarrow$ $u^1 \ge 0, u^2 \ge 0$}
    &\Leftrightarrow
    & (b - A \bar{x})^\top \bar{u} &= 0, \quad b - A \bar{x} \le 0.
  \end{align*}
  Insgesamt gilt damit $(\bar{x}, \bar{u}) \in \real^n_+ \times \real^m_+$ ist
  genau dann Sattelpunkt, wenn $\bar{x}$ und $\bar{u}$ Lösung von (1) bis (3)
  sind.
\end{proof}

\section{Transportoptimierung}
\subsection{Problemstellung}
Es gebe Erzeuger $i \in I = \{1, \ldots, r\}$ und Verbraucher $k \in K = \{1,
\ldots, s\}$. Weiterhin seien die Kosten $c_{ik}$ für den Transport einer
Einheit von $i$ nach $k$ sowie der Vorrat $a_i > 0$ und der Bedarf $b_k > 0$ für
alle $i \in I$ und $k \in K$ bekannt. Wie muss der Transport erfolgen, damit
alle Bedarfe erfüllt werden und die Gesamtkosten minimal sind?

Variablen $x_{ik} \in \real_+$ für alle $i \in I$, $k \in K$.
\begin{equation}\groupequation{
  \begin{aligned}
    & & z = &\sum_{i \in I} \sum_{k \in K} c_{ik} x_{ik} \to \min \\
    &\text{bei} &
    &\sum_{k \in K} x_{ik} = a_i, \quad i \in I, \\
    & &
    &\sum_{i \in I} x_{ik} = b_k, \quad k \in K.
  \end{aligned}
}\end{equation}
Mit $x = (x_ {11}, x_{12}, \ldots, x_{1s}, x_{21}, \ldots, x_{2s}, \ldots,
x_{r1}, \ldots, x_{rs})^\top \in \real^{r \cdot s}$ hat man also $n = r \cdot s$
Variablen und $m = r + s$ Gleichungen. Damit ergibt sich die Form
\begin{equation}\groupequation{
  \begin{aligned}
    &z = c^\top x \to \min \subjto Ax = \tilde{b}, \quad x \ge 0, \\
    &A = \begin{pmatrix}
      1 & \cdots & 1 & 0 & \cdots & 0 \\
      0 & \cdots & 0 & 1 & \cdots & 0 \\
      & & \vdots & \vdots \\
      0 & \cdots & 0 & 0 & \cdots & 1 \\
      1 & & & 1 \\
      & \ddots & & &\ddots \\
      & & 1 & & & 1
    \end{pmatrix},
    \qquad
    \tilde{b} = \begin{pmatrix}
      a_1 \\ a_2 \\ \vdots \\ a_r \\ b_1 \\ \vdots \\ b_s
    \end{pmatrix},
    \qquad
    c = \pmat{c_{11} \\ c_{21} \\ \vdots \\ c_{r,s-1} \\ c_{1s} \\ \vdots \\ c_{rs}}
  \end{aligned}
}\end{equation}
Zeile $A^i$ für $i = 1, \ldots, r$ bildet die
Vorratsbedingungen ab:
\[a_{ij} = \begin{cases}
    1, &j = 1 + (i-1) \cdot s, \ldots, i \cdot s, \\
    0, &\text{sonst.} 
  \end{cases}
\]
Zeile $A^i$ für $i = r+1, \ldots, r+s$ bildet die Bedarfsbedingungen ab. Es sind
also $s$ Einheitsmatrizen.

\begin{rmrk} % 3.9
  Das Transportproblem ist ein (sehr) spezielles Optimierungsproblem. Die
  Koeffizientenmatrix $A$ hat $m = r + s$ Zeilen und $n = r \cdot s$ Spalten und
  ist schwach besetzt. Insbesondere hat die Spalte von $A$, die zu $x_{ik}$
  geört, die Gestalt $A^{ik} = \pmat{ e^i & \ldots & e^k}^\top \in \real^{r+s}$.
\end{rmrk}

\begin{thm} %% 3.14
  Das Transportproblem ist genau dann lösbar, wenn die
  \emph{Sättigungsbedingung}
  \begin{equation} %% 3.20
    \sum_{i=1}^r a_i = \sum_{k=1}^s b_k
  \end{equation}
  gilt.
\end{thm}

\begin{proof}
  \begin{enumerate}[(i)]
  \item Wir zeigen zuerst
    \[ G \ne \emptyset \qLRq  \sum_{i=1}^r a_i = \sum_{k=1}^s b_k. \]
    Ist $G$ nicht leer, so folgt
    \[ \sum_{i=1}^r a_i = \sum_{i=1}^r \sum_{k=1}^s x_{ik} =
      \sum_{k=1}^s \sum_{i=1}^r x_{ik} = \sum_{k=1}^s b_k, \]
    also ist (3.20) erfüllt.

    Gilt (3.20) mit $\delta = \sum_i a_i$, dann ist
    \[ x_{ik} := \frac{ a_i b_k }{\delta} \in G, \]
    also ist $G \ne \emptyset$.
  \item Der zulässige Bereich $G$ ist polyedrisch und damit abgeschlossen. Wegen
    $0 \le x_{ik} \le \min \{a_i, b_k\}$ ist er auch beschränkt und damit
    kompakt. Nach dem Satz von Weierstrass folgt die Lösbarkeit, falls (3.20)
    gilt. \qedhere
  \end{enumerate}
\end{proof}

Die duale Aufgabe zu (3.18) bzw. (3.19) lautet
\begin{equation} %3.21
  w = a^\top u + b^\top v \to \max \subjto A^\top \pmat{u\\v} \le c, \quad
  u \in \real^r_+, \quad v \in \real^s_+.
\end{equation}

Es gilt
\begin{align*}
  &\begin{aligned} (\text{P}) \quad
    &z = c^\top x \to \min \subjto \\
    &Ax = \pmat{a\\b} = \bar{a}, \\
    &x \ge 0,
  \end{aligned}
    & &\Leftrightarrow &
  &\begin{aligned} (\text{P}) \quad
    &z = c^\top x \to \min \subjto \\
    &Ax \ge \bar{a} : u^1 \ge 0,  \\
    &-Ax \le -\bar{a} : u^2 \ge 0,  \\
    &x \ge 0
  \end{aligned}
\intertext{sowie}
  &\begin{aligned} (\text{D}) \quad
    &w = \bar{a}^\top u^1 - \bar{a}^\top u^2 \to \max \subjto \\
    &A^\top u^1 - A^\top u^2 \le c, \\
    &u^1 \ge 0, u^2 \ge 0,
  \end{aligned}
    & &\Leftrightarrow &
  &\begin{aligned} (\text{D}) \quad
    &w = \bar{a}^\top \tilde{u} \to \max \subjto \\
    &A^\top \tilde{u} \le c, \\
    &\tilde{u} = u_1 - u_2 \text{ frei}.
  \end{aligned}
\end{align*}
Damit können wir (3.21) formulieren als
\begin{equation} %% 3.22
  w = \sum_{i\in I} a_i u_i + \sum_{k \in K} b_k v_k \subjto
  u_i + v_k \le c_{ik}, \quad u_i \ge 0, v_k \ge 0 \quad
  \forall i,k.
\end{equation}

\begin{thm}[Optimalitätskriterium] %% 3.15
  Sei $x \in G$, das heißt $x$ ein zulässiger Transportplan, dann gilt
  \[ x \text{ optimal} \qLRq \exists u \in \real^r_+, v \in \real^s_+ \]
  mit $u_i + v_k \le c_{ik}$, $x_{ik}(c_{ik}-u_i-v_k) = 0$ für alle $i \in I$,
  $k \in K$.
\end{thm}

\begin{proof}
  Nach dem Charakterisierungssatz der linearen Optimierung gilt: $x \in G$, $u
  \in \real^r_+$, $v \in \real^s_+$ sind optimal genau dann, wenn
  \[ \underbrace{x^\top}_{\ge 0}
    \underbrace{\left( c - A^\top \pmat{u\\v} \right)}_{\ge 0} = 0, \qquad
    A^\top \pmat{u\\v} \le c \]
  gilt, was unmittelbar die Aussage des Satzes liefert.
\end{proof}

\begin{aus} %% 3.16
  Der Rang von $A$ ist $|I| + |K| - 1 = r + s - 1$.
\end{aus}

\begin{proof}
  Übung.
\end{proof}

\begin{flg}
  Jede Ecke des zulässigen Bereiches $G$ hat höchstens $r+s-1$ positive
  Komponenten.
\end{flg}

\begin{defn} %% 3.5
  Eine Folge von Zellen (Indexpaaren)
  \[ (i_1, k_1), (i_2, k_1), (i_2, k_2), \ldots,
    (i_l, k_l), (i_1, k_l), (i_1, k_1) \]
  mit $i_p \ne i_q$ und $k_p \ne k_q$ für alle $p \ne q$ heißt \emph{Zyklus}
  (der Länge $2l$).
\end{defn}

\begin{exmp} %% 3.5
  \[ \text{Grundschema }
    \begin{array}{c|cccc}
      & 1 & 2 & 3 & 4 \\
      \hline
      1 \\ 2 \\ 3 \\ 4
    \end{array}
    \qquad \text{Zyklus }
    \begin{array}{c|ccccc}
      & 1 & 2 & 3 & 4 & 5 \\
      \hline
      1 & \times & & \times \\
      2 & & \times & & & \times \\
      3 & \times & & & & \times \\
      4 & & \times & \times
    \end{array}
  \]
  Zyklus: $x_{11} \to x_{13} \to x_{43} \to x_{42} \to x_{22} \to x_{25} \to
  x_{35} \to x_{31} \to x_{11}$.
\end{exmp}

\begin{aus}
  \begin{enumerate}[(i)]
  \item Sei $J$ eine Menge von Zellen. Gilt $|J| \ge r + s$, so enthält $J$
    mindestens einen Zyklus.
  \item Sei $x = (x_{ik})$ ein zulässiger Transportplan. $x$ ist genau dann Ecke
    von $G$, wenn
    \[ J_* := \{ (i,k) : x_{ik} > 0 \} \]
    keinen Zyklus enthält.
  \end{enumerate}
\end{aus}

\begin{proof}
  Siehe Großmann/Terno 1997.
\end{proof}

\subsection{Erzeugung eines ersten Transportplans}
Die Erzeugung eines ersten Transportplans entspricht der Phase 1 im
Simplexverfahren, das heißt die Bestimmung einer Ecke von $G$. Aufgrund der
speziellen Struktur beim Transportproblem erfolgt die Darstellung des
Simplexverfahrens in geänderter Form:
\begin{align*}
  &\text{Eingabe-Daten:} &
  &\text{Transportplan:} \\
  &\begin{array}{c|cccc}
    & b_1 & b_2 & \cdots & b_s \\
    \hline
    a_1 & c_{11} & c_{12} & \cdots & c_{1s} \\
    a_2 & c_{21} & c_{22} & \cdots & c_{2s} \\
    \vdots \\
    a_r & c_{r1} & c_{r2} & \cdots & c_{rs}
  \end{array}
  &
  &\begin{array}{c|cccc}
    x &  \\
    \hline
     & x_{11} & x_{12} & \cdots & x_{1s} \\
     & x_{21} & x_{22} & \cdots & x_{2s} \\
     & \vdots \\
     & x_{r1} & x_{r2} & \cdots & x_{rs}
  \end{array}
\end{align*}

Zur Ermittlung eines ersten Transportplans können Heuristiken verwendet werden.
\begin{enumerate}
\item \emph{Nordwest-Ecken-Regel.} Die jeweilige, noch nicht belegte
  Nordwest-Zelle wird mit der maximal verfügbaren Transportmenge belegt.
\item \emph{Regel der minimalen Kosten.} In jedem Schritt wird eine noch nicht
  belegte Zelle mit geringstem Kostenkoeffizient maximal belegt.
\item \emph{Methode von Vogel.} Bestimme in jeder Zeile und Spalte die Differenz
  der zwei kleinsten Kostenkoeffizienten der noch freien Zellen. Wähle eine
  Zeile oder Spalte mit maximaler Differenz und belege die Zelle mit kleinsten
  Kosten maximal.
\end{enumerate}

\begin{exmp} %% 3.6
  \begin{align*}
    & & &\text{Nordwest:} & &\text{Minimale Kosten:} \\
    &\begin{array}{c|ccccc}
      & 12 & 5 & 6 & 7 & 7 \\
      \hline
      4 & 12 & 6 & 10 & 9 & 5 \\
      19 & 10 & 16 & 17 & 3 & 7 \\
      14 & 4 & 11 & 5 & 8 & 10
    \end{array}
    &
    &\begin{array}{c|ccccc}
       & \\
       \hline
       & 4^1 & 0 & 0 & 0 & 0 \\
       & 8^2 & 5 ^3 & 6^4 & 0 & 0 \\
       & 0   & 0 & 0 & 7^5 & 7^6
    \end{array} &
    &\begin{array}{c|ccccc}
       & \\
       \hline
       & 0 & 0 & 0 & 0 & 4^3 \\
       & 0 & 5^5 & 4^6 & 7^1 & 3^4 \\
       & 12^2 & 0 & 2^3 & 0 & 0
    \end{array}
  \end{align*}
  \[ z_{\text{NW}} = 48 + 80 + 80 + 102 + 56 + 70 = 456, \qquad z_{\text{min.
        K}} = 268. \]
\end{exmp}

Motiv:
\[ \sum_{i \in I}\sum_{k \in K} c_{ik} x_{ik} =
  \sum_{i \in I}\sum_{k \in K} \left( c_{ik} -
    \min \{ c_{il} : l \in K \} \right) x_{ik}
  + \sum_{i \in I}\sum_{k \in K} \min \{ c_{il} : l \in K \} x_{ik}
\]

\subsection{Der Transportalgorithmus}
Ein Transportplan $X = (x_k)$ ist optimal, falls duale Variablen $u$ und $v$
existieren mit
\[ z(X) = \sum_{i \in I} \sum_{k \in K} c_{ik} x_{ik}
  = \sum_{i \in I} a_i u_i + \sum_{k \in K} b_k v_k.\]
Das folgt aus dem starken Dualitätssatz. $u$ und $v$ werden mittels des
Charakterisierungssatzes bestimmt.

Verwendung von zwei Schemata, eines für $x$, eines für die Bestimmung von $u$
und $v$.
\[
  \begin{array}{l|cccc}
    X \\
    \hline
    & x_{11} & x_{12} & \cdots & x_{1s} \\
    & \vdots \\
    & x_{r1} & x_{r2} & \cdots & x_{rs}
  \end{array}
  \qquad
  \begin{array}{l|cccc}
    D \\
    \hline
    & d_{11} & d_{12} & \cdots & d_{1s} \\
    & \vdots \\
    & d_{r1} & d_{r2} & \cdots & d_{rs}
  \end{array}
\]
mit $d_{ik} := c_{ik} - u_i - v_k$.
 
Vorgehensweise:
\begin{enumerate}[(1)]
\item Ausgangspunkt ist ein zulässiges, zyklenfreies Transportproblem $X_0$ mit
  genau $r+s-1$ (markierten) Basiszellen (Menge $J$). Trage in Schema $D$ die
  Kostenkoeffizienten für alle $(i,k) \in J$ ein und markiere sie.
\item Bestimme die $u_i$ und $v_k$ aus dem Gleichungssystem
  \begin{equation} %% 3.23
    u_i + v_k = c_{ik} \quad \forall (i,k) \in J.
  \end{equation}
  Da (3.23) unbestimmt ist, kann eine Variable frei gewählt werden.
\item Berechne die transformierten Zielfunktionskoeffizienten durch
  \[ d_{ik} := c_{ik} - u_i - v_k \quad \forall (i,k) \notin J. \]
  Falls $d_{ik} \ge 0$ für alle $(i,k)$, dann ist $X_0$ optimal. Anderenfalls
  wähle eine Zelle $(p,q)$ mit
  \[ d_{p,q} = \min \left\{  d_{ik} : i \in I, k \in K \right\}. \]
\item Markiere die Zelle $(p,q)$ im Schema $X$. Bestimme den (eindeutigen)
  Zyklus $J_{pq}$ in $J \cup \{(p,q)\}$. Markiere bei $(p,q)$ beginnend
  abwechselnd mit ``$+$'' und ``$-$''. Seien $J_{pq}^+$ und $J_{pq}^-$ die
  zugehörigen Indexmengen.
\item Bestimme
  \[ \delta := \min \{ x_{ik} : (i,k) \in J_{pq} \} \]
  mit dem zugehörigen $x_{gh}$. Setze $X^1 := (x_{ik}^{\mathrm{neu}})$ mit
  \[ (x_{ik}^{\mathrm{neu}}) =
    \begin{cases}
      x_{ik} + \delta, & (i,k) \in J_{pq}^+, \\
      x_{ik} - \delta, & (i,k) \in J_{pq}^-, \\
      x_{ik}, & \text{sonst.}
    \end{cases}
  \]
  Die zugehörige Basisindexmenge ist
  \[ J^{\mathrm{neu}} := ( J \cup \{(p,q)\} ) \setminus \{ (g,h) \}. \]
\end{enumerate}

\begin{exmp} %% 3.7
  \[
    \begin{array}{r|ccccc}
      c & 12 & 5 & 6 & 7 & 7 \\
      \hline
      4 & 12 & 6 & 10 & 9 & 5 \\
      19 & 10 & 16 & 17 & 3 & 7 \\
      14 & 4 & 11 & 5 & 8 & 10
    \end{array}
  \]
  Beachte: Linke Spalte von $D_\tau$ sind die $u_i$, erste Zeile sind die $v_k$.
  \begin{align*}
    &\begin{array}{r|ccccc}
      X_0^* & 12 & 5 & 7 & 7 \\
      \hline
      4 &  &  &  &  & 4 \\
      19 &  & 5 & 4 & 7 & 3 \\
      14 & 12 & & 2 
    \end{array}
                     &
    &\begin{array}{r|ccccc}
      D_0 & 16 & 16 & 17 & 3 & 7 \\
      \hline
       -2 & -2 & -8 & -5 & & \boxed{5} \\
       0 & -6 & \boxed{16} & \boxed{17} & \boxed{3} & \boxed{7}  \\
       -12 & \boxed{4} & & \boxed{5} 
    \end{array} \\
    &\begin{array}{r|ccccc}
      X_1 &  \\
      \hline
       & & 4 & & & 0 \\
       & & 1 & 4 & 7 & 7 \\
       & 12 & & 2 & &
    \end{array}
                      &
    &\begin{array}{r|ccccc}
      D_1 & 16 & 16 & 17 & 3 & 7  \\
      \hline
      -10 & & 6 & & & \\
      0 & -6 & 16 & 17 & 3 & 7 \\
      -12 & 4 & & 5 & &
     \end{array}\\
    &\begin{array}{r|ccccc}
      X_2 &  \\
      \hline
       & & 4 & & & \\
       & 4 & 1 & & 7 & 7 \\
       & 8 & & 6 & &
    \end{array}
                      &
    &\begin{array}{r|ccccc}
      D_2 & 10 & 16 & 11 & 3 & 7  \\
      \hline
      -10 & & 6 & & & \\
      0 & 10 & 16 & & 3 & 7 \\
      -6  & 4 & & 5 & &
     \end{array}
  \end{align*}
  Für $D_2$ gilt $d_{ik} \ge 0$ für alle $(i,k)$, also ist $X_2$ optimal.
  \begin{align*}
    z(X_0) &= 268,
    & Z(X_1) &= z(X_0) + \delta d_{pq} = 236,
    & z(X_2) &= 236 - 24 = 212.
  \end{align*}
\end{exmp}

\begin{rmrk*}
  Wegen $d_{ik} = 0$ für $x_{ik}$ mit $(i,k) \in J$ kann ein komprimiertes
  Schema verwendet werden:
  \[
    \begin{array}{r|cccc}
      & v_1 & v_2 \cdots & v_s \\
      \hline
      u_1 & d_{11} & d_{12} & \cdots & d_{1s} \\
      u_2 & \vdots & d_{21} & \cdots & \boxed{x_{2s}} \\
      \vdots \\
      u_r & & \boxed{x_{2r}}
    \end{array}
  \]
\end{rmrk*}

\subsection{Modifizierte Problemstellungen} %% 3.5.4
\begin{enumerate}[(i)]
\item Verbotene Wege: Vom Erzeuger $p$ soll kein Transport zum Verbraucher $q$
  erfolgen. Lösung durch $x_{pq} = 0$ oder $c_{pq} := \infty$.
\item Überkapazität, also $\sum_{i} a_i > \sum_k b_k$. Definiere einen
  Scheinverbraucher $s+1$ durch
  \[ b_{s+1} = \sum_i a_i - \sum_k b_k. \]
  Die Kosten $c_{i,s+1}$ können als Lager- bzw. Entsorgungskosten interpretiert
  werden.
\item Unterkapazität: Definiere einen Scheinerzeuger $r+1$ mit
  \[ a_{r+1} := \sum_k b_k - \sum_i a_i. \]
  Die Kosten $c_{r+1,i}$ können als Vertragsstrafen interpretiert werden.
\item Kapazitätsbeschränkung: $x_{ik} \le u_{ik}$. 
\end{enumerate}

\chapter{Diskrete Optimierung}
\section{Branch and bound-Methode}
\subsection{Grundlagen}
Problem:
\[ (\text{P}) = (\text{P}_0) \qquad f(x) \to \min \subjto x \in D \cap E. \]
Relaxation zu (P):
\[ (\text{Q}) \qquad g(x) \to \min \subjto x \in E, \]
wobei $g(x) \le f(x)$ für alle $x \in D \cap E$.

\begin{thm*}[Wiederholung Satz 1.1]
  Sei $\bar{x}$ Lösung von (Q) und gelte $\bar{x} \in
  D$ sowie $f(\bar{x}) = g(\bar{x})$. Dann ist $\bar{x}$ Lösung von (P).
\end{thm*}

\emph{Prinzip der b\&b-Methode}: Die Relaxationsmenge $E$ wird durch
Separationen in Teilmengen $E_i$ zerlegt. Es entstehen Teilprobleme
\[ (\text{P}_i) \qquad f(x) \to \min \subjto D \cap E_i. \]
Jedem Teilproblem $(\text{P}_i)$ wird eine Zahl $b( \text{P}_i )$, eine untere
Schranke, zugeordnet, so dass gilt
\begin{enumerate}[(a)]
\item $b( \text{P}_i ) \le \min \{ f(x) : x \in D \cap E_i \}$,
\item $b( \text{P}_i ) = f(x)$, falls $|D \cap E_i| = |\{x\}| = 1$,
\item $b( \text{P}_i ) \le b( \text{P}_k )$, falls $E_k \subset E_i$.
\end{enumerate}

\subsection{Allgemeiner b\&b-Algorithmus}
Bezeichnungen: \\
$R$ ... Menge der noch zu betrachtenden Teilprobleme, \\
$bar{z}$ ... Zielfunktionswert der bisher besten gefundenen zulässigen Lösung.

\begin{enumerate}[S1]
  \setcounter{enumi}{-1}
\item \emph{Initialisierung.} Bestimme $b(\text{P}_0)$:
  \begin{enumerate}[(a)]
  \item Falls $\bar{x} \in D \cap E$ bekannt ist mit $f(\bar{x}) = b(\text{P}_0)$,
    \verb+STOP+, $\bar{x}$ ist zulässige Lösung.
  \item Setze $R := \{ \text{P}_0 \}$, $\bar{z} := \infty$ oder
    $\bar{z} := f( \bar{x} )$, wenn $\bar{x} \in D \cap E$ bekannt.
  \end{enumerate}
\item \emph{Abbruchtest.} Falls $R = \emptyset$, \verb+STOP+. Falls $\bar{z} =
  \infty$, dann ist (P) unzulässig, anderenfalls ist $\bar{x}$ Lösung.
\item \emph{Strategie.} Wähle entsprechend einer Auswahlstrategie ein
  $\text{P}_i \in R$ und setze $R := R \setminus \text{P}_i$.
\item \emph{Zerlegung} (branch). Zerlege $\text{P}_i$ durch Separation in
  Teilprobleme $\text{P}_{i,1}, \ldots, \text{P}_{i,k}$. Setze $j := 1$.
\item \emph{Schranken und Dominanztest.}
  \begin{enumerate}[(a)]
  \item Berechne $b(\text{P}_{i,j})$. Falls dabei oder auf andere Weise ein
    $\tilde{x} \in D \cap E$ gefunden wurde mit $f(\tilde{x}) < \bar{z}$, setze
    $\bar{x} :=  \tilde{x}$, $\bar{z} = f(\tilde{x})$.
  \item Falls $b(\text{P}_{i,j}) < \bar{z}$, dann setze $R := R \cup \{
    \text{P}_{i,j} \}$. Falls $j < k_i$, setze $j := j + 1$ und gehe zu (a).
  \item Setze $R := R \setminus \{\text{P}_k\}$ für alle $\text{P}_k \in R$ mit
    $b(\text{P}_k) \ge \bar{z}$.
  \end{enumerate}
  Gehe zu S1.
\end{enumerate}

\begin{rmrk*}
  \begin{enumerate}[(1)]
  \item Die Endlichkeit ist zu sichern durch
    \[ |E_{i,j} \cap D | < | E_i \cap D | \quad \text{für alle } j, \]
    \[ b( \text{P}_{i,j} ) \ge b( \text{P}_i ) + \eps \quad \text{mit } \eps >
      0, \text{ für alle } i, j. \]
  \item Festlegung der Auswahlstrategie:
    \begin{itemize}
    \item Minimalstrategie (best bound search)
      \[ b(\text{P}_i) \le b(\text{P}_k) \quad \text{für alle } \text{P}_k \in
        R. \] 
    \item LIFO (last-in-first-out, depth first search) bzw. FIFO
      (first-in-first-out, breadth first search). Jeweils Auswahl von
      $\text{P}_i$ unter den Teilproblemen mit minimaler bzw. maximaler
      Verzweigungstiefe und kleinster Schranke.
    \end{itemize}
  \item Repräsentation durch Verzweigungsbaum.
  \item Wichtige zusätzliche Tests sind in S4 möglich:
    \begin{itemize}
    \item Zulässigkeitstest: $D \cap E_{i,j} = \emptyset$ $\Rightarrow$ $R := R
      \setminus \{ \text{P}_{i,j} \}$.
    \item Dominanztest: $\text{P}_i$ dominiert $\text{P}_k$, falls eine Lösung
      von $\text{P}_i$ mindestens so gut wie eine Lösung von $\text{P}_k$ ist.
    \end{itemize}
  \end{enumerate}
\end{rmrk*}

\subsection{Beispiele für b\&b-Algorithmen}
\subsubsection{Das 0/1-Rucksackproblem}
Voraussetzung: $c_i > 0$, $0 < a_i \le b$ für alle $i \in I = \{ 1, \ldots, n
\}$. 
\begin{align*}
  f(x) &= c^\top x \to \max \subjto a^\top x \le b, 
         x \in \boole^n = \{0,1\}^n \tag{P} \\
  g(x) &= c^\top x \to \max \subjto a^\top x \le b, 
         x \in [0,1]^n \tag{Q}
\end{align*}
Also
\[ E = \{ x \in \real^n : 0 \le x_i \le 1, i \in I \}, \qquad D = \boole^n
  (=\integer^n). \]
O.B.d.A. Voraussetzung:
\[ \frac{c_i}{a_i} \ge \frac{c_{i+1}}{a_{i+1}}, \qquad i = 1, \ldots, n-1. \]
Definiere Teilprobleme ($\bar{x}_i$, $i = 1, \ldots, k$, seien bereits
fixierte Variablen):
\[ \begin{aligned}
    \text{P}_k(\bar{x}) : \quad
    &\sum_{i=1}^k c_i \bar{x}_i + \sum_{i=k+1}^n c_i x_i \to \max \subjto \\
    &\sum_{i=k+1}^n a_i x_i \le b - \sum_{i=1}^k a_i \bar{x}_i,
    \quad x_i \in \boole, i = k+1, \ldots, n.
  \end{aligned}
\]
Zugehörige Relaxation $=$ stetige Relaxation:
\[ \text{Q}_k(y) : \quad
  z_k(y) := \max \left\{ \sum_{i=k+1}^n c_i x_i : \sum_{i=k+1}^n a_i x_i \le
    y, 0 \le x_i \le 1 \text{ für alle } i \right\}. \]
Damit ergibt sich aber eine Schranke für $\text{P}_i$:
\[ b( \text{P}_i ) := \sum_{i=1}^k c_i \bar{x}_i + z_k(y). \]
Es gilt
\[ z_k(y) = \sum_{i=k+1}^p c_i + \rez{a_{p+1}} \cdot \left( y - \sum_{i=k+1}^p
    a_i \right), \]
wobei
\[ \sum_{i=k+1}^p a_i \le y \le \sum_{i=k+1}^{p+1} a_i. \]

\begin{exmp}
  Beispiel:
  \[ \begin{aligned}
      z = c^\top x &= 8x_1 + 16x_2 + 20x_3 + 12x_4 + 6 x_5 + 10x_6 + 4x_7 \to
      \max \subjto \\
      a^\top x &= 3x_1 + 7x_2 + 9x_3 + 6x_4 + 3x_5 + 5x_6 + 2x_7 \le 17, \quad
      x_i \in \{0,1\}.
    \end{aligned}
  \]
  Es gilt
  \[ b(\text{P}_0) = 8 + 16 + \frac{20}{9} \cdot 7 = \lfloor \ldots \rfloor =
    39. \]
  Mögliche Teilprobleme: $\text{P}_1 : x_1 = 1$, $\text{P}_2 : x_1 = 0$.
  \begin{align*}
    b(\text{P}_1) &= 39, \\
    b(\text{P}_2) &= 16 + 20 + \frac{12}{6} \cdot (17-7-9)
                    = \lfloor \ldots \rfloor = 38, \\
    b(\text{P}_3) &= 39, \\
    b(\text{P}_4) &= 8 + 20 + \frac{12}{6} \cdot 5 = 38, \\
    b(\text{P}_5) &= \infty & D \cap E_5 = \emptyset, \\
    b(\text{P}_6) &= 8 + 16 + 0 + 2 \cdot 7 = 38.
  \end{align*}
  Aus ``Heuristik'' ist
  \[ \bar{x} = (1,1,0,0,0,1,1)^\top, \qquad z(\bar{x}) = 38 \]
  bekannt. 38 ist der Optimalwert, $\bar{x}$ die Lösung.
\end{exmp}

\subsubsection{Ganzzahlige lineare Optimierung}
Generelle Voraussetzung: Alle Daten sind ganzzahlig.

Verfahren von Land/Doig/Dakin zur Lösung von
\[ (\text{P}) = (\text{P}_0) : \quad
  z = c^\top x \to \min \subjto x \in D \cap E \]
mit $D = \integer^n$, $E = E_0 = \{ x \in \real^n : Ax = b, x \ge 0 \}$.

Teilproblem:
\[ (\text{P}_i) : \quad
  c^\top x \to \min \subjto x \in D \cap E_i, \]
wobei $E_i$ durch eine oder mehrere zusätzliche Ungleichungen aus $E$
entsteht.
\[ (\text{Q}): \quad
  z(E_i) := \min \{ c^\top x : x \in E_i \}. \]
Sei $x^{\text{LP}}$ Lösung der LP-Relaxation zu $(\text{P}_i)$ mit
$x_j^{\text{LP}} \notin \integer$.

Verzweigung mittels Alternativkonzept:
\begin{align*}
  (\text{P}_{i,1}) \quad E_{i,1}
  &= \left\{ x \in E_i : x_j \le \lfloor x_j^{\text{LP}} \rfloor \right\}, \\
  (\text{P}_{i,2}) \quad E_{i,2}
  &= \left\{ x \in E_i : x_j \ge \lceil x_j^{\text{LP}} \rceil \right\}. \\
\end{align*}

\setcounter{exmp}{2}
\begin{exmp} %% 4.3
  Betrachte
  \[ z = 7 x_1 + 2 x_2 \to \max \subjto -x_1 + 2 x_2 \le 4, 5x_1 + x_2 \le 20,
    x_1, x_2 \in \integer_+. \]
  Tableau:
  \[
    \begin{array}{c|ccc}
      \ST_1 & x_1 & x_2 & 1 \\
      \hline
      x_3 = & 1 & -2 & 4 \\
      x_4 = & \boxed{-5} & -1 & 20 \\
      \hline
      - z = & -7 & -2 & 0 \\
      \hline
            & \ast & - \rez{5} & 4
    \end{array}
    \quad
    \begin{array}{c|ccc}
      \ST_2 & x_4 & x_2 & 1 \\
      \hline
      x_3 = & -\rez{5} & \boxed{-\frac{11}{5}} & 8 \\
      x_1 = & -\rez{5} & -\rez{5} & 4 \\
      \hline
      - z = & \frac{7}{5} & -\frac{3}{5} & -28 \\
      \hline
            & -\rez{11} & \ast & \frac{40}{11}
    \end{array}
    \quad
    \begin{array}{c|ccc}
      \ST_3 & x_4 & x_3 & 1 \\
      \hline
      x_2 = & -\rez{11} & -\frac{5}{11} & \frac{40}{11} \\
      x_1 = & -\frac{2}{11} & \frac{1}{11} & \frac{36}{11} \\
      \hline
      - z = & \frac{16}{11} & \frac{3}{11} & -\frac{332}{11}
    \end{array}
  \]
  Verzweigung nach $x_2$, da
  \[ \min \left\{ 4 - \frac{40}{11}, \frac{40}{11}-3 \right\} >
    \min \left\{ 4 - \frac{36}{11}, \frac{36}{11}-3 \right\} \]
  \begin{enumerate}[1. TP:]
  \item $x_2 \ge 4$ $\Rightarrow$
    \[ s_2 = x_2 - 4 = -\rez{11} x_4 - \frac{5}{11} x_3 - \frac{4}{11} < 0 \]
    für alle $x_3, x_4 \ge 0$. Der zulässige Bereich ist leer!
  \item $x_2 \le 3$ $\Rightarrow$
    \[ s_2 = 3 - x_2 = \rez{11} x_4 + \frac{5}{11} x_3 - \frac{7}{11}\]
  \end{enumerate}
  Neues Tableau:
  \[
    \begin{array}{c|ccc}
      \ST_3^* & x_4 & x_3 & 1 \\
      \hline
      x_2 = & -\rez{11} & -\frac{5}{11} & \frac{40}{11} \\
      x_1 = & -\frac{2}{11} & \frac{1}{11} & \frac{36}{11} \\
      s_2 = & \frac{1}{11} & \boxed{\frac{5}{11}} & - \frac{7}{11} \\
      \hline
      - z = & \frac{16}{11} & \frac{3}{11} & -\frac{332}{11} \\
      \hline
              & -\rez{5} & \ast & \frac{7}{5}
    \end{array}
    \quad
    \begin{array}{c|ccc}
      \ST_4 & x_4 & s_2 & 1 \\
      \hline
      x_2 = & 0 & -1 & 3 \\
      x_1 = & -\rez{5} & \rez{5} & \frac{17}{5} \\
      x_3 = & -\rez{5} & \frac{11}{5} & \frac{7}{5} \\
      \hline
      - z = & \frac{7}{5} & \frac{3}{5} & -\frac{149}{5}
    \end{array}
  \]
  Dieses Tableau ist optimal für das Teilproblem, aber es sind immer noch
  nicht alle Variablen ganzzahlig.
  
  Verzweigung nach $x_1$:
  \begin{enumerate}[1. TP]
    \setcounter{enumi}{2}
  \item $x_1 \ge 4$ $\Rightarrow$ $s_1 = x_1 - 4$.
    \[
      \begin{array}{c|ccc}
        \ST_4^* & x_4 & s_2 & 1 \\
        \hline
        x_2 = & 0 & -1 & 3 \\
        x_1 = & -\rez{5} & \rez{5} & \frac{17}{5} \\
        x_3 = & -\rez{5} & \frac{11}{5} & \frac{7}{5} \\
        s_1 = & -\rez{5} & \boxed{\rez{5}} & -\frac{3}{5} \\
        \hline
        - z = & \frac{7}{5} & \frac{3}{5} & -\frac{149}{5} \\
        \hline
                & 1 & \ast & 3 
      \end{array}
      \quad
      \begin{array}{c|ccc}
        \ST_5 & x_4 & s_1 & 1 \\
        \hline
        x_2 = & & & 0 \\
        x_1 = & & & 4 \\
        x_3 = & & & 8 \\
        s_2 = & & & 3 \\
        \hline
        - z = & 2 & 3 & -28
      \end{array}
    \]
  \item $x_1 \le 3$ $\Rightarrow$ $s_1 = 3 - x_1$.
    \[
      \begin{array}{c|ccc}
        \ST_4^* & x_4 & s_2 & 1 \\
        \hline
        x_2 = & 0 & -1 & 3 \\
        x_1 = & -\rez{5} & \rez{5} & \frac{17}{5} \\
        x_3 = & -\rez{5} & \frac{11}{5} & \frac{7}{5} \\
        s_1 = & \boxed{\rez{5}} & -\rez{5} & -\frac{2}{5} \\
        \hline
        - z = & \frac{7}{5} & \frac{3}{5} & -\frac{149}{5} \\
        \hline
                & \ast & 1 & 2 
      \end{array}
      \quad
      \begin{array}{c|ccc}
        \ST_5 & s_1 & s_2 & 1 \\
        \hline
        x_2 = & & & 3 \\
        x_1 = & & & 3 \\
        x_3 = & & & 1 \\
        x_4 = & & & 2 \\
        \hline
        - z = & 7 & 2 & -27
      \end{array}
    \]
  \end{enumerate}
  Also ist die Lösung im 3. TP: $z_{\max} = 28$.
\end{exmp}

\subsubsection{Das Rundreise-Problem}
Traveling salesman problem, TSP

Gesucht ist eine Rundreise minimaler Länge zwischen $n$ Orten, die jeden Ort
\emph{genau einmal} besucht. Anwendung zum Beispiel in der
Leiterplattenbestückung.

\begin{itemize}
\item Entfernungsmatrix $C = (c_{ik}) \in \realmat{n}{n}$.
\item Entscheidungsvariable
  \[ x_{ik} =
    \begin{cases}
      1, &\text{falls ovn $i$ nach $k$ gereist wird,} \\
      0, &\text{sonst.}
    \end{cases}
  \]
\end{itemize}
O.B.d.A: $c_{ii} := + \infty$ $\Rightarrow$ $x_{ii} = 0$.
\[ z = \sum_{i=1}^n \sum_{i=1}^n c_{ik} x_{ik} \to \min \subjto
  \sum_{k=1}^n x_{ik} = 1, \quad
  \sum_{i=1}^n x_{ik} = 1, \quad
  x_{ik} \in \{0,1\} \tag{AP} \]
für alle $i,k$. Absicherung, dass jeder Ort einmal getroffen wird und
Verhinderung von ``Subtouren'' durch die weiteren Forderungen
\[ \sum_{i,k \in S} x_{ik} = |S| - 1 \tag{SEB} \]
(Subtoureliminationsbedingungen)
für alle $S \subset \{1, \ldots, n\}$ mit $1 \le |S| \le n-1$. Beachte: Es
ergeben sich also $2^n$ Nebenbedingungen.

Das ist ein spezielles Transportproblem, da alle $a_i$ und $b_k$ den Wert 1
haben. Man bezeichnet es als \emph{Zuordnungsproblem} (assignment problem).

Es können von den nötigen $2n-1$ Basisvariablen nur $n$ Stück $\ne 0$
sein. Das führt zu Zyklen im Simplexverfahren, das heißt es werden immer
wieder die selben Variablen ausgetauscht, ohne dass sich der ZFW ändert.

Für diese Art Problem existieren Algorithmen, die eine Komplexität $O(n^3)$
aufweisen.

Es sind unterschiedliche Relaxationen anwendbar:
\begin{enumerate}[(1)]
\item Zeilen- und Spaltenrelaxation (Algorithmus von
  Little/Sweeney/Karel/Murthy, 1963)
  \[ u_i := \min \left\{ c_{ik} : k \in I=\{1, \ldots, n \} \right\}
    \qRq \tilde{d}_{ik} := c_{ik} - u_i \]
  \[ v_k := \min \left\{ \tilde{d}_{ik} : i \in I \right\} \]
  Es ergibt sich
  \begin{align*}
    z &= \sum_{i \in I} \sum_{k \in I} c_{ik} x_{ik} \\
      &= \sum_{i \in I} \sum_{k \in I}( c_{ik} - u_i ) x_{ik}
        + \underbrace{\sum_{i \in I} \sum_{k \in I} u_i x_{ik}}_{=\sum_{i \in I} u_i} \\
      &= \sum_{i \in I} \sum_{k \in I} (c_{ik} - u_i - v_k) x_{ik}
        + \underbrace{\sum_{i \in I} u_i + \sum_{k \in I} v_k}_{\text{Untere Schranke für $z$}},
  \end{align*}
  eine zulässige Lösung der dualen Aufgabe zu AP.
\item AP-Relaxation, das heißt otimale Lösung von AP
\item LP-Relaxation von AP und SEB.
\end{enumerate}
Für $n \le 40$ ist (1) am wenigsten aufwändig, bis $n = 120$ ist (2) am
besten, für $n > 120$ ist (3) zu empfehlen.

Realisierung von (3): Start mit AP-Relaxation und Hinzufügen der verletzten
SEB (Wie finden?).

\paragraph{Verzweigungsstrategie bei Zeilen- und Spaltenreduktion}
Für jedes Indexpaar $(p,q)$ mit $d_{pq} = c_{pq} - u_p - v_q = 0$ wird ein
Gewicht $w_{pq} \in \real_+$ ermittelt, welches durch Zeilen- und
Spaltenreduktion aus $\tilde{D}$ erhalten wird, wobei $\tilde{D} =
(\tilde{d}_{ik})$ aus $D$ entsteht durch Setzen von
\[ \tilde{d}_{ik} :=
  \begin{cases}
    \infty, &(i,k) = (p,q), \\
    d_{ik}, &\text{sonst.}
  \end{cases}
\]

Aus $d_{ik} := c_{ik} - u_i - v_k$ für alle $i, k$ folgt
\[ z(X) = \sum_{i,k} d_{ik} x_{ik} + \sum_{i}(u_i + v_i) \ge \sum_i (u_i +
  v_i), \]
da $d_{ik} \ge 0$ und $x_{ik} \ge 0$.

Verzweigung:
\begin{itemize}
\item $x_{pq} := 1$ $\Rightarrow$ $x_{pk} := 0$ für alle $k \ne q$,
\item $x_{pq} := 1$ $\Rightarrow$ $x_{iq} := 0$ für alle $i \ne p$, $x_{qp} :=
  0$ bzw. Verbot des zugehörigen Zyklus $\Rightarrow$ $d_{pq} := \infty$.
\end{itemize}

\paragraph{Welches $(p,q)$ sollte man wählen?}
Für jedes Indexpaar $(p,q)$ mit $d_{pq} = 0$ definiere das Gewicht
\[ w_{pq} := \min_{k \ne q} d_{pk} + \min_{i \ne p} d_{iq}. \]
Das entpricht dem Mindestzuwachs für $x_{pq} := 0$.

Sein nun $(p,q)$ ein Indexpaar mit \emph{maximalem} Gewicht, dann verzweige
bezüglich $(p,q)$.

\begin{exmp}
  Little-Algorithmus mit Zeilen- und Spaltenreduktion.
  \[
    \begin{array}{c|ccccc|c}
      (c_{ik}) & & & & & & \\
      \hline
               & \infty & 32 & \underline{22} & 30 & 24 & 22 \\
               & 10 & \infty & \underline{3} & 18 & \infty & 3 \\
               & \infty & \underline{9} & \infty & 14 & 12 & 9 \\
               & 16 & 10 & 7 & \infty & \underline{6} & 6 \\
               & 15 & 19 & 15 & \underline{12} & \infty & 12 \\
      \hline
               & 3 & & & & & \sum 52 + 3
    \end{array}
  \]
  %% Unfertig
\end{exmp}

Aufwand: Little-Algorithmus $O(n^2)$ wegen Zeilen-/Spaltenreduktion. Je
Teilproblem: AP-Relaxation $O(n^2)$, LP-Relaxation: Lösung eines LP uns Suche
von SEB ``$> O(n^2)$''

\begin{rmrk*}
  Die Zeilen-/Spaltenreduktion liefert die Schranke 58.
\end{rmrk*}

\section{Dynamische Optimierung}
(Prinzip wird nur an zwei Beispielen erklärt)

\subsection{Das klassische Rucksackproblem}
Voraussetzung; $c_i > 0$, $0 < a_i \le b$ für alle $i \in I = \{1, \ldots, n\}$,
alle Daten ganzzahlig.
\[ f(x) = c^\top x \to \max \subjto a^\top x \le b, \quad x \in \integer_+^n.
  \tag{P} \]

\begin{defn*}
  Für alle $k \in \{1, \ldots, n\}$ und $y \in \{0,1, \ldots, b\}$ definiere
  \[ F(k,y) := \max \left\{
      \sum_{i=1}^k c_i x_i : \sum_{i=1}^k a_i x_i \le y, x_i \in \integer_+, i
      =1, \ldots, k
    \right\}. \]
\end{defn*}
Gesucht ist $F(n,b)$.

Für $k=1$ gilt
\[ F(1,y) := c_1 \left\lfloor \frac{y}{a_1} \right\rfloor\]
für alle $y \in \{ 0, \ldots, b \}$.

Aufgrund der Linearität der Zielfunktion und der Nebenbedingungen gilt für $k >
1$, $y \ge a_k$:
\[ F(k,y) = \max \left\{ c_k x_k + F(k-1, y - a_k x_k) : x_k = 0, 1, \ldots,
    \left\lfloor \frac{y}{a_k} \right\rfloor 
  \right\} \]
bzw.
\[ F(k,y) = \max \{
  \underbrace{F(k-1,y)}_{\text{d.h. } x_k = 0},
  \underbrace{c_k + F(k, y- a_k)}_{\text{d.h.} x_k \ge 1}
  \} \]

\paragraph{Interpretation} Der Optimalwert $F(k,y)$ zum \emph{Zustand} $(k,y)$
wird mit Hilfe von Optimalwerten für ``kleinere'' Zustände berchnet. Also sollte
man die Reihenfolge der Berechnung dieser Optimalwerte so organisieren, dass
alle benötigten Zwischenwerte vorher ermittelt werden. Das führt zu den
Rekursionsformeln der dynamischen Optimierung (DO).

\begin{rmrk*}
  \begin{enumerate}
  \item Mit $F(1,y) := c_1 \cdot \lfloor y / a_1  \rfloor$ und (GG) für alle
    $y \ge 0$ und $k > 1$ ergibt sich der Algorithmus von Gilmore/Gomory.
  \item Im Unterschied zu rekursiven Programmierung wird in der dynamischen
    Optimierung jeder Optimalwert nur \emph{einmal} berechnet.

    Fibonacci-Zahl: $F(n) = F(n-1) + F(n-2)$, $F(0) = 0$, $F(1) = 1$.
  \end{enumerate}
\end{rmrk*}

\subsubsection*{Algorithmus von Gilmore/Gomory}
\begin{enumerate}
\item[$S_1$] Setze $F(1,y) = c_1 \cdot \lfloor y / a_1 \rfloor$,
  $y = 0, 1, \ldots, b$.
\item[$S_k$] Für $k = 2, \ldots, n$
  \begin{align*}
    &\text{für } y = 0, \ldots, a_{k-1}:
    & F(k,y) &:= F(k-1,y), \\
    &\text{für } y = a_k, a_{k+1}, \ldots, a_b:
    & F(k,y) &:= \max \{ F(k-1,y), c_k + F(k,y-a_k) \}.
  \end{align*}

  Merken von Indexinformation und Identifizierung einer Lösung
  \[ p(k,y) = \begin{cases}
      p(k-1,y), & \text{falls } F(k,y) = F(k+1,y), \\
      k, &\text{sonst.}
    \end{cases}
  \]
  \[ p(1,y) = \begin{cases}
      0, & \text{falls } y < a_1, \\
      1, & \text{falls } y \ge a_1.
    \end{cases}
  \]
\end{enumerate}
$F: \real^2 \to \real$ ist stückweise konstant und monoton nicht fallend.

\begin{exmp} %% 4.5
  \begin{align*}
    &z = 5 x_1 + 10 x_2 + 12 x_3 + 6 x_4 \to \max \\
    \text{bei } 4 x_1 + 7 x_2 + 9 x_3 + 5 x_4 \le 5, \quad x_i \in \integer_+
  \end{align*}

  %% Mehr
\end{exmp}

\subsection{Guillotine-Zuschnitt}
Aus einem Rechteck $L \times W$ sind kleinere Rechtecke $l_i \times w_i$, $i =
1, \ldots, n$, durch Guillotine-Schnitte so zuzuschneiden, dass der Abfall
minimal ist. Gewünschte Rechtecke können mehrfach erhalten werden.

$F(L,W)$ bezeichne den Maximaleintrag bei Guillotine-Zuschnitt aus $L \times W$.
Mit $F(l, 0) = 0$ für $l = 1, \ldots, L$ und $F(0,w) = 0$ für $w = 1, \ldots,
W$, gilt für $l = 1, \ldots, L$, $w = 1, \ldots, W$:
\[ F(l,w) = \max \{ e(l,w), h(l,w), v(l,w) \}, \]
wobei
\begin{align*}
  e(l,w)
  &= \max \left\{ 0 , \max_i \{ l_i w_i : l_i \le l, w_i \le w \} \right\}, \\
  h(l,w)
  &= \max \{ F(r,W) + F(L-r,W) : r = 1, \ldots, \lfloor L/2 \rfloor \}, \\
  v(l,w)
  &= \max \{ F(L,s) + F(L,W-s) : s = 1, \ldots, \lfloor W/2 \rfloor \}.
\end{align*}

Der Aufwand ist $O(LW(L+W))$.

Die Optimalwerte für größere Rechtecke werden aus den Optimalwerten für kleinere
Rechtecke erhalten.

Im Allgemeinen gilt
\[ \max \{ f(x) + g(x) \} \ne \max f(x) + \max g(x). \]

\section{Schnittebenenverfahren für GLO}
\begin{align*}
  c^\top x \to \min \subjto Ax &= a, \quad x \in \integer_+^n \tag{P} \\
  c^\top x \to \min \subjto Ax &= a, \quad x \in \real_+^n \tag{q}
\end{align*}
Für die konvexe Hülle der gültigen Punkte von (P) gilt
\[ \operatorname{conv} \{ x \in \integer_+^n : Ax = a \} \subset G, \]
sie ist immer eine Teilmenge des gültigen Bereichs von (Q).

\subsection{Gomory-Schnitte}
Sei (Q) mittels Simplexverfahren gelöst und das optimale Tableau sei
\[ x_B = P x_N + p, \qquad z = q^\top x_N + q_n \]
mit $P = -A_B^{-1} A_N$.

Alternative Schreibweise:
\[ I_B x_B - P x_N = p. \]

Die Lösung von (Q) ist dann
\[ x = \pmat{x_B \\ x_N} = \pmat{p \\ 0}, \]
beachte $J = J_B \cup J_N = \{ 1, \ldots, n \}$.

Wenn $p$ ganzzahlig ist, dann ist $x$ auch eine Lösung von $p$. Sei also $p$
\emph{nicht} ganzzahlig.

\textbf{Idee:} Eine Nebenbedingung (oder mehrere) in (P) hinzufügen, die von
$x = \pmat{x_B \\ x_N} = \pmat{p \\ 0}$ \emph{nicht} erfüllt wird, aber von
allen zulässigen ganzzahligen Punkten von (P).

Es sei die $i$-te Komponente von $p$ nicht ganzzahlig, also $p_i \notin
\integer$. Wegen
\begin{align*}
  [x_B]_i + \sum_{j \in J_N} (-P_{ij}) x_j &= p_i
  & \text{für alle }  x \in \real^n \text{ mit } Ax = a, \\
  [x_B]_i + \sum_{j \in J_N} \lfloor -P_{ij} \rfloor x_j &= \lfloor p_i \rfloor
  & \text{für alle }  x \in \integer_+^n \text{ mit } Ax = a 
\end{align*}
gilt
\[ \sum_{j \in J_N}
  \underbrace{((-P_{ij}) - \lfloor P_{ij} \rfloor)}_{=(\lceil P_{ij} \rceil -
    P_{ij}) x_j = r_{j}^{(i)}}
  \ge p_i - \lfloor p_i \rfloor = r_i \]
für alle $x \in \integer_+^n$ mit $Ax = a$.

\begin{defn}
  Die Ungleichung
  \[ \sum_{j \in J_n} r_j^{(i)} x_j \ge r_i \]
  heißt \emph{Gomory-Schnitt} zur Zeile $i$.
\end{defn}

\begin{rmrk*}
  Für $x = \pmat{p \\ 0}$ mit $p_i \notin \integer_+$ gilt
  \[ x^\top r^{(i)} = \sum_{j \in J_N} r_j^{(i)} \cdot 0 = 0 < r_i (\ne 0) \]
  mit $r_j^{(i)} = 0$ für $j \in J_B$, das heißt die nicht-ganzzahlige Lösung
  $x$ wird ``abgeschnitten''.
\end{rmrk*}

\begin{flg}
  Es gilt
  \begin{align*}
    \{ x \in \integer_+^n : Ax = a \}
    &= \{ x \in \integer_+^n : x_B = P x_N + p \} \\
    &= \{ x \in \integer_+^n : x_B = P x_N + p, (r^{(i)})^\top x \ge r_i \}.
  \end{align*}
\end{flg}

\subsection{Gomory-Verfahren}
Durch Gomory-Schnitte erhält man eine strengere Relaxation zu (P)
\[ \begin{aligned}
    z = q^\top x_N + q_0 \to \min \subjto
    &x_B = P x_N + p, \\
    &s = (r^{(i)})^\top x_N - r_i, \\
    &x_B, x_N \ge 0. 
  \end{aligned} \tag{$\text{Q}_1$} \]

Die Lösung von ($\text{Q}_1$) erhält man mit dem dualen Simplexverfahren.
\begin{enumerate}[S1]
  \setcounter{enumi}{-1}
\item (Initialisierung) $k := 0$, $(\text{Q}_k) = (\text{Q})$
\item Löse ($\text{Q}_k$).\\
  Falls $(\text{Q}_k)$ nicht zulässig ist, dann ist (P) nicht zulässig.
  $\rightarrow$ \verb+STOP+ \\
  Falls $z$ unbeschränkt ist für $(\text{Q}_k)$, dann auch für (P). $\rightarrow$
  \verb+STOP+ \\
  Falls die Lösung von $(\text{Q}_k)$ ganzzahlig ist, dann ist sie auch
  Lösung von (P). $\rightarrow$ \verb+STOP+
\item Füge einen Gomory-Schnitt zu ($\text{Q}_k$) hinzu. Dies ergibt
  ($\text{Q}_{k+1}$). \\
  Setze $k := k + 1$ und gehe zu S1.
\end{enumerate}

\begin{exmp}
  %% Noch zu füllen
\end{exmp}

\begin{rmrk*}
  Die Anzahl der Gomory-Schnitte, die erforderlich sind, um eine ganzzahlige
  Lösung zu erhalten, ist exponentiell. Die Endlichkeit des Verfahrens ist im
  Allgemeinen unklar (Zyklen).

  Ist die Menge der zulässigen Punkte beschränkt, so liefert der Algorithmus
  nach endlich vielen Schritten eine ganzzahlige Lösung oder es existiert keine
  zulässige Lösung.
\end{rmrk*}

\textbf{Alternative 1.} Das sogenannte \emph{lexikographische} Gomory-Verfahren
ist endlich. Wähle immer die Variable mit kleinstem Index, die keinen
ganzzahligen Wert hat.

\textbf{Alternative 2.} Kombination von b\&b und
Schnittebenenverfahren $\rightarrow$ branch and cut

\textbf{Alternative 3.} Anwendung allgemeinerer, besserer Schnitte
(Chvátal-Gomory cuts, facettendefinierte Schnitte)

%% Viel fehlt

\chapter{Optimierung auf Graphen}
1739 Euler: Königsberger Brückenproblem

\section{Definitionen}
Graph $G = (V,E)$, $V$ ist die Menge der Knoten, $E$ die Menge der Kanten oder
Bögen.
\begin{itemize}
\item Kanten: \emph{ungerichteter} Graph, Bögen: \emph{gerichteter} Graph
\item $G$ heißt \emph{schlicht}, falls er keine Mehrfachkanten bzw. -bögen und
  keine Schleifen enthält.
\item In ungerichteten Graphen ist $E$ eine Teilmenge aller 2-elementigen
  Teilmengen von $V$, in gerichteten Graphen eine Teilmenge von $V \times V$.
\item Bezeichnung: Kante $\{u,v\}$ bzw. Bogen $(u,v)$ mit $u, v \in V$.
\item Zwei Knoten $u$ und $v$ heißen \emph{adjuzent} oder verbunden, falls
  $\{u,v\} \in E$. Ein Knoten $u \in V$ und eine Kante bzw. ein Bogen $e \in
  E$ heißen \emph{inzident}, wenn $u \in e$ bzw. $e = \{ u, v \}$ für ein $v
  \in V$.
\item \emph{Kantengrad} eines Knotens:
  \[ \deg (v) = \delta (v) := | \{ u \in V : \{ u, v \} \in E \} |. \]
  Bei gerichteten Graphen ist der \emph{Eingangsgrad}
  \[ \delta^-(v) := |\{ (u,v) \in E \}| \]
  bzw. \emph{Ausgangsgrad}
  \[ \delta^+(v) := |\{ (v,u) \in E \}|. \]
\item Ist $\delta(v) = 0$, so ist $v$ ein \emph{isolierter Knoten}.
\item $v$ heißt \emph{Vorgänger} bzw. \emph{Nachfolger} von $u \in V$ in einem
  gerichteten Graphen, falls $(v,u) \in E$ bzw. $(u,v) \in E$.
\item Die Menge der Vorgänger von $v$:
  \[ N^-(v) := \{ u \in V : (u,v) \in E \} \]
  bzw. Menge der Nachfolger von $v$:
  \[ N^+(v) := \{ u \in V : (v,u) \in E \}. \]
\item Falls $\delta^-(v) = 0$, dann ist $v$ eine \emph{Quelle}. Falls
  $\delta^+(v) = 0$ , dann ist $v$ eine \emph{Senke}.
\item Sei $G = (V,E)$ ein ungerichteter (gerichteter) Graph und $W = (v_{i_1},
  v_{i_2}, \ldots, v_{i_n})$ eine Folge von Knoten von $V$ mit $\{ v_{i_k},
  v_{i_{k+1}} \} \in E$ ($(v_{i_k}, v_{i_{k+1}}) \in E$) für alle $k = 1,
  \ldots, n-1$. Dann heißt $W$ \emph{Weg} in $G$.
\item $W$ heißt \emph{Kette}, falls $\{v_{i_k}, v_{i_{k+1}}\} \in E$ oder
  $(v_{i_{k+1}}, v_{i_k}) \in E$ für alle $k$ gilt.
\item Ein Weg $W$ heißt \emph{Pfad} oder \emph{elementarer Weg}, falls alle
  Knoten in $W$ enthalten sind.
\item $W$ ist ein \emph{Zyklus}, falls $v_{i_1} = v_{i_n}$. $W$ ist ein
  \emph{Kreis}, falls der Zyklus zusätzlich elementar ist.
\item Die \emph{Länge} von $W = |W| - 1$, die Anzahl der Kanten bzw. Bögen.
\item $G$ heißt \emph{zusammenhängend} (stark zusammenhängend), falls zu jedem
  Knotenpaar eine Kette (Weg) existiert.
\item $G$ heißt \emph{vollständig}, wenn alle möglichen Kanten bzw. Bögen
  vorhanden sind.
\item $G$ heißt \emph{(Kanten-)gewichteter} Graph, falls jede Kante $e \in
  E$ ein Gewicht $c(e) \in \real$ zugeordnet ist, kurz $G = (V, E, c)$.
\end{itemize}

\section{Das Minimalgerüstproblem}
Gegeben sei ein ungerichteter, zusammenhängender und kantengewichteter Graph $G
= (V,E,c)$.

\begin{defn}
  Eine Kantenmenge $T \subset E$ mit $|T| = |V| - 1$ heißt \emph{Gerüst}
  (Spannbaum, spanning tree), falls der induzierte Subgraph $G_T = (V,T)$
  zyklenfrei ist.

  Das \emph{Minimalgerüstproblem} besteht darin, ein Gerüst mit minimalem
  Gesamtgewicht
  \[ \sum_{e \in T} c(e) \]
  zu finden.
\end{defn}

\begin{rmrk*}
  $G_T$ ist zusammenhängend.
\end{rmrk*}

\subsection{Algorithmus von Kruskal}
\emph{Prinzip}. Führe den folgenden Schritt so oft wie möglich aus: \\
Wähle unter den noch nicht betrachteten Kanten von $G$ eine kürzeste Kante, die
mit den schon für das Gerüst gewählten Kanten \emph{keinen} Zyklus bildet.

\paragraph{Algorithmus von Kruskal}
\[ \begin{aligned}
    &T := \emptyset \\
    &\text{Solange } |T| < n - 1: & &\text{Ermittle eine kürzeste Kante } e \in
    E. \\
    & & &\text{Setze } E := E \setminus \{ e \} \text{ und falls } T \cup \{ e
    \} \text{zyklenfrei,} \\
    & & &\text{setze } T := T \cup \{e\}.
  \end{aligned}
\]
Laufzeit: $O(|E| \log |E|)$, wobei $|E| \le |V| ( |V| - 1 )$, also $O(n^2 \log
n)$.

%% Beispiel?
\begin{exmp}
  .
\end{exmp}

\subsection{Algorithmus von Prim/Dijkstra}
\emph{Prinzip.} Wähle einen Knoten $v_0 \in V$ und setze $T := \emptyset$.
Solange $|T| < |V| - 1$ gilt, wähle eine Kante $e \in E$ mit minimalem Gewicht,
so dass $T \cup \{ e \}$ zyklenfrei und zusammenhängend ist.

\paragraph{Algorithmus von Prim/Dijkstra}
$T := \emptyset, S := \{u\}$ mit $u \in V$

Für alle $v \in V \setminus \{u\}$ setze, falls $(u,v) \in E$:
\[ h(v) := c(u,v), p(u,v) := u, \]
anderenfalls $h(v) := \infty$.

Solange $|T| < n - 1$, bestimme $v \in V \setminus S$ mit
\[ h(v) = \min \{ h(t) : t \in V \setminus S \}, \]
setze
\[ T := T \cup \{ (p(v), v) \}, \quad S := S \cup \{ v \}, \]
für alle $t \in T \setminus S$ mit $(v,t) \in E$ und $c(v,t) < h(t)$ setze $h(t)
:= c(v,t)$, $p(t) := v$.

\begin{rmrk*}
  Es gilt $h(v) = \min \{ c(u,v) : u \in S \}$, $v \in V \setminus S$. $p(v)$
  lässt sich als ``Vorgänger'' interpretieren.

  Bei ``vielen'' Kanten ist Prim/Dijkstra besser, Kruskal kann aber
  für ``wenige'' Kanten unter Umständen schneller sein.

  Ein weiterer gebräuchlicher Algorithmus ist der Sollin-Algorithmus $O(n^2)$.
\end{rmrk*}

%% Beispiel
\begin{exmp}
  .
\end{exmp}

\section{Optimale Wege}
\subsection{Das Kürzeste-Wege-Problem}
(shortest path problem)

Gegeben sei ein gerichteter, bogenbewerteter Graph $G = (V,E,c)$ mit $c(e) \ge
0$ für alle $e \in E$.

\begin{defn}
  Das \emph{Kürzeste-Wege-Problem} (KW-Problem) besteht darin, ausgehend von
  einem Startknoten $v_1 \in V$ zu jedem Knoten $v_k \in V \setminus \{ v_1 \}$
  einen kürzesten Weg $v_1 \to v_{k,1} \to v_{k,2} \to \cdots \to  v_{k,t(k)} =
  v_k$ zu finden.
\end{defn}

\begin{aus}
  Es existiert eine Bogenmenge $E_w \subset E$ mit $|E_w| = |V| -1$, die für
  jeden Knoten $v_k \in V$, $v_k \ne v_1$ einen kürzesten Weg von $v_1$ nach
  $v_k$ repräsentiert.
\end{aus}

\begin{proof}
  (Indirekt) Angenommen, die Menge der Bögen, die die kürzesten Wege zu allen
  Knoten $v_k \ne v_1$ bilden, entählt mehr als $|V| - 1$ Bögen. Dann existiert
  ein Knoten $v \ne v_1$ mit Eingangsgrad $\delta^-(v) \ge 2$. Das heißt, es
  gibt zwei optimale Wege von $v_1$ nach $v$. Nur einer der Bögen, die in $v$
  enden, ist erforderlich, um eine Bogenmenge zu erhalten, die kürzeste Wege zu
  allen Knoten $v_k \ne v_1$ repräsentiert.
\end{proof}

\begin{flg}
  $E_w$ ist zusammenhängend, also ein gerichtetes Gerüst (arborescence).
\end{flg}

\subsection*{Algorithmus von Dijkstra}
\emph{Prinzip.} Ausgehend von $v_1$ werden bereits bekannte kürzeste Wege durch
Hinzufügen weiterer Bögen verlängert, um zum Knoten einen Weg zu finden oder
einen kürzeren Weg als die bisher bekannten.
 
\emph{Bezeichnungen.}
\begin{itemize}
\item $v_k$ wird durch $k$ repräsentiert.
\item $K$: Menge der Knoten, zu denen bereits ein kürzester Weg bekannt ist.
\item $S$: Menge der Knoten, zu denen ein Weg gefunden wurde.
\item $p(k)$: Vorgängerknoten von $k$ auf dem bisher kürzesten Weg zu $k$.
\item $l(k)$: Länge des bisher kürzesten Wegs zu $k$.
\end{itemize}

\paragraph{Algorithmus von Dijkstra}
\begin{enumerate}[S1:]
  \setcounter{enumi}{-1}
\item $K := \{v_1\}$, $S := \{ v_k \in V : (v_1, v_k) \in E \}$,
  \[ p(k) := \begin{cases}
      -1, &v_k \in V \setminus S, \\
      1, &v_k \in S,
    \end{cases}
    \qquad
    l(k) := \begin{cases}
      0, & v_k = v_1 \\
      c(v_1,v_k), & (v_1, v_k) \in E, \\
      \infty, & v_k \notin S.
    \end{cases}
  \]
\item Wähle $v_i \in S$ mit $l(i) \le l(k)$ für alle $v_k \in S$. \\
  Setze $K := K \cup \{ v_i \}$, $S := S \setminus \{v_i\}$. \\
  Für alle $v_k \in S$ mit $(v_i, v_k) \in E$: Falls $l(i) + c(v_i,v_k) \le
  l(k)$, dann setze $p(k) := i$ und $l(k) := l(i) + c(v_i,v_k)$. \\
  Für alle $v_k \in V \setminus (K \cup S)$ mit $(v_i, v_k) \in E$: Setze $p(k) =
  i$ und $l(k) := l(i) + l(i) + c(v_i,v_k)$.
\item[] Falls $S \ne \emptyset$, gehe zu $S1$.
\end{enumerate}

Laufzeit: $O(n^2)$.

\begin{aus}
  Der Algorithmus von Dijkstra liefert die kürzeste Weglänge $l(k)$ von $v_k \ne
  v$.
\end{aus}

\begin{proof}
  Im Algorithmus werden kürzeste Wege, nach wachsender Weglänge geordnet,
  erzeugt. In einem beliebigen Schritt seien $l(i)$, $i \in K$, die kürzesten
  Weglängen von $v_1$ zu $v_i$. Von $v_i$ aus Algorithmus aus werden höchstens
  längere Wege konstruiert (da $c(e) \ge 0$), von denen einer zu einem Knoten
  (nicht in $K$) einen weiteren kürzesten Weg liefert.
\end{proof}

\subsection{Das Längste-Wege-Problem}
(longest path problem)

Gegeben sei ein gerichteter, bogenbewerteter Graph $G = (V,E,c)$.

\begin{defn}
  Das \emph{Längste-Wege-Problem} besteht darin, ausgehend von einem Startknoten
  $v_1$ zu jedem Knoten $v_k \ne v_1$ einen längsten Weg zu finden.
\end{defn}

\textbf{Prinzip.} Ausgehend von $v_1$ werden bereits bekannte Wege durch
Hinzufügen weiterer Bögen verlängert, um zu neuen Knoten zu gelangen bzw.
längere Wege zu finden.

\textbf{Bezeichnungen.}
\begin{itemize}
\item $K$: Menge der Knoten, zu denen im vorherigen Iterationsschritt ein
  längerer Weg gefunden wurde.
\item $S$: Menge der Knoten, zu denen im aktuellen Iterationsschritt ein
  längerer Weg gefunden wird.
\item $p(k)$: Vorgängerknoten auf dem Weg zu $v_k$.
\item $l(k)$: Länge des bisher längsten Wegs zu $v_k$.
\item $\nu$: Iterationszähler.
\end{itemize}

\subsubsection*{Algorithmus von Ford/Moore}
\begin{enumerate}[S1]
  \setcounter{enumi}{-1}
\item $K := \{v_1\}$, $S := \{ v_k \in V : (v_1, v_k) \in E \}$, $\nu := 1$.
  \begin{align*}
    p(k) &:= \begin{cases}
      1 & \text{für } v_k \in S, \\
      0 & \text{für } v_k \in V \setminus ( S \cup \{ v_1 \}, )
    \end{cases} \\
    l(k) &:= \begin{cases}
      0 & \text{für } v_k = v_1, \\
      c(v_1, v_k) & \text{für } v_k \in S, \\
      -\infty & \text{sonst.}
    \end{cases}
  \end{align*}
\item Falls $S = \emptyset$ $\rightarrow$ \verb+STOP+.\\
  Falls $\nu = |V|$ $\rightarrow$ \verb+STOP+.
\item Setze $K := S$, $S := \emptyset$, $\nu := \nu + 1$.
  
  Für alle $v_i \in K$ und alle $v_k \in V$ mit $l(i) + c(v_i, v_k) > l(k)$
  setze $S := S \cup \{ v_k \}$, $p(k) := i$, $l(k) := l(i) + c(v_i, v_k)$. Gehe
  zu S1.
\end{enumerate}

\begin{aus}
  Der Algorithmus von Ford/Moore ermittelt zu jedem Knoten, der von $v_1$
  erreichbar ist, einen längsten Weg, sofern der Algorithmus mit $S = \emptyset$
  stoppt. Anderenfalls enthält der Graph Kreise positiver Länge.
\end{aus}

\begin{proof}
  Es werden \emph{alle} bereits betrachteten Wege auf Verlängerung untersucht.
  Existiert kein Kreis positiver Länge, dann erfolgt der Abbruch nach $|V|$
  Iterationsschritten. 
\end{proof}

\begin{exmp}
  .
\end{exmp}

\begin{rmrk*}
  Der Algorithmus von Ford/Moore kann auch zur Bestimmung kürzester Wege benutzt
  werden. Die Bedingung $c(e) \ge 0$ muss nicht erfüllt sein.
\end{rmrk*}

\subsection{Minimale Wege zwischen allen Knoten}
Gegeben sei ein gerichteter, bogenbewerteter Graph $G = (V,E,c)$. Ziel ist nun
die Bestimmung jeweils eines kürzesten Weges von jedem Knoten zu \emph{allen}
anderen.

\textbf{Bezeichnungen.}
\begin{itemize}
\item $P$: Vorgängermatrix, $p_{ik} = h$, $v_k \in V$, wenn $(v_h, v_k)$ der
  letzte Bogen auf dem Weg von $v_i$ nach $v_k$ ist. $p_{ik} = 0$, wenn kein Weg
  von $v_i$ nach $v_k$ existiert.
\item $L$: Entfernungsmatrix, $l_{ik} =$ Länge des kürzesten Weges (bisher
  gefunden) von $v_i$ nach $v_k$, sonst $l_{ik} = \infty$.
\end{itemize}

\subsubsection*{Tripel-Algorithmus von Floyd}
Setze $p_{ik} := i$ und $l_{ik} := c(i,k)$ für $(v_i, v_k) \in E$ und
$p_{ik} := 0$ nud $l_{ik} := \infty$ sonst.

Führe für alle $v_k \in V$ folgende allgemeine Schritte aus:

Bilde für alle $v_i, v_k \in V \setminus \{ v_k \}$ die Summe
\[ l_{ik}^h = l_{ih} + l_{hk}. \]
Falls $l_{ik}^h < l_{ik}$, setze $l_{ik} := l_{ik}^h$ und $p_{ik} := p_{hk}$.

Laufzeit: $O(n^3)$ (ohne Voraussetzung $c(e) \ge 0$).

\subsection{Netzplantechnik: Metra-Potential-Methode}
Ein Gesamtprojekt bestehe aus Teilprojekten (Aktivitäten). Modellierung als
Knoten- und kantenbewerteter Graph.
\begin{itemize}
\item Knoten $\simeq$ Aktivitäten, $i \in I = \{1, \ldots, N \}$, Dauer der
  Aktivitäten: $d_i$ ($\ge 0$), $i \in I$.
\item Kanten $\simeq$ Koppelbedingungen, $\KA(i,j)$ $\simeq$ Abstand
  (Koppelabstand)
  \begin{enumerate}[a)]
  \item$\KA(i,j) \ge 0$ bedeute, dass die Aktivität $j$ \emph{frühestens} nach
    $\KA(i,j)$ Zeiteinheiten nach Beginn von Aktivität $i$ beginnen kann.
  \item $\KA(i,j) \le - 0$ bedeute, dass die Aktivität $j$ \emph{spätestens}
    nach $|\KA(i,j)|$ Zeiteinheiten nach Beginn von $i$ beginnen muss.
  \end{enumerate}
\item $\FT(i)$: Frühestmöglicher Anfangstermin für $i$
\item $\ST(i)$: Spätester zulässiger Anfangstermin für $i$
  \begin{enumerate}[a)]
  \item $\FT(j) \ge \FT(i) + \KA(i,j)$
  \item $\FT(j) \le \FT(i) - \KA(i,j)$
  \end{enumerate}
\item Die Zuordnung eines Bogens ist wie folgt:
  \begin{enumerate}[a)]
  \item Bogen von $i$ nach $j$: $\KA(i,j) \ge 0$.
  \item Wegen
    \[ \FT(i) \ge \FT(j) + \KA(i,j) \]
   vertauschen $i$ und $j$ ihre Rollen, deshalb gilt bei Bögen von $j$ nach $i$:
   $\KA(j,i) \le -0$.
  \end{enumerate}
\end{itemize}

\subsubsection*{Sonderfälle}
\begin{enumerate}
\item $\KA(i,j) = d_i$ bedeutet, $j$ kann frühestens nach Abschluss von $i$
  beginnen.
\item $\KA(i,j) = \pm 0$ bedeutet, dass $i$ und $j$ gleichzeitig beginnen.
\end{enumerate}

Der Knoten $A$ ist der Beginn des Projekts, $E$ ist die Abschlusssektion.

\begin{lem}
  Bei der Aufstellung eines Netzplans sind solche Kreise unzulässig, für die die
  Summe der zugehörigen $\KA$ positiv ist.
\end{lem}

%% Beispiel
\begin{exmp}
  .
\end{exmp}

\begin{aus}
  Der frühestmögliche Anfangstermin $\FT(j)$ für den Beginn der Aktion $j$ ist
  gleich der maximalen Länge aller Wege vom Startknoten $A$ zum Knoten $j$.
\end{aus}

\begin{proof}
  Jeder Weg beschreibt eine Folge von Aktivitäten. Bevor $j$ beginnen kann,
  müssen alle Aktivitäten berücksichtigt worden sein. Die zugehörige Zeit ist
  gleich der Weglänge. Aus der Berücksichtigung der ungünstigsten Folge von $A$
  nach $j$ folgt, dass die maximale Weglänge $\FT(j)$ definiert.
\end{proof}

Analog gilt
\begin{aus}
  Der späteste zulässige Termin $\ST(j)$ ist gleich dem Wert $\FT(E)$ abzüglich
  der maximalen Länge aller Wege von $j$ zum Projektende $E$.
\end{aus}

\begin{flg}
  Die Bestimmung der $\FT(j)$ und $\ST(j)$ kann mit dem Algorithmus
  FML\footnotemark erfolgen.
\end{flg}
\footnotetext{Ford/Moore für längste Wege}

Ein Weg mit der Länge $T = \FT(E)$ heißt \emph{kritischer Weg}.

Die \emph{minimale Projektdauer} ist
\[ \max_j \{ \FT(j) + d_j \}. \]

\subsubsection*{MPM-Algorithmus}
\begin{enumerate}[(i)]
\item Setze $v_a := A$ mit $i = v_a$ und
  \[ c_{ij} := \begin{cases}
      \KA(i,j) &\text{für } \KA(i,j) \ge 0, \\
      \KA(j,i) &\text{für } \KA(i,j) \le - 0. \\
    \end{cases}
  \]
\item Wende den FLM-Algorithmus an (liefert $l(j)$). Setze $\FT(j) := l(j)$,
  $T := \FT(E)$.
\item Setze $v_a := E$ und orientiere die Bögen um, gemäß
  \[ c_{ij} := \begin{cases}
      \KA(i,j) &\text{für } \KA(i,j) \ge 0, \\
      \KA(j,i) &\text{für } \KA(i,j) \le - 0. \\
    \end{cases}
  \]
\item Wende den Algorithmus FML an (liefert wieder $l(j)$). Setze $\ST(j) := T -
  l(j)$.
\end{enumerate}

\begin{lem}
  Es gilt
  \[ \FT(j) \le \ST(j) \]
  für alle $j$.
\end{lem}

Die Größen $\GP(j) := \ST(j) - \FT(j)$ heißen \emph{Pufferzeit}. Eine Aktivität
$j$ mit $\GP(j) = 0$ heißt \emph{kritisch}.

\begin{aus}
  Kritische Aktivitäten liegen genau auf kritischen Wegen.
\end{aus}

\begin{aus}
  Ein Bogen mit $\KA(i,j)$ zwischen $i$ und $j$ liegt genau dann auf einem
  kritischen Weg, wenn
  \[ \GP(i) = \GP(j) = 0 \quad \text{und} \quad \ST(j) - \FT(i) = \KA(i,j). \]
\end{aus}

\begin{rmrk*}
  Identifiziert man $i$ mit dem Start einer Aktivität $(i,j)$ zum Knoten $j$ und
  setzt die Dauer dieser Aktivität zu
  \[ d(i,j) = \KA(i,j) \ge 0, \]
  so erhält man die Methode CPM\footnote{Critical path method}.
\end{rmrk*}

\section{Maximale Flüsse in Graphen}
\subsection{Problemstellung}
Sei $G = (V,E,k)$ ein gerichteter Graph mit Kapazitätsschranken $k_l \ge 0$ für
alle $e \in E$. Weiterhin sei $q \in V$ die \emph{einzige} Quelle und $s \in V$
die \emph{einzige} Senke. Alle Daten seien ganzzahlig.
\begin{align*}
  E^+(v) &:= \{ e \in E: e = (v,u) \}, \\
  E^-(v) &:= \{ e \in E: e = (u,v) \}. \\
\end{align*}

\begin{defn} %% 5.4
  Eine Funktion $x:E \to \real^1$ heißt \emph{Fluss}, wenn
  \[ 0 \le x_e \le k_e \]
  für alle $e \in E$ und
  \[ \sum_{e\in E^-(v)} x_e = \sum_{e \in E^+(v)} x_e \]
  für alle $v \in V \setminus \{ q, s \}$.
\end{defn}



%% Stückware

\subsubsection*{Max-Flow-Problem}
Finde einen Fluss $x$ mit maximaler Flussstärke $f(x)$.

\begin{exmp} %% 5.5
  Bogenmarkierungen $x_l / x_k$

  %% Bild

  Der Fluss ist nicht maximal.

  O.B.d.A. gelte für alle $e = (u,v) \in E$, dass auch $(v,u) \in E$ sei. Falls
  dies nicht der Fall ist für ein $e = (u,v) \in E$, dann wird $(v,u)$
  hinzugefügt  und $k(v,u) := 0$ gesetzt.
\end{exmp}

\begin{defn} %% 5.5
  Sei $x : E \to \real^1$ ein zulässiger Fluss, das heißt unter anderem es gilt
  $0 \le x_e \le k_e$ für alle $e \in E$. Für $e = (u,v) \in E$ definiert
  \[ r(e) = r(u,v) := k_e - x_e \]
  das \emph{Residuum} von $e$ (Restkapazität). Weiterhin sei
  \[ E(x) := \{ e \in E : x(e) > 0 \}. \]
  Der \emph{Residualgraph} (Restgraph) ist definiert als
  \[ G(x) := (V, E(x), r(e)_{e \in E}). \]
\end{defn}

\begin{rmrk*}
  Ist $x$ ein  Fluss, so ist für eine Kante $(u,v) \in E$  das Residuum die
  maximal zusätzliche Flussmenge, die von $q$ über $(u,v)$ und $(v,u)$ nach $s$
  geschickt werden kann. Offenbar gilt bei Flüssen $r(e) \ge 0$ für alle $e \in
  E$. Der Graph $G(x)$ enthält gerade jene Bögen, die Möglichkeiten zur
  Flusserhöhung repräsentieren.
\end{rmrk*}

\subsection{Algorithmus von Ford/Fulkerson (1956)}
\begin{enumerate}
\item Start mit $x = 0$.
\item Finde einen Fluss $x'$ von $q$ nach $s$ in $G(x)$ bzw. eine
  flussvergrößernde Kette in $G(x)$ mit Flussstärke $f'$ (maximale Flussstärke
  entlang des Weges). Aktualisiere $x$ durch $x := x + x'$.
\item Wiederhole Schritt 2, solange ein positiver Fluss in $G(x)$ existiert.
\end{enumerate}

\textbf{Aufwand:} Bei ganzzahligen Kapazitäten erfolgt eine Flussvergrößerung um
mindestens eine Einheit, damit ist der Algorithmus endlich.

Unter Umständen proportional zum Optimalwert, das heißt pseudopolynomial.

\subsection{Algorithmus von Edmonds/Karp (1972)}
Wie im Algorithmus von Ford/Fulkerson, aber in Schritt 2 erfolgt stets eine
Breitensuche, das heißt es wird ein Weg von $q$ nach $s$ in $G(x)$ mit kleinster
Bogenzahl ermittelt.

Es bezeichne $\delta(u,v)$ die Abstand zwischen $u \in V$ und $v \in V$ in
$G(x)$, also die Anzahl der Bögen auf einem kürzesten Weg von $u$ nach $v$.

\begin{lem} %% 5.11
  Im Ablauf des Edmonds/Karp-Algorithmus sei $x'$ ein Fluss, der aus dem Fluss
  $x$ durch eine flussvergrößernde Kette $P$ erhalten wird. Dann gilt
  \[ \delta_x (q,v) \le \delta_{x'}(q,v) \]
  für alle $v \in V \setminus \{q\}$.
\end{lem}

\begin{proof}
  Angenommen, es gilt $\delta_x(q,v) > \delta_{x'}(q,v)$ für ein $v \in V$.
  O.B.d.A. habe $v$ zusätzlich minimales $\delta_{x'}(q,v)$. Also gilt für alle
  anderen $u \in V$:
  \[ \delta_x(q,u) \le \delta_x(q,v) \qLRq \delta_x(q,u) \le \delta_{x'}
    (q,u). \tag{$*$} \]
  Sei $P'$ ein kürzester Weg von $q$ nach $v$ in $G(x)$ und $u$ der letzte
  Knoten vor $v$ auf $P'$. Dann gilt
  \[ \delta_x (q,u) \le \delta_{x'}(q,u). \tag{$**$} \]
  Betrachten wir $x(u,v)$ vor der Flussvergrößerung:

  Fall a): Falls $x(u,v) < k(u,v)$, dann ist $(u,v) \in G(x)$ und es gilt
  \[ \delta_x(q,v) \le \delta_x(q,u) + 1
    \le \delta_{x'}(q,u) + 1 \le \delta_{x'}(q,v) \]

  Fall b): Falls $x(u,v) = k(u,v)$, dann ist $(u,v)$ \emph{nicht} in $G(x)$,
  aber in $G(x')$. Daraus folgt, dass die flussvergrößernde Katte den Bogen
  $(u,v)$ enthalten muss und es gilt:
  \[ \delta_x(q,v) = \delta_x(q,u) - 1
    \overset{(**)}{\le} \delta_{x'}(q,u) - 1
    =\footnotemark \delta_{x'}(q,v) - 2
    \le \delta_{x'}(q,v).
  \]
  \footnotetext{Weil $u$ direkter Vorgänger von $v$ ist.}
  Das ist ein Widerspruch.
\end{proof}

\begin{lem} %% 5.12
  Der Edmonds/Karp-Algorithmus führt höchstens $O(|V| \cdot |E|)$
  Flussvergrößerungen (Augmentierungen) durch.
\end{lem}

\begin{proof}
  Ein Bogen $(u,v)$ heißt \emph{kritisch} auf dem augmentierten Weg $P$ genau
  dann, wenn
  \[ r_x(u,v) = k_x(P) := \min \{ r_x(e) : e \in P \}. \]
  Ein kritischer Bogen verschwindet bei einer Augmentierung aus dem
  Residualgraph. Wie kann ein Bogen $(u,v)$ kritisch werden? Ein Bogen $(u,v)$
  kann in einem späteren Residualgraphen wieder auftreten, wenn er irgendwie
  wieder Restkapazität $> 0$ erhält. Das heißt, $(v,u)$ liegt dann auf einem
  augmentierenden Weg.

  Sei $x$ ein Fluss, bei dem $(u,v)$ kritisch war, das heißt
  \[ \delta_x(q,v) = \delta_x(q,u) + 1. \]
  $x'$ sei der Fluss, bei dem der Bogen $(v,u)$ das nächste Mal wieder auf dem
  kritischen Weg liegt. Dann gilt:
  \[ \delta_{x'}(q,u) = \delta_{x'} (q,v) + 1
    \ge\footnotemark \delta_x(q,v) + 1 = \delta_x(q,u) + 2.  \]
  \footnotetext{Wegen Lemma 5.11}
  Also gilt: Zwischen zwei Schritten, in denen ein Bogen $(u,v)$ kritisch ist,
  erhöht sich der Abstand von $q$ um mindestens 2. Der Abstand kann höchstens
  $|V| - 2$ sein, damit wird jeder Bogen höchstens $(|V|-2)/2 = O(|V|)$ mal
  kritisch.
\end{proof}

%% ...

\begin{aus} %% 5.15
  Ein zulässiger Fluss $x$ mit Stärke $\omega$ ist genau dann optimal, wenn der
  Residualgraph $G(x)$ keinen Zyklus negativer Länge enthält.
\end{aus}

\begin{proof}[Beweisidee]
  Man zeigt: Ein zulässiger Fluss der Stärke $\omega$ ist genau dann nicht
  $\omega$-optimal, wenn im Residualgraphen $G(x)$ eine Flusszirkulation
  existiert, deren Gesamtkosten negativ sind.
\end{proof}

\begin{exmp} %% 5.8
  Bogenmarkierungen $x_e / k_e / c_e$.  Start mit $x_e = 0$ für alle $e \in E$.

  %% Bild

  Der kürzeste Weg ist in $G(x)$ ist: $v_1 \to v_2 \to v_3 \to v_4 \to v_5$.

  %% G(x)

  $v_1 \to v_3 \to v_2 \to v_4 \to v_5 \to v_6$ mit Stärke 2

  %% Mehr Graphen

  Maximaler Fluss: $\omega = 5$, Kosten $c(x) = 38$.
\end{exmp}

\begin{rmrk*}
  Der Algorithmus von Busacker/Gower ist pseudopolynomial. Es gibt auch
  polynomiale Algorithmen, siehe Ahuja/Magnanti/Orlin ``Network Flows''.
\end{rmrk*}

\appendix
%% Nochmal zum Simplex-Verfahren
\chapter*{Anmerkungen}
\addcontentsline{toc}{chapter}{Anmerkungen}

\section*{Klee-Minty-Cube}
\addcontentsline{toc}{section}{Klee-Minty-Cube}
\begin{align*}
  x_1 &\le 5 \\
  4 x_1 + x_2 &\le 25 \\
  8 x_1 + 4 x_2 + x_5 &\le 125 \\
      &\vdots \\
  2^D x_1 + 2^{D-1} + \ldots + 4 x_{D-1} + x_D &\le 5^D \\
  x_1, x_2, \ldots, x_D &\ge 0
\end{align*}
\[ z = 2^{D-1} x_1 + 2^{D-2} x_2 + \ldots + 2 x_{D-1} + x_D \to \max. \]

Behauptung: Das Simplex-Verfahren benötigt $2^{D-1}$ Schritte, falls die
sogenannte Dantzig-Rule\footnote{%
  Dabei wird immer der kleinste transformierte Zielfunktionskoeffizient zur Wahl
  der Pivotspalte bei Min-Formulierung benutzt.
} verwendet wird. Also ist das Simplexverfahren nicht polynomial.

\section*{Packungsprobleme}
\addcontentsline{toc}{section}{Packungsprobleme}
%% Bild
Fragestellungen:
\begin{enumerate}
\item Finde alle Rechtecke (mit Mindestgröße) $\to \Omega_{\max}$, diese müssen
  nicht disjunkt sein.
\item Finde eine Überdeckung der Nutzfläche mit minimaler Anzahl von Rechtecken
  (aus $\Omega_{\max}$).
\item Finde eine Packung mit maximaler Fläche von Rechtecken mit Mindestgrößen
  oder aus $\Omega_{\max}$.
\item Finde eine Packung mit maximaler Fläche mit höchstens $k$ Rechtecken ($k
  \in \nat$).
\end{enumerate}
\end{document}