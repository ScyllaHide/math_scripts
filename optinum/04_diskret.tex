\chapter{Diskrete Optimierung}
\section{Branch and bound-Methode}
\subsection{Grundlagen}
Problem:
\[ (\text{P}) = (\text{P}_0) \qquad f(x) \to \min \subjto x \in D \cap E. \]
Relaxation zu (P):
\[ (\text{Q}) \qquad g(x) \to \min \subjto x \in E, \]
wobei $g(x) \le f(x)$ für alle $x \in D \cap E$.

\begin{thm*}[Wiederholung Satz 1.1]
  Sei $\bar{x}$ Lösung von (Q) und gelte $\bar{x} \in
  D$ sowie $f(\bar{x}) = g(\bar{x})$. Dann ist $\bar{x}$ Lösung von (P).
\end{thm*}

\emph{Prinzip der b\&b-Methode}: Die Relaxationsmenge $E$ wird durch
Separationen in Teilmengen $E_i$ zerlegt. Es entstehen Teilprobleme
\[ (\text{P}_i) \qquad f(x) \to \min \subjto D \cap E_i. \]
Jedem Teilproblem $(\text{P}_i)$ wird eine Zahl $b( \text{P}_i )$, eine untere
Schranke, zugeordnet, so dass gilt
\begin{enumerate}[(a)]
\item $b( \text{P}_i ) \le \min \{ f(x) : x \in D \cap E_i \}$,
\item $b( \text{P}_i ) = f(x)$, falls $|D \cap E_i| = |\{x\}| = 1$,
\item $b( \text{P}_i ) \le b( \text{P}_k )$, falls $E_k \subset E_i$.
\end{enumerate}

\subsection{Allgemeiner b\&b-Algorithmus}
Bezeichnungen: \\
$R$ ... Menge der noch zu betrachtenden Teilprobleme, \\
$bar{z}$ ... Zielfunktionswert der bisher besten gefundenen zulässigen Lösung.

\begin{enumerate}[S1]
  \setcounter{enumi}{-1}
\item \emph{Initialisierung.} Bestimme $b(\text{P}_0)$:
  \begin{enumerate}[(a)]
  \item Falls $\bar{x} \in D \cap E$ bekannt ist mit $f(\bar{x}) = b(\text{P}_0)$,
    \verb+STOP+, $\bar{x}$ ist zulässige Lösung.
  \item Setze $R := \{ \text{P}_0 \}$, $\bar{z} := \infty$ oder
    $\bar{z} := f( \bar{x} )$, wenn $\bar{x} \in D \cap E$ bekannt.
  \end{enumerate}
\item \emph{Abbruchtest.} Falls $R = \emptyset$, \verb+STOP+. Falls $\bar{z} =
  \infty$, dann ist (P) unzulässig, anderenfalls ist $\bar{x}$ Lösung.
\item \emph{Strategie.} Wähle entsprechend einer Auswahlstrategie ein
  $\text{P}_i \in R$ und setze $R := R \setminus \text{P}_i$.
\item \emph{Zerlegung} (branch). Zerlege $\text{P}_i$ durch Separation in
  Teilprobleme $\text{P}_{i,1}, \ldots, \text{P}_{i,k}$. Setze $j := 1$.
\item \emph{Schranken und Dominanztest.}
  \begin{enumerate}[(a)]
  \item Berechne $b(\text{P}_{i,j})$. Falls dabei oder auf andere Weise ein
    $\tilde{x} \in D \cap E$ gefunden wurde mit $f(\tilde{x}) < \bar{z}$, setze
    $\bar{x} :=  \tilde{x}$, $\bar{z} = f(\tilde{x})$.
  \item Falls $b(\text{P}_{i,j}) < \bar{z}$, dann setze $R := R \cup \{
    \text{P}_{i,j} \}$. Falls $j < k_i$, setze $j := j + 1$ und gehe zu (a).
  \item Setze $R := R \setminus \{\text{P}_k\}$ für alle $\text{P}_k \in R$ mit
    $b(\text{P}_k) \ge \bar{z}$.
  \end{enumerate}
  Gehe zu S1.
\end{enumerate}

\begin{rmrk*}
  \begin{enumerate}[(1)]
  \item Die Endlichkeit ist zu sichern durch
    \[ |E_{i,j} \cap D | < | E_i \cap D | \quad \text{für alle } j, \]
    \[ b( \text{P}_{i,j} ) \ge b( \text{P}_i ) + \eps \quad \text{mit } \eps >
      0, \text{ für alle } i, j. \]
  \item Festlegung der Auswahlstrategie:
    \begin{itemize}
    \item Minimalstrategie (best bound search)
      \[ b(\text{P}_i) \le b(\text{P}_k) \quad \text{für alle } \text{P}_k \in
        R. \] 
    \item LIFO (last-in-first-out, depth first search) bzw. FIFO
      (first-in-first-out, breadth first search). Jeweils Auswahl von
      $\text{P}_i$ unter den Teilproblemen mit minimaler bzw. maximaler
      Verzweigungstiefe und kleinster Schranke.
    \end{itemize}
  \item Repräsentation durch Verzweigungsbaum.
  \item Wichtige zusätzliche Tests sind in S4 möglich:
    \begin{itemize}
    \item Zulässigkeitstest: $D \cap E_{i,j} = \emptyset$ $\Rightarrow$ $R := R
      \setminus \{ \text{P}_{i,j} \}$.
    \item Dominanztest: $\text{P}_i$ dominiert $\text{P}_k$, falls eine Lösung
      von $\text{P}_i$ mindestens so gut wie eine Lösung von $\text{P}_k$ ist.
    \end{itemize}
  \end{enumerate}
\end{rmrk*}

\subsection{Beispiele für b\&b-Algorithmen}
\subsubsection{Das 0/1-Rucksackproblem}
Voraussetzung: $c_i > 0$, $0 < a_i \le b$ für alle $i \in I = \{ 1, \ldots, n
\}$. 
\begin{align*}
  f(x) &= c^\top x \to \max \subjto a^\top x \le b, 
         x \in \boole^n = \{0,1\}^n \tag{P} \\
  g(x) &= c^\top x \to \max \subjto a^\top x \le b, 
         x \in [0,1]^n \tag{Q}
\end{align*}
Also
\[ E = \{ x \in \real^n : 0 \le x_i \le 1, i \in I \}, \qquad D = \boole^n
  (=\integer^n). \]
O.B.d.A. Voraussetzung:
\[ \frac{c_i}{a_i} \ge \frac{c_{i+1}}{a_{i+1}}, \qquad i = 1, \ldots, n-1. \]
Definiere Teilprobleme ($\bar{x}_i$, $i = 1, \ldots, k$, seien bereits
fixierte Variablen):
\[ \begin{aligned}
    \text{P}_k(\bar{x}) : \quad
    &\sum_{i=1}^k c_i \bar{x}_i + \sum_{i=k+1}^n c_i x_i \to \max \subjto \\
    &\sum_{i=k+1}^n a_i x_i \le b - \sum_{i=1}^k a_i \bar{x}_i,
    \quad x_i \in \boole, i = k+1, \ldots, n.
  \end{aligned}
\]
Zugehörige Relaxation $=$ stetige Relaxation:
\[ \text{Q}_k(y) : \quad
  z_k(y) := \max \left\{ \sum_{i=k+1}^n c_i x_i : \sum_{i=k+1}^n a_i x_i \le
    y, 0 \le x_i \le 1 \text{ für alle } i \right\}. \]
Damit ergibt sich aber eine Schranke für $\text{P}_i$:
\[ b( \text{P}_i ) := \sum_{i=1}^k c_i \bar{x}_i + z_k(y). \]
Es gilt
\[ z_k(y) = \sum_{i=k+1}^p c_i + \rez{a_{p+1}} \cdot \left( y - \sum_{i=k+1}^p
    a_i \right), \]
wobei
\[ \sum_{i=k+1}^p a_i \le y \le \sum_{i=k+1}^{p+1} a_i. \]

\begin{exmp}
  Beispiel:
  \[ \begin{aligned}
      z = c^\top x &= 8x_1 + 16x_2 + 20x_3 + 12x_4 + 6 x_5 + 10x_6 + 4x_7 \to
      \max \subjto \\
      a^\top x &= 3x_1 + 7x_2 + 9x_3 + 6x_4 + 3x_5 + 5x_6 + 2x_7 \le 17, \quad
      x_i \in \{0,1\}.
    \end{aligned}
  \]
  Es gilt
  \[ b(\text{P}_0) = 8 + 16 + \frac{20}{9} \cdot 7 = \lfloor \ldots \rfloor =
    39. \]
  Mögliche Teilprobleme: $\text{P}_1 : x_1 = 1$, $\text{P}_2 : x_1 = 0$.
  \begin{align*}
    b(\text{P}_1) &= 39, \\
    b(\text{P}_2) &= 16 + 20 + \frac{12}{6} \cdot (17-7-9)
                    = \lfloor \ldots \rfloor = 38, \\
    b(\text{P}_3) &= 39, \\
    b(\text{P}_4) &= 8 + 20 + \frac{12}{6} \cdot 5 = 38, \\
    b(\text{P}_5) &= \infty & D \cap E_5 = \emptyset, \\
    b(\text{P}_6) &= 8 + 16 + 0 + 2 \cdot 7 = 38.
  \end{align*}
  Aus ``Heuristik'' ist
  \[ \bar{x} = (1,1,0,0,0,1,1)^\top, \qquad z(\bar{x}) = 38 \]
  bekannt. 38 ist der Optimalwert, $\bar{x}$ die Lösung.
\end{exmp}

\subsubsection{Ganzzahlige lineare Optimierung}
Generelle Voraussetzung: Alle Daten sind ganzzahlig.

Verfahren von Land/Doig/Dakin zur Lösung von
\[ (\text{P}) = (\text{P}_0) : \quad
  z = c^\top x \to \min \subjto x \in D \cap E \]
mit $D = \integer^n$, $E = E_0 = \{ x \in \real^n : Ax = b, x \ge 0 \}$.

Teilproblem:
\[ (\text{P}_i) : \quad
  c^\top x \to \min \subjto x \in D \cap E_i, \]
wobei $E_i$ durch eine oder mehrere zusätzliche Ungleichungen aus $E$
entsteht.
\[ (\text{Q}): \quad
  z(E_i) := \min \{ c^\top x : x \in E_i \}. \]
Sei $x^{\text{LP}}$ Lösung der LP-Relaxation zu $(\text{P}_i)$ mit
$x_j^{\text{LP}} \notin \integer$.

Verzweigung mittels Alternativkonzept:
\begin{align*}
  (\text{P}_{i,1}) \quad E_{i,1}
  &= \left\{ x \in E_i : x_j \le \lfloor x_j^{\text{LP}} \rfloor \right\}, \\
  (\text{P}_{i,2}) \quad E_{i,2}
  &= \left\{ x \in E_i : x_j \ge \lceil x_j^{\text{LP}} \rceil \right\}. \\
\end{align*}

\setcounter{exmp}{2}
\begin{exmp} %% 4.3
  Betrachte
  \[ z = 7 x_1 + 2 x_2 \to \max \subjto -x_1 + 2 x_2 \le 4, 5x_1 + x_2 \le 20,
    x_1, x_2 \in \integer_+. \]
  Tableau:
  \[
    \begin{array}{c|ccc}
      \ST_1 & x_1 & x_2 & 1 \\
      \hline
      x_3 = & 1 & -2 & 4 \\
      x_4 = & \boxed{-5} & -1 & 20 \\
      \hline
      - z = & -7 & -2 & 0 \\
      \hline
            & \ast & - \rez{5} & 4
    \end{array}
    \quad
    \begin{array}{c|ccc}
      \ST_2 & x_4 & x_2 & 1 \\
      \hline
      x_3 = & -\rez{5} & \boxed{-\frac{11}{5}} & 8 \\
      x_1 = & -\rez{5} & -\rez{5} & 4 \\
      \hline
      - z = & \frac{7}{5} & -\frac{3}{5} & -28 \\
      \hline
            & -\rez{11} & \ast & \frac{40}{11}
    \end{array}
    \quad
    \begin{array}{c|ccc}
      \ST_3 & x_4 & x_3 & 1 \\
      \hline
      x_2 = & -\rez{11} & -\frac{5}{11} & \frac{40}{11} \\
      x_1 = & -\frac{2}{11} & \frac{1}{11} & \frac{36}{11} \\
      \hline
      - z = & \frac{16}{11} & \frac{3}{11} & -\frac{332}{11}
    \end{array}
  \]
  Verzweigung nach $x_2$, da
  \[ \min \left\{ 4 - \frac{40}{11}, \frac{40}{11}-3 \right\} >
    \min \left\{ 4 - \frac{36}{11}, \frac{36}{11}-3 \right\} \]
  \begin{enumerate}[1. TP:]
  \item $x_2 \ge 4$ $\Rightarrow$
    \[ s_2 = x_2 - 4 = -\rez{11} x_4 - \frac{5}{11} x_3 - \frac{4}{11} < 0 \]
    für alle $x_3, x_4 \ge 0$. Der zulässige Bereich ist leer!
  \item $x_2 \le 3$ $\Rightarrow$
    \[ s_2 = 3 - x_2 = \rez{11} x_4 + \frac{5}{11} x_3 - \frac{7}{11}\]
  \end{enumerate}
  Neues Tableau:
  \[
    \begin{array}{c|ccc}
      \ST_3^* & x_4 & x_3 & 1 \\
      \hline
      x_2 = & -\rez{11} & -\frac{5}{11} & \frac{40}{11} \\
      x_1 = & -\frac{2}{11} & \frac{1}{11} & \frac{36}{11} \\
      s_2 = & \frac{1}{11} & \boxed{\frac{5}{11}} & - \frac{7}{11} \\
      \hline
      - z = & \frac{16}{11} & \frac{3}{11} & -\frac{332}{11} \\
      \hline
              & -\rez{5} & \ast & \frac{7}{5}
    \end{array}
    \quad
    \begin{array}{c|ccc}
      \ST_4 & x_4 & s_2 & 1 \\
      \hline
      x_2 = & 0 & -1 & 3 \\
      x_1 = & -\rez{5} & \rez{5} & \frac{17}{5} \\
      x_3 = & -\rez{5} & \frac{11}{5} & \frac{7}{5} \\
      \hline
      - z = & \frac{7}{5} & \frac{3}{5} & -\frac{149}{5}
    \end{array}
  \]
  Dieses Tableau ist optimal für das Teilproblem, aber es sind immer noch
  nicht alle Variablen ganzzahlig.
  
  Verzweigung nach $x_1$:
  \begin{enumerate}[1. TP]
    \setcounter{enumi}{2}
  \item $x_1 \ge 4$ $\Rightarrow$ $s_1 = x_1 - 4$.
    \[
      \begin{array}{c|ccc}
        \ST_4^* & x_4 & s_2 & 1 \\
        \hline
        x_2 = & 0 & -1 & 3 \\
        x_1 = & -\rez{5} & \rez{5} & \frac{17}{5} \\
        x_3 = & -\rez{5} & \frac{11}{5} & \frac{7}{5} \\
        s_1 = & -\rez{5} & \boxed{\rez{5}} & -\frac{3}{5} \\
        \hline
        - z = & \frac{7}{5} & \frac{3}{5} & -\frac{149}{5} \\
        \hline
                & 1 & \ast & 3 
      \end{array}
      \quad
      \begin{array}{c|ccc}
        \ST_5 & x_4 & s_1 & 1 \\
        \hline
        x_2 = & & & 0 \\
        x_1 = & & & 4 \\
        x_3 = & & & 8 \\
        s_2 = & & & 3 \\
        \hline
        - z = & 2 & 3 & -28
      \end{array}
    \]
  \item $x_1 \le 3$ $\Rightarrow$ $s_1 = 3 - x_1$.
    \[
      \begin{array}{c|ccc}
        \ST_4^* & x_4 & s_2 & 1 \\
        \hline
        x_2 = & 0 & -1 & 3 \\
        x_1 = & -\rez{5} & \rez{5} & \frac{17}{5} \\
        x_3 = & -\rez{5} & \frac{11}{5} & \frac{7}{5} \\
        s_1 = & \boxed{\rez{5}} & -\rez{5} & -\frac{2}{5} \\
        \hline
        - z = & \frac{7}{5} & \frac{3}{5} & -\frac{149}{5} \\
        \hline
                & \ast & 1 & 2 
      \end{array}
      \quad
      \begin{array}{c|ccc}
        \ST_5 & s_1 & s_2 & 1 \\
        \hline
        x_2 = & & & 3 \\
        x_1 = & & & 3 \\
        x_3 = & & & 1 \\
        x_4 = & & & 2 \\
        \hline
        - z = & 7 & 2 & -27
      \end{array}
    \]
  \end{enumerate}
  Also ist die Lösung im 3. TP: $z_{\max} = 28$.
\end{exmp}

\subsubsection{Das Rundreise-Problem}
Traveling salesman problem, TSP

Gesucht ist eine Rundreise minimaler Länge zwischen $n$ Orten, die jeden Ort
\emph{genau einmal} besucht. Anwendung zum Beispiel in der
Leiterplattenbestückung.

\begin{itemize}
\item Entfernungsmatrix $C = (c_{ik}) \in \realmat{n}{n}$.
\item Entscheidungsvariable
  \[ x_{ik} =
    \begin{cases}
      1, &\text{falls ovn $i$ nach $k$ gereist wird,} \\
      0, &\text{sonst.}
    \end{cases}
  \]
\end{itemize}
O.B.d.A: $c_{ii} := + \infty$ $\Rightarrow$ $x_{ii} = 0$.
\[ z = \sum_{i=1}^n \sum_{i=1}^n c_{ik} x_{ik} \to \min \subjto
  \sum_{k=1}^n x_{ik} = 1, \quad
  \sum_{i=1}^n x_{ik} = 1, \quad
  x_{ik} \in \{0,1\} \tag{AP} \]
für alle $i,k$. Absicherung, dass jeder Ort einmal getroffen wird und
Verhinderung von ``Subtouren'' durch die weiteren Forderungen
\[ \sum_{i,k \in S} x_{ik} = |S| - 1 \tag{SEB} \]
(Subtoureliminationsbedingungen)
für alle $S \subset \{1, \ldots, n\}$ mit $1 \le |S| \le n-1$. Beachte: Es
ergeben sich also $2^n$ Nebenbedingungen.

Das ist ein spezielles Transportproblem, da alle $a_i$ und $b_k$ den Wert 1
haben. Man bezeichnet es als \emph{Zuordnungsproblem} (assignment problem).

Es können von den nötigen $2n-1$ Basisvariablen nur $n$ Stück $\ne 0$
sein. Das führt zu Zyklen im Simplexverfahren, das heißt es werden immer
wieder die selben Variablen ausgetauscht, ohne dass sich der ZFW ändert.

Für diese Art Problem existieren Algorithmen, die eine Komplexität $O(n^3)$
aufweisen.

Es sind unterschiedliche Relaxationen anwendbar:
\begin{enumerate}[(1)]
\item Zeilen- und Spaltenrelaxation (Algorithmus von
  Little/Sweeney/Karel/Murthy, 1963)
  \[ u_i := \min \left\{ c_{ik} : k \in I=\{1, \ldots, n \} \right\}
    \qRq \tilde{d}_{ik} := c_{ik} - u_i \]
  \[ v_k := \min \left\{ \tilde{d}_{ik} : i \in I \right\} \]
  Es ergibt sich
  \begin{align*}
    z &= \sum_{i \in I} \sum_{k \in I} c_{ik} x_{ik} \\
      &= \sum_{i \in I} \sum_{k \in I}( c_{ik} - u_i ) x_{ik}
        + \underbrace{\sum_{i \in I} \sum_{k \in I} u_i x_{ik}}_{=\sum_{i \in I} u_i} \\
      &= \sum_{i \in I} \sum_{k \in I} (c_{ik} - u_i - v_k) x_{ik}
        + \underbrace{\sum_{i \in I} u_i + \sum_{k \in I} v_k}_{\text{Untere Schranke für $z$}},
  \end{align*}
  eine zulässige Lösung der dualen Aufgabe zu AP.
\item AP-Relaxation, das heißt otimale Lösung von AP
\item LP-Relaxation von AP und SEB.
\end{enumerate}
Für $n \le 40$ ist (1) am wenigsten aufwändig, bis $n = 120$ ist (2) am
besten, für $n > 120$ ist (3) zu empfehlen.

Realisierung von (3): Start mit AP-Relaxation und Hinzufügen der verletzten
SEB (Wie finden?).

\paragraph{Verzweigungsstrategie bei Zeilen- und Spaltenreduktion}
Für jedes Indexpaar $(p,q)$ mit $d_{pq} = c_{pq} - u_p - v_q = 0$ wird ein
Gewicht $w_{pq} \in \real_+$ ermittelt, welches durch Zeilen- und
Spaltenreduktion aus $\tilde{D}$ erhalten wird, wobei $\tilde{D} =
(\tilde{d}_{ik})$ aus $D$ entsteht durch Setzen von
\[ \tilde{d}_{ik} :=
  \begin{cases}
    \infty, &(i,k) = (p,q), \\
    d_{ik}, &\text{sonst.}
  \end{cases}
\]

Aus $d_{ik} := c_{ik} - u_i - v_k$ für alle $i, k$ folgt
\[ z(X) = \sum_{i,k} d_{ik} x_{ik} + \sum_{i}(u_i + v_i) \ge \sum_i (u_i +
  v_i), \]
da $d_{ik} \ge 0$ und $x_{ik} \ge 0$.

Verzweigung:
\begin{itemize}
\item $x_{pq} := 1$ $\Rightarrow$ $x_{pk} := 0$ für alle $k \ne q$,
\item $x_{pq} := 1$ $\Rightarrow$ $x_{iq} := 0$ für alle $i \ne p$, $x_{qp} :=
  0$ bzw. Verbot des zugehörigen Zyklus $\Rightarrow$ $d_{pq} := \infty$.
\end{itemize}

\paragraph{Welches $(p,q)$ sollte man wählen?}
Für jedes Indexpaar $(p,q)$ mit $d_{pq} = 0$ definiere das Gewicht
\[ w_{pq} := \min_{k \ne q} d_{pk} + \min_{i \ne p} d_{iq}. \]
Das entpricht dem Mindestzuwachs für $x_{pq} := 0$.

Sein nun $(p,q)$ ein Indexpaar mit \emph{maximalem} Gewicht, dann verzweige
bezüglich $(p,q)$.

\begin{exmp}
  Little-Algorithmus mit Zeilen- und Spaltenreduktion.
  \[
    \begin{array}{c|ccccc|c}
      (c_{ik}) & & & & & & \\
      \hline
               & \infty & 32 & \underline{22} & 30 & 24 & 22 \\
               & 10 & \infty & \underline{3} & 18 & \infty & 3 \\
               & \infty & \underline{9} & \infty & 14 & 12 & 9 \\
               & 16 & 10 & 7 & \infty & \underline{6} & 6 \\
               & 15 & 19 & 15 & \underline{12} & \infty & 12 \\
      \hline
               & 3 & & & & & \sum 52 + 3
    \end{array}
  \]
  %% Unfertig
\end{exmp}

Aufwand: Little-Algorithmus $O(n^2)$ wegen Zeilen-/Spaltenreduktion. Je
Teilproblem: AP-Relaxation $O(n^2)$, LP-Relaxation: Lösung eines LP uns Suche
von SEB ``$> O(n^2)$''

\begin{rmrk*}
  Die Zeilen-/Spaltenreduktion liefert die Schranke 58.
\end{rmrk*}

\section{Dynamische Optimierung}
(Prinzip wird nur an zwei Beispielen erklärt)

\subsection{Das klassische Rucksackproblem}
Voraussetzung; $c_i > 0$, $0 < a_i \le b$ für alle $i \in I = \{1, \ldots, n\}$,
alle Daten ganzzahlig.
\[ f(x) = c^\top x \to \max \subjto a^\top x \le b, \quad x \in \integer_+^n.
  \tag{P} \]

\begin{defn*}
  Für alle $k \in \{1, \ldots, n\}$ und $y \in \{0,1, \ldots, b\}$ definiere
  \[ F(k,y) := \max \left\{
      \sum_{i=1}^k c_i x_i : \sum_{i=1}^k a_i x_i \le y, x_i \in \integer_+, i
      =1, \ldots, k
    \right\}. \]
\end{defn*}
Gesucht ist $F(n,b)$.

Für $k=1$ gilt
\[ F(1,y) := c_1 \left\lfloor \frac{y}{a_1} \right\rfloor\]
für alle $y \in \{ 0, \ldots, b \}$.

Aufgrund der Linearität der Zielfunktion und der Nebenbedingungen gilt für $k >
1$, $y \ge a_k$:
\[ F(k,y) = \max \left\{ c_k x_k + F(k-1, y - a_k x_k) : x_k = 0, 1, \ldots,
    \left\lfloor \frac{y}{a_k} \right\rfloor 
  \right\} \]
bzw.
\[ F(k,y) = \max \{
  \underbrace{F(k-1,y)}_{\text{d.h. } x_k = 0},
  \underbrace{c_k + F(k, y- a_k)}_{\text{d.h.} x_k \ge 1}
  \} \]

\paragraph{Interpretation} Der Optimalwert $F(k,y)$ zum \emph{Zustand} $(k,y)$
wird mit Hilfe von Optimalwerten für ``kleinere'' Zustände berchnet. Also sollte
man die Reihenfolge der Berechnung dieser Optimalwerte so organisieren, dass
alle benötigten Zwischenwerte vorher ermittelt werden. Das führt zu den
Rekursionsformeln der dynamischen Optimierung (DO).

\begin{rmrk*}
  \begin{enumerate}
  \item Mit $F(1,y) := c_1 \cdot \lfloor y / a_1  \rfloor$ und (GG) für alle
    $y \ge 0$ und $k > 1$ ergibt sich der Algorithmus von Gilmore/Gomory.
  \item Im Unterschied zu rekursiven Programmierung wird in der dynamischen
    Optimierung jeder Optimalwert nur \emph{einmal} berechnet.

    Fibonacci-Zahl: $F(n) = F(n-1) + F(n-2)$, $F(0) = 0$, $F(1) = 1$.
  \end{enumerate}
\end{rmrk*}

\subsubsection*{Algorithmus von Gilmore/Gomory}
\begin{enumerate}
\item[$S_1$] Setze $F(1,y) = c_1 \cdot \lfloor y / a_1 \rfloor$,
  $y = 0, 1, \ldots, b$.
\item[$S_k$] Für $k = 2, \ldots, n$
  \begin{align*}
    &\text{für } y = 0, \ldots, a_{k-1}:
    & F(k,y) &:= F(k-1,y), \\
    &\text{für } y = a_k, a_{k+1}, \ldots, a_b:
    & F(k,y) &:= \max \{ F(k-1,y), c_k + F(k,y-a_k) \}.
  \end{align*}

  Merken von Indexinformation und Identifizierung einer Lösung
  \[ p(k,y) = \begin{cases}
      p(k-1,y), & \text{falls } F(k,y) = F(k+1,y), \\
      k, &\text{sonst.}
    \end{cases}
  \]
  \[ p(1,y) = \begin{cases}
      0, & \text{falls } y < a_1, \\
      1, & \text{falls } y \ge a_1.
    \end{cases}
  \]
\end{enumerate}
$F: \real^2 \to \real$ ist stückweise konstant und monoton nicht fallend.

\begin{exmp} %% 4.5
  \begin{align*}
    &z = 5 x_1 + 10 x_2 + 12 x_3 + 6 x_4 \to \max \\
    \text{bei } 4 x_1 + 7 x_2 + 9 x_3 + 5 x_4 \le 5, \quad x_i \in \integer_+
  \end{align*}

  %% Mehr
\end{exmp}

\subsection{Guillotine-Zuschnitt}
Aus einem Rechteck $L \times W$ sind kleinere Rechtecke $l_i \times w_i$, $i =
1, \ldots, n$, durch Guillotine-Schnitte so zuzuschneiden, dass der Abfall
minimal ist. Gewünschte Rechtecke können mehrfach erhalten werden.

$F(L,W)$ bezeichne den Maximaleintrag bei Guillotine-Zuschnitt aus $L \times W$.
Mit $F(l, 0) = 0$ für $l = 1, \ldots, L$ und $F(0,w) = 0$ für $w = 1, \ldots,
W$, gilt für $l = 1, \ldots, L$, $w = 1, \ldots, W$:
\[ F(l,w) = \max \{ e(l,w), h(l,w), v(l,w) \}, \]
wobei
\begin{align*}
  e(l,w)
  &= \max \left\{ 0 , \max_i \{ l_i w_i : l_i \le l, w_i \le w \} \right\}, \\
  h(l,w)
  &= \max \{ F(r,W) + F(L-r,W) : r = 1, \ldots, \lfloor L/2 \rfloor \}, \\
  v(l,w)
  &= \max \{ F(L,s) + F(L,W-s) : s = 1, \ldots, \lfloor W/2 \rfloor \}.
\end{align*}

Der Aufwand ist $O(LW(L+W))$.

Die Optimalwerte für größere Rechtecke werden aus den Optimalwerten für kleinere
Rechtecke erhalten.

Im Allgemeinen gilt
\[ \max \{ f(x) + g(x) \} \ne \max f(x) + \max g(x). \]

\section{Schnittebenenverfahren für GLO}
\begin{align*}
  c^\top x \to \min \subjto Ax &= a, \quad x \in \integer_+^n \tag{P} \\
  c^\top x \to \min \subjto Ax &= a, \quad x \in \real_+^n \tag{q}
\end{align*}
Für die konvexe Hülle der gültigen Punkte von (P) gilt
\[ \operatorname{conv} \{ x \in \integer_+^n : Ax = a \} \subset G, \]
sie ist immer eine Teilmenge des gültigen Bereichs von (Q).

\subsection{Gomory-Schnitte}
Sei (Q) mittels Simplexverfahren gelöst und das optimale Tableau sei
\[ x_B = P x_N + p, \qquad z = q^\top x_N + q_n \]
mit $P = -A_B^{-1} A_N$.

Alternative Schreibweise:
\[ I_B x_B - P x_N = p. \]

Die Lösung von (Q) ist dann
\[ x = \pmat{x_B \\ x_N} = \pmat{p \\ 0}, \]
beachte $J = J_B \cup J_N = \{ 1, \ldots, n \}$.

Wenn $p$ ganzzahlig ist, dann ist $x$ auch eine Lösung von $p$. Sei also $p$
\emph{nicht} ganzzahlig.

\textbf{Idee:} Eine Nebenbedingung (oder mehrere) in (P) hinzufügen, die von
$x = \pmat{x_B \\ x_N} = \pmat{p \\ 0}$ \emph{nicht} erfüllt wird, aber von
allen zulässigen ganzzahligen Punkten von (P).

Es sei die $i$-te Komponente von $p$ nicht ganzzahlig, also $p_i \notin
\integer$. Wegen
\begin{align*}
  [x_B]_i + \sum_{j \in J_N} (-P_{ij}) x_j &= p_i
  & \text{für alle }  x \in \real^n \text{ mit } Ax = a, \\
  [x_B]_i + \sum_{j \in J_N} \lfloor -P_{ij} \rfloor x_j &= \lfloor p_i \rfloor
  & \text{für alle }  x \in \integer_+^n \text{ mit } Ax = a 
\end{align*}
gilt
\[ \sum_{j \in J_N}
  \underbrace{((-P_{ij}) - \lfloor P_{ij} \rfloor)}_{=(\lceil P_{ij} \rceil -
    P_{ij}) x_j = r_{j}^{(i)}}
  \ge p_i - \lfloor p_i \rfloor = r_i \]
für alle $x \in \integer_+^n$ mit $Ax = a$.

\begin{defn}
  Die Ungleichung
  \[ \sum_{j \in J_n} r_j^{(i)} x_j \ge r_i \]
  heißt \emph{Gomory-Schnitt} zur Zeile $i$.
\end{defn}

\begin{rmrk*}
  Für $x = \pmat{p \\ 0}$ mit $p_i \notin \integer_+$ gilt
  \[ x^\top r^{(i)} = \sum_{j \in J_N} r_j^{(i)} \cdot 0 = 0 < r_i (\ne 0) \]
  mit $r_j^{(i)} = 0$ für $j \in J_B$, das heißt die nicht-ganzzahlige Lösung
  $x$ wird ``abgeschnitten''.
\end{rmrk*}

\begin{flg}
  Es gilt
  \begin{align*}
    \{ x \in \integer_+^n : Ax = a \}
    &= \{ x \in \integer_+^n : x_B = P x_N + p \} \\
    &= \{ x \in \integer_+^n : x_B = P x_N + p, (r^{(i)})^\top x \ge r_i \}.
  \end{align*}
\end{flg}

\subsection{Gomory-Verfahren}
Durch Gomory-Schnitte erhält man eine strengere Relaxation zu (P)
\[ \begin{aligned}
    z = q^\top x_N + q_0 \to \min \subjto
    &x_B = P x_N + p, \\
    &s = (r^{(i)})^\top x_N - r_i, \\
    &x_B, x_N \ge 0. 
  \end{aligned} \tag{$\text{Q}_1$} \]

Die Lösung von ($\text{Q}_1$) erhält man mit dem dualen Simplexverfahren.
\begin{enumerate}[S1]
  \setcounter{enumi}{-1}
\item (Initialisierung) $k := 0$, $(\text{Q}_k) = (\text{Q})$
\item Löse ($\text{Q}_k$).\\
  Falls $(\text{Q}_k)$ nicht zulässig ist, dann ist (P) nicht zulässig.
  $\rightarrow$ \verb+STOP+ \\
  Falls $z$ unbeschränkt ist für $(\text{Q}_k)$, dann auch für (P). $\rightarrow$
  \verb+STOP+ \\
  Falls die Lösung von $(\text{Q}_k)$ ganzzahlig ist, dann ist sie auch
  Lösung von (P). $\rightarrow$ \verb+STOP+
\item Füge einen Gomory-Schnitt zu ($\text{Q}_k$) hinzu. Dies ergibt
  ($\text{Q}_{k+1}$). \\
  Setze $k := k + 1$ und gehe zu S1.
\end{enumerate}

\begin{exmp}
  %% Noch zu füllen
\end{exmp}

\begin{rmrk*}
  Die Anzahl der Gomory-Schnitte, die erforderlich sind, um eine ganzzahlige
  Lösung zu erhalten, ist exponentiell. Die Endlichkeit des Verfahrens ist im
  Allgemeinen unklar (Zyklen).

  Ist die Menge der zulässigen Punkte beschränkt, so liefert der Algorithmus
  nach endlich vielen Schritten eine ganzzahlige Lösung oder es existiert keine
  zulässige Lösung.
\end{rmrk*}

\textbf{Alternative 1.} Das sogenannte \emph{lexikographische} Gomory-Verfahren
ist endlich. Wähle immer die Variable mit kleinstem Index, die keinen
ganzzahligen Wert hat.

\textbf{Alternative 2.} Kombination von b\&b und
Schnittebenenverfahren $\rightarrow$ branch and cut

\textbf{Alternative 3.} Anwendung allgemeinerer, besserer Schnitte
(Chvátal-Gomory cuts, facettendefinierte Schnitte)

%% Viel fehlt