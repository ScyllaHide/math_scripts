\documentclass{scrartcl}
\usepackage[utf8]{inputenc}
\usepackage[ngerman]{babel}

\usepackage{pgfplots}
\usepackage{tikz}
\usetikzlibrary{arrows, shapes}

\usepackage[active,tightpage]{preview}
\PreviewEnvironment{tikzpicture}
\setlength{\PreviewBorder}{0mm}
\pagestyle{empty}

\begin{document}
\tikzstyle{blk} = [
  draw, fill=blue!20, text width=10em, 
  text centered, minimum height=2.5em,
  rounded corners ]
\tikzstyle{root_blk} = [
  draw, fill=blue!20, text width=10em, 
  text centered, minimum height=2.5em,
  minimum width = 12.5em,
  rounded corners ]
\tikzstyle{txtabove} = [
  above, text centered, font=\scriptsize,
  text width=6em ]
\tikzstyle{txtright} = [
  right, text centered, font=\scriptsize,
  text width=6em ]
\tikzstyle{txtleft} = [
  left, text centered, font=\scriptsize,
  text width=6em ]
  
\def\blockdist{5}
\def\rowdist{2.0}
\def\edgedist{2.5}

\begin{tikzpicture}
  \node (root) [root_blk] {Anwendungsproblem};
  \path (root)+(1.5*\blockdist,0) node (1) [blk] {%
    Modellierungsziel;\\Teilproblem,\\Vereinfachte Darst.%
  };
  \path (1)+(0,-\rowdist) node (2) [blk] {Mathematisches Modell};
  \path (2)+(0,-\rowdist) node (3) [blk] {%
    Eigenschaften,\\Lösungsstrategie%
  };
  \path (3)+(-1.5*\blockdist,0) node (4) [blk] {%
    Simulationsergebnisse,\\Auswertung%
  };
  \path (4)+(0,\rowdist) node (5) [blk] {Bewertung\\der Resultate};

  \draw [->] (root.east) -- node [txtabove] {Analyse des Anwendungsproblems} (1.west);
  \draw [->] (1.south) -- node [txtright] {Modellbildung} (2.north);
  \draw [->] (2.south) -- node [txtright] {Mathematische Analyse} (3.north);
  \draw [->] (3.west) -- node [txtabove] {Berechnung und Simulation} (4.east);

  \draw [->] (4.north) -- node [txtleft] {Interpretation, Validierung} (5.south);
  \draw [->] (5.north) -- node [txtleft] {Ziel noch nicht erreicht?} (root.south);
  
\end{tikzpicture}

\end{document}