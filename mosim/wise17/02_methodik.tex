\chapter{Methodik der mathematischen Modellierung}
\section{Modellierungszyklus}
Mathematische Modellierung ist ein iterativer Prozess.

\begin{center}
  \includegraphics{img/modellierung}
\end{center}

Beispiel Basketball-Freiwurf: Zweifaches Durchlaufen des Zyklus.

Analyse des Anwendungsproblems: Beschränkung auf Wurftrajektorie,
vereinfachende Annahmen, Flugbahn nach Abwurf festgelegt durch Bewegungsgleichungen.

Modellbildung: Bewegungsgleichungen, geometrische Bedingungen an Flugbahn.

Mathematisches Modell: Flugbahn
\[ \pmat{x(t)\\y(t)}, \qquad \pmat{v_x(t)\\v_y(t)}. \]
Optimierungsproblem (erstes Modell):
\[ \vartheta_{\mathrm{opt}}^0 = \argmax_{\vartheta^0} e( \vartheta^0). \]

Mathematische Analyse: Zielfunktional $e(\vartheta^0)$ ist nicht stetig
differenzierbar $\rightsquigarrow$ Optimierung ohne Verwendung von Gradienten.

Interpretation und Validierung: Vergleich mit experimentellen Daten.

\section{Mathematisches Modell}
\begin{defn*}
  Ein \emph{mathematisches Modell} stellt formale Zusammenhänge zwischen
  (physikalischen bzw. Modell-) Größen her. Eine solche Größe ist eine
  quantitative Eigenschaft eines physikalischen Objektes, Zustands oder
  Prozesses. Man unterscheidet zwischen Parametern und Variablen.

  Aus der Analyse bzw. Simulation eines mathematischen Modells sollen
  Erkenntnisse über das reale (physikalische) System gewonnen werden.
\end{defn*}

\subsubsection*{Parameter}
\begin{itemize}
\item Systemparameter (z.B. Korbhöhe beim Basketball-Freiwurf)
\item Modellparameter (z.B. Schrittweite beim diskreten Modell)
\end{itemize}

\subsubsection*{(Zustands-) Variablen} (z.B. Trajektorie $\pmat{x(t)\\y(t)}$,
$\pmat{v_x(t)\\v_y(t)}$) bei Berücksichtung der Drehung eines (starren) Körpers
zusätzlich Rotation, beschrieben typischerweise durch Quaternionen)

\subsubsection*{Mathematische Analyse eines Modells}
\begin{itemize}
\item Dimensionsanalyse und Skalierung
\item Existenz und Eindeutigkeit einer Lösung
\item Untersuchung von Sonderfällen
\item Falls explizite Lösung angegeben werden kann: Validierung der numerischen
  Lösung
\item Vereinfachung, Linearisierung; Störungstheorie
\item Untersuchung von qualitativen Eigenschaften, etwa asymptotisches Verhalten
\item Sensitivitätsanalyse: Wie wirken sich Störungen von Parametern auf die
  Lösung aus?
\item Auswahl von numerischen Verfahren
\end{itemize}

\section{Dimensionsanalyse und Skalierung}
Größe $X$, Dimension $[X]$, z.B:
\[ [l] = \text{Länge}, \qquad [t] = \text{Zeit}, \qquad [v] =
  \text{Geschwindigkeit} = \frac{\text{Länge}}{\text{Zeit}}. \]

Iternationales Einheitensystem SI

\subsubsection*{Basiseinheiten}
\begin{center}
  \begin{tabular}{l|c|c|c|c}
    Dimensionsname & Größensymbol & Dimensionssymbol & Einheit & Zeichen \\
    \hline
    Länge & $l$ & $L$ & Meter & \si{\m} \\
    Masse & $m$ & $M$ & Kilogramm & \si{\kg} \\
    Zeit  & $t$ & $T$ & Sekunde & \si{\s} \\
    Stromstärke & $I$ & $I$ & Ampere & \si{\ampere} \\
    Temperatur & $T$ & $\Theta$ & Kelven & \si{\kelvin} \\
    Stoffmenge & $n$ & $N$ & Mol & \si{\mol} \\
    Lichtstärke & $Ir$ & $J$ & Candela & \si{\candela}
  \end{tabular}
\end{center}

Abgeleitete Größen und Einheiten, zum Beispiel:
\begin{center}
  \begin{tabular}{l|c|c|c|c}
    Geschwindigkeit & $v$ & $\frac{L}{T}$
    & $\frac{\text{Meter}}{\text{Sekunde}}$ & \si{\m \per \s} \\
    Kraft & $F$ & $M \cdot \frac{L}{T^2}$
    & Newton & $\si{\newton} = \si{\kg \m \per \s \squared}$ \\
    Energie/Arbeit & $E$ & $M \cdot \frac{L^2}{T^2}$
    & Joule & $\si{\joule} = \si{\newton \m} = \si{\kg \m \squared \per \second \squared}$ \\
    Leistung & $P$ & $M \cdot \frac{L^2}{T^3}$
    & Watt & $\si{\joule \per \s} = \si{\kg \m \squared \per \second \cubed}$ \\
    Elektrische Leistung & $Q$ & $J \cdot T$
    & Couloumb & $\si{\coulomb} = \si{\ampere \s}$
  \end{tabular}
\end{center}

Grundlegende Eigenschaften und Regeln:
\begin{itemize}
\item Multiplikation, Potenzieren, Division von dimensionsbehafteten Größen
  überträgt sich auf Dimensionen.

  Beispiel: \\
  Volumen eines Kubus
  \[ V = (\text{Kantenlänge } a)^3, \qquad [V] = L^3. \]
  Geschwindigkeit
  \[ \left[ \diff{}{t} x(t) \right] = \left[ \lim_{\Delta t \to 0} \frac{x(t +
        \Delta t) - x(t)}{\Delta t} \right] = \frac{L}{T}. \]
  Beschleunigung
  \[ \left[ \frac{\diffop^2}{\diffop t^2} x(t) \right] = \frac{L}{T^2}. \]
\item Nur Größen gleicher Dimension dürfen addiert, subtrahiert und verglichen
  werden.

  Beispiel: Zweites Newtonsches Gesetz, ``Kraft $=$ zeitliche Änderung des
  Impulses''
  \[ F = \diff{}{t} p(t), \qquad p = m \cdot v, \]
  \[ [ F ] = \frac{[p]}{[t]} = \frac{M \cdot L/T}{T} = M \cdot \frac{L}{T^2}. \]
\item Argumente von Funktionen\footnote{%
    Ausnahme: Monome. Das sind Polynome, die nur aus einem Glied bestehen. Ein
    Monom ist ein Produkt, bestehend aus einem Koeffizienten und einer Potenz
    einer Variablen.}%
  müssen dimensionslos sein. Zum Beispiel ist $\exp(x)$ mit $[x] = L$
  \emph{nicht} zulässig, da
  \[ \exp(x) = 1 + x + \rez{2} x^2 + \rez{6} x^3 + \ldots, \]
  das ergibt keinen Sinn. Zulässig ist aber
  \[ \exp( k \cdot x ) \qquad \text{mit} \qquad [k] = \rez{L}, [x] = L. \]
  Dann ist $k \cdot x$ wieder dimensionslos.
  \[ F(t) = F_0 \cdot \sin (\omega t), \quad [t] = T, \quad [\omega] =
    \rez{T}. \]
\end{itemize}

\section{Entdimensionalisierung}
Beobachtung:  Mathematische Aussagen über ein mathematisches (physikalisches)
Modell können nicht von der Wahl der Einheiten der jeweiligen Größen abhängen.
$\rightsquigarrow$ Transformation auf eine dimensionslose Form.

Beispiel: senkrechter Wurf, Gravitationskraft auf einen geworfenen Ball.
Gravitationsgesetz: 
\[ F = \gamma \frac{ m_1 m_2 }{ r^2 }. \]
$\gamma = \SI{6.674e-3}{\m \cubed \per \kg \per \s \squared}$ ...
Gravitationskonstante. $[\gamma] = \frac{L^3}{MT^2}$.

Berücksichtige den Erdradius $R$:
\[ F = - \gamma \frac{ m \cdot m_E }{ (x(t) + R)^2 } \]
Anwendung des zweiten Newtonschen Gesetzes:
\begin{align*}
  \diff{}{t} \underbrace{m \dot{x}(t)}_{\text{Impuls}} &= F \\
  m \ddot{x}(t) &= - \gamma \frac{ m \cdot m_E }{ (x(t) + R)^2 }
\end{align*}
Speziell für die Erde:
\[ R = \SI{6371}{\km}, \qquad m_E = \SI{5.9722e24}{\kg}, \qquad g = \frac{\gamma
    \cdot m_E}{R^2} = \SI{9.81}{\m \per \second \squared} \]
Einsetzen:
\[ \ddot{x}(t) = -g \frac{R^2}{(x(t) + R)^2} \]

Anfangsbedingungen $x(0) = 0$, $\dot{x}(0) = \bar{v}$.

\subsubsection*{Entdimensionalisierung}
Referenzlänge $\ell_0$, $[\ell_0] = L$ \\
Referenzzeit $\tau_0$, $[\tau_0] = T$ \\
Neue dimensionslose Variablen:
\[ \hat{x} = \frac{x}{\ell_0}, \qquad \hat{t} = \frac{t}{\tau_0} \qRq
  \hat{x}(\hat{t}) = \rez{\ell_0} x ( \tau_0 \hat{t} ). \]
Also folgt
\[ \diff{}{\hat{t}} \hat{x}(\hat{t})
  = \frac{\tau_0}{\ell_0} \dot{x}(\tau_0 \hat{t}), \qquad
  \frac{\diffop^2}{\diffop \hat{t}^2} \hat{x}(\hat{t}) = \frac{\tau_0^2}{\ell_0}
  \ddot{x}( \tau_0 \hat{t}). \]
Einsetzen:
\[ \frac{\diffop^2}{\diffop \hat{t}^2} \hat{x}(\hat{t}) =
  - \frac{\tau_0^2}{\ell_0} g \rez{\left( \frac{\ell_0}{R} \hat{x}(\hat{t}) + 1
    \right)^2}. \]
Anfangsbedingungen $\hat{x}(0) = 0$, $\diff{}{\hat{t}}\hat{x}(0) =
\frac{\tau_0}{\ell_0} \bar{v}$.

Wir bezeichnen:
\[ \Pi_1 := \frac{\tau_0^2}{\ell_0} g, \qquad
  \Pi_2 := \frac{\ell_0}{R}, \qquad
  \Pi_3 := \frac{\tau_0}{\ell_0} \bar{v}. \]
$\Pi_1$, $\Pi_2$ und $\Pi_3$ sind drei Gruppen von Parametern. Wir können
$\ell_0$ und $\tau_0$ so wählen, dass zwei der drei $\Pi$-Terme gleich 1 sind.

Beobachtung: Für realistische Wurfhöhe $h \ll R$ gilt $\Pi_2 \ll 1$
\[ \left. \begin{aligned}
      \Pi_1 &= 1: & \ell_0 &= g \cdot \tau_0^2 \\
      \Pi_2 &= 1: & \ell_0 &= \bar{v} \tau_0
    \end{aligned}
  \right\} \quad \tau_0 = \frac{\bar{v}}{g}, \quad \ell_0 =
  \frac{\bar{v}^2}{g}. \]
Also gilt
\[ \Pi_3 = \frac{\bar{v}}{gR} =: \eps. \]
Einsetzen:
\[ \frac{\diffop^2}{\diffop \hat{t}^2} \hat{x}(\hat{t}) = - \rez{(\eps
    \hat{x}(\hat{t}) + 1)^2} \]
mit $\hat{x}(0) = 0$, $\diff{}{\hat{t}} \hat{x}(0) = 1$.

Die näherungsweise Lösung für $\eps = 0$ ist
\[ \frac{\diffop^2}{\diffop \hat{t}^2} \hat{x}(\hat{t}) \approx -1, \]
das heißt wir erhalten die ``übliche'' Bewegungsgleichung
\[ \ddot{x}(t) = - g. \]

\subsubsection*{Systematisches Vorgehen}
Zur Bestimmung der Referenzgrößen in komplizierteren Systemen ist es
nicht immer möglich, wie oben intuitiv die richtigen Größen zu bestimmen. Formal
geht man daher wie folgt vor:
\[ \ddot{x}(t) = - g \frac{R^2}{(x(t) + R)^2}, \]
Anfangsbedingungen $x(0) = 0$, $\dot{x}(0) = \bar{v}$.

\begin{enumerate}
\item Unabhängige Dimensionen: Länge $L$, Zeit $T$.
\item Variablen und Parameter: \\
  Variablen: $x$, $[x] = L$; $t$, $[t] = T$.
  Parameter: $\bar{v}$ ($LT^{-1}$), $g$ ($LT^{-2}$), $R$ ($L$).
\item Dimensionslose Größen:
  Ansatz
  \[ \left[ \bar{v}^{\alpha_1} \cdot g^{\alpha_2} \cdot R^{\alpha_3} \right] =
    1. \]
  Also
  \[ \left( \frac{L}{T} \right)^{\alpha_1} \cdot
    \left( \frac{L}{T^2} \right)^{\alpha_2} \cdot
    L^{\alpha_3} = 1 \qRq L^{\alpha_1 + \alpha_2 + \alpha_3} \cdot T^{- \alpha_1
      - 2 \alpha_2} = 1. \]
  Ausgedrückt als lineares Gleichungssystem
  \begin{center}
  \begin{tabular}{l|ccc}
    & $\bar{v}$ & $g$ & $R$ \\
    \hline
    L & 1 & 1 & 1 \\
    T & -1 & -2 & 0
  \end{tabular}
  $\Rightarrow$
  $S = \pmat{ 1 & 1 & 1 \\ -1 & -2 & 0}$.
  \end{center}
  \[ S \alpha = \pmat{ 1 & 1 & 1 \\ -1 & -2 & 0} \pmat{ \alpha_1 \\ \alpha_2 \\
      \alpha_3} = \pmat{0\\0} \]
  Allerdings gilt $\dim \ker S = 3 - 2 = 1$,
  \[ \ker S = \spn \left\{ \pmat{ 2 \\ -1 \\ -1} \right \}. \]

  $\bar{v}^2 g^{-1} R^{-1} =: \eps$ ist dimensionslos (i.A. können mehrere
  ``entdimensionalisierte'' Parameter auftreten).
\item Referenzgrößen:
  \[ \hat{x} = \frac{x}{\ell_0}, \qquad \hat{t} = \frac{t}{\tau_0}. \]
  Ansatz:
  \[ \ell_0 = \bar{v}^{\alpha_1} g^{\alpha_2} R^{\alpha_3}, \]
  Gleichheit der Dimensionen erfordert
  \[ S \alpha = \pmat{1 \\ 0} \qRq \alpha = c \cdot \pmat{2 \\ -1 \\ -1} +
    \pmat{0 \\ 0 \\ 1} \]
  (Lösung nicht eindeutig!). Also folgt
  \[ \ell_0 = R \cdot \eps^c, \]
  im Beispiel oben galt $c = 1$, also $\ell_0 = \frac{\bar{v}^2}{g}$.

  Analog für die Referenzzeit:
  \[ \tau_0 = \bar{v}^{\alpha_1} g^{\alpha_2} R^{\alpha_3}, \qquad S \alpha =
    \pmat{0 \\ 1}. \]
  Also folgt
  \[ \alpha = c' \pmat{2 \\ -1 \\ -1} + \pmat{1 \\ -1 \\ 0 } \]
  und damit $\tau_0 = \frac{\bar{v}}{g} \eps^{c'}$, im Beispiel oben $c' = 0$,
  $\tau_0 = \frac{\bar{v}}{g}$.
\item Aufstellen der entdimensionalisierten Modellgleichung (selbe Rechnung wie
  oben)
  \[ \ddot{\hat{x}}( \hat{t} ) = -\rez{(\eps \hat{x}(\hat{t}) + 1)^2}, \qquad
    \hat{x}(0) = 0, \qquad \dot{\hat{x}}(0) = 1. \]
\end{enumerate}

\begin{rmrk*}
  Für die Wahl $\ell_0 = R$ und $\tau_0 = \frac{\bar{v}}{g}$ erhält man
  \[ \ddot{\hat{x}}(\hat{t}) = - \eps \rez{(\hat{x}(\hat{t}) + 1)^2}, \qquad
    \hat{x}(0) = 0, \qquad \dot{\hat{x}}(0) = \eps. \]
  Das ist mathematisch korrekt, aber wir erhalten keine Näherungslösung für
  $\eps \approx 0$.
\end{rmrk*}

Mit dem obigen Verfahren haben wir die Anzahl der freien Parameter von drei
(Gravitationskonstante, Erdradius, Erdmasse) auf eins ($\eps$) reduziert. Im
Allgemeinen lässt sich immer die Anzahl der Parameter um die Anzahl der
vorhandenen Dimensionen (hier Länge, Zeit) reduzieren.

\begin{thm}[Buckinghamsches $\Pi$-Theorem]
  Gegeben sei ein mathematisches Modell mit $k$ Variablen $x_1, \ldots, x_k$ und
  $l$ Parametern $x_{k+1}, \ldots, x_{k+l}$, in dem $m$ unabhängige Dimensionen
  vorkommen. Dann können $k + l - m$ dimensionslose Größen $q_1, \ldots,
  q_{k+l-m}$ als Produkte und Quotienten von $x_1, \ldots, x_{k+l}$ definiert
  werden und jede skalare Modellgleichung $f(x_1, \ldots, x_{k+l}) = 0$ kann auf
  eine Relation $\hat{f}( q_1, \ldots, q_{k+l-m} ) = 0$ reduziert werden.
\end{thm}

Für unser Beispiel: $m = 2$
\begin{align*}
  f( x, t, \bar{v}, g, R) = 0 \qquad
  &\left\{ \begin{aligned}
      \ddot{x}(t) = -g \frac{R^2}{(x(t) + R)^2} \\
      x(0) = 0, \dot{x}(0) = \bar{v}
    \end{aligned}
  \right. \\ 
  \rightsquigarrow \qquad
  f( \hat{x}, \hat{t}, \eps) = 0 \qquad
  &\left\{ \begin{aligned}
      \ddot{\hat{x}}(\hat{t}) = - \rez{(\eps \hat{x}(\hat{t}) + 1)^2} \\
      \hat{x}(0) = 0, \dot{\hat{x}}(0) = 1
    \end{aligned}
  \right.
\end{align*}

Konsequenz: Aussagen über das Modell, ohne Modellgleichungen zu kennen!

Beispiel Pendel. Schwingungsdauer $t_{sd}$?

Vorwissen/Intuition: Für kleine Auslenkungen ($\vartheta \ll 1$) sollte $t_{sd}$
nicht von den Anfangsbedingungen abhängen (Auslenkung, Geschwindigkeit). Es
sollte eine Relation zwischen $t_{sd}$ und den Parametern $l$, $m$ und $g$ der
Form
\[ f( t_{sd}, g, m, l ) = 0 \]
existieren. Also haben wir $m=3$ unabhängige Dimensionen $T, L, M$.

Buckinghamsches $\Pi$-Theorem $\rightsquigarrow$ Reduktion auf $\hat{f}(q) = 0$
mit der dimensionlosen Größe
\[ q = t_{sd}^{\alpha_1} \cdot l^{\alpha_2} \cdot m^{\alpha_3} \cdot
  g^{\alpha_4} \]
Dimensionsanalyse:
\begin{center}
  \begin{tabular}{r|cccc}
    & $t_{sd}$ & $l$ & $m$ & $g$ \\
    \hline
    $T$ & 1 & 0 & 0 & -2 \\
    $L$ & 0 & 1 & 0 & 1 \\
    $M$ & 0 & 0 & 1 & 0
  \end{tabular}
\end{center}
$q$ muss dimensionslos sein, also:
\[ S\alpha = 0 \qRq \alpha = c \pmat{1 \\ -1/2 \\ 0 \\ 1/2 } \]
und damit
\[ q = t_{sd} l^{-1/2} m^0 g^{1/2}, \qquad t_{sd} = q \sqrt{\frac{l}{g}}. \]
Die Schwingungsdauer ist unabhängig von $m$! Diese Aussage ist ohne Kenntnis der
genauen Systemgleichungen möglich.

Insbesondere ändert eine Skalierung $l \to s \cdot l$, $g \to s \cdot g$ die
Schwingungsdauer nicht.

Bemerkung: Die tatsächliche Lösung ist $t_{sd} = 2 \pi \sqrt{l/g}$.

%%% Local Variables:
%%% TeX-master: "master"
%%% End: