\documentclass[
 a4paper,
 12pt,
 parskip=half
 ]{scrartcl}

\usepackage{../.tex/settings}

\usepackage{../.tex/mathpkgs}
\usepackage{../.tex/mathcmds}

\numberwithin{equation}{section}

\theoremstyle{plain}
\newtheorem{thm}{Satz}[section] % reset theorem numbering for each chapter

\theoremstyle{definition}
\newtheorem{defn}[thm]{Definition} % definition numbers are dependent on theorem numbers
\newtheorem{exmp}{Beispiel}[section] % same for example numbers
\newtheorem{rmrk}{Bemerkung}[section] % same for example numbers

\numberwithin{equation}{section}

%opening
\title{Vorlesung\\Einführung in die Numerik}
\subtitle{Wintersemester 2016/2017}
\author{Vorlesung: Prof. Dr. Andreas Fischer\\Mitschrift: Jonas Hippold}

\begin{document}

\maketitle

\tableofcontents

\section{Interpolation}
\subsection{Aufgabenstellung}
\textbf{Gegeben:} $n+1$ Datenpaare $(x_0, f_0), \ldots, (x_n,f_n)$ reeller Zahlen. Die Zahlen $x_0, \ldots, x_n$ heißen \emph{Stützstellen} und die Zahlen $f_0, \ldots, f_n$ \emph{Stützwerte}.

Die Grundaufgabe der Interpolation besteht darin, eine Funktion $F: \real \to \real$ zu finden, die den folgenden \emph{Interpolationsbedingungen}
\begin{equation}
 F(x_0) = f_0,\, \ldots,\, F(x_n) = f_n \label{eq:int_bedingung}
\end{equation}
genügt. In Erweiterung dessen können etwa auch Bedingungen an Ableitungen von $F$ an bestimmten Stellen vorkommen.

Eine solche Funktion wird \emph{Interpolierende} genannt. Es ist sehr leicht, diese Aufgabe zu lösen. Zum Beispiel ist
\[ f(x) = \begin{cases}
           0, &\text{falls } x \notin \{ x_0, \ldots, x_n \}, \\
           f_0, &\text{falls } x = x_0, \\
            &\vdots \\
           f_n, &\text{falls } x = x_n.
          \end{cases} \]
immer eine Interpolierende. Zur sinnvollen Bearbeitung der Interpolationsaufgabe muss man also die Menge der als Interpolierende zugelassenen Funktionen einschränken, zum Beispiel auf Polynome, Winkel- oder Exponentialfunktionen oder Spline-Funktionen. Im Zusammenhang mit der Wahl der Funktionenmenge (eigentlich des Funktionenraums) sind folgende Fragen von Bedeutung:
\begin{itemize}
 \item Existenz einer Interpolierenden,
 \item Eindeutigkeit der Interpolierenden,
 \item Weitere Eigenschaften (zum Beispiel hinsichtlich Krümmung, Approximation einer Funktion $f$ (wenn $f = f(x_k)$ für $k=0,\ldots,n$)
 \item Wie sollte man die Stützstellen wählen, wenn sie nicht vorgegeben sind?
 \item Effizienz der Bestimmung einer Interpolierenden.
\end{itemize}

\begin{exmp}
 Die Temperatur bei Erwärmung eines Kessels wurde zu diskreten Zeitpunkten tabelliert:
 
 \begin{center}
 \begin{tabular}{c|cccccc}
  k & 0 & 1 & 2 & 3 & 4 & 5\\
  \hline
  $x_k$ in \si{\s} & 0 & 1 & 2 & 3 & 4 & 5\\
  $f_k$ in \si{\celsius} & 80 & 85.8 & 86.4 & 93.6 & 98.3 & 99.1
 \end{tabular}
 \end{center}

 Beispielsweise soll die Interpolierende im
 \begin{enumerate}[a)]
  \item Raum der stetigen stückweise affinen Funktionen,
  \item Raum der Polynome höchstens 5. Grades,
  \item Raum der Polynome höchstens 4. Grades
 \end{enumerate}
 ermittelt werden.
\end{exmp}

\subsection{Interpolation durch Polynome}
Sei $\Pi_n$ der Vektorraum der Polynome vom Höchstgrad $n$ mit der üblichen Addition $(p_1+p_2)(x) := p_1(x) + p_2(x)$ und der üblichen skalaren Multiplikation $(\alpha p)(x) := \alpha \cdot p(x)$.

Für jedes $p \in \Pi_n$ gibt es dann $a_0, \ldots, a_n \in \real$, so dass 
\begin{equation} p(x) = a_n x^n + a_{n-1} x^{n-1} + \ldots + a_1 x + a_0 \text{ für alle } x \in \real. \end{equation}

\subsubsection{Existenz und Eindeutigkeit}
\begin{thm}
 Zu $n+1$ beliebigen Datenpaaren $(x_0, f_0), \ldots, (x_n, f_n)$ mit paarweise verschiedenen Stützstellen existiert genau ein Polynom $p \in \Pi_n$, das die Interpolationsbedingungen (\ref{eq:int_bedingung}) erfüllt.
\end{thm}

Dieses Polynom heißt \emph{Interpolationspolynom} zu den Datenpaaren $(x_0, f_0), \ldots, (x_n, f_n)$.

\begin{proof}
 \begin{enumerate}[a)]
  \item Existenznachweis durch Konstruktion.
  
  Sei $j \in \{ 0, \ldots, n \}$ und bezeichne $L_j: \real \to \real$ mit
  \[ L_j(x) := \prod_{i=0,i \ne j}^n \frac{x-x_i}{x_j-x_i} = \frac{(x-x_0)\cdots(x-x_{j-1})(x-x_{j+1})\cdots(x-x_n)}{(x_j-x_0)\cdots(x_j-x_{j-1})(x_j-x_{j+1})\cdots(x_j-x_n)} \]
  das \emph{Lagrange-Basispolynom} vom Grad $n$. Offenbar gilt $L_j \in \Pi_n$ und
  \begin{equation} L_j(x_k) = \delta_{jk} = \begin{cases}
                 1, &\text{falls } k = j, \\
                 0, &\text{falls } k \ne j.
                \end{cases} \end{equation}
  Definiert man $p: \real \to \real$ durch
  \begin{equation} p(x) = \sum_{j=0}^n f_j \cdot L_j(x) \text{ für alle } x \in \real, \end{equation}
  so ist $p \in \Pi_n$. Außerdem erfüllt $p$ wegen (1.3) die Interpolationsbedingung (1.1).
  \item Nachweis der Eindeutigkeit.
  
  Angenommen, es existieren zwei Interpolierende $p$ und $\tilde{p} \in \Pi_n$ und $p \ne \tilde{p}$. Dann folgt 
  \[ p - \tilde{p} \in \Pi_n \text{ und } (p-\tilde{p})(x_k) = p(x_k) - \tilde{p}(x_k) = 0 \text{ für } k = 0, \ldots, n. \]
  Das heißt, $p - \tilde{p}$ ist vom Höchstgrad $n$ und hat mindestens $n+1$ Nullstellen. Also muss $p-\tilde{p}$ das Nullpolynom sein. Widerspruch zur Annahme $p \ne \tilde{p}$!
 \end{enumerate}
\end{proof}

\begin{rmrk}
 \begin{enumerate}[a)]
  \item Die Darstellung (1.4) heißt \emph{Lagrange-Form} des Interpolationspolynoms.
  \item Um mit (1.4) einen ersten Funktionswert $p(x)$ zu berechnen, werden $O(n^2)$ benötigt. Bei äquidistanten Stützstellen genügen $O(n)$ Operationen.
  
  Ändern sich nur die Stützwerte (die Stützstellen und die Stelle $x$ bleiben gleich), dann kann man $p(x)$ in $O(n)$ ermitteln.
  \item Man kann zeigen, dass die Lagrange-Basispolynome $L_0, \ldots, L_n$ eine Basis des Vektorraums $\Pi_n$ bilden.
 \end{enumerate}
\end{rmrk}

\subsubsection{Newton-Form des Interpolationspolynoms}
Ein Polynom $p \in \Pi_n$ von der Form 
\[ p(x) = c_0 + c_1(x-x_0) + c_2(x-x_0)(x-x_1) + \ldots + c_n (x-x_0)(x-x_1) \ldots (x-x_{n-1}) \] mit geeigneten Koeffizienten $c_0, \ldots, c_n \in \real$, das die Bedingung (1.1) erfüllt, wird als Interpolationspolynom in \emph{Newton-Form} bezeichnet.

Die Berechnung der Koeffizienten $c_0, \ldots, c_n$ kann rekursiv durch Ausnutzung der Interpolationsbedingungen erfolgen:
\[ f_0 \overset{!}{=} p(x_0) = c_0, \text{ also } c_0 = f_0. \] 
Seien $c_0, \ldots, c_{j-1}$ bereits bestimmt. Dann folgt
\begin{align*}
 f_j \overset{!}{=} p(x_j) = c_0\, &+ c_1 (x_j-x_0) + \ldots + c_{j-1} (x_j-x_0) \ldots (x_j-x_{j-2}) \\
   &+ c_j (x_j-x_0)(x_j-x_1) \ldots (x_j-x_{j-1}) \\
   &+ c_{j+1} (x_j-x_0)(x_j-x_1) \ldots \underbrace{(x_j-x_j)}_{=0}(x_j-x_{j+1} ) + \ldots . 
\end{align*}
Alle Terme ab $c_{j+1}$ entfallen, da sie $(x_j-x_j)=0$ enthalten. Damit lässt sich $c_j$ berechnen:
\[ c_j = \frac{f_j - c_0 + \sum_{k=1}^{j-1}(x_j-x_0)\ldots(x_j-x_{k-1})}{(x_j-x_0) \ldots (x_j-x_{j-1})} \]

\end{document}