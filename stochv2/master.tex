\documentclass[
 a4paper,
 12pt,
 parskip=half
 ]{scrreprt}

\usepackage{../.tex/settings}

\usepackage{../.tex/mathpkgs}
\usepackage{../.tex/mathcmds}

\usepackage[numbers,with_chapter]{fancy_thm}

\swapnumbers
\theoremstyle{plain}
% \newtheorem{thm}{Satz}[section] % reset theorem numbering for each chapter
\newtheorem*{thm*}{Satz}
%\newtheorem{lem}[thm]{Lemma}    

\theoremstyle{definition}
\newtheorem{defn}[thm]{Definition} 
\newtheorem{folg}[thm]{Folgerung} 
\newtheorem{rmrk}[thm]{Bemerkung} 
\newtheorem{deno}[thm]{Bezeichnungen}
\newtheorem{exmp}[thm]{Beispiel}
%\newtheorem{aufg}[thm]{Aufgabe} 
\newtheorem{prgp}[thm]{} % Numbered paragraph

\newtheorem*{rmrk*}{Bemerkung}
\newtheorem*{exmp*}{Beispiel}
\newtheorem*{defn*}{Definition}
\newtheorem*{deno*}{Bezeichnung}
\newtheorem*{denos*}{Bezeichnungen}

\numberwithin{equation}{section}

\hypersetup{
  pdftitle={Stochastik},
  pdfauthor={Jonas Hippold},
  hidelinks
}

%opening
\title{%
  Vorlesung\\
  Stochastik Vertiefung\\
  Stationäre Prozesse
}
\subtitle{Sommersemester 2018}
\author{Vorlesung: Prof. Dr. Zoltán Sasvári\\Mitschrift: Jonas Hippold}

\begin{document}

\maketitle

\tableofcontents

\clearpage

%%\section*{Organisatorisches}
Aufgaben:\\
\url{http://www.math.tu-dresden.de/~sasvari/Download/BaStoch-Aufgaben/}

Abgabetermin für die Hausaufgaben in der Regel Mittwoch 13:00.

Prüfungsvorleistung: 50 \% der Aufgabenpunkte.

Schriftliche Prüfung: 90 Minuten

Formulierung von Definitionen, Sätzen; Beweise, Aufgaben ähnlich zu den
Übungsaufgaben.

Formelsammlung: Ein A4-Blatt, beidseitig beschrieben

Ausweichtermin für die Übung am 14. April ist der 12. April.

\subsection*{Literatur}
\begin{itemize}
\item H. Bauer: Wahrscheinlichkeitstheorie und Grundzüge der Maßtheorie
\item A. Rényi: Wahrscheinlichkeitsrechnung
\item K. D. Schmidt: Maß und Wahrscheinlichkeit
\end{itemize}
\section*{Organisatorisches}
Aufgaben:\\
\url{http://www.math.tu-dresden.de/~sasvari/Download/StatProz/}

Prüfungsvorleistung: Vorrechnen von zwei Aufgaben in der Übung (evtl. auch
abgeben). 

Klausur für beide Teile des Moduls STOCHV: 110 Minuten

Teil Stationäre Prozesse: Formulierung von Definitionen, Sätzen; Beweise,
Aufgaben ähnlich zu den Übungsaufgaben.

Übung: Nico Uhlig

\subsection*{Literatur}
\begin{itemize}
\item H. Bauer: Wahrscheinlichkeitstheorie und Grundzüge der Maßtheorie
\item A. Rényi: Wahrscheinlichkeitsrechnung
\item K. D. Schmidt: Maß und Wahrscheinlichkeit
\end{itemize}

\chapter{Einleitung}
\section{Bezeichnungen}
\begin{prgp}
  Grundlegende Bezeichnungen:
  
  \begin{tabular}{lcl}
    $\nat$ & = & $\{1,2,3, \ldots\}$ \\
    $\nat_0$ & = & $\{0,1,2, \ldots\}$ \\
    $\integer$ & = & $\{ 0, \pm 1, \pm 2, \ldots \}$ \\
    $\rat$ & & rationale Zahlen \\
    $\real$ & & reelle Zahlen \\
    $\complex$ & & komplexe Zahlen \\
    $x_+$ & = & $\max(0,x)$, $x \in \real$ \\
    $\ind_A$ & & Indikatorfunktion der Menge $A$ \\
    $\delta_x$ & & Einpunkt- oder Dirac-Maß, das in $x$ konzentriert ist \\
    $\delta_{x,y}$ & & Kronecker'sche Delta-Funktion \\
    $\tilde{f}(x)$ & = & $\obar{f(-x)}$ \\
    $\pE(X)$ & & Erwartungswert der Zufallsgröße bzw. des Zufallsvektors $X$ \\
    $\intf_p(\real^d)$ & & Menge der komplexwertigen Lebesgue-messbaren
                           Funktionen $f$ auf $\real^d$, \\
           & & für die $|f|^p$ Lebesgue-integrierbar ist \\
    $\intf_p(\integer^d)$ & & Menge der komplexwertigen Funktionen $f$
                              auf $\integer^d$, \\
           & & die bezüglich des Zählmaßes integrierbar sind. \\
    $\| \cdot \|_p$ & & $L_p$-Norm \\
    $L_p(\real^d)$ & & $L_p$-Raum bezüglich des Lebesgue-Maßes auf $\real^d$ \\
    $L_p(\integer^d)$ & & $L_p$-Raum bezüglich des Zählmaßes auf $\integer^d$
  \end{tabular}
\end{prgp}

\begin{prgp}[Mehrdimensionale Bezeichnungen]
  Ist $x \in \real^d$ ($\complex^d$, $\nat_0^d$ usw.), so bezeichnet $x_i$ ($1
  \le i \le d$) die $i$-te Koordinate von $x$ und wir schreiben $x$ als $x =
  (x_1, \ldots, x_d)$. In Ausdrücken, wo Matrix-Operationen\footnote{%
    Zum Beispiel $Ax$ mit der Matrix $A$}
  vorkommen, betrachten wir $x$ als Spalten-Vektor. Das Nullelement von
  $\real^d$ wird auch mit 0 bezeichnet.

  Sei $x \in \real^d$ oder $\complex^d$ und $\alpha \in \nat_0^d$. Dann
  schreiben wir
  \begin{align*}
    x^\alpha
    &= x_1^{\alpha_1} \cdot x_1^{\alpha_2} \cdot \ldots \cdot x_d^{\alpha_d}, \\
    \| x \|
    &= \sqrt{|x_1|^2 + \ldots + |x_d|^2}, \\
    |\alpha|
    &= \alpha_1 + \ldots + \alpha_d, \\
    \alpha!
    &= \alpha_1! \cdot \ldots \cdot \alpha_d!, \\
    \binom{\alpha}{\beta}
    &= \frac{\alpha!}{(\alpha-\beta)! \beta!}
      = \binom{\alpha_1}{\beta_1} \cdot \ldots \cdot
      \binom{\alpha_d}{\beta_d}, \qquad \beta \le \alpha, \\
    D^\alpha
    &= \frac{\partial^{|\alpha|}}
      {\partial x_1^{\alpha_1} \cdot \ldots \cdot \partial x_d^{\alpha_d}}.
  \end{align*}
  Wenn $\alpha = (0, \ldots, 0)$, dann ist $x^\alpha := 1$.
\end{prgp}

\begin{rmrk*}
  Es gilt
  \[ D^\alpha D^\beta g = D^{\alpha + \beta} g, \]
  wobei $g$ eine beliebige reell- oder komplexwertige Funktion auf $\real^d$
  ist, für die die partiellen Ableitungen $D^{\alpha + \beta} g$ existieren.

  Es gilt $D^\alpha g = g$, wenn $\alpha = (0, \ldots, 0)$.

  Sei $\beta = (\beta_1, \ldots, \beta_d)$ ein anderes $d$-Tupel von
  nichtnegativen ganzen Zahlen. Wir schreiben $\beta \le \alpha$, wenn $\beta_j
  \le \alpha_j$ für alle $j$.

  Mit $\diffop x$, $\diffop \lambda(x)$ oder $\diffop \lambda_d(x)$ bezeichnen
  wir die Integration bezüglich des Lebesgue'schen Maßes $\lambda = \lambda_d$
  auf $\real^d$ und $(x,y)$ ist das Skalarprodukt von $x, y \in \real^d$ oder
  $\complex^d$.
\end{rmrk*}

\section{Historische Bemerkungen}
Geschichte:
\begin{itemize}
\item Albert Einstein (1914), stochastische Beschreibung von Molekülbewegungen
\item Älteste Zeitreihe: Zählung der Sonnenflecken, Maxima treten in regelmäßigen
\item Abständen auf, aber es gibt zufällige Abweichungen.
\item Khinchin (1934)
\end{itemize}
Zusammenfassung von L. Cohen: The History of Noise, IEEE Signal Proc. Mag.,
2005.

\chapter{Zufällige Felder zweiter Ordnung}
In diesem Abschnitt:
\begin{itemize}
\item $V \ne \emptyset$ ist eine beliebige nichtleere Menge.
\item $(\Omega, \mA, \pP)$ ist ein Wahrscheinlichkeitsraum.
\end{itemize}

\section{Zufallsfelder}
\begin{defn}
  Ein (reelles) \emph{Zufallsfeld} $Z$ auf $V$ ist eine Abbildung, die jedem
  Element $t \in V$ eine Zufallsgröße $Z(t)$ auf $\Omega$ zuordnet.

  Sind die Zufallsgrößen komplexwertig, dann heißt auch $Z$ komplexes
  Zufallsfeld.
\end{defn}

\begin{rmrk*}
  \begin{itemize}
  \item Anstelle von $Z(t)$ wird auch $Z_t$ geschrieben.
  \item Das Feld $Z$ wird auch mit $\{ Z_t \}$ oder $\{ Z_t : t \in V \}$ oder
    $\{ Z_t \}_{t \in V}$ bezeichnet.
  \item $V = \real, [0, \infty),[0,1]$: zeitstetiger stochastischer Prozess.
  \item $V = \nat, \nat_0$: zeitdiskreter stochastischer Prozess,
    \emph{Zeitreihe}.
  \item Die Funktionen $t \to Z_t(\omega)$ heißen \emph{Realisierungen} des
    Feldes.
  \end{itemize}
\end{rmrk*}

\begin{exmp}
  \begin{enumerate}[(a)]
  \item Sind $X_j$, $j \in \nat_0$, beliebige Zufallsgrößen (auf $\Omega$), so
    sind
    \[ \{X_j, j \in \nat_0\} \quad \text{und} \quad S_n = \sum_{j=0}^n X_j,
      \quad n \in \nat_0 \]
    Zeitreihen.
  \item Seien $a$ und $\varphi$ Zufallsgrößen und $u \in \real$. Dann ist
    \[ X_t = a \cdot \cos (ut + \varphi), \qquad t \in \real \]
    ein stochastischer Prozess auf $\real$, die \emph{harmonische Schwingung}
    mit zufälligen Parametern; Amplitude und Phase sind zufällig, die Frequenz
    ist gegeben.
  \end{enumerate}

  Allgemeiner:
  \[ X_t(\omega) = \sum_{j=1}^n a_j(\omega) \cdot
    \cos(u_j t + \varphi_j(\omega)), \qquad t \in \real. \]
  
  Viele Prozesse in Anwendungen lassen sich so modellieren. Die Menge $\{ u_1,
  \ldots, u_n \}$ heißt \emph{Spektrum} von $X$.

  Wichtige Aufgabe: Näherungsweise Bestimmung des Spektrums mit Hilfe von
  endlich vielen Beobachtungen $X_{t_1}(\omega), \ldots, X_{t_n}(\omega)$.

  Oft ist es vorteilhaft, komplexwertige Schwingungen zu betrachten:
  \[ Z_t = \sum_{j=1}^n c_j \cdot e^{i u_j t}, \qquad t \in \real. \]
  Hier ist $c_j$ eine komplexe Zufallsgröße. Wir schreiben $c_j$ in der Form
  \[ c_j = | c_j | \cdot e^{i \varphi_j}, \]
  wobei $\varphi_j$ eine reelle Zufallsgröße ist. Dann gilt
  \[ Z_t = \sum_{j=1}^n |c_j| \cdot e^{i(u_j t + \varphi_j)} \]
  und 
  \begin{align*}
    \Re Z_t &= \sum_{j=1}^n |c_j| \cdot \cos (u_j t + \varphi_j) \\
    \Im Z_t &= \sum_{j=1}^n |c_j| \cdot \sin (u_j t + \varphi_j)
  \end{align*}
\end{exmp}

\begin{defn}
  Ein komplexes \emph{Zufallsfeld zweiter Ordnung} $Z$ auf $V$ ist eine
  Abbildung $Z: V \to \intf_2(\Omega,\mA,\pP)$.

  Wird $\intf_2$ durch $\intf_2^r$ ersetzt, so heißt $Z$ ein \emph{reelles
    Zufallsfeld zweiter Ordnung}.
\end{defn}

In beiden Fällen bezeichnet $(\cdot, \cdot)$ das Skalarprodukt
\[ (X,Y) = \int_{\Omega} X \cdot \obar{Y} \diffop \pP = \pE(X \cdot \obar{Y}). \]
Speziell: $\pE(X) = (X,\ind)$

Weiterhin gilt:
\[ (X,Y) = (\ind, \obar{X} \cdot Y) = (X \cdot \obar{Y}, \ind). \]
$\| \cdot \|$ bezeichnet die Norm
\[ \| X \| := \sqrt{(X,X)}. \]
Damit können wir eine Metrik definieren:
\[ d(X,Y) := \| X - Y \|. \]
Cauchy-Schwartz:
\[ |\pE(X \cdot Y)| \le \| X \| \cdot \| Y \|. \]
Orthogonalität:
\[ X \perp Y \quad :\Leftrightarrow \quad (X,Y) = \pE(X \cdot \obar{Y}) = 0.\]
Zwei Zufallsgrößen mit Erwartungswert 0 sind genau dann unkorreliert, wenn sie
orthogonal sind.

Wir werden ein Zufallsfeld $Z$ zweiter Ordnung auch als eine Abbildung $Z : V
\to L_2(\Omega, \mA, \pP)$ oder $Z : V \to L^r_2(\Omega, \mA, \pP)$ betrachten.
In diesem Fall ist $Z_t$ keine Zufallsgröße, sondern eine Äquivalenzklasse von
Zufallsgrößen.

Vorteil dieser Betrachtung: $L_2$ und $L_2^r$ sind Hilberträume.

Vorsicht: Zum Beispiel ist $Z_t(\omega)$ kein sinnvoller Ausdruck, aber
\[ (Z_t, Z_s) = \int_\Omega Z_t \cdot \obar{Z}_s \diffop \pP, \]
da $\int X \diffop \pP = \int Y \diffop \pP$, wenn $X = Y$ fast sicher.

Ausnahme: $Z_t(\omega)$ kann doch sinnvoll sein.
\[ \pP(A) = 0  \qRq A = \emptyset \tag{$*$} \]
gilt genau dann, wenn $\Omega = \{ \omega_1, \omega_2, \ldots \}$ höchstens
abzählbar viele Elemente hat und $\pP(\omega_j) > 0$ für alle $j$. Das heißt,
aus ($*$) folgt $\pP(\omega) > 0$ für alle $\omega \in \Omega$ und damit auch,
dass $\Omega$ höchstens abzählbar unendlich ist, weil
\[ \Omega = \bigcup_n
  \underbrace{%
    \left\{ \omega : \pP(\{\omega\}) \ge \rez{n} \right\}
  }_{%
    \text{Höchstens $n$ Elemente}
  }.
\]

\begin{deno*}
  Sei $Z$ ein Feld auf $V$ mit Werten in $L_2(\Omega,\mA,\pP)$ oder in
  $\intf(\Omega,\mA,\pP)$. Mit $H(Z)$ bzw. $\mathcal{H}(Z)$ bezeichnen wir die
  abgeschlossene lineare Hülle (bezüglich $\| \cdot \|$) der ``Vektoren'' $Z_t$,
  $t \in V$:
  \[ H(Z) = \left\{
      \sum_{j=1}^n c_j Z_{t_j} : c_j \in \complex, t_j \in V, n \in \nat
    \right\}^-. \]
  Sie besteht also aus allen Zufallsgrößen, die man durch endliche
  Linearkombinationen erzeugen kann. $\mathcal{H}(Z)$ analog.

  Im reellen Fall schreibt man $H^r(Z)$, $\mathcal{H}^r(Z)$.

  $H(Z)$ und $H_2^r(Z)$ sind Hilberträume.
\end{deno*}

\begin{exmp}
  \begin{enumerate}[(a)]
  \item Sei $X \in L_2(\pP)$ und $f: V \to \complex$ beliebig. Dann ist
    \[ Z(t) = f(t) \cdot X, \quad t \in V \]
    ein Feld zweiter Ordnung. Weiterhin gilt:
    \[ H(Z) = \complex \cdot X. \]
    Falls $f \equiv 0$, dann ist $H(Z) = \{0\}$.

    \emph{Allgemeiner:} Seien $X_n \in L_2(\pP)$, $n \in \nat$, unkorreliert mit
    $\pE(X_n) = 0$ und Varianz 1, das heißt $\| X_n \| = 1$. Dann ist $\{X_1,
    X_2, \ldots \}$ ein \emph{orthonormiertes System}\footnote{%
      Das heißt $(X_i, X_j) = 0$ für $i \ne j$ und $(X_j, X_j) = 1$.
    } im Hilbertraum $L_2(\pP)$.

    Weiterhin seien $f_n : V \to \complex$ so, dass
    \[ \sum_{n=1}^\infty |f_n(t)|^2 < \infty. \]
    Dann ist
    \[ Z(t) = \sum_{n=1}^\infty f_n(t) \cdot X_n \]
    ein Feld zweiter Ordnung.
  \item Umkehrung von (a). Spezialfall: $L_2(\pP)$ separabel\footnote{%
      Ein metrischer Raum $M$ heißt \emph{separabel}, wenn eine abzählbare
      Teilmenge $N = \{m_1, m_2, \ldots \} \subset M$ existiert mit $\obar{N} =
      M$. Zum Beispiel ist $\real^d$ separabel, $N = \rat^d$.
    }. Dann existiert eine abzählbare orthonormale Basis
    $\{ e_1, e_2, \ldots \}$. Dann gilt:
    \[ Z(t) = \sum_{n=1}^\infty f_n(t) \cdot e_n, \quad t \in V, \]
    wobei $f_n(t) = (Z_t, e_n)$. Weiterhin ist
    \[ \| Z(t) \|^2 = \sum_{n=1}^\infty | f_n(t) |^2. \]
  \end{enumerate}
\end{exmp}

Bemerkung:
Sei $e_1, e_2, \ldots \in \mathcal{H}$ Hilbert-Raum ein orthonormiertes
System.
\[ \sum_{j=1}^\infty c_j e_j := \lim_{n \to \infty} \sum_{j=1}^n c_j e_j, \]
falls der Grenzwert existiert.

$\lim_n x_n$ existiert $\Leftrightarrow$ $\{ x_n\}$ ist eine Cauchy-Folge in
beliebigen metrischen Räumen.
\[ x_n = \sum_{j=1}^n c_j \cdot e_j \]
ist eine Cauchy-Folge, wenn
\[ \| x_n - x_m \| \xrightarrow{n,m \to \infty} 0. \]
Sei zum Beispiel $m \ge n$, dann ist
\[ x_n - x_m = \sum_{j=n}^m c_j e_j. \]
Also folgt
\[ \| x_n - x_m \|^2 = \sum_{j=n}^m |c_j|^2 \xrightarrow{m,n \to \infty}
  0 \]
$\Leftrightarrow$ $\sum_{j=1}^\infty |c_j|^2 < \infty$.

Wiederholung:
Feld 2. Ordnung ist eine Abb. $V \to \intf^2(\pP)$ bzw. $L^2(\pP)$.

Das Skalarprodukt $(X,Y) := \pE( X \cdot \obar{Y})$ ist für beide Varianten
sinnvoll. Es ist positiv semidefinit:
\[ (X,X) \ge 0, \]
für $L^2(\pP)$ sogar
positiv definit:
\[ (X,X) > 0 \qquad \text{für} \qquad X \ne 0. \]

Bekannt aus der Maßtheorie: Wenn $X \ge 0$ und $\int X \diffop \pP = 0$, dann
ist $X = 0$ fast sicher.

\begin{defn}
  Seien $X$ und $Y$ Felder 2. Ordnung auf $V$ mit dem selben $\pW$-Raum
  $(\Omega, \mA, \pP)$. Gilt $(X(t),Y(s)) = 0$ für alle $t,s \in V$, so heißen
  $X$ und $Y$ orthogonal: $X \perp Y$.

  Zum Beispiel $X(t)$, $Y(s)$ unabhängig und $\pE(X(t)) = \pE(Y(s)) = 0$, dann
  \[ (X(t),Y(s)) = \pE(X(t) \obar{Y(s)}) = \pE(X(t)) \pE(\obar{Y(s)}) = 0. \]
\end{defn}

\begin{defn}
  Sei $Z$ ein Zufallsfeld auf $V$ und $t_1, \ldots, t_n \in V$. Dann ist $(Z_{t_1},
  \ldots, Z_{t_n})$ ein Zufallsvektor und besitzt eine Verteilung. Verteilungen
  dieser Form heißen \emph{endlichdimensionale Verteilungen} von $Z$.

  Ist $Z$ reell oder komplex und sind alle endlichdimensionalen Verteilungen
  Gauß'sch, so nennt man auch $Z$ \emph{Gauß'sch}.

  Simples Beispiel: $V = \{1\}$, $Z_1$ beliebige normalverteilte Zufallsgröße.
  Oder $V = \{1, 2, \ldots\} = \nat$, $Z_1, Z_2, \ldots$ Gauß'sch und
  unabhängig. Dann ist die Verteilung von $(Z_1, \ldots, Z_n)$ das Produkt der
  Verteilungen der $Z_j$.
\end{defn}

\section{Die Korrelationsfunktion}
\begin{defn}
  Sei $Z$ ein Feld zweiter Ordnung auf $V$. Die Funktion $C: V \times V \to
  \complex$, die durch
  \[ C(X,Y) = \pE[ Z(X) \cdot \obar{Z(Y)} ) = ( Z(X), Z(Y) ] \]
  definiert ist, heißt \emph{Korrelationsfunktion} von $Z$.

  Die \emph{Kovarianzfunktion} ist durch
  \[ \sigma(X,Y) = \pE [ (Z(X) - M(X)) \cdot \obar{(Z(Y) - M(Y))}] \]
  definiert, wobei $M(X) := \pE(Z(X))$ der Mittelwert bzw. Erwartungswert des
  Feldes ist.
\end{defn}

Es ist leicht zu sehen:
\[ \sigma( X, Y ) = C(X,Y) - M(X) \cdot \obar{M(Y)}. \]
Wenn $M \equiv 0$, dann ist $\sigma = C$.

\begin{thm}
  Eine komplexwertige (reellwertige) Funktion $K$ auf $V \times V$ ist genau
  dann die Kovarianz- oder Korrelationsfunktion eines komplexen (reellen) Felder
  zweiter Ordnung, wenn für beliebige $X_1, \ldots, X_n \in V$, $n \in \nat$,
  die Matrix
  \[ (K(X_i,X_j))_{i,j=1}^n \]
  positiv semidefinit (positiv semidefinit und symmetrisch) ist.
\end{thm}

\begin{proof}
  Nehmen wir an, dass $K$ die Kovarianzfunktion des Feldes $Z$ ist. Dann gilt
  \[ \sum_{i,j=1}^n K(X_i, X_j) a_i \obar{a_j}
    = \sum_{i,j=1}^n \pE[ (Z(X_i) - M(X_i)) \cdot \obar{(Z(X_j) - M(X_j))}] a_i
    \obar{a_j}. \]
  Es gilt
  \[ \pE \left| \sum_{j=1}^n (Z(X_j) - M(X_j)) a_j \right|^2 \ge 0, \]
  denn
  \[ |w|^2 = w \cdot \obar{w} \]
  für alle $w \in \complex$, also auch für
  \[ w := \sum_{j=1}^n (Z(X_j) - M(X_j)) a_j. \]

  Die Fälle $Z$ reell und $K$ Korrelationsfunktion folgen analog. Der zweite
  Teil des Beweises folgt aus Satz 2.2.4.
\end{proof}

\begin{lem} %% 2.2.3
  Sei $C = (c_{jk})_{j,k=1}^n$ eine positiv semidefinite Matrix und
  \[ A = (a_{jk}) = \Re C, \qquad B = (B_{jk}) = \Im C. \]
  Dann ist die $(2n) \times (2n)$-Matrix
  \[ D = (d_{jk}) = \begin{bmatrix}
      A & -B \\ B & A
    \end{bmatrix}
  \]
  ebenfalls positiv semidefinit.
\end{lem}

\begin{proof}
  Aus $c_{jk} = \obar{c_{kj}}$ folgt, dass $A^\top = A$ ud $B^\top = -B$, damit
  muss $D$ symmetrisch sein.

  Für beliebige $r_1, \ldots, r_{2n} \in \real$ gilt
  \[ 0 \le \sum_{j,k = 1}^n c_{jk}( r_j - i r_{n+j}) \cdot (r_k + r_{n+k}), \]
  da $C$ positiv semidefinit ist.

  Es gilt
  \begin{align*}
    &\sum_{j,k=1}^n c_{jk} (r_j r_k + r_{n+j} r_{n+k}) - i \sum_{j,k=1}^n c_{jk}
    (r_{n+j} r_k + r_j r_{n+k}) \\
    = &\sum_{j,k=1}^n a_{jk} (r_j r_k + r_{n+j} r_{n+k}) + \sum_{j,k=1}^n b_{jk}
        (r_{n+j} r_k + r_j r_{n+k}) \\
    = &\sum_{j,k = 1}^n d_{jk} r_j r_k.
  \end{align*}
  Die erste Summe ist reell, die zweite rein imaginär, da
  $c_{jk} = \obar{c_{kj}}$. Die zweite Identität folgt aus der Definition von
  $D$.
\end{proof}

\clearpage

\begin{thm}
  Sei $M : V \to \complex$ beliebig und $K : V \times V \to \complex$ so, dass
  für beliebige $x_1, \ldots, x_n \in V$ die Matrix
  \[ (K(x_i, x_j) )_{i,j=1}^n \tag{1} \]
  positiv semidefinit ist. Dann existiert ein Gauß-Feld $Z$ auf $V$ mit
  \begin{align*}
    \pE(Z(x))
    &= M(x), \tag{2} \\
    \pE(Z(x) \cdot \obar{Z(y)}) - M(x) \obar(M(y))
    &= K(x,y), \tag{3} \\
    \pE(Z(x) \cdot Z(y))
    &= M(x) \cdot M(y). \tag{4}
  \end{align*}
  Sind $M$ und $K$ reellwertig, so existiert ein reelles Gauss-Feld $Z$ so, dass
  (2) und (3) gelten.
\end{thm}

\begin{rmrk*}
  Wenn ein Feld $Z$ (2) und (3) erfüllt, dann ist
  \[ \pE[ (Z(x) - M(x))^2 ] = 0. \]
  Wenn also $Z$ reell ist, folgt $Z(x) = M(x)$.
\end{rmrk*}

\appendix

\setcounter{chapter}{14}
\chapter{Orthogonalität}
Sei $X$ ein linearer Raum, $(\cdot, \cdot)$ das Skalarprodukt. $x,y \in X$
heißen \emph{orthogonal}, wenn
\[ (x,y) = 0. \]
Schreibweise $x \perp y$.

Satz von Pythagoras: Für $x \perp y$ gilt
\[ \| x + y \|^2 = \| x \|^2 + \| y \|^2. \]

\begin{proof}
  \[ \|x+y\|^2 = (x+y,x+y) = \ldots \qedhere \]
\end{proof}

Parallelogrammidentität:
\[ \|x+y\|^2 + \|x-y\|^2 = 2(\|x\|^2 + \|y\|^2). \]

Sei $x \in X$, $A,B \subset X$. $x$ ist \emph{orthogonal} zu $A$, $x \perp A$,
wenn $x \perp a$ für alle $a \in A$. $A$ ist \emph{orthogonal} zu $B$, $A \perp
B$, wenn $a \perp b$ für alle $a \in A$, $b \in B$.

Das \emph{orthogonale Komplement} von $A$ ist
\[ A^\perp := \{ x \in X : x \perp A \}. \]

\renewcommand{\thethm}{O.\arabic{thm}}
\setcounter{thm}{0}
\begin{aufg} %% O.1
  \begin{enumerate}
  \item $A^\perp$ ist ein abgeschlossener Teilraum von $X$.
  \item $\{ 0 \}^\perp = X$ und $X^\perp = \{ 0 \}$.
  \item Aus $A \subset B$ folgt $B^\perp \subset A^\perp$.
  \item $A^\perp = (\obar{A})^\perp$ (Abschluss) und $A^\perp = L(A)^\perp =
    (\obar{L(A)})^\perp$ (lineare Hülle).
  \end{enumerate}
\end{aufg}

$M \subset X$ heißt \emph{total}, wenn die lineare Hülle $L(M)$ dicht in $X$
ist:
\[ \obar{L(M)} = X. \]

Beispiel: $X = \real^n$, $M := \{e_1, \ldots, e_n\}$ eine Basis, dann ist
$L(M) = \real^n$.

\clearpage

\begin{thm}[Approximationssatz] %% O.2
  Sei $X$ ein Hilbertraum, $A$ eine abgeschlossene konvexe Teilmenge von $X$.
  Dann gibt es zu jedem $x \in X$ genau eine beste Approximation in $A$, das
  heißt es gibt genau ein $y \in A$ mit
  \[ \| x - y \| = d(x,A) := \inf \{ \| x - z \| : z \in A \}. \]
\end{thm}

\begin{proof}
  Nach der Definition von $d(A,x)$ gibt es eine Folge $\{y_n\}$ aus $A$ mit
  \[ \| x - y_n \| \xrightarrow{n \to \infty} d(x,A). \]
  Wir zeigen: $\{y_n\}$ ist eine Cauchy-Folge.

  Parallelogrammidentität $x \to x - y_n$ und $y \to x - y_n$.
  \begin{align*}
    \| y_n - y_m \|^2
    &= 2 \big( \|x-y_n\|^2 + \|x-y_m\|^2 \big) - \| 2x - (y_n + y_m) \|^2 \\
    &= 2 \big( \|x-y_n\|^2 + \|x-y_m\|^2 \big)
      - 4 \left\| x - \frac{y_n+y_m}{2} \right\|^2 \\
    &= 2 \big( \|x-y_n\|^2 + \|x-y_m\|^2 \big)
      - 4 d(x,A)^2 \xrightarrow{n,m \to \infty} 0. \\
  \end{align*}
  Es gilt $\frac{y_n+y_m}{2} \in A$ wegen der Konvexität von $A$.

  Also ist $\{y_n\}$ eine Cauchy-Folge und damit existiert ein $y \in A$ mit
  $y_n \to y$ und
  \[ \| x - y \| = \lim_{n \to \infty} \| x - y_n \| = d(x,A). \]
  Damit ist die Existenz bewiesen.

  Es fehlt noch die Eindeutigkeit: Angenommen, für $y_1$ und $y_2$ gilt
  \[ \| x - y_1 \| = \| x - y_2 \| = d(x,A), \qquad y_1 \ne y_2. \]

  Oder: Der erste Teil des Beweises zeigt: $y_1, y_2, y_1, y_2, \ldots$ ist eine
  Cauchy-Folge $\Rightarrow$ $y_1 = y_2$.
\end{proof}

\begin{thm}[Projektionssatz] % O.3
  Sei $X$ ein Hilbertraum, $M$ ein abgeschlossener Teilraum. Dann gilt:
  \begin{enumerate}[a)]
  \item Jedes $x \in X$ lässt sich eindeutig in der Form $x = y + z$ schreiben
    mit $y \in M$ und $z \in M^\perp$. $y$ heißt die \emph{orthogonale
      Projektion} von $x$ auf $M$.
  \item $(M^\perp)^\perp = M$.
  \end{enumerate}
\end{thm}

\clearpage

\begin{aufg} % O.4
  Sei $X$ ein Hilbertraum und $A \subset X$.
  \begin{enumerate}[a)]
  \item $(A^\perp)^\perp = \obar{L(A)}$.
  \item $A^\perp = \{0\}$ $\Leftrightarrow$ $\obar{L(A)} = X$.
  \end{enumerate}
\end{aufg}

\begin{proof}[Beweis zu Satz O.3]
  Existenz: Wir wählen für $y$ die beste Approximation von $x$ in $M$. Definiere
  $z := x - y$.

  Noch zu zeigen: $z \in M^\perp$, das heißt $(z,w) = 0$ für alle $w \in M$.
  O.B.d.A. $w \ne 0$. Es gilt $y + aw \in M$ für alle $a \in K$ ($=\complex$
  oder $\real$).
  \begin{align*}
    d(x,M)^2
    &\le \| x - (y+aw) \|^2 = \| z - aw \|^2 \\
    &= \underbrace{\| z \|^2}_{=d(x,M)^2} - 2 \Re( a(z,w) + |a|^2 \|w\|^2 )
  \end{align*}
  für alle $a \in K$. Mit $a = \| w \|^{-2} (w,z)$ folgt
  \[ 0 \le -2 \frac{|(z,w)|^2}{\|w\|^2} + \frac{|(z,w)|^2}{\|w\|^2}
    = - \frac{|(z,w)|^2}{\|w\|^2}. \]
  Also muss $(z,w) = 0$ sein.

  Eindeutigkeit: Gilt auch $x = y' + z'$ mit $y' \in M$, $z' \in M^\perp$, so
  gilt $y - y' \in M$ und $z - z' \in M^\perp$. Wegen
  \[ y + z = y' + z' \]
  gilt
  \[ y - y' = z' - z \in M \cap M^\perp = \{ 0 \}. \]
  Also gilt $y = y'$ und $z = z'$.
\end{proof}

\begin{denos*}
  Seien $M_1$, $M_2$ Teilräume von $X$ mit $M_1 \cap M_2 = \{ 0 \}$.
  \[ M_1 + M_2 = \{ y_1 + y_2 : y_1 \in M_1, y_2 \in M_2 \} \]
  ist die sogenannte \emph{direkte Summe} $M_1 + M_2$ (das heißt, jedes Element
  aus $M_1 + M_2$ hat \emph{genau eine} Darstellung der Form $y_1 + y_2$, $y_j
  \in M_j$).
  Sind $M_1$ und $M_2$ orthogonal, so gilt $M_1 \cap M_2 = \{0\}$. In diesem
  Fall schreibt man $M_1 \oplus M_2$, die \emph{orthogonale Summe}.

  Man schreibt auch $M_1 = M \ominus M_2$, das heißt $M_1$ ist das
  \emph{orthogonale Komplement} von $M_2$ bezüglich $M$.

  Spezialfall: $Y^\perp = X \ominus Y$.
\end{denos*}

\clearpage

\begin{thm}
  Sei $X$ ein Hilbertraum.
  \begin{enumerate}[a)]
  \item Sind $M_1$ und $M_2$ orthogonale Teilräume, so ist $M_1 \oplus M_2$
    genau dann abgeschlossen, wenn $M_1$ und $M_2$ abgeschlossen sind.
  \item Sind $M_1$ und $M$ abgeschlossene Teilräume mit $M_1 \subset M$, so
    existiert ein abgeschlossener Teilraum $M_2 \subset M$ mit $M = M_1 \oplus
    M_2$. 
  \end{enumerate}
\end{thm}

\begin{proof}
  a) folgt aus der Tatsache, dass $\{ y_{1,n} + y_{2,n} \}_{n=1}^\infty$ mit
  $y_{j,n} \in M_J$ genau dann eine Cauchy-Folge ist, wenn
  $\{y_{j,n}\}_{n=1}^\infty$ Cauchy-Folgen sind.

  b) O.B.d.A. $M = X$. $M_2 = M_1^\perp$ hat die gewünschte Eigenschaft.
\end{proof}

\begin{rmrk*}
  Die Orthogonalität in a) ist notwendig. Sei $H = \ell_2$, das heißt
  \[ x \in \ell_2 : \{ x = (x_0, x_1, \ldots ) \} \]
  mit $\sum_j |x_j|^2 < \infty$.

  Seien $X_1 := \{$ alle Folgen in $\ell_2$ mit $ x_{2n} = 0\}$ und $X_2 := \{$
  alle Folgen in $\ell_2$ mit $x_{2n+1} = n x_{2n} \}$. Dann gilt
  \begin{enumerate}
  \item $X_1$ und $X_2$ sind abgeschlossene, lineare Teilräume.
  \item $X_1 \cap X_2 = \{ 0 \}$.
  \item $X_1 \dot{+} X_2$ ist dicht in $\ell_2$ (Hinweis: Enthält alle Folgen
    mit endlichem Träger)
  \item $X_1 \dot{+} X_2 \ne \ell_2$ (Hinweis: $x_n := \rez{n+1} \in \ell_2$,
    aber $\notin X_1 \dot{+} X_2$)
  \end{enumerate}
  Aufgabe.
\end{rmrk*}

\begin{exmp}
  Ist $X = L^2(a,b)$, $a < c < b$, so gilt
  \[ L^2(a,b) = L^2(a,c) \oplus L^2(c,b), \]
  da
  \[ f = f \cdot \ind_{[a,c]} + f \cdot \ind_{[c,b]} \]
  und
  \[ \int_a^b f \cdot \ind_{[a,c]} \cdot f \cdot \ind_{[c,b]} \diffop \lambda =
    0. \]
\end{exmp}

\begin{exmp}
  In $X = L^2(-a,a)$ seien
  \[ L_g^2(-a,a) := \{ f \in L^2(-a,a) :
    f(x) = f(-x) \text{ fast überall} \} \]
  und
  \[ L_g^2(-a,a) = \{ f \in L^2(-a,a) :
    f(x) = -f(-x) \text{ fast überall} \}. \]
  Dann gilt
  \[ L^2(-a,a) = L^2_g(-a,a) \oplus L^2_u(-a,a), \]
  denn
  \[ f(x) = f_g + f_u = \frac{f(x) + f(-x)}{2} + \frac{f(x) - f(-x)}{2} \]
  und
  \[ \int_{-a}^a f_g \cdot \obar{f_u} \diffop \lambda = 0, \]
  weil $f_g \cdot \obar{f_u}$ ungerade ist.
\end{exmp}

Eine Familie $M = \{ e_\alpha : \alpha \in A \}$ in einem Hilbertraum heißt ein
\emph{Orthonormalsystem}, wenn $e_\alpha \perp e_\beta$ für $\alpha \ne \beta$
und $\| e_\alpha \| = 1$. Ein totales Orthonormalsystem heißt
\emph{Orthonormalbasis}.

\begin{thm}
  \begin{enumerate}[a)]
  \item Jedes Orthonormalsystem ist linear unabhängig.
  \item Jede Orthonormalbasis ist ein maximales Orthonormalsystem.
  \item Im Hilbertraum ist jedes maximale Orthonormalsystem eine
    Orthonormalbasis.
  \end{enumerate}
\end{thm}

\begin{proof}
  Aufgabe.
\end{proof}

\begin{exmp}
  In $\ell^2(\nat)$, $\ell^2(\integer)$. Orthonormalbasis:
  \[ \{ e_n \}_{n=1}^\infty; \qquad e_n(n) = 1; \quad e_n(m) = 0, \quad n \ne
    m. \]
  Zu zeigen: $(x,e_n) = 0$ für alle $n$ $\Rightarrow$ $x = 0$.
\end{exmp}

\begin{aufg}
  Sei $X = L^2(0,1)$.
  \begin{enumerate}[a)]
  \item $M = \{e_n : n \in \integer \}$, wobei $e_n(x) = e^{2 \pi i n x}$, $x
    \in [0,1]$.

    Zu zeigen: $M$ ist Orthonormalbasis.
  \item $c_n(x) = \sqrt{2} \cos (2 \pi n x)$, $s_n(x) = \sqrt{2} \cos (2 \pi n
    x)$.

    Zu zeigen: $\{ c_n : n \in \nat_0 \} \cup \{ s_n : n \in \nat \}$ ist
    Orthonormalbasis.
  \end{enumerate}
\end{aufg}

Hinweis: Satz von Weierstraß für trig. Polynomräume.

\clearpage

\begin{thm} %% O.11
  Sei $X$ ein Prähilbertraum\footnotemark, $\{ e_{\alpha} : \alpha \in A \}$ ein
  Orthonormalsystem in $X$.
  \begin{enumerate}[a)]
  \item Ist $\{ \alpha_n \}$ eine Folge paarweise verschiedener Elemente aus $A$
    und $\{c_n\}$ eine Folge aus $\mathbb{K}$, so gilt
    \begin{enumerate}[(i)]
    \item Ist $\sum_n c_n e_{\alpha_n}$ konvergent bzw.
      \[ \left\{ \sum_{n=1}^m c_n e_{\alpha_n} \right\} \]
      eine Cauchy-Folge, so ist $\{ c_n \} \in \ell_2$.
    \item isr $\{c_n\} \in \ell_2$, so ist
      \[ \left\{ \sum_{n=1}^m c_n e_{\alpha_n} \right\} \]
      eine Cauchy-Folge.
    \end{enumerate}
  \item Ist $y = \sum_n c_n e_{\alpha_n}$, so gilt $c_n = (e_{\alpha_n}, y)$ und
    \[ \| y \|^2 = \sum_n |c_n|^2, \qquad
      (y,x) = \sum_n (y, e_{\alpha_n}) \cdot (e_{\alpha_n}, x)\]
    für alle $x \in X$.
  \end{enumerate}
\end{thm}
\footnotetext{Ein Raum mit positiv definitem Skalarprodukt, der möglicherweise
  nicht vollständig ist.}

\begin{thm}[Entwicklungspunktsatz] %% O.12
  \begin{enumerate}[a)]
  \item Ist $\{ e_\alpha : \alpha \in A\}$ ein Orthonormalsystem im
    Prähilbertraum $X$, so gilt für alle $x \in X$ die \emph{Bessel'sche
      Ungleichung}
    \[ \sum_{\alpha \in A} |(e_{\alpha},x)|^2 \le \| x \|^2, \]
    wobei in der Summe über höchstens abzählbar viele Terme $\ne 0$ sind.
  \item Ein Orthonormalsystem $\{ e_\alpha : \alpha \in A \}$ in $X$ ist genau
    dann eine Orthonormalbasis, wenn die \emph{Parseval'sche Gleichung}
    \[ \sum_{\alpha \in A} |(e_\alpha,x)|^2 = \| x \|^2 \]
    gilt. Es ist dann
    \[ x = \sum_{\alpha \in A} (e_\alpha, x) \cdot e_\alpha \]
    (Summe abzählbar).
  \item ist $\{ e_\alpha : \alpha \in A \}$ ein Orthonormalsystem im Hilbertraum
    $X$, so ist
    \[ \sum_{\alpha \in A} (e_\alpha, x) \cdot e_\alpha \]
    die orthogonale Projektion von $x$ auf den abgeschlossenen Teilraum
    $\obar{L \{ e_\alpha : \alpha \in A\}}$.
  \end{enumerate}
\end{thm}

\begin{proof}
Zu a): Ist $y = \sum_{n=1}^m c_n \cdot e_{\alpha_n} \in L\{ e_\alpha : \alpha
\in A \}$, so gilt
\begin{align*}
  \| x - y \|^2
  &= \| x \|^2 - 2 \Re \sum_{n=1}^m c_n (x, e_{\alpha_n})
    + \sum_{n=1}^m | c_n |^2 \\
  &= \| x \|^2 - \sum_{n=1}^n |(e_{\alpha_n}, x)^2
    + \sum_{n=1}^m |c_n - (e_{\alpha_n}, x) |^2 \tag{$*$}\\
  &\ge \| x \|^2 - \sum_{n=1}^m | (e_\alpha, x) |^2.
\end{align*}
$\| x - y \|$ also minimal genau dann, wenn $c_n = (e_{\alpha_n}, x)$ (bei
festen $\alpha_1, \ldots, \alpha_n$). Dann gilt
\[ \| x - y \|^2 = \| x \|^2 - \sum_{n=1}^m | (e_{\alpha_n}, x) |^2,\]
also folgt
\[ \sum_{n=1}^m |(e_{\alpha_n}, x)|^2 \le \| x \|^2. \]
Da dies für beliebige endliche Familien $\{\alpha_1, \ldots, \alpha_n\}$ aus $A$
gilt, folgt die Behauptung.

Zu b): Ist $\{ e_\alpha : \alpha \in A \}$ eine Orthonormalbasis, also
$\obar{L\{e_\alpha : \alpha \in A\}} = X$, dann existieren für alle $\eps > 0$
eine Familie $\{ \alpha_1, \ldots, \alpha_m \} \subset A$ und $c_1, \ldots, c_m
\in \mathbb{K}$ mit
\[ \| x - y \| < \eps \]
für $y = \sum_{n=1}^m c_n e_n$ und damit folgt mit $(*)$
\[ \| x \| - \sum_{n=1}^m |(e_{\alpha_n},x)|^2 \le \| x - y \|^2 \le \eps^2, \]
also
\[ \sum_{n=1}^m |(e_{\alpha_n},x)^2| \ge \| x \|^2 - \eps^2. \]
Mit der Bessel'schen Ungleichung folgt damit die Parseval'sche Gleichung.

Ist $\{ \alpha_1, \alpha_2, \ldots \}$ die Menge der $\alpha \in A$ mit
$(e_\alpha, x) \ne 0$ und
\[ y_m := \sum_{n=1}^m (e_{\alpha_n}, x) e_{\alpha_n}, \]
so folgt mit $(*)$ und der Parseval'schen Gleichung
\[ \| x - y_m \|^2 = \|x\|^2 - \sum_{n=1}^m |(e_{\alpha_m},x)|^2 \xrightarrow{m
    \to \infty} 0, \]
das heißt
\[ x = \lim_{m \to \infty} y_m = \sum_{n=1}^\infty (e_{\alpha_n},x)
  e_{\alpha_n} = \sum_{\alpha \in A} (e_{\alpha}, x) e_\alpha. \]

Zu c): Sei
\[ y := \sum_{\alpha \in A} (e_\alpha, x) e_\alpha = \sum_{n=1}^\infty
  (e_{\alpha_n}, x) e_{\alpha_n}. \]
Mit $(*)$ folgt für beliebige $c_1, \ldots, c_m \in \mathbb{K}$
\begin{align*}
  \| x - y \|^2
  &= \| x \|^2 - \sum_{\alpha \in A} |(e_\alpha,x)|^2 \\
  &\le \| x \|^2 - \sum_{n=1}^m |(e_{\alpha_n},x)|^2 \\
  &\le \left\| x - \sum_{n=1}^m c_n e_{\alpha_n} \right\|^2.
\end{align*}
Es gilt $\sum_{n=1}^m c_n e_{\alpha_n} \in \obar{L \{ e_\alpha : \alpha =
  \alpha_1, \ldots, \alpha_m\}}$. Also
\begin{align*}
  \| x - y \|
  &= \inf \left\{ \| x - z \| : z \in L\{e_\alpha\} \right\} \\
  &= \inf \left\{ \| y - z \| : z \in \obar{L\{e_\alpha\}} \right\}.
\end{align*}
Das heißt $y$ ist die orthogonale Projektion von $x$ auf $\obar{L\{e_\alpha\}}$.
\end{proof}

Einfaches Beispiel: Sei $X = \real^2$, $e_1 = (1,0)$, $e_2 = (0,1)$ ist eine
Orthonormalbasis. Die orthogonale Projektion von $x = (2,2) \in X$ auf $\real
e_2$ ist $(0,2)$.

\begin{thm}[Gram-Schmidt'sche Orthogonalisierung] %% O.13
  Sei $X$ ein Prähilbertraum, $F = \{ x_n : n \in \nat \}$ oder $F = \{ x_n : n
  = 1, \ldots, m \}$.
  Dann existiert ein Orthonormalsystem $M = \{ e_n \}$ mit $L(F) = L(M)$.

  Ist $F$ linear unabhängig, dann kann $M$ so gewählt werden, dass
  \[ L(\{x_1, \ldots, x_n\}) = L( \{e_1, \ldots, e_n\}) \]
  für alle $n$. In diesem Fall lassen sich die $e_n$ in der Form
  \[ e_n = \sum_{j=1}^n c_{n,j} x_j \]
  schreiben. Mit der zusätzlichen Forderung $c_{n,n} > 0$ werden sie eindeutig
  bestimmt. 
\end{thm}

\begin{proof}
  Aufgabe.

  Hinweis: O.B.d.A. $F$ linear unabhängig.
  \[ e_1 := \frac{x_1}{\| x_1 \|}, \qquad
    e_{n+1} := \frac{x_{n+1} + P_n x_{n+1}}{ \| x_{n+1} - P_n x_{n+1} \|}, \]
wobei $P_n x$ die orthogonale Projektion von $x$ auf $L\{ e_1, \ldots e_n\}$
ist. $x_{n+1} - P_n x_{n+1} \ne 0$, weil $F$ linear unabhängig ist.
\end{proof}

\clearpage

\begin{thm}[Existenz von Orthormalbasen] % O.14
  \begin{enumerate}[a)]
  \item Jeder separable Prähilbertraum besitzt eine endliche oder abzählbar
    unendliche Orthonormalbasis.
  \item Jeder Hilbertraum besitzt eine Orthonormalbasis.
  \end{enumerate}
\end{thm}

\begin{proof}
  a): Sei $X$ separabel, dann gibt es eine höchstens abzählbare totale Menge
  $\{x_n\}$. Wir dürfen annehmen, dass $\{x_n\}$ linear unabhängig ist.
  Anwendung von O.13 liefert eine orthonormale Basis der gleichen Mächtigkeit.

  b): Sei $\mM$ die Menge aller Orthonormalsysteme in $X$. $\mM$ ist bezüglich
  der Inklusion ``$\subseteq$'' halbgeordnet.

  \emph{Lemma von Zorn:} Besitzt in einer halbgeordneten Menge jede Teilmenge
  eine obere Schranke, so existiert (mindestens) ein maximales Element. Ist
  $\mN$ eine geordnete Teilmenge von $\mM$, so ist die Vereinigung aller $N \in
  \mN$ eine obere Schranke für alle $M \in \mM$.

  $M$ ist ein Orthonormalsystem: Für $e_1, e_2 \in M$ mit $e_1 \ne e_2$
  existieren $N_1, N_2 \in \mN$ mit $e_j \in N_j$. Wegen $N_1 \subset N_2$ oder
  $N_2 \subset N_1$ gilt $e_1, e_2 \in N_2$ oder $e_1, e_2 \in N_1$. Also ist
  $e_1 \perp e_2$.

  Nach dem Lemma von Zorn existiert ein maximales Element $M_{\max} \in \mM$,
  dieses ist eine orthonormale Basis: Wäre $\obar{L(M_{\max})} \ne X$, so gäbe
  es ein $e \in \obar{L(M_{\max})}^\perp$ mit $\| e \| = 1$, das heißt $M_{\max}
  \cup \{e\}$ wäre ein Orthonormalsystem. Widerspruch zur Maximalität!
\end{proof}

\begin{folg} % O.15
  \begin{enumerate}[a)]
  \item In einem separablen Prähilbertraum kann jedes endliche Orthonormalsystem
    zur einer Orthonormalbasis ergänzt werden.
  \item Im Hilbertraum kann jedes Orthonormalsystem zu einer Orthonormalbasis
    ergänzt werden.
  \end{enumerate}
\end{folg}

\begin{proof}
  a): Zu dem endlichen Orthonormalsystem nimmt man eine abzählbare, dichte
  Teilmenge dazu. Man streicht eventuelle linear abhängige Elemente heraus und
  wendet das Gram-Schmidt'sche Verfahren an.

  b): Sei $M_1$ das gegebene Orthonormalsystem und $M_2$ eine Orthonormalbasis
  von $M_1^\perp$. Dann ist $M_1 \cup M_2$ eine Orthonormalbasis von $X$.
\end{proof}

\begin{thm} % O.16
  Alle Orthonormalbasen in einem Prähilbertraum haben die gleiche Mächtigkeit.
\end{thm}

\begin{proof}
  Aufgabe.
\end{proof}

Dieser Satz erlaubt es, die \emph{Hilbertraumdimension} zu definieren: Die
Mächtigkeit einer Orthonormalbasis.

\begin{rmrk*}
  Es gibt Hilberträume mit beliebiger Dimension. Ist $A$ eine Menge mit
  vorgegebener Mächtigkeit, so hat $\ell^2(A)$ (alle quadratisch summierbaren
  Funktionen auf dieser Menge) die Dimension $|A|$.

  Orthonormalbasis: $\{ \ind_{\{a\}} : a \in A \}$.

  $\ell^2(A) = \{ f : A \to \complex \}$ mit $\sum_{a \in A} |f(a)|^2 < \infty$.
  Skalarprodukt:
  \[ (f,g) = \sum_{a \in A} f(a) \cdot \obar{g(a)}. \]
  Es gilt $(\ind_{\{a\}}, \ind_{\{b\}}) = 0$ für $a \ne b$ und $\| \ind_{\{a\}} \|
  = 1$.
\end{rmrk*}

\subsubsection*{Beispiel für die Gram-Schmit-Orthogonalisierung}
Sei $X$ der lineare Raum aller stetigen Funktionen auf $[0,1]$. Das
Skalarprodukt sei
\[ (f,g) = \int_0^1 f(x) \obar{g(x)} \diffop x \]
oder allgemeiner
\[ (f,g) = \int_0^1 f(x) \obar{g(x)} \cdot p(x) \diffop x, \]
wobei $p(x) \ge 0$ stetig.

Die Polynome $1, x, x^2, x^3, \ldots$ sind linear unabhängig.

Aus $c_0 + c_1 x + \ldots + c_n x^n = 0$ für alle $x \in [0,1]$ folgt, dass $c_j
= 0$ für alle $j$. Anwendung von Gram-Schmidt liefert orthogonale Polynome
bezüglich $p$.

\setcounter{chapter}{1}
\chapter{Semidefinite Matrizen}
\renewcommand{\thethm}{B.\arabic{thm}}

\begin{defn} %% B.1
  $X \ne \emptyset$ sei eine beliebige Menge und $\Phi: X \times X \to \complex$
  ein Kern. Wir definieren $\Phi^*$ durch
  \[ \Phi(x,y) = \obar{\Phi(y,x)} \]
  für alle $x, y \in X$.

  $\Phi$ heißt \emph{hermitesch}, wenn $\Phi = \Phi^*$.

  $\Phi$ heißt \emph{positiv definit}, wenn
  \[ \sum_{j,k = 1}^n \Phi(x_j, x_k) c_j \obar{c_k} > 0 \]
  für alle $n \in \nat$, $x_1, \ldots, x_n \in X$ mit $x_j \ne x_k$ und $c_1,
  \ldots, c_n \in \complex$ nicht alle $=0$.

  $\Phi$ heißt \emph{positiv semidefinit}, wenn
  \[ \sum_{j,k = 1}^n \Phi(x_j, x_k) c_j \obar{c_k} \ge 0. \]
  In diesem Fall gilt $ge$ auch ohne die Bedingungen $x_j \ne x_k$ und $c_j \ne
  0$.

  Eine \emph{quadratische Matrix} der Ordnung $n$ ist ein Kern auf $\{1, \ldots,
  n\} \times \{1, \ldots, n\}$.

  $\diag(\lambda_1, \ldots, \lambda_n)$ ist eine Diagonalmatrix $(d_{ij})$ mit
  \[ d_{ii} = \lambda_i, \qquad d_{ij} = 0, \quad i \ne j. \]
\end{defn}

\begin{thm} %% B.2
  Jeder positiv semidefinite Kern $\phi$ auf $X$ ist hermitesch und
  \[ \phi(x, x) \ge 0, \qquad x \in X. \]
\end{thm}

\begin{exmp*}
  $A = (a_{jk}) = \pmat{ 1 & 1 \\ -1 & 1}$ ist nicht hermitesch, aber
  \[ \sum_{j,k=1}^2 a_{jk} r_j r_k = r_1^2 + r_2^2 > 0 \]
  wenn nicht beide $= 0$ sind.
\end{exmp*}

\clearpage

\begin{thm} %% B.3
  Ein reeller Kern $\phi$ auf $X$ ist genau dann positiv semidefinit, wenn
  $\phi$ symmterisch ist, das heißt $\phi(x,y) = \phi(y,x)$ und
  \[ \sum_{j,k = 1}^n \phi(x_j, x_k) r_j r_k \ge 0 \]
  für alle $n \in \nat$, $x_j \in X$, $r_j \in \real$.
\end{thm}

\begin{defn}
  Sei $A = (a_{jk})$ eine hermitesche Matrix der Ordnung $n$. Die \emph{Anzahl
    der negativen Quadrate} ist die Anzahl der negativen Eigenwerte, gezählt mit
  Vielfachheit.

  Sei $E \subseteq \complex^n$ ein linearer Raum mit
  \[ \sum_{j,k=1}^n a_{jk} x_j \obar{x_k} = (Ax, x) < 0 \]
  für alle $x \in E \setminus \{ 0 \}$. Dann heißt $E$ \emph{negativer
    Unterraum}.

  Analog für $\le 0$, $\ge 0$, $> 0$.
\end{defn}

\begin{rmrk}
  Bekannt ist bereits, dass eine hermitesche $n \times n$-Matrix $A$ $n$ reelle
  Eigenwerte $r_1, \ldots, r_n$ besitzt. Weiterhin existiert eine orthonormale
  Basis von $complex$, die aus Eigenvektoren von $A$ besteht. Eigenvektoren mit
  negativem Eigenwert erzeugen negative Unterräume.
\end{rmrk}

\begin{rmrk*}
  $\det(A) = r_1 \cdot \ldots \cdot r_n$.
\end{rmrk*}

\begin{thm}
  Ist $A$ eine hermitesche Matrix, so ist die Anzahl der negativen Quadrate
  gleich der maximalen Dimension von negativen Unterräumen von $A$.
\end{thm}

\begin{thm}
  Ist $A$ eine nicht singuläre $n \times n$ hermitesche Matrix, so ist die
  Anzahl der negativen Quadrate gleich der Anzahl der Vorzeichenwechsel der
  Folge
  \[ 1, \det A_1, \ldots, \det A_n, \]
  wobei $A_k = (a_{ij})_{i,j = 1}^k$.
\end{thm}

\begin{folg}
  Eine hermitesche Matrix $A$ der Ordnung $n$ ist genau dann positiv definit,
  wenn $\det(A_k) = 0$ für $k = 1, \ldots, n$.
\end{folg}

\begin{exmp*}
  $A = \pmat{0 & 0 \\ 0 & -1}$ ist hermitesch, $\det( A_k ) \equiv 0$, aber nicht
  positiv semidefinit.
\end{exmp*}

\begin{thm} %% B.9
  Ein hermitescher Kern $\Phi : X \times X \to \complex$ ist genau dann positiv
  semidefinit, wenn
  \[ \det (( \Phi(x_j, x_k))_{j,k \le n}) \ge 0 \]
  für alle $n$ und für alle $x_1, \ldots, x$
\end{thm}

\begin{thm} %% B.10
  Eine $n \times n$-Matrix $A$ ist genau dann positiv semidefinit, wenn eine $n
  \times n$ hermitesche Matrix $B$ existiert, so dass $A = B^2$. Ist $A$ reell,
  so kann $B$ auch reell gewählt werden.
\end{thm}

\begin{proof}
  Wenn $A = B^2$, dann ist
  \[ (Ax,x) = (BBx,x) = (Bx,B^* x) = (Bx, Bx) \ge 0. \]

  Sei $A$ positiv semidefinit und sei $\{e_1, \ldots, e_n\}$ eine orthonormale
  Basis von $\complex^n$ (bzw. $\real^n$), wobei $e_j$ ein Eigenvektor von $A$
  zum Eigenwert $r_j \ge 0$\footnote{%
    Wegen $(A e_j, e_j) = (r_j e_j, e_j) = r_j$ muss $r_j \ge 0$ gelten, wenn
    $A$ positiv semidefinit ist.
  } ist. Definiere
  \[ Q := [ e_1, \ldots, e_n] \]
  mit den $e_j$ als Spaltenvektoren. Das ist eine $n \times n$-Matrix und es
  gilt
  \[ Q^* Q = I_n, \]
  wobei $I_n$ die $n \times n$-Einheitsmatrix ist. Es folgt
  \[ Q \diag (r_1, \ldots, r_n) Q^* e_j = r_j e_j = A e_j \]
  für alle $j$. Also ist
  \[ A = Q \diag( r_1, \ldots, r_n) Q^*, \]
  damit ist
  \[ B := Q \diag(\sqrt{r_1}, \ldots, \sqrt{r_n}) Q^* \]
  hermitesch und es gilt $BB = A$.
\end{proof}

\begin{thm} %% B.11
  Sind $A = (a_{ij})$ und $B = (b_{ij})$ positiv semidefinite $n \times
  n$-Matrizen, so ist es auch das \emph{Hadamard-Produkt}
  \[ C = (a_{ij} \cdot b_{ij})_{i,j = 1}^n. \]
\end{thm}

\setcounter{chapter}{6}
\chapter{Charakteristische Funktionen von mehreren Variablen}
\renewcommand{\thethm}{G.\arabic{thm}}

\begin{defn}
  Sei $X$ eine $d$-dimensionale Zufallsvariable und sie $\mu$ die Verteilung von
  $X$ (ein $\pW$-Maß auf $\real^d$). Die \emph{charakteristische Funktion} $f =
  f_X$ von $X$ ist definiert durch
  \[ f(t) := \pE( e^{i(t,X)}) = \int_{\real^d} e^{i(t,x)} \diffop \mu(x),
    \qquad t \in \real^d. \]
\end{defn}

\begin{thm}
  Jede charakteristische Funktion $f$ auf $\real^d$ hat die folgenden
  Eigenschaften:
  \begin{enumerate}[(i)]
  \item $f(0) = 1$,
  \item $|f(t)| \le 1$ für alle $t \in \real^d$,
  \item $f(-t) = \obar{f(t)}$ für alle $t \in \real^d$,
  \item $f$ ist gleichmäßig stetig auf $\real^d$,
  \item $\sum_{i,j=1}^n f(t_i - t_j) c_i \obar{c_j} \ge 0$ für alle $n$, $t_j
    \in \real^d$, $c_j \in \complex$.
  \end{enumerate}
\end{thm}

\begin{proof}
  Aufgabe.
\end{proof}

\begin{rmrk*}
  Falls $X$ eine Dichte $p$ besitzt, dann gilt
  \[ f(t) = \int_{\real^d} e^{i(t,x)} p(x) \diffop x. \]
  Ist $X$ diskret mit $P(X = x_j) = p_j$, $\sum p_j = 1$, dann gilt
  \[ f(t) = \sum_{j=1}^n e^{i(t,x_j)}. \]
\end{rmrk*}

\begin{exmp*}
  \begin{itemize}
  \item $t \to e^{i(t,x)}$ ist eine charakteristische Funktion für alle $x \in
    \real^d$.
  \item $t \to \cos((t,x)) = \rez{2} e^{i(t,x)} + \rez{2} e^{i(t,-x)}$,
    das heißt $p_1 = p_2 = \rez{2}$, $x_1 = x$, $x_2 = -x$.
  \end{itemize}
\end{exmp*}

Sei $X$ gleichmäßig verteilt auf
\[ K = [a_1, b_1] \times \ldots \times [a_d, b_d] \]
mit Dichte
\[ p = \rez{\lambda(K)} \ind_K, \qquad \lambda(K) = \prod_{j=1}^d (b_j - a_j). \]
dann gilt
\begin{align*}
  f(t)
  &= \int_{\real^d} e^{i(t,x)} p(x) \diffop x \\
  &= \int_K \rez{\lambda(K)} e^{i(t,x)} \diffop x \\
  &= \rez{\lambda(K)} \int_K e^{i(t,x)} \diffop x \\
  &= \rez{\lambda(K)} \int_K \prod_{j=1}^d e^{i t_j x_j} \diffop x \\
  \intertext{mit dem Satz von Fubini:}
  &= \rez{\lambda(K)} \int_{a_d}^{b_d} \cdots \int_{a_1}^{b_1}
    \prod_{j=1}^d e^{i t_j x_j} \diffop x_1 \diffop x_2 \cdots \diffop x_d \\
  &= \rez{\lambda(K)} \prod_{j=1}^d \int_{a_j}^{b_j} e^{i t_j x_j} \diffop x_j \\
  &= \rez{\lambda(K)} \prod_{j=1}^d
    \left. \frac{e^{i t_j x_j}}{i t_j} \right|_{x_j = a_j}^{b_j} \\
  &= \prod_{j=1}^d \frac{e^{i t_j b_j} - e^{i t_j a_j}}{i t_j (b_j - a_j)}
\end{align*}
für $t_j \ne 0$. Für $t = 0$ vereinfacht sich die Berechnung und es gilt
\[ f(0) = \rez{\lambda(K)} \int_K e^{i(0,x)} \diffop \lambda
  = \rez{\lambda(K)} \int_K 1 \diffop \lambda = 1. \]

Die Beweise der folgenden Sätze sind Hausaufgaben.

\begin{thm}
  Sind $X$ und $Y$ unabhängige, $d$-dimensionale Zufallsvariablen, so gilt
  \[ f_{X+Y} = f_X \cdot f_Y. \]
  Weiterhin gilt:
  \[ f_{AX + b}(t) = e^{i(b,t)} \cdot f_X(A^\top t), \qquad t \in \real^n \]
  für eine beliebige lineare Abbildung $A : \real^d \to \real^n$ und $b \in
  \real^n$.
\end{thm}

\begin{folg}
  Sind $X$ und $Y$ i.i.d. mit charakteristischer Funktion $f$, dann ist
  $\obar{f}$ die charakteristische Funktion von $-X$ und $|f|^2$ ist die
  charakteristische Funktion von $X - Y$.
\end{folg}

\begin{folg}
  Das Produkt von zwei charakteristischen Funktionen von $d$-Variablen ist
  wieder eine charakteristische Funktion.
\end{folg}

\begin{rmrk*}
  $f(t) = \cos(t)$ ist eine charakteristische Funktion, $|f|$ aber nicht.
\end{rmrk*}

\begin{thm}
  Seien $X_1$ und $X_2$ unabhängige, $d_1$- bzw. $d_2$-dimensionale
  Zufallsvariablen und $d := d_1 + d_2$. Die charakteristische Funktion von $X
  := (X_1, X_2)$ ist
  \[ f_X(t) = f_{X-1}(t_1) \cdot f_{X_2}(t_2), \]
  wobei $t_1 \in \real^{d_1}$, $t_2 \in \real^{d_2}$, $t := (t_1, t_2) \in
  \real^d$.
\end{thm}

\begin{folg} %% G.7
  Seien $f_1$ und $f_2$ charakteristische Funktionen von $d_1$- bzw.
  $d_2$-Variablen. Dann ist
  \[ f(t) = f_1 (t_1) \cdot f_2(t_2) \]
  eine charakteristische Funktion einer $d_1 + d_2$-Variablen.
\end{folg}

\begin{thm}
  Sei $\alpha \in \nat_0^d$ so, dass das Moment\footnotemark $M_\alpha$
  existiert. Dann existiert die partielle Ableitung $D^\alpha f$ und
  \begin{enumerate}[(i)]
  \item $D^\alpha f(t) = i^{|\alpha|} {\displaystyle \int_{\real^d}} x^\alpha
    e^{i(t,x)} \diffop \mu(x)$ für alle $t \in \real^d$,
  \item $D^\alpha f(0) = i^{|\alpha|} \cdot M_\alpha$,
  \item $D^\alpha f$ ist beschränkt und gleichmäßig stetig auf $\real^d$.
  \end{enumerate}
\end{thm}
\footnotetext{Siehe Anhang A,
  \[ M_\alpha(X) = \int_{\real^d} x^d \diffop \mu(x) \]
  mit
  \[ x^\alpha = x_1^{\alpha_1} \cdot x_2^{\alpha_2} \cdot \ldots \cdot
    x_d^{\alpha_d}. \]
}

Hinweis: (I.1).

\begin{folg}
  Sei $X = (X_1, \ldots, X_d)$ ein $d$-dimensionaler Zufallsvektor mit
  charakteristischer Funktion $f$. Ist $X_j$ quadratisch integrierbar für alle
  $j$, so ist die Kovarianzmatrix $\cov(X) = (c_{jk})_{j,k=1}^d$ gegeben durch
  \[ c_{jk} = - \pdiff{f}{t_j \partial t_k}(0) + \pdiff{}{t_j}f(0) +
    \pdiff{}{t_k} f(0). \]
\end{folg}

\begin{exmp*}
  \begin{enumerate}[a)]
    \item Exponentialverteilung, Parameter $\alpha > 0$, Dichte:
    \[ p(x) = \alpha e^{-\alpha x}, \qquad x \ge 0.\]
    \begin{align*}
      f(t)
      &= \int_0^\infty e^{itx} \cdot \alpha \cdot e^{-\alpha x} \diffop x \\
      &= \alpha \int_0^\infty e^{(it - \alpha) x}  \diffop x \\
      &= \alpha
        \left. \frac{e^{(it-\alpha)x}}{it - \alpha} \right|_{x=0}^\infty \\
      &= \frac{\alpha}{\alpha - it},
    \end{align*}
    für alle $t \in \real$.
  \item Laplace-Verteilung, Parameter $\beta > 0$, Dichte:
    \[ p(x) = \rez{2\beta} e^{-|x|/\beta}, \qquad x \in \real. \]
    Wie bei a) erhalten wir
    \[ f(t) = \rez{2 \beta} \int_{-\infty}^\infty e^{(- \rez{\beta} + it)x}
      \diffop x =
      \rez{1 + \beta^2 t^2}\]
    für alle $t \in \real$. $f$ ist reellwertig.
  \item Binomialverteilung:
    \[ \mu = \sum_{k=0}^n \binom{n}{k} p^k (1-p)^{n-k} \delta_k, \]
    es gilt
    \[ f(t) = \sum_{k=0}^n \binom{n}{k} p^k (1-p)^{n-k} e^{itk}. \]

    Sei $X$ binomialverteilt mit Parametern $n$ und $p$. Dann existieren
    unabhängige Zufallsgrößen $X_1, \ldots, X_n$ mit $\pP(X_j = 0) = 1-p$,
    $\pP(X_j = 1) = p$ und
    \[ X = X_1 + \ldots + X_n. \]
    Wegen $f_X = f_{X_1} \cdot \ldots \cdot f_{X_n} = f_{X_1}^n$ gilt
    \[ f_X(t) = ((1-p) + p \cdot e^{it})^n. \]
  \item Poisson-Verteilung:
    \[ e^{-\lambda} \sum_{k=0}^\infty \frac{\lambda^k}{k!} \delta_k. \]
    Es gilt
    \[ f(t) = e^{-\lambda} \sum_{k=0}^\infty \frac{\lambda^k}{k!} e^{itk}
      = e^{-\lambda} \sum_{k=0}^\infty \frac{(\lambda e^{it})^k}{k!}
      = e^{\lambda(e^{it} - 1)} \]
  \end{enumerate}
\end{exmp*}

\end{document}