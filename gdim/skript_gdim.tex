\documentclass[
 a4paper,
 12pt,
 parskip=half
 ]{scrartcl}

\usepackage{../.tex/settings}

\usepackage{../.tex/mathpkgs}
\usepackage{../.tex/mathcmds}

\theoremstyle{plain}
\newtheorem{thm}{Satz}[section] % reset theorem numbering for each chapter
\newtheorem*{thm*}{Satz}

\theoremstyle{definition}
\newtheorem{defn}[thm]{Definition} % definition numbers are dependent on theorem numbers
\newtheorem{exmp}{Beispiel}[section] % same for example numbers
\newtheorem{rmrk}{Bemerkung}[section] % same for example numbers

\numberwithin{equation}{section}

\renewcommand{\thesection}{\Roman{section}} 
\renewcommand{\thesubsection}{\arabic{subsection}}

%opening
\title{Vorlesung\\Analysis}
\subtitle{Wintersemester 2016/2017}
\author{Vorlesung: Prof. Dr. Peter Hornung\\Mitschrift: Jonas Hippold}

\begin{document}

\maketitle

\tableofcontents

\clearpage

\setcounter{section}{10}
\section{Abriss der Maßtheorie}
\subsection*{Motivation und Idee des Lebesque-Integrals}
\addcontentsline{toc}{subsection}{Motivation und Idee des Lebesque-Integrals}
\subsubsection*{Erinnerung an das Riemann-Integral auf $[a,b] \subset \real$}
\addcontentsline{toc}{subsubsection}{Erinnerung an das Riemann-Integral auf \texorpdfstring{$[a,b] \subset \real$}{[a,b] in R}}
Bausteine sind die Treppenfunktionen $\varphi \in \stair$ mit zugehöriger Unterteilung $a = x_0 < x_1 < \ldots < x_n = b$ und $\varphi = c_k$ auf $(x_{k_1},x_k)$:
\[ \int_a^b \varphi := \sum_{k=1}^n c_k \cdot (x_k - x_{k-1}). \]

Das Integral für beliebige (Riemann-integrierbare) $f:[a,b]\to\real$: Approximation von $f$ durch folgendermaßen gewählte Treppenfunktionen $\varphi \in \stair$:
\begin{enumerate}
 \item Wähle (immer feinere) Unterteilung $a = x_0 < x_1 < \ldots < x_n = b$ des \emph{Definitionsbereichs} $[a,b]$.
 \item Wähle $c_1, \ldots, c_n$ in Abhängigkeit von $f$.
\end{enumerate}

\subsubsection*{Lebesque-Integral auf $[a,b] \subset \real$}
\addcontentsline{toc}{subsubsection}{Erinnerung an das Lebesque-Integral auf \texorpdfstring{$[a,b] \subset \real$}{[a,b] in R}}
Bausteine sind \emph{einfache Funktionen}.

Eine Funktion $\varphi: [a,b] \to \real$ heißt \emph{einfach} $:\Leftrightarrow$ $\varphi$ nimmt nur endlich viele verschiedene Werte an:
\[ \sharp \varphi([a,b]) < \infty. \]
Unterschied zu Treppenfunktionen: $\varphi^{-1}(c_k)$ ist im Allgemeinen kein Intervall.

Lebesque-Integral für einfache Funktionen $\varphi:[a,b] \to \{ c_1, \ldots, c_n \}$:
\[ \int_a^b \varphi := \sum_{k=1}^n c_k \cdot [ \text{Länge von } \varphi^{-1}(c_k) ]. \]

Lebesque-Integral für beliebige (Lebesque-integrierbare) $f:[a,b]\to\real$: Approximation von $f$ durch einfache Funktionen $\varphi$:
\begin{enumerate}
 \item Wähle (immer feinere) Unterteilung des \emph{Wertebereichs} von $f$.
 \item Wähle $A_k := f^{-1}([c_{k-1},c_k])$ und setze $\varphi := c_k$ auf $A_k$.
\end{enumerate}

Diese Definition überträgt sich sofort auf $f: \real^n \to \real$ und sogar auf auf $f: X \to \real^n$ mit beliebigem merischen Raum $X$. Problem dabei ist die Frage: 
\begin{itemize}
 \item Wie ``lang'' ist $f^{-1}((c_{k-1},c_k))$?
 \item Welches ``Volumen'' hat diese Menge? (zum Beispiel $f: \real^3 \to \real^3$)
 \item Allgemein: Welches \emph{Maß} hat sie?
\end{itemize}
Ist ein sinnvoller ``Volumen''-Begriff für jede Menge in $\real^n$ überhaupt möglich? Nein! Siehe Banach-Tarski. Aber für eine geeignete Klasse von (sogenannten messbaren) Teilmengen schon.

\subsection{Äußere Maße}
Im Folgenden bezeichnet $2^X$ die Potenzmenge einer Menge $X$ (Bezeichnung in MINT: $\pot(X)$).

Wir bezeichnen außerdem $A \subset B$ $:\Leftrightarrow$ $x \in A \Rightarrow x \in B$, insbesondere ist $A = B$ möglich.

\subsubsection{Erweiterte Zahlengerade}
Wir werden Funktionen betrachten, die Werte in $[-\infty, \infty]$ annehmen. Die Ordnung von $\real$ wird durch folgende Konvention fortgesetzt:
\[ - \infty < a < \infty \text{ für alle } a \in \real. \]
Desweiteren ist 
\[ a + (\pm \infty) = \pm \infty \text{ für alle } a \in \real \]
sowie 
\[ a \cdot (\pm \infty) = \begin{cases}
                           \pm \infty, &\text{falls } a \in (0, \infty] \\
                           \mp \infty, &\text{falls } a \in [-\infty, 0) \\
                           0, &\text{falls } a = 0.
                          \end{cases} \]

\subsubsection{Äußere Maße}
Sei $X$ eine nichtleere Menge. Eine Funktion $\mu: 2^X \to [0,\infty]$ heißt \emph{äußeres Maß} auf $X$ $:\Leftrightarrow$
\begin{enumerate}
 \item $\mu(\emptyset) = 0$.
 \item Monotonie: $A \subset B \subset X\, \Rightarrow \, \mu(A) \le \mu(B)$.
 \item $\sigma$-Subadditivität: Für jede Folge $(A_n)_{n \in \nat} \subset 2^X$ gilt:
  \[ \mu \left( \bigcup_{n=0}^\infty A_n \right) \le \sum_{n=0}^\infty \mu( A_n ) \]
\end{enumerate}

\subsubsection{\texorpdfstring{$\mu$}{Mu}-messbare Mengen}
Sei $\mu$ ein äußeres Maß auf $X$. Eine Menge $E \subset X$ heißt $\mu$-messbar [Forster Analysis III: ``Caratheodory''-messbar] $:\Leftrightarrow$
 \[ \mu(F) = \mu( F \cap E ) + \mu( F \setminus E ) \text{ für alle } F \subset X. \]

\begin{bem}
 Wegen der Subadditivität ist das genau dann der Fall, wenn $\mu(F) \ge \mu(F \cap E) + \mu(F \setminus E)$ ist für alle $F$.
\end{bem}

Die Menge aller $\mu$-messbaren Mengen wird mit $\sigma(\mu) \subset 2^X$ bezeichnet.

\subsubsection{\texorpdfstring{$\sigma$}{Sigma}-Algebra}
Sei $X$ eine nichtleere Menge. Eine Familie $\mA \subset 2^X$ heißt \emph{$\sigma$-Algebra} $:\Leftrightarrow$
\begin{enumerate}
 \item $X \in \mA$.
 \item $A \in \mA \Rightarrow X \setminus A \in \mA$.
 \item $(A_n)_{n \in \nat} \subset \mA \Rightarrow \bigcup\limits_{n=0}^\infty A_n \in \mA$.
\end{enumerate}
Das Paar $(X, \mA)$ heißt \emph{Messraum}. 

Analogie: Topologie $\mathcal{T}$ auf einem metrischen Raum.
\[ \mathcal{T} = \{ \text{ offene Mengen auf } X \}. \]

\subsubsection{Borel-Mengen}
Sei $X$ ein metrischer Raum (zum Beispiel $X = \real^n$). 

Erinnerung: $U \subset X$ offen $:\Leftrightarrow$
\[ \forall x \in U \exists \tau > 0: B_\tau(x) \subset U. \]
\begin{itemize}
 \item \emph{Topologie} auf $X$ $:=$ Familie aller offenen Teilmengen von $X$.
 \item $A \subset X$ \emph{abgeschlossen} $:\Leftrightarrow$ $X \setminus A$ offen.
\end{itemize}

Definitionen:
\begin{itemize}
 \item \emph{Borel-$\sigma$-Algebra} $\borel(X)$ $:=$ kleinste $\sigma$-Algebra, die alle offenen Teilmengen von $X$ enthält (Existenz: MINT).
 \item Elemente von $\borel(X)$ heißen \emph{Borel-Mengen}.
 \item Borel-Mengen auf $[-\infty,\infty]$ sind per Definition von der Form $B$, $B \cup \{ \pm \infty \}$, $B \cup \{ -\infty, \infty \}$, wobei $B \in \borel(\real)$.
\end{itemize}

\begin{bem}
 Alle offenen und alle abgeschlossenen Teilmengen von $X$ sind Borel-messbar. Ebenso $F_\sigma$-Mengen ($=:$ Vereinigungen abzählbar vieler abgeschlossener Mengen) und $G_\delta$-Mengen ($=:$ Durchschnitt abzählbar vieler offener Mengen).
\end{bem}

\begin{proof}
 Das folgt direkt aus der Definition der $\sigma$-Algebra.
\end{proof}

\subsubsection{Maß und Maßraum}
Sei $(X,\mA)$ ein Messraum. Ein \emph{Maß} auf $\mA$ ist eine Funktion $\mu: \mA \to [0,\infty]$ mit folgenden Eigenschaften:
\begin{enumerate}
 \item $\mu(\emptyset)=0$.
 \item $\sigma$-Additivität: für jede Folge $(A_n)_{n \in \nat} \subset \mA$ von \emph{paarweise disjunkten} Mengen gilt:
 \[ \mu\left( \bigcup_{n=0}^\infty A_n \right) = \sum_{n=0}^\infty \mu(A_n). \]
\end{enumerate}

Das Tripel $(X, \mA, \mu)$ heißt dann \emph{Maßraum}.

Weitere Definitionen:
\begin{itemize}
 \item Ein Maß $\mu$ auf einem metrischen Raum $X$ heißt \emph{Borel-Maß} $:\Leftrightarrow$ $(X, \borel(X), \mu)$ ist ein Maßraum.
 \item $A \in \mA$ mit $\mu(A) = 0$ heißt \emph{$\mu$-Nullmenge}.
 \item Ein Maßraum $(X, \mA, \mu)$ heißt \emph{vollständig} $:\Leftrightarrow$ Für jede Nullmenge $A \in \mA$ gilt: $F \subset A$, dann ist $F \in \mA$ (und damit auch Nullmenge).
 \item Eine Aussage über Elemente $x \in X$ gilt \emph{fast überall} $:\Leftrightarrow$
 $\{ x \in X :$  Aussage gilt für  $x$ nicht$\}$
 ist eine $\mu$-Nullmenge.
\end{itemize}

\begin{bem}
 Abzählbare Vereinigungen von Nullmengen sind wieder Nullmengen.
\end{bem}

\subsubsection{Maß und äußeres Maß}
\begin{thm*}
 Jedes Maß definiert ein äußeres Maß und jedes äußere Maß definiert ein Maß. Genauer:
 \begin{enumerate}
  \item Sei $(X,\mA,\mu)$ ein Maßraum. Definiere $\mu^\ast:2^X \to [0,\infty]$ durch
  \[ \mu^\ast(A) := \inf \{ \mu(E) : E \subset \mA \text{ und } A \subset E \} \]
  für jedes $A \in 2^X$.
  
  Dann ist $\mu^\ast$ ein äußeres Maß auf $X$ und jede Menge $E \in \mA$ ist $\mu^\ast$-messbar.
  \item Sei $\mu$ ein äußeres Maß auf (nichtleerer) Menge $X$. Dann ist $(X, \sigma(\mu), \mu)$ ein voll\-ständiger Maßraum.
 \end{enumerate}
\end{thm*}

Beweis und weitere Eigenschaften von Maßen siehe MINT.

\subsubsection{Lebesque-Maß auf \texorpdfstring{$\real^n$}{Rn}}
Setze $Q_r(a) := a + \left[ - \frac{r}{2}, \frac{r}{2} \right)^n$ für jedes $a \in \real^n$ und jedes $r > 0$.\footnote{Das ist ein $n$-dimensionaler Würfel mit Kantenlänge $r$ und Mittelpunkt $a$. Das ``Volumen'' ist $r^n$.}

Für jedes $E \subset \real^n$ definiere
\[ \lebesque^n (E) := \inf \left\{ \sum_{i=0}^\infty r_i^n : E \subset \bigcup_{i=0}^\infty Q_{r_i}(a_i) \right\}.\footnote{Das lässt sich als ``Volumen'' der ``kleinsten Überdeckung'' von $E$ mit Würfeln interpretieren.} \]

\begin{thm*}
 $\lebesque^n$ ist ein äußeres Maß auf $\real^n$, und jede Borel-Menge ist $\lebesque^n$-messbar.
\end{thm*}

Definitionen:
\begin{itemize}
 \item $\lebesque^n$ heißt (äußeres) \emph{Lebesque-Maß} auf $\real^n$.
 \item Die Elemente von $\sigma( \lebesque^n )$ werden als \emph{Lebesque-messbare} Mengen bezeichnet\footnote{Satz besagt: Jede Borelmenge in $\real^n$ ist Lebesque-messbar}.
 \item Das zu $\lebesque^n$ gehörige Maß auf $\sigma( \lebesque^n )$ (und auf $\borel( \real^n )$ ) bezeichnen wir ebenfalls mit $\lebesque^n$.
\end{itemize}

\begin{bem}
\begin{itemize}
 \item $\lebesque^n$ ist das einzige Borel-Maß mit 
 \[ \lebesque^n( Q_r(a) ) = r^n \text{ für alle } a \in \real^n, r > 0. \]
 \item Es gilt $\lebesque^n( \partial Q_r(a) ) = 0$\footnote{Allgemein: alle $m$-dimensionalen Mengen mit $m < n$ sind $\lebesque^n$-Nullmengen.}. Insbesondere also
\[ \lebesque^n \left(a + \left( - \frac{r}{2}, \frac{r}{2} \right)^n \right) = \lebesque^n \left(a + \left[ - \frac{r}{2}, \frac{r}{2} \right]^n \right) = \lebesque^n \left(a + \left( - \frac{r}{2}, \frac{r}{2} \right]^n \right) = r^n. \]
Also für $n=1$ und $b \ge a$ gilt:
\[ \lebesque^1( (a,b) ) = \lebesque^1( [a,b) ) = \lebesque^1( (a,b] ) = \lebesque^1( [a,b] ) = b-a. \]
\end{itemize}
\end{bem}

\subsection{Lebesque-Integral}
\subsubsection{Charakteristische Funktionen}
Sei $X$ eine nichtleere Menge. Für jedes $A \subset X$ definiert man die \emph{charakteristische Funktion} $\chi_A : X \to \{ 0, 1 \}$ durch 
\[ \chi_A(x) := \begin{cases}
              1, &\text{falls } x \in A \\
              0, &\text{sonst.}
             \end{cases} \]

\subsubsection{Einfache Funktionen}
Sei $X$ eine nichtleere Menge. Eine Funktion $\varphi: X \to \real$ heißt \emph{einfach} $:\Leftrightarrow$ $\varphi$ nimmt nur endlich viele verschiedene Werte an. Mit anderen Worten: Es existiert $m \in \nat$ und $A_1, \ldots, A_m \subset X$ und $c_1, \ldots, c_m \in \real$, so dass
\[ \varphi = \sum_{i=1}^m c_i \chi_{A_i}. \]

\subsubsection{Messbare Funktionen}
Sei $(X, \mA)$ ein Messraum.
\begin{itemize}
 \item Eine Funktion $f: X \to [0, \infty]$ heißt \emph{$\mA$-messbar} $:\Leftrightarrow$ $f^{-1}(E) \in \mA$ für alle Borel-Mengen $E \subset [-\infty,\infty]$.
 \item $f: \real^n \to [-\infty,\infty]$ heißt \emph{Borel-messbar} (bzw. Lebesque-messbar) $:\Leftrightarrow$ $f$ ist $\borel(\real^n)$-messbar (bzw. $\lebesque^n$-messbar).
 \item $f: \real^n \to [-\infty,\infty]^n$ heißt \emph{$\mA$-messbar} $:\Leftrightarrow$ jede Komponente von $f$ ist messbar.
\end{itemize}

\begin{bem}
 Sei $(X, \mA)$ ein Messraum und $f: X \to [-\infty,\infty]$. Äquivalente Aussagen:
 \begin{enumerate}
  \item $f$ ist messbar.
  \item $\{ x \in X: f(x) > c \} \in \mA$ für alle $c \in \real$.
 \end{enumerate}
 Ebenso für $\le$, $\ge$, $<$ statt $>$.
\end{bem}

\begin{thm*}
 Sei $(X, \mA)$ ein Messraum und für alle $k \in \nat$ seien $f_k, f, g: X \to [-\infty,\infty]$ messbar. Dann sind ebenfalls messbar:
 \begin{align*}
 f+g,\quad fg,\quad \max\{ f, g \},\quad \min\{ f, g \}, \\ \sup_{k \in \nat} f_k,\quad \inf_{k \in \nat} f_k,\quad \limsup_{k \in \nat} f_k,\quad \liminf_{k \in \nat} f_k.
 \end{align*}
 Ebenso $\frac{f}{g}$ sofern $g \ne 0$. Die Grenzwerte und Supremum sind hier \emph{punktweise} zu verstehen.
\end{thm*}

\begin{bem}
 Jede stetige Funktion $f:\real^n \to \real$ ist Borel-messbar (also auch Lebesque-messbar).
\end{bem}

\begin{proof}
 Wegen der Stetigkeit ist $f^{-1}((c,\infty))$ offen, also Borel-messbar.
\end{proof}

\end{document}