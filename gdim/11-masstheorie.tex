\setcounter{section}{10}
\section{Abriss der Maßtheorie}
\subsection*{Motivation und Idee des Lebesgue-Integrals}
\addcontentsline{toc}{subsection}{Motivation und Idee des Lebesgue-Integrals}
\subsubsection*{Erinnerung an das Riemann-Integral auf $[a,b] \subset \real$}
\addcontentsline{toc}{subsubsection}{Erinnerung an das Riemann-Integral auf \texorpdfstring{$[a,b] \subset \real$}{[a,b] in IR}}
Bausteine sind die Treppenfunktionen $\varphi \in \stair$ mit zugehöriger Unterteilung $a = x_0 < x_1 < \ldots < x_n = b$ und $\varphi = c_k$ auf $(x_{k_1},x_k)$:
\[ \int_a^b \varphi := \sum_{k=1}^n c_k \cdot (x_k - x_{k-1}). \]

Das Integral für beliebige (Riemann-integrierbare) $f:[a,b]\to\real$: Approximation von $f$ durch folgendermaßen gewählte Treppenfunktionen $\varphi \in \stair$:
\begin{enumerate}
 \item Wähle (immer feinere) Unterteilung $a = x_0 < x_1 < \ldots < x_n = b$ des \emph{Definitionsbereichs} $[a,b]$.
 \item Wähle $c_1, \ldots, c_n$ in Abhängigkeit von $f$.
\end{enumerate}

\subsubsection*{Lebesgue-Integral auf $[a,b] \subset \real$}
\addcontentsline{toc}{subsubsection}{Lebesgue-Integral auf \texorpdfstring{$[a,b] \subset \real$}{[a,b] in IR}}
Bausteine sind \emph{einfache Funktionen}.

Eine Funktion $\varphi: [a,b] \to \real$ heißt \emph{einfach} $:\Leftrightarrow$ $\varphi$ nimmt nur endlich viele verschiedene Werte an:
\[ \sharp \varphi([a,b]) < \infty. \]
Unterschied zu Treppenfunktionen: $\varphi^{-1}(c_k)$ ist im Allgemeinen kein Intervall.

Lebesgue-Integral für einfache Funktionen $\varphi:[a,b] \to \{ c_1, \ldots, c_n \}$:
\[ \int_a^b \varphi := \sum_{k=1}^n c_k \cdot [ \text{Länge von } \varphi^{-1}(c_k) ]. \]

Lebesgue-Integral für beliebige (Lebesgue-integrierbare) $f:[a,b]\to\real$: Approximation von $f$ durch einfache Funktionen $\varphi$:
\begin{enumerate}
 \item Wähle (immer feinere) Unterteilung des \emph{Wertebereichs} von $f$.
 \item Wähle $A_k := f^{-1}([c_{k-1},c_k])$ und setze $\varphi := c_k$ auf $A_k$.
\end{enumerate}

Diese Definition überträgt sich sofort auf $f: \real^n \to \real$ und sogar auf auf $f: X \to \real^n$ mit beliebigem merischen Raum $X$. Problem dabei ist die Frage: 
\begin{itemize}
 \item Wie ``lang'' ist $f^{-1}((c_{k-1},c_k))$?
 \item Welches ``Volumen'' hat diese Menge? (zum Beispiel $f: \real^3 \to \real^3$)
 \item Allgemein: Welches \emph{Maß} hat sie?
\end{itemize}
Ist ein sinnvoller ``Volumen''-Begriff für jede Menge in $\real^n$ überhaupt möglich? Nein! Siehe Banach-Tarski. Aber für eine geeignete Klasse von (sogenannten messbaren) Teilmengen schon.

\subsection{Äußere Maße}
Im Folgenden bezeichnet $2^X$ die Potenzmenge einer Menge $X$ (Bezeichnung in [MINT]: $\pot(X)$).

Wir bezeichnen außerdem $A \subset B$ $:\Leftrightarrow$ $x \in A \Rightarrow x \in B$, insbesondere ist $A = B$ möglich.

\subsubsection{Erweiterte Zahlengerade}
Wir werden Funktionen betrachten, die Werte in $[-\infty, \infty]$ annehmen. Die Ordnung von $\real$ wird durch folgende Konvention fortgesetzt:
\[ - \infty < a < \infty \text{ für alle } a \in \real. \]
Desweiteren ist 
\[ a + (\pm \infty) = \pm \infty \text{ für alle } a \in \real \]
sowie 
\[ a \cdot (\pm \infty) = \begin{cases}
                           \pm \infty, &\text{falls } a \in (0, \infty] \\
                           \mp \infty, &\text{falls } a \in [-\infty, 0) \\
                           0, &\text{falls } a = 0.
                          \end{cases} \]

\subsubsection{Äußere Maße}
Sei $X$ eine nichtleere Menge. Eine Funktion $\mu: 2^X \to [0,\infty]$ heißt \emph{äußeres Maß} auf $X$ $:\Leftrightarrow$
\begin{enumerate}
 \item $\mu(\emptyset) = 0$.
 \item Monotonie: $A \subset B \subset X\, \Rightarrow \, \mu(A) \le \mu(B)$.
 \item $\sigma$-Subadditivität: Für jede Folge $(A_n)_{n \in \nat} \subset 2^X$ gilt:
  \[ \mu \left( \bigcup_{n=0}^\infty A_n \right) \le \sum_{n=0}^\infty \mu( A_n ) \]
\end{enumerate}

\subsubsection{\texorpdfstring{$\mu$}{Mu}-messbare Mengen}
Sei $\mu$ ein äußeres Maß auf $X$. Eine Menge $E \subset X$ heißt $\mu$-messbar\footnote{Forster Analysis III: ``Caratheodory''-messbar} $:\Leftrightarrow$
 \[ \mu(F) = \mu( F \cap E ) + \mu( F \setminus E ) \text{ für alle } F \subset X. \]

\begin{rmrk}
 Wegen der Subadditivität ist das genau dann der Fall, wenn $\mu(F) \ge \mu(F \cap E) + \mu(F \setminus E)$ ist für alle $F$.
\end{rmrk}

Die Menge aller $\mu$-messbaren Mengen wird mit $\sigma(\mu) \subset 2^X$ bezeichnet.

\subsubsection{\texorpdfstring{$\sigma$}{Sigma}-Algebra}
Sei $X$ eine nichtleere Menge. Eine Familie $\mA \subset 2^X$ heißt \emph{$\sigma$-Algebra} $:\Leftrightarrow$
\begin{enumerate}
 \item $X \in \mA$.
 \item $A \in \mA \Rightarrow X \setminus A \in \mA$.
 \item $(A_n)_{n \in \nat} \subset \mA \Rightarrow \bigcup\limits_{n=0}^\infty A_n \in \mA$.
\end{enumerate}
Das Paar $(X, \mA)$ heißt \emph{Messraum}. 

Analogie: Topologie $\mathcal{T}$ auf einem metrischen Raum.
\[ \mathcal{T} = \{ \text{ offene Mengen auf } X \}. \]

\subsubsection{Borel-Mengen}
Sei $X$ ein metrischer Raum (zum Beispiel $X = \real^n$). 

Erinnerung: $U \subset X$ offen $:\Leftrightarrow$
\[ \forall x \in U \exists \tau > 0: B_\tau(x) \subset U. \]
\begin{itemize}
 \item \emph{Topologie} auf $X$ $:=$ Familie aller offenen Teilmengen von $X$.
 \item $A \subset X$ \emph{abgeschlossen} $:\Leftrightarrow$ $X \setminus A$ offen.
\end{itemize}

Definitionen:
\begin{itemize}
 \item \emph{Borel-$\sigma$-Algebra} $\borel(X)$ $:=$ kleinste $\sigma$-Algebra, die alle offenen Teilmengen von $X$ enthält (Existenz: [MINT]).
 \item Elemente von $\borel(X)$ heißen \emph{Borel-Mengen}.
 \item Borel-Mengen auf $[-\infty,\infty]$ sind per Definition von der Form $B$, $B \cup \{ \pm \infty \}$, $B \cup \{ -\infty, \infty \}$, wobei $B \in \borel(\real)$.
\end{itemize}

\begin{rmrk}
 Alle offenen und alle abgeschlossenen Teilmengen von $X$ sind Borel-mess\-bar. Ebenso $F_\sigma$-Mengen ($=:$ Vereinigungen abzählbar vieler abgeschlossener Mengen) und $G_\delta$-Mengen ($=:$ Durchschnitt abzählbar vieler offener Mengen).
\end{rmrk}

\begin{proof}
 Das folgt direkt aus der Definition der $\sigma$-Algebra.
\end{proof}

\subsubsection{Maß und Maßraum}
Sei $(X,\mA)$ ein Messraum. Ein \emph{Maß} auf $\mA$ ist eine Funktion $\mu: \mA \to [0,\infty]$ mit folgenden Eigenschaften:
\begin{enumerate}
 \item $\mu(\emptyset)=0$.
 \item $\sigma$-Additivität: für jede Folge $(A_n)_{n \in \nat} \subset \mA$ von \emph{paarweise disjunkten} Mengen gilt:
 \[ \mu\left( \bigcup_{n=0}^\infty A_n \right) = \sum_{n=0}^\infty \mu(A_n). \]
\end{enumerate}

Das Tripel $(X, \mA, \mu)$ heißt dann \emph{Maßraum}.

Weitere Definitionen:
\begin{itemize}
 \item Ein Maß $\mu$ auf einem metrischen Raum $X$ heißt \emph{Borel-Maß} $:\Leftrightarrow$ $(X, \borel(X), \mu)$ ist ein Maßraum.
 \item $A \in \mA$ mit $\mu(A) = 0$ heißt \emph{$\mu$-Nullmenge}.
 \item Ein Maßraum $(X, \mA, \mu)$ heißt \emph{vollständig} $:\Leftrightarrow$ Für jede Nullmenge $A \in \mA$ gilt: $F \subset A$, dann ist $F \in \mA$ (und damit auch Nullmenge).
 \item Eine Aussage über Elemente $x \in X$ gilt \emph{fast überall} $:\Leftrightarrow$
 $\{ x \in X :$  Aussage gilt für  $x$ nicht$\}$
 ist eine $\mu$-Nullmenge.
\end{itemize}

\begin{rmrk}
 Abzählbare Vereinigungen von Nullmengen sind wieder Nullmengen.
\end{rmrk}

\subsubsection{Maß und äußeres Maß}
\begin{thm}
 Jedes Maß definiert ein äußeres Maß und jedes äußere Maß definiert ein Maß. Genauer:
 \begin{enumerate}
  \item Sei $(X,\mA,\mu)$ ein Maßraum. Definiere $\mu^\ast:2^X \to [0,\infty]$ durch
  \[ \mu^\ast(A) := \inf \{ \mu(E) : E \subset \mA \text{ und } A \subset E \} \]
  für jedes $A \in 2^X$.
  
  Dann ist $\mu^\ast$ ein äußeres Maß auf $X$ und jede Menge $E \in \mA$ ist $\mu^\ast$-messbar.
  \item Sei $\mu$ ein äußeres Maß auf (nichtleerer) Menge $X$. Dann ist $(X, \sigma(\mu), \mu)$ ein voll\-ständiger Maßraum.
 \end{enumerate}
\end{thm}

Beweis und weitere Eigenschaften von Maßen siehe [MINT].

\subsubsection{Lebesgue-Maß auf \texorpdfstring{$\real^n$}{IRn}}
Setze $Q_r(a) := a + \left[ - \frac{r}{2}, \frac{r}{2} \right)^n$ für jedes $a \in \real^n$ und jedes $r > 0$.\footnote{Das ist ein $n$-dimensionaler Würfel mit Kantenlänge $r$ und Mittelpunkt $a$. Das ``Volumen'' ist $r^n$.}

Für jedes $E \subset \real^n$ definiere
\[ \lebesgue^n (E) := \inf \left\{ \sum_{i=0}^\infty r_i^n : E \subset \bigcup_{i=0}^\infty Q_{r_i}(a_i) \right\}.\footnote{Das lässt sich als ``Volumen'' der ``kleinsten Überdeckung'' von $E$ mit Würfeln interpretieren.} \]

\begin{thm}
 $\lebesgue^n$ ist ein äußeres Maß auf $\real^n$, und jede Borel-Menge ist $\lebesgue^n$-messbar.
\end{thm}

Definitionen:
\begin{itemize}
 \item $\lebesgue^n$ heißt (äußeres) \emph{Lebesgue-Maß} auf $\real^n$.
 \item Die Elemente von $\sigma( \lebesgue^n )$ werden als \emph{Lebesgue-messbare} Mengen bezeichnet\footnote{Satz besagt: Jede Borelmenge in $\real^n$ ist Lebesgue-messbar}.
 \item Das zu $\lebesgue^n$ gehörige Maß auf $\sigma( \lebesgue^n )$ (und auf $\borel( \real^n )$ ) bezeichnen wir ebenfalls mit $\lebesgue^n$.
\end{itemize}

\begin{rmrk}
\begin{itemize}
 \item $\lebesgue^n$ ist das einzige Borel-Maß mit 
 \[ \lebesgue^n( Q_r(a) ) = r^n \text{ für alle } a \in \real^n, r > 0. \]
 \item Es gilt $\lebesgue^n( \partial Q_r(a) ) = 0$\footnote{Allgemein: alle $m$-dimensionalen Mengen mit $m < n$ sind $\lebesgue^n$-Nullmengen.}. Insbesondere also
\[ \lebesgue^n \left(a + \left( - \frac{r}{2}, \frac{r}{2} \right)^n \right) = \lebesgue^n \left(a + \left[ - \frac{r}{2}, \frac{r}{2} \right]^n \right) = \lebesgue^n \left(a + \left( - \frac{r}{2}, \frac{r}{2} \right]^n \right) = r^n. \]
Also für $n=1$ und $b \ge a$ gilt:
\[ \lebesgue^1( (a,b) ) = \lebesgue^1( [a,b) ) = \lebesgue^1( (a,b] ) = \lebesgue^1( [a,b] ) = b-a. \]
\end{itemize}
\end{rmrk}

\subsection{Lebesgue-Integral}
\subsubsection{Charakteristische Funktionen}
Sei $X$ eine nichtleere Menge. Für jedes $A \subset X$ definiert man die \emph{charakteristische Funktion} $\chi_A : X \to \{ 0, 1 \}$ durch 
\[ \chi_A(x) := \begin{cases}
              1, &\text{falls } x \in A \\
              0, &\text{sonst.}
             \end{cases} \]

\subsubsection{Einfache Funktionen}
Sei $X$ eine nichtleere Menge. Eine Funktion $\varphi: X \to \real$ heißt \emph{einfach} $:\Leftrightarrow$ $\varphi$ nimmt nur endlich viele verschiedene Werte an. Mit anderen Worten: Es existiert $m \in \nat$ und $A_1, \ldots, A_m \subset X$ und $c_1, \ldots, c_m \in \real$, so dass
\[ \varphi = \sum_{i=1}^m c_i \chi_{A_i}. \]

\subsubsection{Messbare Funktionen}
Sei $(X, \mA)$ ein Messraum.
\begin{itemize}
 \item Eine Funktion $f: X \to [0, \infty]$ heißt \emph{$\mA$-messbar} $:\Leftrightarrow$ $f^{-1}(E) \in \mA$ für alle Borel-Mengen $E \subset [-\infty,\infty]$.
 \item $f: \real^n \to [-\infty,\infty]$ heißt \emph{Borel-messbar} (bzw. Lebesgue-messbar) $:\Leftrightarrow$ $f$ ist $\borel(\real^n)$-messbar (bzw. $\lebesgue^n$-messbar).
 \item $f: \real^n \to [-\infty,\infty]^n$ heißt \emph{$\mA$-messbar} $:\Leftrightarrow$ jede Komponente von $f$ ist messbar.
\end{itemize}

\begin{rmrk}
 Sei $(X, \mA)$ ein Messraum und $f: X \to [-\infty,\infty]$. Äquivalente Aussagen:
 \begin{enumerate}
  \item $f$ ist messbar.
  \item $\{ x \in X: f(x) > c \} \in \mA$ für alle $c \in \real$.
 \end{enumerate}
 Ebenso für $\le$, $\ge$, $<$ statt $>$.
\end{rmrk}

\begin{thm}
 Sei $(X, \mA)$ ein Messraum und für alle $k \in \nat$ seien $f_k, f, g: X \to [-\infty,\infty]$ messbar. Dann sind ebenfalls messbar:
 \begin{align*}
 &f+g,& &fg,& &\max\{ f, g \},& &\min\{ f, g \}, \\ 
 &\sup_{k \in \nat} f_k, & &\inf_{k \in \nat} f_k,& &\limsup_{k \in \nat} f_k,& &\liminf_{k \in \nat} f_k.
 \end{align*}
 Ebenso $\frac{f}{g}$ sofern $g \ne 0$ auf $X$. Die Grenzwerte und Supremum sind hier \emph{punktweise} zu verstehen.
\end{thm}

\begin{rmrk}
 Jede stetige Funktion $f:\real^n \to \real$ ist Borel-messbar (also auch Lebesgue-messbar).
\end{rmrk}

\begin{proof}
 Wegen der Stetigkeit ist $f^{-1}((c,\infty))$ offen, also Borel-messbar.
\end{proof}

\subsubsection{Integral für nichtnegative einfache Funktionen}
Sei $(X,\mA,\mu)$ ein Maßraum und $\varphi: X \to [0, \infty)$ einfach; genauer: Seien $c_i \ge 0$ und $A_i \in \mA$ für $i=1,\ldots,m$ und $\varphi = \sum_{i=1}^m c_i \chi_{A_i}$. Dann definiert man folgende Zahl\footnote{{$\infty$ ist möglich, da unter Umständen $\mu(A_i) = \infty$ für ein $i$}} in $[0,\infty]$:
\[ \int_X \varphi \diffop \mu := \sum_{i=1}^m c_i \mu( A_i ). \]

\begin{rmrk}
 Das Integral ist wohldefiniert, das heißt wenn
 \[ \sum_{i=1}^m c_i \chi_{A_i} = \sum_{i=1}^m c'_i \chi_{A'_i}, \]
 dann ist 
 \[ \sum_{i=1}^m c_i \mu(A_i) = \sum_{i=1}^m c'_i \mu(A'_i). \]
\end{rmrk}

\subsubsection{Integral für nichtnegative Funktionen}
Sei $(X,\mA,\mu)$ ein Maßraum und $f: X \to [0, \infty]$ messbar. Man definiert
\[ \int_X f \diffop \mu := \sup \left\{ \int_x \varphi \diffop \mu : \varphi: X \to \real \text{ messbar und einfach mit } 0 \le \varphi \le f \right\}. \]

\begin{rmrk}
 Das Integral ist linear und monoton, das heißt
 \[ f \le g \Rightarrow \int f \diffop \mu \le \int g \diffop \mu \]
 auf der Klasse der nichtnegativen messbaren Funktionen.
\end{rmrk}

\subsubsection{Integrierbare Funktionen}
Sei $(X,\mA,\mu)$ ein Maßraum. Eine Funktion $f:X \to [-\infty,\infty]$ heißt \emph{$\mu$-integrierbar} $:\Leftrightarrow$ $f$ ist $\mA$-messbar und für die beiden (dann ebenfalls messbaren\footnote{Das folgt aus Satz 2.3}) Funktionen $f_\pm : X \to [0,\infty]$ gilt:
\[ \int f_\pm \diffop \mu < \infty. \]

Man definiert dann folgende Zahl in $\real$:
\[ \int_X f \diffop \mu := \int_X f_+ \diffop \mu - \int_X f_- \diffop \mu. \]

Hierbei ist $f_\pm := \max \{ \pm f, 0 \}$.\footnote{Beachte: $f_-$ ist nach Definition auch eine nichtnegative Funktion}

\begin{rmrk}
 $f$ ist integrierbar genau dann, wenn $|f|$ integrierbar ist.
 \begin{proof}
  Sei $\int |f| \diffop \mu < \infty$. Da $0 \le f_\pm \le |f|$ folgt aus der Monotonie auf nichtnegativen Funktionen, dass $\int f_\pm \le \int |f| < \infty$.
  
  Umgekehrt sei $f$ integrierbar, das heißt $\int f_\pm < \infty$. Da $|f| = f_+ + f_-$ folgt aus der Linearität $\int |f| < \infty$.
 \end{proof}
\end{rmrk}

\begin{thm}
 Sei $(X,\mA,\mu)$ ein Maßraum und seien $f,g:X \to [-\infty, \infty]$ messbar. Dann gilt
 \begin{enumerate}[(a)]
  \item $\int_X |f| \diffop \mu = 0$ $\Leftrightarrow$ $f=0$ fast überall.
  \item $f$ ist $\mu$-integrierbar $\Rightarrow$ $f(x) \notin \{ -\infty, \infty \}$ für $\mu$-fast alle\footnotemark $x \in X$.
  \item Wenn $\mu$-fast überall $f=g$ gilt, dann ist $f$ integrierbar genau dann, wenn $g$ integrierbar ist. Dann ist
  \[ \int_X f \diffop \mu = \int_X g \diffop \mu. \]
  \item $f,g$ integrierbar und $c \in \real$ $\Rightarrow$ $cf$, $f+g$ integrierbar.
  \item Das Integral ist linear und monoton auf dem Vektorraum der integrierbaren Funktionen.
 \end{enumerate}
\end{thm}
\footnotetext{$\mu(\{ x \in X: f(x) = \infty$ oder $f(x) = -\infty \} ) = 0$}

\begin{rmrk}
 Keine Äquivalenz in (b), zum Beispiel $f: \real \to \real$ gegeben durch $f(x)$ für alle $x$ mit Maßraum $(\real, \borel(\real), \lebesgue^1 )$.
\end{rmrk}

\subsubsection{Integration über Teilmengen}
 Sei $(X,\mA,\mu)$ ein Maßraum und $Z \in X$ messbar. Eine messbare Funktion $f:X \to [-\infty,\infty]$ heißt \emph{über $Z$ integrierbar} $:\Leftrightarrow$ $\chi_Z f$ ist integrierbar. 
 
 In diesem Fall definiert man
 \[ \int_Z f \diffop \mu := \int_X \chi_Z f \diffop \mu. \]

\begin{exmp}
 $(X,\mA,\mu) = (\real, \borel(\real), \lebesgue^1 )$, $f \equiv 1$ und $Z = [a,b]$ mit $a,b \in \real$, $a \le b$. Dann ist $f$ nicht integrierbar (auf $\real$), aber über $Z$ integrierbar, da $\chi_Z f = \chi_{[a,b]}$ und
 \[ \int_\real \chi_{[a,b]} \diffop \lebesgue^1 = \lebesgue^1( [a,b] ) = b - a. \]
\end{exmp}

\subsection{Konvergenzsätze}
 Riemann-Integral: $f_k \to f$ \emph{gleichmäßig} $\Rightarrow$ $\int f_k \to \int f$. Das ist eine sehr starke Bedingung.
 
\subsubsection{Monotone Konvergenz}
\begin{thm}
 Sei $(X,\mA,\mu)$ ein Maßraum und $f_0 \le f_1 \le \ldots$ eine monoton wachsende Folge integrierbarer Funktionen, $f_k : X \to \real$. Zudem sei
 \[ \sup_{k \in \nat} \int f_k \diffop \mu < \infty. \]
 Dann ist der \emph{punktweise} Grenzwert\footnotemark $f: X \to (-\infty, \infty]$ ebenfalls integrierbar und es gilt
 \[ \lim_{k \to \infty} \int_X f_k \diffop \mu = \int_X f \diffop \mu. \]
\end{thm}
\footnotetext{Das heißt $f(x) := \lim_{k \to \infty} f_k(x)$ für alle $x \in X$.}

Beweis siehe [MINT].

\begin{exmp}
 Siehe nächster Abschnitt (\ref{sect:riemann-lebesgue}) und Tonelli (\ref{sect:tonelli}).
\end{exmp}

\subsubsection{Riemann vs Lebesgue}\label{sect:riemann-lebesgue}
\begin{kor}
 Sei $f:[a,b] \to \real$ Riemann-integrierbar. Dann gilt
 \begin{enumerate}
  \item $f$ ist $\lebesgue^1$ messbar.
  \item Riemann- und Lebesgue-Integral von $f$ stimmen überein:
   \[ \int f(x) \diffop x = \int_{[a,b]} f \diffop \lebesgue^1. \]
 \end{enumerate}
\end{kor}

\begin{proof}
 Da $f$ Riemann-integrierbar ist, existieren Folgen $(\varphi_k), (\psi_k) \subset \stair$, so dass $\varphi_k \le f \le \psi_k$ und
 \[ \int_a^b f = \lim_{k\to\infty} \int_a^b \varphi_k = \lim_{k\to\infty} \int_a^b \psi_k. \tag{$\circ$} \]

 Ohne Einschränkung sei $\varphi_0 \le \varphi_1 \le \ldots$ und $\psi_0 \ge \psi_1 \ge \ldots$. Aber jede Treppenfunktionen ist Borel-messbar und Lebesgue-integrierbar und ihr Riemann- und Lebesgue-Integral stimmen überein, da $\lebesgue^1( (a,b) ) = b - a$.

 Wegen Satz 3.1 und ($\circ$) sind auch die punktweisen Grenzwerte $\varphi := \lim \varphi_k$ und $\psi := \lim \psi_k$ integrierbar und Borel, und
 \[ \int_{[a,b]} \varphi \diffop \lebesgue^1 = \lim \int_{[a,b]} \varphi_k \diffop \lebesgue = \lim \int_a^b \varphi_k = \int_a^b f. \]
 Analog für $\psi$. Da $\psi \ge \varphi$ also
 \[ \int | \psi - \varphi | \diffop \lebesgue^1 = \int (\psi - \varphi) \diffop \lebesgue^1 = \int \psi \diffop \lebesgue^1 - \int \varphi \diffop \lebesgue^1 = 0. \]
 Also ist $\varphi = \psi$ $\lebesgue^1$-integrierbar fast überall wegen Satz 2.6, also ist $\psi = f = \varphi$ $\lebesgue^1$-integrierbar fast überall, weil ja $\varphi \le f \le \psi$.
\end{proof}

Beispiel einer $\lebesgue^1$-integrierbaren, aber nicht Riemann-integrierbaren Funktion $f:[a,b] \to [0,\infty)$:
\[ f = \chi_{\rat \cap [a,b]} = \begin{cases} 1, &\text{wenn } x \in \rat \text{ und } x \in [a,b], \\
                                 0, &\text{sonst.}
                                \end{cases}
   \qRq \int_{[a,b]} f \diffop \lebesgue^1 = 0. \]
   
\subsubsection*{Uneigentliches Riemann-Integral}
Eine Funktion $f:(0,\infty) \to \real$ heißt uneigentlich Riemann-integrierbar $:\Leftrightarrow$ $\forall 0 < \eps < R < \infty$ gilt $f$ ist Riemann-integrierbar auf $[\eps,R]$. Dann:
\[ \int_0^\infty f := \lim_{\substack{\eps \to 0 \\ R \to \infty}} \int_\eps^R f \]
Es gilt: wenn $f \ge 0$ und uneigentlich Riemann-integrierbar ist auf $(0,\infty)$, dann ist $f$ auch $\lebesgue^1$-integrierbar. auf $(0,\infty)$.

\begin{proof}
 Wende monotone Konvergenz an auf $\chi_{[\eps,R]} f \uparrow f$ für $\eps \to 0$, $R \to \infty$.
\end{proof}

Aber ohne Voraussetzung $f \ge 0$ ist die Aussage im Allgemeinen falsch. Zum Beispiel ist $f(x) := \frac{\sin x}{x}$ uneigentlich Riemann-integrierbar auf $(0,\infty)$, aber nicht $\lebesgue^1$-integrierbar, denn $|f(x)| \ge \rez{2 |x|}$ auf einer Menge $E := \{ x \in \real: |\sin x| \ge \rez{2} \}$. Man kann leicht zeigen, dass
\[ \int_E \rez{|x|} \diffop x = \sum_{i=1}^\infty ( \log b_i - \log a_i ) = \infty. \]

\subsubsection{Majorierte Konvergenz}
\begin{thm}
 Sei $(X, \mA, \mu)$ ein Maßraum und $(f_k)_{k \in \nat}$ eine Folge messbarer Funktionen $X \to \real$, die \emph{punktweise} gegen $f:X \to \real$ konvergiert. Zudem existiere eine Majorante, das heißt eine \emph{integrierbare} Funktion $F: X \to [0, \infty]$, so dass punktweise fast überall $|f_k| \le F$ für alle $k \in \nat$. Dann ist $f$ integrierbar\footnotemark und
 \[ \lim_{k \to \infty} \int_X f_k \diffop \mu = \int_X f \diffop \mu. \]
\end{thm}
\footnotetext{Auch alle $f_k$, das folgt sofort aus der Ungleichung}

\begin{rmrk}
 Wegen $|f_k -f| \le 2F$ und der punktweisen Konvergenz $|f_k - f| \to 0$ folgt durch nochmalige Anwendung des Satzes der stärkere Schluss
 \[ \int_x |f_k - f| \diffop \mu \to 0. \]
\end{rmrk}

\subsubsection*{Notwendigkeit der Majorante}
Zugleich Notwendigkeit der \emph{monotonen} Konvergenz-Annahme in Satz 3.1:

Suche eine Folge $(f_k)$ mit $\int_\real f_k = 1$, aber $f_k \to 0$ punktweise. Wähle zum Beispiel $f_k = k \cdot \chi_{[-1/2k,1/2k]\setminus \{ 0 \}}$. (``Rechtecke'' zentriert um 0). Es gibt keine Majorante, da $f_{k+1} > f_k$ auf $\left[ -\rez{2(k+1)}, \rez{2k+1} \right]$. Es gilt
\[ \int_\real f_k = k \cdot \lebesgue^1 \left( -\rez{2k}, \rez{2k} \right) = k \cdot \rez{k} = 1 \]
und für alle $x \ne 0$ gilt $f_k(x) \to 0$. Damit gilt aber $\int_\real f = 0$!

\subsection{Parameterabhängige Integrale}
Situation: $f(x,t)$ und betrachte
\[ \int_X f(x,t) \diffop \mu(x) := \int_X f( \cdot, t ) \diffop \mu \]
wobei $f( \cdot, t)(x) := f(x,t)$. Wie hängt $\int_X f(x,t) \diffop \mu(x)$ von $t$ ab?

\subsubsection{Stetige Abhängigkeit}
\begin{thm}
 Sei $(X, \mA, \mu)$ ein Maßraum, $U \subset \real^n$ offen und $a \in U$. Weiter sei
 \[ f: X \times U \to \real, \quad (x,t) \mapsto f(x,t) \]
 eine Funktion mit folgenden Eigenschaften:
 \begin{enumerate}
  \item Für jedes feste $t \in U$ ist $f( \cdot, t ): x \mapsto f(x,t)$ integrierbar auf $X$.
  \item Für jedes feste $x \in X$ ist $f( x, \cdot ): t \mapsto f(x,t)$ stetig im Punkt $a$.
  \item Es existiert eine integrierbare Majorante $F: X \to [0,\infty]$ mit $|f(x,t)| \le F(x)$ für alle $(x,t) \in X \times U$.
 \end{enumerate}
 Dann ist die durch
 \[ g(t) := \int_X f(x,t) \diffop \mu(x) := \int_X f( \cdot, t ) \diffop \mu \]
 definierte Funktion $g: U \to \real$ stetig in $a$, das heißt
 \[ \lim_{t \to a} \int_X f(x,t) \diffop \mu(x) = \int_X f(x,a) \diffop \mu(x). \]
\end{thm}

\begin{proof}
 Sei $(t_k)_{k \in \nat} \subset U$ mit $t_k \to a$ und definiere $f_k(x) := f(x, t_k)$ und $f_\infty(x) := f(x, a)$. Nach Voraussetzung ist jedes $f_k$ messbar und majoriert durch $f$:
 \[ |f_k(x)| = |f(x,t_k)| \le F(x) \text{ für alle } x \in X. \]
 Zudem gilt wegen 2. $f_k \to f$ punktweise auf $X$. Daher folgt die Behauptung aus der majorierten Konvergenz.
\end{proof}

\subsubsection{Differenzierbare Abhängigkeit}
\begin{thm}
 Sei $(X, \mA, \mu)$ ein Maßraum und $I \subset \real$ ein nichttriviales Intervall. Weiter sei 
 \[ f: X \times I \to \real, \quad (x,t) \mapsto f(x,t) \]
 eine Funktion mit folgenden Eigenschaften:
 \begin{enumerate}
  \item Für jedes feste $t \in I$ ist $f( \cdot, t )$ integrierbar.
  \item Für jedes feste $x \in X$ ist $f( x, \cdot )$ differenzierbar auf $I$.
  \item Es existiert eine integrierbare Majorante $F: X \to [0,\infty]$, so dass $\pdiff{f}{t}(x,t) \le F(x)$ für alle $(x,t) \in X \times I$.
 \end{enumerate}
 Dann ist die durch
 \[ g(t) := \int_X f(x,t) \diffop \mu(x)  \]
 definierte Funktion $g: I \to \real$ differenzierbar auf $I$ und
 \[ g'(t) = \int_X \pdiff{f}{t}(x,t) \diffop \mu(x) \text{ für alle } t \in I. \]
\end{thm}

\begin{proof}
 Sei $t \in I$ fest und $h \ne 0$, so dass $t+h \in I$. Definiere
 \[ F_h(x) := \rez{h}( f(x, t+h) - f(x,t) ). \]
 Nach dem Mittelwertsatz existiert $\Theta \in [0,1]$, so dass
 \[ |F_h(x)| = \left| \pdiff{f}{t} (x, t + \Theta h) \right| \le F(x) \]
 wegen 3. für alle $x \in X$. Aufgrund der Differenzierbarkeit von $f(x, \cdot)$ konvergiert $F_h(x) \to \pdiff{f}{t}(x,t)$ für alle $x \in X$ (mit $h \to 0$). Also folgt wegen majorierter Konvergenz
 \[ \int_X \pdiff{f}{t}(x,t) \diffop \mu(x) = \lim_{h \to 0} \int_X F_h \diffop \mu = g'(t). \qedhere \]
\end{proof}

\clearpage

\subsection{Eigenschaften des Lebesgue-Maßes}
\subsubsection{Definition (Erinnerung)}
 \textbf{Äußeres Lebesgue-Maß.} Für jede Menge $E \subset \real^n$ definiert man
 \[ \lebesgue^n( E ) := \inf \left\{ \sum_{k=0}^\infty r_k^n : E \subset \bigcup_{i=0}^\infty Q_{r_k}(a_k) \right\}, \]
 wobei das Infimum über alle Folgen $(r_k)_{k \in \nat} \subset (0, \infty)$ genommen wird, für die eine zugehörige Folge $(a_k)_{k \in \nat} \subset \real^n$ existiert, so dass $E \subset \bigcup_{k=0}^\infty Q_{r_k}(a_k)$. Hierbei ist $Q_{r}(a) := a + \left(-\frac{r}{2}, \frac{r}{2} \right]^n$.

 Lebesgue-Maß: Einschränkung des so definierten äußeren Lebesgue-Maßes auf $\sigma( \lebesgue^n )$.
 
\subsubsection{``Maßraum''-Eigenschaften}
\begin{thm}
 Bezeichne $\lebesgue^n: 2^{\real^n} \to [0, \infty]$ das äußere Lebesgue-Maß auf $\real^n$. Dann gilt:
 \begin{enumerate}
  \item $\lebesgue( \emptyset ) = 0$.
  \item $\sigma$-Subadditivität: Sind $E_0, E_1, \ldots \subset \real^n$, dann
   \[ \lebesgue^n \left( \bigcup_{k=0}^\infty E_k \right) \le \sum_{k=0}^\infty \lebesgue^n (E_k). \]
   Insbesondere ist $\lebesgue^n (E \cup F) \le \lebesgue^n (E) + \lebesgue^n (F)$.
  \item Monotonie: Wenn $E \subset F \subset \real^n$, dann ist $\lebesgue^n (E) \le \lebesgue^n (F)$.
  \item $\sigma$-Additivität: Seien $E_0, E_1, \ldots \in \sigma(\lebesgue^n)$ paarweise disjunkt. Dann ist
   \[ \lebesgue^n \left( \bigcup_{k=0}^\infty E_K \right) = \sum_{k=0}^\infty \lebesgue^n (E_k). \]
   Insbesondere ist $\lebesgue^n (E \cup F) = \lebesgue^n (E) + \lebesgue^n (F)$ für alle $E,F \in \sigma(\lebesgue^n)$ mit $E \cap F = \emptyset$.
  \item Stetigkeit von unten: Seien $E_0, E_1, \ldots \in \sigma(\lebesgue^n)$, sodass $E_k \uparrow E$, das heißt $E_0 \subset E_1 \subset \ldots$ und $E := \bigcup_{k=0}^\infty E_k$, dann ist
   \[ \lim_{k \to \infty} \lebesgue^n(E_k) =  \lebesgue^n (E). \]
  \item Stetigkeit von oben: Seien $E_0, E_1, \ldots \in \sigma(\lebesgue^n)$, sodass $E_k \downarrow E$, das heißt $E_0 \supset E_1 \supset \ldots$ und $E := \bigcap_{k=0}^\infty E_k$ und $\lebesgue^n(E_0) < \infty$, dann ist
   \[ \lim_{k \to \infty} \lebesgue^n(E_k) =  \lebesgue^n (E). \]
 \end{enumerate}
\end{thm}

Beweis siehe [MINT].

\subsubsection{Metrische Eigenschaften}
\begin{thm}
 Bezeichne $\lebesgue^n$ das äußere Lebesgue-Maß.
 \begin{enumerate}
  \item Jede Borelmenge ist $\lebesgue^n$-messbar, das heißt $\borel( \real^n ) \subset \sigma(\lebesgue^n)$. Genauer existiert für jedes $E \in \sigma(\lebesgue^n)$ eine $\lebesgue^n$-Nullmenge $N$ und eine $F_\sigma$-Menge\footnotemark $F$, so dass $E = F \cup N$.
  \item Wenn $E \subset \real^n$ beschränkt ist, dann ist $\lebesgue^n(E) < \infty$. Insbesondere ist $\lebesgue^n (K) < \infty$ für alle $K \subset \real^n$ kompakt.
 \end{enumerate}
\end{thm}
\footnotetext{Eine abzählbare Vereinigung abgeschlossener Mengen. Insbesondere sind Borel-Mengen $F_\sigma$-Mengen.}

\begin{proof}
 \begin{enumerate}
  \item Siehe [MINT] und vergleiche Übungsblatt 1.
  \item Sei $E \subset \real^n$ beschränkt. Dann existiert $r > 0$, so dass $E \subset Q_r(0)$. Also $\lebesgue^n(E) \le \lebesgue^n( Q_r(0) ) = r^n < \infty$. Zudem wissen wir aus Analysis 2: kompakte Teilmengen des $\real^n$ sind beschränkt. \qedhere
 \end{enumerate}
\end{proof}

Es folgt eine weitere maßtheoretische Eigenschaft:
\begin{kor}
 Das äußere Lebesgue-Maß ist $\sigma$-endlich, das heißt für jedes $E \subset \real^n$ existiert $E_0 \subset E_1 \subset \ldots$ mit $\lebesgue(E_k) < \infty$ und $E = \bigcup_{k=0}^\infty E_k$.
\end{kor}

\begin{proof}
 Sei $E \subset \real^n$. Sei $E_k$ der Schnitt von $E$ mit einer Kugel vom Radius $k$, also
 \[ E_k := B_k(0) \cap E. \]
 Dann ist wegen Satz (Teil 2) $\lebesgue^n(E_k) < \infty$ und offensichtlich $E_k \uparrow E$.
\end{proof}

\subsubsection*{Weiterführendes}
\textbf{Literatur:}
\begin{itemize}
 \item Forster: Analyis 2, Analysis 3
 \item Hildebrandt: Analyis 1, Analyis 2
 \item Königsberger: Analysis 1, Analysis 2
 \item Bauer: Maß- und Integrationstheorie
\end{itemize}

\textbf{Ergänzendes Seminar: ``Geometrische Maßtheorie''}
\begin{itemize}
 \item Hausdorff-Maße $\mathscr{H}^k \rightsquigarrow \int f \diffop \mathscr{H}^k$
 \item Flächen- und Ko-Flächenformeln
 \item Rektifizierbarkeit
 \item Teilnahme freiwillig, Vortrag halten möglich, Mail an \href{mailto:peter.hornung@tu-dresden.de}{peter.hornung@tu-dresden.de}
 \item Evans, Gariepy bzw. Ambrosio, Fusco, Pallara
\end{itemize}

\subsubsection{Euklidische Eigenschaften}
\textbf{Formale Anmerkung.} Wenn sich das Volumen einer Menge $E \subset \real^n$ elemtargeometrisch berechnen lässt, dann stimmt es mit dem Lebesgue-Maß $\lebesgue^n(E)$ überein. Mit anderen Worten: Das $n$-dimensionale Lebesgue-Maß ist die natürliche Fortsetzung des Volumenbegriffs auf die Klasse der Borel-messbaren Mengen $\borel(\real^n)$.

\begin{thm}[Verhalten unter affinen Transformationen]
 Sei $T \in \realmat{n}{n}$ invertierbar, sei $a \in \real^n$ und definiere $\Phi:\real^n \to \real^n$ durch $\Phi(x) = a + Tx$. Dann gilt für jede Borel-messbare Menge $E \in \real^n$:
 \begin{enumerate}
  \item $\Phi(E)$ ist Borel-messbar.
  \item $\lebesgue^n( \Phi(E) ) = | \det T | \lebesgue^n (E)$.
 \end{enumerate}
\end{thm}

\emph{Beweisidee.} 
\begin{itemize}
 \item Für $T = \lambda I$ ($I$ ist die Einheitsmatrix) siehe Übung $\lebesgue^n (a + \lambda E) = \lebesgue^n ( \lambda E ) = \lambda^n \lebesgue^n (E)$. 
 \item $\rightsquigarrow$ Naheliegend, dass auch für 
\[ T = \begin{pmatrix} 
        \lambda_0 & 0 & \cdots & \\
        0 & \lambda_1 & \ddots & \\
        \vdots  & \ddots & \ddots & 0 \\
          &  & 0 & \lambda_n
       \end{pmatrix} \]
 die behauptete Formel gilt.
 \item $\lebesgue^n$ ist invariant unter Rotationen und Spiegelungen ($T \in O(n)$).
 \item Jede Matrix $T$ ist von der Form $T = QRDR^{-1}$, wobei $D$ diagonal und $Q,R \in O(n)$.
\end{itemize}

Insbesondere für $T \in SO(n)$ $\Rightarrow$ $\det T = 1$ $\Rightarrow$ $\lebesgue^n$ ist rotationsinvariant.

\clearpage

\subsubsection{Transformationsformel}
Erinnerung an die Substitutionsregel für Riemann-Funktionen in Analysis 2: Sei $I \subset \real$ Intervall, $f:I \to \real$ und $\varphi:[a,b] \to \real$ stetig differenzierbar mit $\varphi([a,b]) \subset I$. Dann
\[ \int_{\varphi(a)}^{\varphi(b)} f = \int_a^b f(\varphi(x)) \varphi'(x) \diffop x. \]

Gegenstück für höhere Dimensionen:
\begin{thm}
 Seien $U,V \subset \real^n$ offen und sei $\Phi: U \to V$ ein $C^1$-Diffeomorphismus\footnotemark. Eine Funktion $f: V \to \real$ ist genau dann integrierbar, wenn $(f \circ \Phi) \cdot | \det \nabla \Phi |$ integrierbar ist und es gilt dann
 \[ \int_V f \diffop \lebesgue^n = \int_U (f \circ \Phi) | \det \nabla \Phi | \diffop \lebesgue^n. \]
\end{thm}
\footnotetext{$\Phi$ ist bijektiv, $\Phi \in C^1$, $\Phi^{-1} \in C^1$}

\begin{proof}
 Siehe [MINT]. Hier Spezialfall $U=V=\real^n$, $f = \chi_A$ und $\Phi(x) = a + Tx$ mit $T \in \realmat{n}{n}$ invertierbar. Aus Satz 5.4 folgt
 \begin{align*}
    \int_{\real^n} f \diffop \lebesgue^n = \lebesgue^n(A)
    &= \lebesgue^n ( \Phi( \Phi^{-1}( A ) )) \\
    &= | \det \nabla \Phi | \lebesgue^n( \Phi^{-1}(A) ) \\
    &= | \det \nabla \Phi | \int_{\real^n} \underbrace{\chi_{\Phi^{-1}(A)}}_{\chi_A \cdot \Phi} \diffop \lebesgue^n \\
    &= \int_{\real^n} | \det\nabla \Phi | f \circ \Phi \diffop \lebesgue^n. \qedhere
 \end{align*}
\end{proof}

\begin{exmp}
 \begin{enumerate}
  \item Translationen: $\Phi(x) = a + x$. Dann $\nabla \Phi = I$ und 
  \[ \int_{a+U} f \diffop \lebesgue^n = \int_U f(a+x) \diffop \lebesgue^n(x). \]
  \item $\Phi(x) = \lambda x$ für $\lambda > 0$. Dann ist $\nabla \Phi = \lambda I$, also $\det \nabla \Phi = \lambda^n$, zum Beispiel mit $U = B_1(0)$ folgt
  \[ \int_{B_\lambda(0)} f \diffop \lebesgue^n = \lambda^n \int_{B_1(0)} f(\lambda x) \diffop \lebesgue^n(x). \]
  \item Polarkoordinaten eines Punktes $x=(x_1,x_2,x_3) \in \real^3$ mit $(x_1,x_2) \ne 0$:
   \begin{itemize}
    \item $r := |x| =$ Länge des Vektors $x$
    \item $\theta :=$ Winkel zwischen $x_3$-Achse und Vektor $x$
    \item $\varphi :=$ Winkel zwischen $x_1$-Achse und Vektor $x$
   \end{itemize}
   Transformation von Polar- in euklidische Koordinaten:$\Phi: [0, \infty) \times [0,\pi] \times [0,2\pi] \to \real^3$, wobei
   \[ (r,\theta, \varphi) \mapsto ( r \sin \theta \cos \varphi, r \sin \theta \sin \varphi, r \cos \theta ). \]
   Rechnung (Übung) ergibt
   \[ \det \nabla \Phi( r, \theta, \varphi) = r^2 \sin \theta. \]
   Setze $U := (0, \infty) \times (0,\pi) \times (0,2\pi)$. Dann ist $\Phi: U \to V$ ein $C^1$-Diffeomorphismus, wobei
   \[ V := \Phi(U) = \real^3 \setminus ( [0,\infty) \times \{ 0 \} \times \real ) \]
   Transformationsformel $\Rightarrow$ eine Funktion $f: \real^3 \to \real$ ist genau dann $\lebesgue^3$-integrierbar, wenn $(f \circ \Phi) | \det \nabla \Phi| : U \to \real$ mit
   \[ (r, \theta, \varphi) \mapsto f( r \sin \theta \cos \varphi, r \sin \theta \sin \varphi, r \cos \theta ) r^2 \sin \theta \]
   integrierbar ist. Dann gilt
   \[ \int_{\real^3} f \diffop \lebesgue^3 = \int_0^\infty \int_0^\pi \int_0^{2 \pi} f( r \sin \theta \cos \varphi, r \sin \theta \sin \varphi, r \cos \theta ) r^2 \sin \theta \diffop \varphi \diffop \theta \diffop r. \]
   Hier wurde Satz \ref{sect:tonelli} verwendet.
   
   Anwendung dieser Formel: 
   \[ \lebesgue^3( B_R(0) ) = \int_{\real^3} \chi_{B_R(0)} \diffop \lebesgue^3 = \int_{\real^3} \chi_{[0,R)}(|x|) \diffop \lebesgue^3 (x). \]
   Hier verwendet: $\chi_{B_R(0)}(0) = \chi_{[0,R)}(|x|)$, denn $x \in B_R(0) \Leftrightarrow |x| \in [0,R)$. Das ist klar, denn
   \[ B_R(0) := \{ x \in \real^n : |x| < R \}. \]
   Also folgt
   \[ \begin{aligned}       
      \lebesgue^3( B_R(0) ) 
      &= \int_0^R \int_0^\pi \int_0^{2 \pi} r^2 \sin \Theta \diffop \varphi \diffop \Theta \diffop r \\
      &= 2 \pi \left( \int_0^R r^2 \diffop r \right) \left( \int_0^\pi \sin \Theta \diffop \Theta \right) \\
      &= \frac{4 \pi}{3} R^3.
      \end{aligned} \]
 \end{enumerate}
\end{exmp}

\subsubsection{Satz von Fubini-Tonelli}\label{sect:tonelli}
``Mehrdimensionale Integrale als iterierte eindimensionale Integrale''

Wir identifizieren in diesem Abschnitt $\real^{k+\ell} = \real^k \times \real^\ell$ und bezeichnen Punkte in $\real^{k + \ell}$ mit $(x,y)$, wobei $x \in \real^k$, $y \in \real^\ell$.

Funktionen $f: \real^{k+\ell} \to \real$ fassen wir auf als Funktionen $\real^k \times \real^\ell \to \real$, $(x,y) \mapsto f(x,y)$. Für $y \in \real^\ell$ definiere 
\[ f( \cdot, y ) : \real^k \to \real, \quad x \mapsto f(x,y). \]
Analog $f( x, \cdot ) := f(x,y)$.

Wir legen den Maßraum $(\real^n, \borel(\real^n), \lebesgue^n)$ zugrunde\footnote{``Integrierbar'' heißt also $\lebesgue^n$-integrierbar und Borel-messbar.}.

\begin{lem}[Cavalieri]
Sei $A \in \real^{k+\ell}$. Dann gilt
\begin{enumerate}
 \item Für alle $y \in \real^\ell$ ist der Schnitt $A_y := \{ x \in \real^k : (x,y) \in A \}$ eine Borelmenge in $\real^k$.
 \item Die Funktion $\real^\ell \to [0,\infty]$, $y \mapsto \lebesgue^n(A_y)$ ist Borel-messbar.
 \item $\lebesgue^{k+\ell} (A) = \int_{\real^\ell} \lebesgue^k (A_y) \diffop \lebesgue^\ell(y)$.
\end{enumerate}
\end{lem}

\begin{proof}
 Siehe [MINT].
\end{proof}

Betrachte zum Beispiel Würfel in $\real^3$
\begin{itemize}
 \item $\real^3 = \real \times \real^2$: $\int_{\real^2} \lebesgue^1 (A_y) \diffop \lebesgue^2(y)$, wir betrachten eine Fläche ($\real^2$) und ermitteln in jedem Punkt die Länge der zugehörigen Linie ($\real^1$).
 \item $\real^3 = \real^2 \times \real$: $\int_{\real^1} \lebesgue^2 (A_y) \diffop \lebesgue^1(y)$, hier gehen wir entlang einer Linie und ermitteln die zugehörige Fläche.
\end{itemize}

\begin{thm}[Tonelli]{}
 Sei $f: \real^{k+\ell} \to [0, \infty]$ Borel-messbar. Dann ist $f(\cdot,y):\real^3 \to [0,\infty]$ Borel-messbar für jedes $y \in \real^\ell$ und die Funktion
 \[ \real^\ell \to [0,\infty], \quad y \mapsto \int_{\real^k} f(x,y) \diffop \lebesgue^k (x) \]
 ist Borel-messbar. Es gilt
 \[ \begin{aligned} \int_{\real^{k+\ell}} f \diffop \lebesgue^{k+\ell} 
    &= \int_{\real^\ell} \left( \int_{\real^k} f(x,y) \diffop \lebesgue^k (x) \right) \diffop \lebesgue^\ell (y) \\
    &= \int_{\real^k} \left( \int_{\real^\ell} f(x,y) \diffop \lebesgue^\ell (y) \right) \diffop \lebesgue^k (x).
 \end{aligned} \]
 Insbesondere ist eine Borel-messbare Funktion $F: \real^k \times \real^\ell \to [-\infty, \infty]$ genau dann $\lebesgue^{k+\ell}$-integrierbar, wenn
 \[ \int_{\real^\ell} \left( \int |F(x,y)| \diffop \lebesgue^k (x) ) \right) \diffop \lebesgue^\ell (y) < \infty. \]
 Analog gilt das auch für $x \leftrightarrow y$ und $k \leftrightarrow \ell$.
\end{thm}

\begin{proof}
 Aufgrund des Lemmas gilt der Satz für die Funktion $f = \chi_A$ mit Borel-Menge $A \subset \real^{k+\ell}$. Wegen Linearität also auch für einfache Funktionen $f$. Nun sei $f$ wie im Satz. Gemäß Übungsblatt 2 existieren einfache Funktionen $\varphi_n : \real^{k+\ell} \to [0, \infty)$, so dass $\varphi_n (x,y) \uparrow f(x,y)$ für alle $(x,y) \in \real^{k+\ell}$ und $\int \varphi_n \diffop \lebesgue^{k+\ell} \rightarrow \int f \diffop \lebesgue^{k+\ell}$.
 
 Sei $y \in \real^\ell$ fest. Dann ist jedes $\varphi_n(\cdot, y)$ Borel-messbar in $\real^k$, weil der Satz für einfach Funktionen $\varphi_n$ gilt. Da $\varphi_n(\cdot, y) \uparrow f( \cdot, y)$ punktweise auf $\real^k$, folgt mit der monotonen Konvergenz:
 \[ \Phi_n(y) := \int \varphi_n(x,y) \diffop \lebesgue^k (x) \uparrow \underbrace{\int f(x,y) \diffop \lebesgue^k (x)}_{:= F(y)}. \]
 
 Nun sei $y$ wieder beliebig. $\Phi_n$ ist messbar wegen Satz für einfache Funktionen $\varphi_n$, daher ist $F$ auch messbar.
 
 Monotone Konvergenz $\Rightarrow$ $\int \Phi_n \diffop \lebesgue^\ell \rightarrow \int F \diffop \lebesgue^\ell$. Daher überträgt sich die Integralzerlegung von $\varphi_n$ auf $f$.
\end{proof}

\begin{kor}[Fubini]
 Sei $f:\real^{k+\ell} \to [-\infty,\infty]$ integrierbar. Dann existiert eine Nullmenge $N \subset \borel(\real^\ell)$, sodass $f(\cdot, y): \real^k \to [-\infty, \infty]$ für jedes $y \in \real^\ell \setminus N$ integrierbar ist, und auch 
 \[ \real^\ell \to [-\infty, \infty],\quad y \mapsto \int_{\real^k} f(x,y) \diffop \lebesgue^k (x) \]
 ($:= 0$ für $y \in N$) integrierbar über $\real^k$. (Dasselbe gilt für $x \leftrightarrow y$).
 
 Zudem gilt
 \[ \begin{aligned}
    \int_{\real^{k+\ell}} f \diffop \lebesgue^{k+\ell} 
    &= \int_{\real^k} \left( \int_{\real^\ell} f(x,y)  \diffop \lebesgue^\ell (y) \right) \diffop \lebesgue^k(x) \\
    &= \int_{\real^\ell} \left( \int_{\real^k} f(x,y)  \diffop \lebesgue^k (x) \right) \diffop \lebesgue^\ell(y).
    \end{aligned} \]
\end{kor}

\begin{proof}
 Wegen Tonelli sind die Funktionen $F_\pm: \real^\ell \to [0,\infty]$, $F_\pm := \int_{\real^k} f_\pm (x,y) \diffop \lebesgue^k(x)$ Borel-messbar mit 
 \[ \int_{\real^\ell} F_\pm \diffop \lebesgue^\ell = \int_{\real^\ell} \left( \int_{\real^k} f_\pm (x,y) \diffop \lebesgue^k(x) \right) \diffop \lebesgue^\ell = \int_{\real^{k+\ell}} f_\pm \diffop \lebesgue^{k+\ell} < \infty. \]
 Also existiert gemäßt Satz 2.6(b) eine Nullmenge $N \subset \real^\ell$, sodass $F_+(y) < \infty$ und $F_-(y) < \infty$ für alle $y \in \real^\ell \setminus N$. Also ist die im Korollar definierte Funktion integrierbar. Die Gleichung am Schluss folgt aus Tonelli, angewandt auf $f_+$ und $f_-$.
\end{proof}

\begin{exmp}
 Aus Cavalieri folgt für jede Borelmenge $A \in \real^{k+\ell}$: $A$ ist Nullmenge genau dann, wenn $\lebesgue^\ell( A_x ) = 0$ für $\lebesgue^k$-fast alle $x \in \real^k$. Hierbei ist $A_x := \{y \in \real^\ell:(x,y) \in A \}$.
\end{exmp}

\begin{folg}
 Graphen messbarer Funktionen sind Nullmengen. Genauer: Ist $U \subset \real^{n-1}$ eine Borelmenge und $f:U \to \real$ Borel-messbar, dann ist der $\mathrm{graph}(f) := \{ (x,f(x)) : x \in U \} \subset \real^n$ Borel-messbar und $\lebesgue^n (\mathrm{graph}(f) ) = 0$.
\end{folg}

\begin{proof}
 Die Borel-Messbarkeit folgt aus $\mathrm{graph}(f) = \phi^{-1}( \{ 0 \} )$, wobei $\phi: U \times \real \to \real$ mit $\phi(x,y) = y -f(x)$ eine Borel-messbare Funktion ist. Er ist eine Nullmenge, weil für alle $x \in U$ $\mathrm{graph}(f)_x = \{ f(x) \}$ eine $\lebesgue^1$-Nullmenge ist.
\end{proof}